\documentclass{article}

\usepackage{lmodern}
\usepackage{pgfgantt}

\newcommand{\ganttmilestonewithdate}[3]{\ganttmilestone{#1}{#3}\ganttmilestone[inline]{\footnotesize\bfseries\color{red}#2}{#3}}

\makeatletter
\newcommand*{\centerfloat}{%
  \parindent \z@
  \leftskip \z@ \@plus 1fil \@minus \textwidth
  \rightskip\leftskip
  \parfillskip \z@skip}
\makeatother

\title{Systematic Techniques for Testing Concurrent and Distributed Functional Programs \\ \large Thesis Outline}
\author{Michael Walker}
\date{September, 2016}

\begin{document}

\maketitle

\begin{abstract}
  This document provides an outline for my Ph.D thesis, along with a
  plan of work for the rest of the programme.
\end{abstract}

\tableofcontents

\pagebreak

\section{Introduction}

% Motivation
Concurrency is notoriously difficult to get right~\cite{yang2013}, and
programmer mistakes can have dire consequences~\cite{leveson1993}. The
problem stems from the nondeterminism of scheduling: the same program
with the same inputs may produce different results depending on the
schedules chosen at execution time. This makes it difficult to use
traditional testing techniques with concurrent programs, which rely on
test execution to be deterministic, and to match ``real''
execution. So-called ``Heisenbugs'' make it difficult to be confident
of the correctness of concurrent programs: no bug has been observed
during the testing process, but how do we \emph{know} that there
aren't any?

% Explain SCT
\paragraph{Systematic Concurrency Testing}
SCT~\cite{emmi2011,musuvathi2007,musuvathi2008,thomson2014} is a
family of techniques for avoiding the problem of nondeterminism when
writing tests. These techniques aim to execute a large number of
schedules, whilst typically also making use of local knowledge of the
program to reduce the number of schedules needed to be confident of an
accurate result. By trying many schedules, we can be confident that
any bugs which have not been found are unlikely to be exhibited in
practice.

SCT overcomes the scheduling problem by forcing a concurrent program
to use instead a scheduler implemented as part of the testing
framework: either by overriding the concurrency primitives of the
language, or by modifying the program under test to call out to this
new scheduler. Once the scheduler is under control, schedules can be
recorded and replayed, giving reproducibility. Furthermore, by
observing which scheduling decisions are available at each decision
point, possible schedules can be systematically explored, making
different decisions on subsequent executions.

% State thesis
\paragraph{Concurrent Functional Programs}
I am investigating applying SCT techniques to concurrent functional
programs, and am doing my work in Haskell. This is an interesting
arena because concurrency testing algorithms are typically presented
in a very minimal setting, whereas GHC Haskell has a very rich set of
concurrency primitives which makes efficient testing difficult.

Furthermore, distributed computing is becoming more important, as
systems need to scale beyond what a single machine can provide. Cloud
Haskell~\cite{epstein2011} is an implementation of distributed
computing using Erlang-style message passing. Prior work on applying
SCT techniques to distributed systems has assumed that networks are
perfectly reliable, which is not true in practice; other prior work on
model checking distributed systems results in very lengthy testing
periods as all possible cases are considered. We develop an
\emph{incomplete} testing approach which is biassed towards exploring
diverse network conditions.

Finally, there is an important question any programmer must ask
themselves when testing a system: \emph{what should be tested?} We
plan to investigate applying property-based testing, and in
particularly automated property discovery, to concurrency libraries.

% Roadmap
This document provides an outline for my Ph.D thesis along with a plan
of work for the rest of my programme. Section~\ref{sec:outline}
details the provisional thesis structure. Section~\ref{sec:plan}
details the plan of work.

\subsection{Scope change}

Due to my internship this summer, I had the idea of applying
systematic concurrency testing to distributed systems, and have
achieved promising results in this area.

In order to give adequate time to explore the other topics, the
``types for concurrency'' (statically enforcing correctness properties
in concurrent programs) and ``concurrency introduction''
(heuristic-based automated transformation of sequential code into
concurrent code) topics suggested in the qualifying dissertation have
been removed.

\section{Thesis structure}
\label{sec:outline}

The thesis will be divided into two broad parts. The first part
explores the testing of concurrent and distributed Haskell programs,
and the second explores generating test cases.

\paragraph{Chapter 1: Introduction} This chapter will present
motivating issues, give a high-level overview of systematic
concurrency testing, give an overview of the thesis, and list
publications that contributed to the thesis.

\paragraph{Part I: Testing Concurrent Haskell Programs}
\begin{itemize}
\item \textbf{Chapter 2: Systematic Concurrency Testing} This
  chapter will describe more thoroughly what systematic concurrency
  testing is. It will present a literature review of works in the
  area. It will be divided into three sections:

  \begin{itemize}
  \item \emph{Section 2.1: Fundamentals of Systematic Concurrency Testing}

    This section will present the fundamental concepts and techniques
    behind systematic concurrency testing.

  \item \emph{Section 2.2: Using Incomplete Scheduling}

    This section will review incomplete approaches to SCT based on
    controlled scheduling. Incomplete methods greatly reduce the
    number of executions tested, but are imprecise: there has
    therefore been much work on the development of incomplete
    approaches which tend to have good bug-finding abilities.

  \item \emph{Section 2.3: Using Complete Reduction}

    This section will review complete approaches to SCT based on
    partial-order reduction. Partial-order reduction, common in model
    checking, consists of observing the behaviour of the program under
    test and using this information to prune needless executions
    before executing them.
  \end{itemize}

\item \textbf{Chapter 3: Concurrent Haskell} This chapter will
  review the rich concurrency abstraction provided by GHC Haskell. It
  will begin with a brief review of the monadic approach to I/O used
  in Haskell, so that reading the thesis will not require in-depth
  Haskell knowledge. This chapter will also include an overview of
  Cloud Haskell~\cite{epstein2011}.

\item \textbf{Chapter 4: Testing Concurrent Programs} The first
  contribution of the thesis. This chapter will describe an SCT tool
  for GHC Haskell~\cite{walker2015}. This chapter will focus on:

  \begin{itemize}
  \item The aspects of Haskell which made SCT particularly easy or
    difficult to implement. For example, the typeclass mechanism in
    Haskell made abstracting over the concurrency primitives simple,
    but the focus on immutability prevented a straightforward and
    direct translation of a standard SCT algorithm.
  \item The structure of the tool.
  \item A number of case studies of its use.
  \item An evaluation of the completeness and usefulness of the tool.
  \end{itemize}

\item \textbf{Chapter 5: Testing Distributed Programs} The second
  contribution of the thesis. This chapter will describe our algorithm
  for incomplete, but behaviourally varied, testing of distributed
  systems where in-order reliable message delivery cannot be
  guaranteed. If message delivery is reliable, a distributed system
  can be thought of as a concurrent program using message-passing and
  no shared state. We propose a model of the network, inspired by work
  on applying SCT to relaxed-memory systems~\cite{zhang2015}, to
  remove that reliability requirement. The implementation in Haskell
  will be discussed, but the technique is not Haskell-specific.
\end{itemize}

\paragraph{Part II: Test Case Generation}
\begin{itemize}
\item \textbf{Chapter 6: Property-based Testing} This chapter will
  review the literature of property-based testing and automated
  property discovery.

\item \textbf{Chapter 7: Properties for Concurrency} The third
  contribution of the thesis. This chapter will develop techniques to
  generate test-cases for concurrency libraries, such as collections
  of concurrent data structures. The cases will be in the form of
  properties, similar to what a property-based testing tool such as
  QuickSpec~\cite{claessen2010} produces.
\end{itemize}

\paragraph{Part III: Conclusions and Future Directions}
\begin{itemize}
\item \textbf{Chapter 8: Conclusions} This chapter will give an
  overview and draw conclusions from the work set out in the previous
  chapters, including a summary of results achieved and thesis
  limitations.
\item \textbf{Chapter 9: Future Directions} This chapter will
  examine avenues for future work.
\end{itemize}

\section{Plan}
\label{sec:plan}

This section summarises the planned work throughout the rest of my
programme.

\subsection{Current progress}

\paragraph{Testing Concurrent Programs} Work on this topic is
complete. It gave rise to one publication~\cite{walker2015}, and a
later technical report~\cite{YCS-2016-503} covering developments since
the paper. A journal version of the technical report is now being
revised with an intent to submit by the end of the calendar
year. \emph{Risks:} the journal version may be rejected, but even if
it is the process of writing it will lead to a chapter version.

\paragraph{Testing Distributed Programs} During August, I visited
Imperial College London, where Alastair Donaldson and Paul Thomson,
who have both done the only (that I am aware of) work on applying SCT
to distributed systems~\cite{deligiannis2015,deligiannis2016} are, to
discuss my ideas. Promising results have been found, and a paper has
been submitted to PLDI. I am currently investigating applying a
component of this work (a scheduling algorithm inspired by swarm
testing~\cite{groce2012}) to normal non-distributed concurrent
programs.

\paragraph{Properties for Concurrency} I have some initial thoughts,
but nothing has been started yet. \emph{Risks:} Even in the restricted
space of property-testing, it may not be possible to generate
interesting test-cases without a combinatorial explosion. This could
be resolved by exploring a restricted property generation in the form
of instantiating templates, which may still be expressive enough to be
interesting.

\subsection{Schedule}

\begin{figure}[h!]
  \centerfloat
  \begin{ganttchart}[
    y unit title=0.5cm,
    y unit chart=0.7cm,
    x unit=0.7cm,
    vgrid,hgrid,
    title height=1,
    milestone inline label node/.append style={right=3mm}]{1}{16}
    \gantttitle{2016}{4}
    \gantttitle{2017}{12}\\
    \gantttitle{Sep}{1}
    \gantttitle{Oct--Dec}{3}
    \gantttitle{Jan--Mar}{3}
    \gantttitle{Apr--Jun}{3}
    \gantttitle{Jul--Sep}{3}
    \gantttitle{Oct--Dec}{3}\\

    \ganttbar{Distributed systems}{1}{3}\\
    \ganttbar{D\'{e}j\`{a} Fu journal article}{4}{4}\\
    \ganttbar{Swarm testing}{4}{4}\\
    \ganttbar{Test-case generation}{5}{10}\\
    \ganttbar{Thesis writing}{5}{16}\\
    \ganttmilestonewithdate{Thesis audit}{31st}{7}\\
    \ganttmilestonewithdate{Thesis seminar}{31st}{14}\\
    \ganttmilestonewithdate{Thesis submission}{31st}{16}
  \end{ganttchart}
  \caption{A Provisional Schedule of Work.}
\end{figure}

The different areas for exploration are listed in the first four
lines. The writing of papers, other than the D\'{e}j\`{a} Fu journal
article, is not explicitly listed, as this will happen after the
successful exploration of the proposed areas of research. The next
three formally required assessments are also shown.  The planned
thesis submission date is 31st December 2017, the thesis submission
deadline is 31st December 2018.

\bibliographystyle{plain}
\bibliography{references}

\end{document}
