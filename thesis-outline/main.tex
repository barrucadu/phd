\documentclass{article}

\usepackage{lmodern}
\usepackage{pgfgantt}

\newcommand{\ganttmilestonewithdate}[3]{\ganttmilestone{#1}{#3}\ganttmilestone[inline]{\footnotesize\bfseries\color{red}#2}{#3}}

\makeatletter
\newcommand*{\centerfloat}{%
  \parindent \z@
  \leftskip \z@ \@plus 1fil \@minus \textwidth
  \rightskip\leftskip
  \parfillskip \z@skip}
\makeatother

\title{Systematic Techniques for Testing Concurrent and Distributed Functional Programs \\ \large Thesis Outline}
\author{Michael Walker}
\date{September, 2016}

\begin{document}

\maketitle

\begin{abstract}
  This document provides an outline for my Ph.D thesis, along with a
  plan of work for the rest of the programme.
\end{abstract}

\tableofcontents

\pagebreak

\section{Introduction}

% Motivation
Concurrency is notoriously difficult to get right\cite{yang2013}, and
programmer mistakes can have dire consequences\cite{leveson1993}. The
problem stems from the nondeterminism of scheduling: the same program
with the same inputs may produce different results depending on the
schedules chosen at execution time. This makes it difficult to use
traditional testing techniques with concurrent programs, which rely on
the result of executing a test to be deterministic. So-called
``Heisenbugs'' make it difficult to be confident of the correctness of
concurrent programs: no bug has been observed during the testing
process, but how do we \emph{know} that there aren't any? This is a
concern in any testing regime, but nondeterminism makes it
particularly different when concurrency is involved.

Despite the difficulty, concurrency is important for producing many
real-world applications. For example, applications with a lot of input
and output can be more responsive by executing I/O asynchronously.
Concurrency is a useful program structuring technique, and it is here
to stay.

% Explain SCT
\emph{Systematic concurrency testing}
(SCT)\cite{emmi2011,musuvathi2007,musuvathi2008,thomson2014} is a
family of related techniques for avoiding the problem of
nondeterminism when writing tests. SCT techniques aim to execute a
large number of schedules, whilst typically also making use of local
knowledge of the program to reduce the number of schedules needed to
be confident of an accurate result. By trying many schedules, we can
be confident that any bugs which have not been found are unlikely to
be exhibited.

SCT overcomes the scheduling problem by forcing a concurrent program
to use instead a scheduler implemented as part of the testing
framework: either by overriding the concurrency primitives of the
language, or by modifying the program under test to call out to this
new scheduler.

Once the scheduler is under control, schedules can be recorded and
replayed, giving reproducibility. Furthermore, by observing which
scheduling decisions are available at each decision point, possible
schedules can be systematically explored, making different decisions
on subsequent executions.

% State thesis
My thesis is:

\begin{quote}
  Systematic concurrency testing can be applied to functional
  programs. It can be applied to variants of the actor model of
  concurrency, such as distributed systems. It can be used to discover
  useful test cases for existing code. There are cases in idiomatic
  code where deterministic concurrency can be safely introduced
  automatically. Furthermore, testing is not necessary: behavioural
  properties of concurrent programs can be encoded at the type level
  and discharegd statically.
\end{quote}

% Roadmap
This document provides an outline for my Ph.D thesis along with a plan
of work for the rest of my programme. Section 2 details the
provisional thesis structure. Section 3 details the plan of work.

\subsection{Scope change}

Due to my internship this summer, I had the idea of applying
systematic concurrency testing to distributed systems, and have begun
investigating.

\section{Thesis structure}

This section gives a description of each chapter in my thesis.

\paragraph{Chapter 1: Introduction} This chapter will explain
systematic concurrency testing, give an overview of the thesis, and
list publications that contributed to the thesis.

\paragraph{Chapter 2: Systematic Testing of Concurrent Programs} This
chapter will describe more thoroughly what systematic concurrency
testing is. It will present a literature review of works in the
area. It will be divided into three sections:

\begin{itemize}
\item \emph{Section 2.1: Fundamentals of systematic concurrency testing}

  This section will present the fundamental concepts and techniques
  behind systematic concurrency testing.

\item \emph{Section 2.2: Complete and incomplete methods}

  There are two approaches to reducing the schedule-space explored
  when performing SCT. Complete methods mostly focus on applying
  partial-order techniques, common in model checking, to the testing
  setting. This has the benefit of reducing the space greatly, but
  without sacrificing completeness of testing. Incomplete methods do
  make this sacrifice, to obtain even greater reductions. The
  hypothesis behind incomplete methods is that \emph{most} program
  errors can be found with only a small subset of the potential
  executions.
\end{itemize}

\paragraph{Chapter 3: A Concurrency Testing Tool for Haskell} This
chapter will describe an SCT tool for GHC Haskell\cite{walker2015}.
This was a challenging effort, as SCT algorithms are typically
presented in a very minimal setting, whereas GHC Haskell has a very
rich set of concurrency primitives. This chapter will focus on:

\begin{itemize}
\item Haskell concurrency and its operational behaviour, including the
  memory model.
\item The aspects of Haskell which made SCT particularly easy or
  difficult to implement. For example, the typeclass mechanism in
  Haskell made abstracting over the concurrency primitives simple, but
  the great focus on immutable data structures prevented a
  straightforward and direct translation of a standard SCT algorithm.
\item The structure of the tool.
\item A number of case studies of its use.
\item An evaluation of the completeness and usefulness of the tool.
\end{itemize}

\paragraph{Chapter 4: Finding Faults in Distributed Systems} This
chapter will develop techniques for the systematic testing of
distributed systems where in-order reliable message delivery cannot be
guaranteed. If message delivery is reliable, a distributed system can
be thought of as a concurrent program using message-passing and no
shared state. We propose a novel model of the network, inspired by
work on applying SCT to relaxed-memory systems\cite{zhang2015}, to
remove that reliability requirement.

\paragraph{Chapter 5: Types for Concurrency} This chapter will develop
new a new type-level approach to encode behavioural properties for
concurrent programs in the type system. Prior examples show using
indexed monads to ensure correct usage of shared communication
channels\cite{pucella2008} and to guarantee that locks held are
released\cite{kiselyov2010}. Indexed monads allow extra type-level
information to be carried through a computation, and it is possible
there is scope for further development here.

\paragraph{Chapter 6: Test-case Generation} This chapter will develop
techniques to generate test-cases for concurrency libraries, such as
collections of concurrent data structures. The cases will be in the
form of properties, similar to what a property-based testing tool such
as QuickSpec\cite{claessen2010} produces.

\paragraph{Chapter 7: Concurrency Introduction} This chapter will
develop a tool to automatically transform sequential generate-and-test
code into concurrent code with the same semantics. As a purely static
approach is not feasible in general\cite{calderon2015}, developing
heuristics to determine where this transformation will be beneficial
is necessary. It is hoped that the restricted setting of
generate-and-test, rather than the general case of whole-program
automated parallelisation, will make this feasible. This work will
then be examined as a large test-case for the work developed in
Chapter 3.

\paragraph{Chapter 8: Conclusions and Future Work} This chapter will
give an overview and draw conclusions from the work set out in the
previous chapters, including a summary of results achieved and thesis
limitations. Then, the chapter will examine some avenues for future
work.

\section{Plan}

This section summarises the planned work throughout the rest of my
programme.

\subsection{Current progress}

\paragraph{Testing Concurrent Programs} I have written a literature
review, which was included as a chapter in my Qualifying Dissertation.
This will be expanded and updated, and included as a chapter in the
thesis.

\paragraph{A Concurrency Testing Tool for Haskell} Work on this topic
is complete. It gave rise to one publication\cite{walker2015}, and a
later technical report\cite{YCS-2016-503} covering developments since
the paper. I began writing a journal version of the technical report,
which will be continued after the distributed systems
work. \emph{Risks:} the journal version may be rejected, but even if
it is the process of writing it will lead to a chapter version.

\paragraph{Finding Faults in Distributed Systems} Work on this topic
is ongoing. During August, I visited Imperial College London, where
Alastair Donaldson and Paul Thomson, who have both done the only (that
I am aware of) work on applying SCT to distributed
systems\cite{deligiannis2015,deligiannis2016} are. They thought my
ideas promising, and I have preliminary results showing feasibility. I
have had an extended abstract detailing my approach accepted by the
York Doctoral Symposium. I plan to write a fuller account and submit
to a conference.

\paragraph{Types for Concurrency} I have some initial thoughts, but
have not done any work yet. \emph{Risks:} There may just be too much
prior work on session typing to catch up to a point where contributing
is possible in three months. This can be mitigated by idenfiying and
only reading the key papers in the field; if there is just too much
work to do, then it may be better to drop this topic entirely and
further development of another topic, sucha s the distributed systems
work.

\paragraph{Test-case generation} I have some initial thoughts, but
nothing has been started yet. \emph{Risks:} Even in the restricted
space of property-testing, it may not be possible to generate
interesting test-cases without a combinatorial explosion. This could
be resolved by exploring a restricted property generation in the form
of instantiating templates, which may still be expressive enough to be
interesting.

\paragraph{Concurrency introduction} I have previously developed a
library for expressing concurrent generate-and-test computations, but
nothing has been started for the heuristics yet. \emph{Risks:} It may
not be possible to find heuristics that work well. This can be
mitigated by making the concurrency transformation optional, or by
allowing the programmer to indicate where concurrency would be useful.

\subsection{Schedule}

\begin{figure}[h!]
  \centerfloat
  \begin{ganttchart}[
    y unit title=0.5cm,
    y unit chart=0.7cm,
    x unit=0.7cm,
    vgrid,hgrid,
    title height=1,
    milestone inline label node/.append style={right=3mm}]{1}{16}
    \gantttitle{2016}{4}
    \gantttitle{2017}{12}\\
    \gantttitle{Sep}{1}
    \gantttitle{Oct--Dec}{3}
    \gantttitle{Jan--Mar}{3}
    \gantttitle{Apr--Jun}{3}
    \gantttitle{Jul--Sep}{3}
    \gantttitle{Oct--Dec}{3}\\

    \ganttbar{Distributed systems}{1}{3}\\
    \ganttbar{D\'{e}j\`{a} Fu journal article}{4}{4}\\
    \ganttbar{Types for concurrency}{5}{7}\\
    \ganttbar{Test-case generation}{8}{10}\\
    \ganttbar{Concurrency introduction}{11}{13}\\
    \ganttbar{Thesis writing}{4}{16}\\
    \ganttmilestonewithdate{Thesis audit}{31st}{6}\\
    \ganttmilestonewithdate{Thesis seminar}{31st}{14}\\
    \ganttmilestonewithdate{Thesis submission}{31st}{16}
  \end{ganttchart}
  \caption{A Provisional Schedule of Work.}
\end{figure}

The different areas for exploration are listed in the first five
lines. The writing of papers, other than the D\'{e}j\`{a} Fu journal
article, is not explicitly listed, as this will happen after the
successful exploration of the proposed areas of research. The next
three formally required assessments are also shown.  The planned
thesis submission date is 31st December 2017, the thesis submission
deadline is 31st December 2018.

\bibliographystyle{plain}
\bibliography{references}

\end{document}
