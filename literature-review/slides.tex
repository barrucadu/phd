\documentclass[12pt]{beamer}
\usetheme{default}
\usefonttheme{serif}

\usepackage{lmodern}
\usepackage[T1]{fontenc}

\author{Michael Walker}
\title{Runtime Verification}
\institute{Department of Computer Science\\
  University of York\\
  \texttt{msw504@york.ac.uk}
}

\begin{document}

\begin{frame}[plain]
  \titlepage
\end{frame}

%%%%% 1 Minute (1)

\begin{frame}{Outline}
  \tableofcontents

  \begin{center}
    See the handout for literature references.
  \end{center}
\end{frame}

%%%%% 5 Minutes (6)

\section{Static Analysis}
\label{sec:statann}

\begin{frame}{Static Analysis}
  % Very quick overview, cover formal specification, proof assistants,
  % proof-carrying code, and certifying compilers. Get a quite saying
  % it's hard.

  % PAPERS: 
\end{frame}

%%%%% 15 Minutes (21)

\section{Runtime Verification}
\label{sec:runver}

\begin{frame}{Runtime Verification}
  % Overview of RV, quote seminal paper

  % PAPERS: 
\end{frame}

%%%%%%%%%% 7 Minutes (13)

\subsection{Design by Contract}
\label{sec:runver-dbc}

\begin{frame}{Design by Contract}
  % Overview of DbC, quote seminal paper (Eiffel!)

  % PAPERS: 
\end{frame}

\subsubsection{Annotation Techniques}
\label{sec:runver-dbc-ann}

\begin{frame}{Design by Contract: Annotation Techniques}
  % Talk mostly about JML for source-level, compare and contract
  % BCSL/BML for bytecode-level

  % PAPERS: 
\end{frame}

\begin{frame}{Design by Contract: Annotation Techniques}
  % Example contract, talk through
\end{frame}

\subsubsection{Aspect-Oriented Programming Techniques}
\label{sec:runver-sbc-aop}

\begin{frame}{Design by Contract: AOP Techniques}
  % Briefly explain AOP (maybe find a seminal paper?), say this can be
  % used with annotation techniques (eg, AspectJML), but here I'm
  % focusing on writing contracts in the host language and then
  % combining the program logic with the contract checking with
  % AOP. Explain and compare jContractor and ezContractor.

  % PAPERS: 
\end{frame}

%%%%%%%%%% 7 Minutes (20)

\subsection{Trace Analysis}
\label{sec:runver-trace}

\begin{frame}{Trace Analysis}
  % Overview of traces, quote seminal paper (Hoare?), explain how we
  % can generate traces by triggering events at program points
  % (find good paper), and how we can specify properties of programs
  % as properties of traces.

  % PAPERS: 
\end{frame}

\subsubsection{Logics}
\label{sec:runver-trace-log}

\begin{frame}{Trace Analysis: Logics}
  % Overview of regular properties and LTL, find seminal papers using
  % those for RV. Example LTL predicate, explain.

  % PAPERS: 
\end{frame}

\subsubsection{Monitor-Oriented Programming}
\label{sec:runver-trace-mop}

\begin{frame}{Trace Analysis: Monitor-Oriented Programming}
  % Explain MOP, cite MOP/JavaMOP papers.

  % PAPERS: 
\end{frame}

%%%%%%%%%% 1 Minutes (21)

\subsection{Trace Analysis by Contract?}
\label{sec:runver-tbc}

\begin{frame}{Trace Analysis by Contract?}
  % Mention how abstract data types & ghost/model fields in JML/BML
  % let us treat traces, at specification time, as a sequence of
  % events, and reason about that.
\end{frame}

%%%%% 4 Minutes (25)

\section{Conclusions}
\label{sec:conc}

\begin{frame}{Conclusions}
  % Briefly summarise RV and state typical downsides. Maybe find a
  % paper comparing overheads of various approaches.

  % PAPERS: 
\end{frame}

\subsection{Runtime Verification has High Overheads}
\label{sec:conc-over}

\begin{frame}{High Overheads}
  % Explain that instrumenting at runtime inevitably leads to
  % overheads. Mention recent JavaMOP work on reducing that.

  % PAPERS: 
\end{frame}

\subsection{It's not an either/or choice}
\label{sec:conc-dich}

\begin{frame}{It's not an either/or choice}
  % Mention that we can use static analysis / runtime verification to
  % inform each other. eg, not generating monitors for things we can
  % statically check (find a paper), and using monitors to filter
  % static analysis results (mention recent work)

  % PAPERS: 
\end{frame}

\begin{frame}{References}
  \bibliographystyle{plain}
  \bibliography{references}
\end{frame}

\end{document}