\chapstart Concurrency is not in standard Haskell, so here we restrict our
interest to GHC.  In this chapter we give an overview of the concurrency
functionality we use.  These operations are available in the
\package{concurrency} library\footnote{Available at
\url{https://hackage.haskell.org/package/concurrency}.  It is a collection of
functionality provided by the \package{atomic-primops}, \package{base}, and
\package{stm} libraries.}.  We cover the basic use of
concurrency~\sref{concurrent_haskell-threads}, the memory
model~\sref{concurrent_haskell-mmodel}, software transactional
memory~\sref{concurrent_haskell-stm}, and finally
exceptions~\sref{concurrent_haskell-exc}.

Readers already familiar with Haskell's concurrency functionality may wish to
skip this chapter.

\section{Threads and Mutable Variables}
\label{sec:concurrent_haskell-threads}

Threads let a program do multiple things at once.  Every program has at least
one thread, which starts where \verb|main| does and runs until the program
terminates.  A thread is the basic unit of concurrency.  It lets us pretend
(with parallelism, it might even be true!) that we're computing multiple things
at once.

\paragraph{Basic threading}
A thread can be started using the \verb|forkIO| function:

\begin{verbatim}
forkIO :: IO () -> IO ThreadId
\end{verbatim}

This starts executing its argument in a separate thread.  It also gives us back
a \verb|ThreadId| value, which can be used to kill the thread.  A thread can get
its own \verb|ThreadId| with the \verb|myThreadId| function:

\begin{verbatim}
myThreadId :: IO ThreadId
\end{verbatim}

\paragraph{Capabilities}
In a real machine, there are multiple processors and cores.  It may be that a
particular application of concurrency is only a net gain if every thread is
operating on a separate core, so that threads are not interrupting each other.
GHC uses a \emph{green threading} model, where Haskell threads are multiplexed
onto a much smaller number of operating system threads.  The number of operating
system threads is referred to as the number of \emph{capabilities} or
\emph{Haskell execution contexts}.  Only operating system threads have the
possibility of executing truly in parallel.

We can fork a thread to a particular capability with the \verb|forkOn| function:

\begin{verbatim}
forkOn :: Int -> IO () -> IO ThreadId
\end{verbatim}

The capability number is interpreted modulo the total number of capabilities,
which can be queried (and set):

\begin{verbatim}
getNumCapabilities :: IO Int
setNumCapabilities :: Int -> IO ()
\end{verbatim}

\paragraph{Scheduling}
The GHC scheduler is good, but sometimes we have domain knowledge which lets us
do better.  There are two ways to influence how threads are scheduled: we can
yield control to another thread, or delay the current thread for a period of
time:

\begin{verbatim}
yield       :: IO ()
threadDelay :: Int -> IO ()
\end{verbatim}

\section{Shared References and the Memory Model}
\label{sec:concurrent_haskell-mmodel}

There are two main types of shared variable in GHC Haskell, with different
semantics.

\paragraph{The \texttt{IORef} type}
An \verb|IORef| is just a mutable location in memory holding a normal Haskell
value:

\begin{verbatim}
newIORef   :: a -> IO (IORef a)
readIORef  :: IORef a -> IO a
writeIORef :: IORef a -> a -> IO ()
\end{verbatim}

\paragraph{The \texttt{MVar} type}
An \verb|MVar| is a shared variable with two possible states: \emph{full} or
\emph{empty}.  An \verb|MVar| can be created in either state:

\begin{verbatim}
newMVar      :: a -> IO (MVar a)
newEmptyMVar :: IO (MVar a)
\end{verbatim}

Writing to a full \verb|MVar| blocks until it is empty, and reading or taking
from an empty \verb|MVar| blocks until it is full.  There are also non-blocking
functions which return an indication of success:

\begin{verbatim}
putMVar     :: MVar a -> a -> IO ()
tryPutMVar  :: MVar a -> a -> IO Bool
readMVar    :: MVar a -> IO a
tryReadMVar :: MVar a -> IO a
takeMVar    :: MVar a -> IO a
tryTakeMVar :: MVar a -> IO (Maybe a)
\end{verbatim}

The mutual exclusion behaviour of \verb|MVar|s means that computations can
become \emph{deadlocked}.  For example, deadlock occurs if every thread tries to
take from the same \verb|MVar|.  This can sometimes be detected, as we shall see
in \chpref{dejafu}.

\paragraph{Memory model}
Unlike the \verb|MVar|, \verb|IORef| operations are not \emph{synchronised}.
Reads and writes between threads may be re-ordered.  The
documentation\footnote{\url{https://hackage.haskell.org/package/base-4.10.0.0/docs/Data-IORef.html\#g:2}}
has this to say:

\begin{quote}
  In a concurrent program, \verb|IORef| operations may appear out-of-order to
  another thread, depending on the memory model of the underlying processor
  architecture. For example, on x86, loads can move ahead of stores.

  The implementation is required to ensure that reordering of memory operations
  cannot cause type-correct code to go wrong. In particular, when inspecting the
  value read from an \verb|IORef|, the memory writes that created that value
  must have occurred from the point of view of the current thread.
\end{quote}

For testing purposes, we support the Total Store Order~(TSO) and Partial Store
Order~(PSO) models~\sref{dejafu-execution}.  Many other operations are
synchronised, and act as a \emph{barrier} to re-ordering.  Reading or writing to
an \verb|MVar| does; executing an STM transaction does; throwing an asynchronous
exeception does; and the atomic \verb|IORef| operations do:

\begin{verbatim}
atomicWriteIORef  :: IORef a -> a -> IO ()
atomicModifyIORef :: IORef a -> (a -> (a, b)) -> IO b
\end{verbatim}

\paragraph{Compare-and-swap}
Modern processor architectures provide an atomic \emph{compare-and-swap}
instruction, which is typically used in implementing high-performance lock-free
algorithms.  The \package{atomic-primops} package exposes this to Haskell code:

\begin{verbatim}
readForCAS :: IORef a -> IO (Ticket a)
peekTicket :: Ticket a -> a
\end{verbatim}

A \verb|Ticket| is a proof that a value has been observed inside an \verb|IORef|
at some prior point.  Given this proof, the programmer can atomically change the
value inside the \verb|IORef| later if it has not been modified:

\begin{verbatim}
casIORef :: IORef a -> Ticket a -> a -> IO (Bool, Ticket a)
\end{verbatim}

There is also a variant of \verb|atomicModifyIORef| using compare-and-swap:

\begin{verbatim}
atomicModifyIORefCAS :: IORef a -> (a -> (a, b)) -> Io b
\end{verbatim}

Both \verb|casIORef| and \verb|atomicModifyIORefCAS| are synchronised operations
which act as a barrier to re-ordering.

\section{Software Transactional Memory}
\label{sec:concurrent_haskell-stm}

Shared variables are nice, until we need more than one.  As we can only claim
\emph{one} \verb|MVar| atomically, it seems we need to introduce an \verb|MVar|
to control access to \verb|MVar|s!  This is unwieldy and prone to bugs.

Software transactional memory~(STM) is the solution.  STM is based upon the idea
of atomic \emph{transactions}.  An STM transaction consists of one or more
operations over a collection of \emph{transaction variables}, where a
transaction may be aborted part-way through, with all its effects rolled back.
Arbitrary effects are not permitted, so STM actions take place in the \verb|STM|
monad.

\paragraph{The \texttt{TVar} type}
Transaction variables, or \verb|TVar|s, are yet another type of shared variable,
but with the difference that operating on them has a transactional effect:

\begin{verbatim}
newTVar   :: a -> STM (TVar a)
readTVar  :: TVar a -> STM a
writeTVar :: TVar a -> a -> STM ()
\end{verbatim}

Transactions are atomic, so all reads will see a consistent state, and in the
presence of writes, intermediary states cannot be observed by another thread.

\paragraph{Aborting and retrying}
If we read a \verb|TVar| and don't like the value it has, the transaction can be
aborted, and the thread will block until any of the referenced \verb|TVar| have
been mutated:

\begin{verbatim}
retry :: STM a
check :: Bool -> STM ()
\end{verbatim}

We can also try executing a transaction, and do something else if it
fails:

\begin{verbatim}
orElse :: STM a -> STM a -> STM a
\end{verbatim}

\paragraph{Executing transactions}
STM transactions compose.  We can take small transactions and build bigger
transactions from them, and the whole is still executed atomically.

\begin{verbatim}
atomically :: STM a -> IO a
\end{verbatim}

This means we can do complex state operations involving multiple shared
variables without worrying!

\section{Exceptions}
\label{sec:concurrent_haskell-exc}

Exceptions are a way to bail out of a computation early.  Exceptions can be
explicitly thrown within a single thread, these are \emph{synchronous}
exceptions, or thrown from one thread to another, these are \emph{asynchronous}
exceptions.

\paragraph{Throwing and catching}
The basic functions for dealing with exceptions are:

\begin{verbatim}
catch :: Exception e => IO a -> (e -> IO a) -> IO a
throw :: Exception e => e -> IO a
\end{verbatim}

Where \verb|throw| causes the computation to jump back to the nearest enclosing
\verb|catch| capable of handling the particular exception. As exceptions belong
to a typeclass, rather than being a concrete type, different \verb|catch|
functions can be nested, to handle different types of exceptions.

Asynchronous exceptions can be thrown to another thread:

\begin{verbatim}
throwTo    :: Exception e => ThreadId -> e -> IO ()
killThread :: ThreadId -> IO ()
\end{verbatim}

These functions block until the target thread is in an appropriate state to
receive the exception.  Asynchronous exceptions can be caught with \verb|catch|,
just like synchronous exceptions thrown with \verb|throw|.

\paragraph{Masking}
A thread has a \emph{masking state}, which can be used to block exceptions from
other threads.  There are three masking states: \emph{unmasked}, in which a
thread can have exceptions thrown to it; \emph{interruptible}, in which a thread
can only have exceptions thrown to it if it is blocked; and
\emph{uninterruptible}, in which a thread cannot have exceptions thrown to it.

There are two functions to set the masking state.  These each execute a
computation in the new state, and pass it a function to run a subcomputation
with the original masking state:

\begin{verbatim}
mask                :: ((forall a. IO a -> IO a) -> IO b) -> IO b
uninterruptibleMask :: ((forall a. IO a -> IO a) -> IO b) -> IO b
\end{verbatim}

When a thread is started, it inherits the masking state of its parent.  As the
parent may be masked, we can fork a thread with a function to run a
subcomputation with exceptions unmasked:

\begin{verbatim}
forkIOWithUnmask :: ((forall a. IO a -> IO a) -> IO ()) -> IO ThreadId
forkOnWithUnmask :: Int -> ((forall a. IO a -> IO a) -> IO ()) -> IO ThreadId
\end{verbatim}

\paragraph{Software transactional memory}
STM can also use exceptions.  If an exception propagates uncaught to the top of
a transaction, that transaction is aborted.

\begin{verbatim}
throwSTM :: Exception e => e -> STM a
catchSTM :: Exception e => STM a -> (e -> STM a) -> STM a
\end{verbatim}
