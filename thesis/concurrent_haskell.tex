Concurrent Haskell\cite{peytonjones1996} is not in the Haskell
standard, so here we restrict our interest to GHC\@.  In this chapter
we give an overview of the concurrency functionality we use.  These
operations are available in the \package{concurrency} library.  We
cover the basic use of concurrency~\sref{concurrent_haskell-threads},
the memory model~\sref{concurrent_haskell-mmodel}, software
transactional memory~\sref{concurrent_haskell-stm}, and finally
exceptions~\sref{concurrent_haskell-exc}.

Throughout, we compare with the concurrency abstractions of Java and
Rust.  Java because it is a popular language which, like Haskell, has
exceptions.  Rust because its design borrows from the spirit of
functional languages.

\section{Multithreading}
\label{sec:concurrent_haskell-threads}

Threads let a program do multiple things at once.  Every program has
at least one thread, which runs the main action of the program.  A
thread is the basic unit of concurrency.  It lets us pretend that
we're computing multiple things at once.

\begin{listing}
\centering
\begin{cminted}{haskell}
forkIO     :: IO () -> IO ThreadId
myThreadId :: IO ThreadId
\end{cminted}
\caption{Basic threading operations in Haskell.}\label{lst:basic_haskell}
\end{listing}

A thread can be started using the \verb|forkIO| function, which starts
executing its argument in a separate thread and also gives us back a
\verb|ThreadId| value, which can be used to kill the thread.  A thread
can get its own \verb|ThreadId| using \verb|myThreadId|.

% [layout hack]: avoid the first line (and nothing else) of the
% following paragraph from ending up on this page.
\pagebreak

\begin{listing}
\centering
\begin{cminted}{java}
/* forkIO */
Runnable runnable = /* action */;
Thread thread = new Thread(runnable);
thread.start();

/* myThreadId */
Thread me = Thread.currentThread();
\end{cminted}
\caption{Basic threading operations in Java.}\label{lst:basic_java}
\end{listing}

In Java, threads are created from classes implementing the
\verb|Runnable| interface.  The \verb|Thread| constructor creates a
new thread object from a \verb|Runnable|, but it does not start until
\verb|Thread.start| is called.  The thread object itself fulfils the
role of the Haskell \verb|ThreadId| type.  A thread can get a
reference to itself with the \verb|Thread.currentThread| static
method.

\begin{listing}
\centering
\begin{cminted}{rust}
/* forkIO */
let thread = thread::spawn(/* closure */);

/* Java-style */
let thread = thread::Builder::new().spawn(/* closure */);

/* myThreadId */
let me = thread::current();
\end{cminted}
\caption{Basic threading operations in Rust.}\label{lst:basic_rust}
\end{listing}

Rust supports both the Haskell and Java thread creation styles.  The
Haskell-style \verb|thread::spawn| function takes a closure to
execute, creates and immediately begins executing a thread, and
returns an identifier.  The alternative Java-style
\verb|thread::Builder| interface allows creating a thread without
starting it.  In both cases, the compiler gives an error if the
closure captures a variable from its outer scope without the
appropriate move semantics.

\paragraph{Capabilities}
In a real machine, there are multiple processors and cores.  It may be that a
particular application of concurrency is only a net gain if every thread is
operating on a separate core, so that threads are not interrupting each other.
GHC uses a \emph{green threading} model, where Haskell threads are multiplexed
onto a much smaller number of operating system threads.  The number of operating
system threads is referred to as the number of \emph{capabilities} or
\emph{Haskell execution contexts}.  Only operating system threads have the
possibility of executing truly in parallel.

% [layout hack]: avoid the first line (and nothing else) of the
% following paragraph from ending up on this page.
\pagebreak

\begin{listing}
\centering
\begin{cminted}{haskell}
forkOn             :: Int -> IO () -> IO ThreadId

getNumCapabilities :: IO Int
setNumCapabilities :: Int -> IO ()
\end{cminted}
\caption{Operating system threads in Haskell.}\label{lst:caps_haskell}
\end{listing}

We can fork a thread to run on a particular capability with the
\verb|forkOn| function, which takes a number identifying the
capability to use.  This capability number is interpreted modulo the
total number of capabilities, which can be queried and set.

Neither Java nor Rust provide green threading.  Java does not specify
how its threads are mapped to OS threads but, on Linux, each Java
thread is an OS thread.  Rust specifies that its threads are OS
threads.

\paragraph{Scheduling}
The GHC scheduler is good, but sometimes we have domain knowledge
which lets us do better.

\begin{listing}
\centering
\begin{cminted}{haskell}
yield       :: IO ()
threadDelay :: Int -> IO ()
\end{cminted}
\caption{Controlling thread scheduling in Haskell.}\label{lst:schedule_haskell}
\end{listing}

There are two ways to influence how threads are scheduled: we can
yield control to another thread, or delay the current thread for a
period of time.

\begin{listing}
\centering
\begin{cminted}{java}
Thread thread = /* ... */;
thread.setPriority(/* new priority */);
\end{cminted}
\caption{Thread priority in Java.}\label{lst:schedule_java}
\end{listing}

In Java, we can use the \verb|Thread.yield| and \verb|Thread.sleep|
methods to affect scheduling.  We can also adjust the \emph{priority}
of a thread, where the initial priority is inherited from its creator.
Threads with higher priority are executed in preference to threads
with lower priority:

\begin{listing}[h!]
\centering
\begin{cminted}{rust}
thread::park() /* execution stops now */

/* from another thread */
reference_to_thread::unpark();
\end{cminted}
\caption{Thread parking and unparking in Rust.}\label{lst:schedule_rust}
\end{listing}

Rust has three ways to control scheduling.  In addition to yielding
and delaying, it can also \emph{park} the current thread.  When
parked, a thread will not execute until it is unparked by another
thread.  There is a variant of \verb|thread::park| with a timeout,
which provides a delay-unless-woken construct.  Haskell threads have
no notion of priority or parking.

\paragraph{Termination}
Both Java and Rust can use a thread handle to block until that thread
terminates.  This is called \emph{joining}.  Haskell provides no join
operation.

\section{Shared State and the Memory Model}
\label{sec:concurrent_haskell-mmodel}

There are two main types of shared variable in GHC Haskell, with different
semantics.

\paragraph{Shared mutable references}
An \verb|IORef| is a mutable location in memory holding a Haskell
value.

\begin{listing}
\centering
\begin{cminted}{haskell}
newIORef   :: a -> IO (IORef a)
readIORef  :: IORef a -> IO a
writeIORef :: IORef a -> a -> IO ()
\end{cminted}
\caption{Shared mutable references in Haskell.}\label{lst:smref_haskell}
\end{listing}

As Java is an impure language with no restriction on sharing, it has
no need for a type like \verb|IORef|.  Any thread can mutate any
reference that is in scope.  Rust does impose restrictions on
mutability and sharing, and provides a few different shared variable
types.  The closest to \verb|IORef| is a reference-counting box
containing an atomically modifiable pointer.

\begin{listing}
\centering
\begin{cminted}{rust}
let ptr = &mut /* initial value */;
let shared = Arc::new(AtomicPtr::new(ptr));

let shared_clone = shared.clone();
let thread = thread::spawn(move|| {
    shared_clone.store(/* new value */, Ordering::SeqCst);
});
\end{cminted}
\caption{Shared mutable references in Rust.}\label{lst:smref_rust}
\end{listing}

Threads can modify the pointer by cloning the shared \verb|Arc| value,
extracting the inner \verb|AtomicPtr|, and updating the value inside.
All mutation operations take as a parameter the type of memory
consistency to enforce, which we shall discuss shortly.

\paragraph{Shared references under mutual exclusion}
An \verb|MVar| is a mutable location in memory with two possible
states: \emph{full}, holding a Haskell value, and \emph{empty},
holding no value.  An \verb|MVar| can be created in either state.

\begin{listing}
\centering
\begin{cminted}{haskell}
newMVar      :: a -> IO (MVar a)
newEmptyMVar :: IO (MVar a)

putMVar      :: MVar a -> a -> IO ()
readMVar     :: MVar a -> IO a
takeMVar     :: MVar a -> IO a

tryPutMVar   :: MVar a -> a -> IO Bool
tryReadMVar  :: MVar a -> IO (Maybe a)
tryTakeMVar  :: MVar a -> IO (Maybe a)
\end{cminted}
\caption{Mutual exclusion in Haskell.}\label{lst:mute_haskell}
\end{listing}

Writing to a full \verb|MVar| blocks until it is empty, and reading or
taking from an empty \verb|MVar| blocks until it is full.  There are
also non-blocking functions which return an indication of success.
The blocking behaviour of \verb|MVar|s means that computations can
become deadlocked.  For example, deadlock occurs if every thread tries
to take from the same \verb|MVar|, with no threads writing to it.
This can be detected, as we shall see in \cref{chp:dejafu}.  As there
are no blocking \verb|IORef| primitives, use of them cannot cause a
deadlock.

\begin{listing}
\centering
\begin{cminted}{java}
Semaphore sem = new Semaphore(/* initial quantity */);

/* from another thread */
sem.acquire(/* quantity */);
/* ... */
sem.release(/* quantity */);
\end{cminted}
\caption{Mutual exclusion in Java.}\label{lst:mute_java}
\end{listing}

Java does not provide an exact analogue of \verb|MVar|, but it does
provide mutexes and semaphores which can be used to control access to
a shared resource.

\begin{listing}
\centering
\begin{cminted}{rust}
let shared = Arc::new(Mutex::new(/* initial value */));

let shared_clone = shared.clone();
let thread = thread::spawn(move|| {
    let mut unlocked = shared_clone.lock();
    /* ... */
});
\end{cminted}
\caption{Mutual exclusion in Rust.}\label{lst:mute_rust}
\end{listing}

The Rust \verb|Mutex| type is more like the Haskell \verb|MVar| type,
in that it does not merely function as a lock but also guards a
reference.  Locks are released when the unlocked value falls out of
scope.  This ensures that a thread cannot unlock a mutex and terminate
with the mutex still locked.  There is also a non-blocking
\verb|Mutex::try_lock| function.  Unlike Java, there is no way to
explicitly lock an unlocked mutex.  Mutexes are not exactly the same
as a Haskell \verb|MVar|, however they can be used in the
implementation of an \verb|MVar|.

\paragraph{Memory model}
Unlike the \verb|MVar|, \verb|IORef| operations are not
\emph{synchronised}.  Reads and writes between threads may be
re-ordered.  The documentation has this to say:

\begin{bquote}{Data.IORef module documentation\footnote{\url{https://hackage.haskell.org/package/base-4.10.0.0/docs/Data-IORef.html\#g:2}}}
  In a concurrent program, \verb|IORef| operations may appear out-of-order to
  another thread, depending on the memory model of the underlying processor
  architecture.  For example, on x86, loads can move ahead of stores.

  The implementation is required to ensure that reordering of memory operations
  cannot cause type-correct code to go wrong.  In particular, when inspecting
  the value read from an \verb|IORef|, the memory writes that created that value
  must have occurred from the point of view of the current thread.
\end{bquote}

For testing purposes, we support the Total Store Order~(TSO) and Partial Store
Order~(PSO) models~\sref{dejafu-execution}.  Many other operations are
synchronised, and act as a \emph{barrier} to re-ordering.  Reading or writing to
an \verb|MVar| does; executing an STM transaction does; throwing an asynchronous
exception does; and the atomic \verb|IORef| operations do.

\begin{listing}
\centering
\begin{cminted}{haskell}
atomicWriteIORef  :: IORef a -> a -> IO ()
atomicModifyIORef :: IORef a -> (a -> (a, b)) -> IO b
\end{cminted}
\caption{Atomic operations in Haskell.}\label{lst:atomic_haskell}
\end{listing}

Java allows controlling the synchronisation on a per-variable basis.
Operations on normal shared variables may appear out-of-order to
different threads, however a \verb|volatile| variable will be
in-order:

\begin{listing}
\centering
\begin{cminted}{java}
public volatile int sequentiallyConsistent = 0;
\end{cminted}
\caption{Atomic operations in Java.}\label{lst:atomic_java}
\end{listing}

As we saw in the \verb|IORef| segment, Rust operations which mutate
atomic values specify the memory consistency desired.  The weakest is
\verb|Relaxed|, which imposes no constraints, and the strongest is
\verb|SeqCst|, which imposes sequential consistency.

\paragraph{Compare-and-swap}
Modern processor architectures provide an atomic
\emph{compare-and-swap} instruction, which is typically used in
implementing high-performance lock-free algorithms.  The
\package{atomic-primops} package exposes this to Haskell code.

\begin{listing}
\centering
\begin{cminted}{haskell}
readForCAS :: IORef a -> IO (Ticket a)
peekTicket :: Ticket a -> a
casIORef   :: IORef a -> Ticket a -> a -> IO (Bool, Ticket a)
\end{cminted}
\caption{Compare-and-swap in Haskell}\label{lst:cas_haskell}
\end{listing}

A \verb|Ticket| is a proof that a value has been observed inside an
\verb|IORef| at some prior point.  Given this proof, the programmer
can efficiently and atomically change the value inside the
\verb|IORef| later if it has not been modified.  The \verb|casIORef|
function is partially synchronised, acting as a barrier to re-ordering
on that particular \verb|IORef|, but not for others.

\begin{listing}
\centering
\begin{cminted}{java}
private AtomicInteger count = new AtomicInteger(0);

public void increment() {
  count.incrementAndGet();
}
\end{cminted}
\caption{Compare-and-swap in Java}\label{lst:cas_java}
\end{listing}

Java provides variants of all the primitive types which support
compare-and-swap.

\begin{listing}
\centering
\begin{cminted}{rust}
ptr.compare_and_swap(other, another, Ordering::SeqCst);
ptr.compare_exchange(other, another, Ordering::SeqCst, Ordering::Relaxed);
\end{cminted}
\caption{Compare-and-swap in Rust.}\label{lst:cas_rust}
\end{listing}

The Rust atomic types provide compare-exchange in addition to
compare-and-swap.  Compare-exchange differs from compare-and-swap in
that the programmer specifies the desired memory consistency on
failure:

\section{Software Transactional Memory}
\label{sec:concurrent_haskell-stm}

Shared variables are nice, until we need more than one.  As we can
only claim one \verb|MVar| atomically (or write to one \verb|IORef|
atomically), it seems we need to introduce additional synchronisation.
This is unwieldy and prone to bugs.

Software transactional memory~(STM) is the solution.  STM is based on
the idea of atomic \emph{transactions}.  An STM transaction consists
of one or more operations over a collection of \emph{transaction
  variables}, where a transaction may be aborted part-way through,
with all its effects rolled back.  Arbitrary effects are not
permitted, which is enforced by having a distinct type for STM
actions.

Neither Java nor Rust provide an STM implementation in their standard
libraries, but there are third-party implementations.  However, as
Java and Rust are impure, these libraries cannot prevent the
programmer from performing arbitrary effects inside a transaction.
These STM library implementations provide atomic transactions for
specified operations, but they \emph{cannot} provide the same
guarantees as STM in Haskell.

\paragraph{Transactional variables}
The \verb|TVar| type is yet another type of shared variable, but with
the difference that operating on them has a transactional effect.

\begin{listing}
\centering
\begin{cminted}{haskell}
newTVar   :: a -> STM (TVar a)
readTVar  :: TVar a -> STM a
writeTVar :: TVar a -> a -> STM ()
\end{cminted}
\caption{Transactional variables in Haskell.}\label{lst:tvars_haskell}
\end{listing}

Transactions are atomic, so all reads will see a consistent state, and
in the presence of writes, intermediate states cannot be observed by
another thread.

\paragraph{Aborting and retrying}
If we read a \verb|TVar| and don't like the value it has, the
transaction can be aborted, and the thread will block until any of the
referenced \verb|TVar|s have been mutated.  We can also try executing
a transaction, and do something else if it retries.

\begin{listing}
\centering
\begin{cminted}{haskell}
retry  :: STM a
orElse :: STM a -> STM a -> STM a
\end{cminted}
\caption{Aborting and retrying transactions in Haskell.}\label{lst:orelse_haskell}
\end{listing}

\paragraph{Executing transactions}
Transactions compose.  We can take small transactions and build bigger
transactions from them, and the whole is still executed atomically.

\begin{listing}
\centering
\begin{cminted}{haskell}
atomically :: STM a -> IO a
\end{cminted}
\caption{Executing transactions in Haskell}\label{lst:atomically_haskell}
\end{listing}

This means we can do complex state operations involving multiple shared
variables without worrying!

\section{Exceptions}
\label{sec:concurrent_haskell-exc}

Exceptions are a way to bail out of a computation early.  Exceptions can be
explicitly thrown within a single thread, these are \emph{synchronous}
exceptions, or thrown from one thread to another, these are \emph{asynchronous}
exceptions.

\paragraph{Throwing and catching}
The basic functions for dealing with exceptions are throwing and
catching.

\begin{listing}
\centering
\begin{cminted}{haskell}
catch :: Exception e => IO a -> (e -> IO a) -> IO a
throw :: Exception e => e -> IO a
\end{cminted}
\caption{Exceptions in Haskell.}\label{lst:excs_haskell}
\end{listing}

Throwing an exception causes the computation to jump back to the
nearest enclosing suitable exception handler.  If there is none, the
thread terminates.  Haskell exceptions belong to a typeclass, rather
than being a concrete type, so different \verb|catch| functions can be
nested, to handle different types of exception.

\begin{listing}
\centering
\begin{cminted}{java}
public void createFile(String path, String text) throws IOException {
  FileWriter writer = new FileWriter(path, true);
  writer.write(text);
  writer.close();
}
\end{cminted}
\caption{Checked exceptions in Java.}\label{lst:excs_java}
\end{listing}

In addition to Haskell-style exceptions, Java supports \emph{checked
  exceptions}.  If a method can throw (or propagate) a checked
exception, it appears in the type.  Checked exceptions statically
enforce exception handling, but are often regarded as cumbersome.  The
Haskell type system has no equivalent of checked exceptions.  If a
Haskell programmer wants something like a checked exception, they use
a type such as \verb|Either| to indicate success or failure.

\begin{listing}
\centering
\begin{cminted}{rust}
let result = panic::catch_unwind(|| {
    panic!("oh no!");
});
\end{cminted}
\caption{Panics in Rust.}\label{lst:excs_rust}
\end{listing}

Rust does not really have exceptions.  The \verb|panic| function which
raises an error which, if uncaught, kills the current thread.  The
\verb|catch_unwind| function can be used to execute a closure and
recover from a panic, but it is not guaranteed to catch all
panics\footnote{\url{https://doc.rust-lang.org/1.9.0/std/panic/fn.catch_unwind.html}},
making panics unsuitable as a general control-flow mechanism.  The
typical Rust approach is, like Haskell, to return a type indicating
success or failure.

\begin{listing}
\centering
\begin{cminted}{haskell}
throwTo    :: Exception e => ThreadId -> e -> IO ()
killThread :: ThreadId -> IO ()
\end{cminted}
\caption{Asynchronous exceptions in Haskell.}\label{lst:excsa_haskell}
\end{listing}

In addition to \emph{synchronous} exceptions, Haskell has
\emph{asynchronous} exceptions which can be thrown to another thread.
These functions block until the target thread is in an appropriate
state to receive the exception.  Asynchronous exceptions can be caught
with \verb|catch|, just like synchronous exceptions thrown with
\verb|throw|.

The Java \verb|Thread.stop| method is like \verb|killThread|, but is
considered a bad idea and
deprecated\footnote{\url{https://docs.oracle.com/javase/7/docs/technotes/guides/concurrency/threadPrimitiveDeprecation.html}}.
The preferred approach is the \verb|Thread.interrupt| method, which
will either throw an exception or set a flag, depending on what the
target thread is doing.  For example, if the target thread is blocked
inside a \verb|Thread.sleep| call, it will receive an
\verb|InterruptedException|.

Rust does not provide any way to tell a thread to terminate.

\paragraph{Masking}
A thread has a masking state, which can be used to block exceptions
from other threads.  There are three masking states: \emph{unmasked},
in which a thread can have exceptions thrown to it;
\emph{interruptible}, in which a thread can only have exceptions
thrown to it if it is blocked; and \emph{uninterruptible}, in which a
thread cannot have exceptions thrown to it.

\begin{listing}
\centering
\begin{cminted}{haskell}
forkIOWithUnmask    :: ((forall a. IO a -> IO a) -> IO ()) -> IO ThreadId
forkOnWithUnmask    :: Int -> ((forall a. IO a -> IO a) -> IO ()) -> IO ThreadId

mask                :: ((forall a. IO a -> IO a) -> IO b) -> IO b
uninterruptibleMask :: ((forall a. IO a -> IO a) -> IO b) -> IO b
\end{cminted}
\caption{Masking exceptions in Haskell.}\label{lst:excm_haskell}
\end{listing}

There are two functions to set the masking state.  These each execute
a computation in the new state, and pass it a function to run a
subcomputation with the original masking state.  When a thread is
started, it inherits the masking state of its parent.  As the parent
may be masked, we can fork a thread with a function to run a
subcomputation with exceptions unmasked.

\paragraph{Software transactional memory}
STM can also use exceptions.  If an exception propagates uncaught to the top of
a transaction, that transaction is aborted.

\begin{listing}
\centering
\begin{cminted}{haskell}
throwSTM :: Exception e => e -> STM a
catchSTM :: Exception e => STM a -> (e -> STM a) -> STM a
\end{cminted}
\caption{STM exceptions in Haskell.}\label{lst:excstm_haskell}
\end{listing}

The \verb|orElse| function does not catch exceptions.
