A common approach to testing in Haskell is to give functional
properties about the code.  Property testing tools are used to
generate input values, and check that these properties hold, or
display a counterexample if they do not.  The popularity of property
testing stems from the difficulty of writing good tests.  In this
chapter we give an overview of using property testing tools.  We first
give a general introduction to specifying and using properties as
tests~\sref{property_testing-intro}, then discuss specific
tools~\sref{property_testing-tools}.

Readers already familiar with property testing may wish to skip this
chapter.

\section{Properties as Tests}
\label{sec:property_testing-intro}

Property-based testing\cite{claessen2000}, sometimes called \emph{parameterised
  unit testing}, is an approach to testing where the programmer gives general
laws (or properties) which should hold for all input values.  This is unlike
typical unit testing, which can be thought of as checking specific pairs of
input--output values.  For example, a sort function should preserve length:

\begin{minted}{haskell}
prop_sort_len xs = length xs == length (sort xs)
\end{minted}

Often we do not want to check a property for arbitrary input values.
Perhaps we know something about how the functions we are testing are
expected to be used, or we are only interested in how they behave in a
certain case.  For this, we can use implication:

\begin{minted}{haskell}
prop_ord_insert x xs = ordered xs ==> ordered (insert x xs)
\end{minted}

Implication is typically provided as part of property DSL of a tool,
rather than being a normal boolean function.  This allows the tool to
use the precondition as part of its input generation strategy.

While implication is a useful technique, it can skew the input distribution.
All empty lists and lists of length one satisfy \verb|ordered xs|, but only 50\%
of all lists of length two do.  An alternative approach is to use a custom value
generator.  For example, suppose we have a suitable function for producing
ordered lists, we could restate the implication property as:

\begin{minted}{haskell}
prop_ord_insert x = forAll orderedList (\xs -> ordered (insert x xs))
\end{minted}

By only generating input values which satisfy the precondition, we can avoid
skewing the distribution, and improve our confidence that the property does
hold.

In the absence of a programmer-supplied generator function, input values are
generated in a type-directed process.  There is a typeclass, typically called
\verb|Arbitrary| or \verb|Enumerable| or \verb|Listable| or something, which has
functions to generate values.  This typeclass will typically have instances for
most common types, but if a programmer wishes to have input values of custom
types, they will need to supply a suitable instance.

\section{Tools}
\label{sec:property_testing-tools}

Property-based testing tools mainly differ along two axes: the
expressiveness of the property DSL, and the strategy for generating
input values. \tblref{proptools} summarises the differences between
several tools for Haskell.

\begingroup
\newcommand{\Y}{\CIRCLE}
\newcommand{\N}{\Circle}
\newcommand{\YN}{\LEFTcircle}
\newcommand{\Q}{\N$^p$}
\newcommand{\VBS}{\rotatebox{90}{\scriptsize BSD3}}
\newcommand{\VGP}{\rotatebox{90}{\scriptsize GPLv3}}

\begin{figure}
\centering
\begin{tabularx}{\textwidth}{lXXXXXXXXXX}
&\rotatebox{90}{QuickCheck\hphantom{a}}
&\rotatebox{90}{SmartCheck}
&\rotatebox{90}{SmallCheck}
&\rotatebox{90}{Lazy SmallCheck}
&\rotatebox{90}{LeanCheck}
&\rotatebox{90}{Feat}
&\rotatebox{90}{Neat}
&\rotatebox{90}{GenCheck}
&\rotatebox{90}{Irulan}
&\rotatebox{90}{Reach}
\\ \toprule
\textbf{Input value generation}          &&&&&&&&&&\\
~~random                              &\Y&\Y&\N&\N&\N&\Y&\N&\Y&\Y&\N\\
~~enumerative                         &\N&\N&\Y&\Y&\Y&\Y&\Y&\Y&\Y&\N\\
~~mixed random \& enumerative         &\N&\N&\N&\N&\N&\Y&\N&\Y&\Y&\N\\ \midrule
\textbf{Property DSL}          &&&&&&&&&&\\
~~existential                 &\N&\N&\Y&\Y&\Y&\Q&\Q&\Q&\N&\N\\
~~higher order                &\Y&\Y&\Y&\Y&\Y&\Q&\Q&\Q&\N&\N\\ \midrule
\textbf{Output}          &&&&&&&&&&\\
~~generalized counterexamples            &\N&\Y&\N&\YN&\N&\N&\N&\N&\N&\N\\
\bottomrule
\multicolumn{11}{l}{
\footnotesize
Legend:\hspace{1em}
\Y{} Yes/Good.\hspace{1em}
\N{} No/Poor.\hspace{1em}
\YN{} Partial/Median.\hspace{1em}
$^p$ Potential support} \\
\end{tabularx}
\caption{Summary of differences between property-based testing tools for Haskell.}\label{tbl:proptools}
\end{figure}

\paragraph{Input value generation}
Inputs can either be generated randomly or enumerated.  Randomisation
is a simple technique which tends to work well in practice,
QuickCheck\cite{claessen2000} is an example of a randomised property
testing tool.  Alternatively, we may assume that there is some
enumeration likely to expose useful counterexamples.
SmallCheck\cite{runciman2008} enumerates values in size order, on the
assumption that most bugs are exhibited by simple counterexamples.  As
simple counterexamples are more useful to the programmer than large
ones, random approaches must have an additional \emph{shrinking} step,
to try and remove unnecessary complexity from counterexamples, which
enumerative approaches may not need.

\paragraph{Property DSL}
A more expressive property language complicates implementation, but
allows the programmer to say more about their tests.  Two important
types of property are \emph{existential} properties and
\emph{higher-order} properties.

Existential properties allow the programmer to assert that some input
exists which the property holds for.  Existential properties are
difficult to implement reliably in a randomised tool, such as
QuickCheck, because the nondeterminism means that the property may
sometimes pass and sometimes fail.  Existential properties are more
commonly supported by enumerative tools.  This is an example of an
existential property:

\begin{minted}{haskell}
prop_gt_5 = exists (\x -> x > 5)
\end{minted}

Higher-order properties are properties where some of the inputs are,
themselves, functions.  To test such a property requires the tool to
be able to generate functions.  Higher-order properties are invaluable
in the testing of higher-order functions.  This is an example of a
higher-order property:

\begin{minted}{haskell}
prop_map_fuse xs f g = map g (map f xs) == map (g . f) xs
\end{minted}

\paragraph{Output}
Despite not affecting the expressiveness of the tool, presenting
output clearly is essential.  This is why
QuickCheck\cite{claessen2000} shrinks counterexamples to find a local
minimum.  However, shrinking and enumeration are not the only ways to
produce small counterexamples.  Both SmartCheck\cite{pike2014} and
Lazy SmallCheck\cite{runciman2008} can generalise counter\-examples.
Generalising counterexamples directly can be more efficient than a
QuickCheck-style shrinking process\cite{pike2014}.  This is an example
of a generalised counterexample:

\begin{minted}{text}
> check $ \xs -> nub xs == (xs::[Int])
*** Failed! Falsifiable (after 3 tests):
[0,0]

Generalization:
x:x:_
\end{minted}

\noindent where the \verb|nub| function removes duplicates from a
list.

\paragraph{Beyond Haskell}
While this is a thesis using Haskell, the interest in property-based testing is
wider than that.

\begin{itemize}
\item QuviQ provide a commercial version of QuickCheck for
  Erlang\cite{arts2006}.
\item The popular JUnit library for Java provides built-in support for parameterised
  tests\footnote{\url{https://github.com/junit-team/junit4/wiki/Parameterized-tests}},
  whereas the
  junit-quickcheck\footnote{\url{https://github.com/pholser/junit-quickcheck}}
  library provides a more traditional property testing experience.
\item The Go standard library provides a
  testing/quick\footnote{\url{https://golang.org/pkg/testing/quick/}} module.
\item The
  Hypothesis\footnote{\url{https://github.com/HypothesisWorks/hypothesis-python}}
  tool for Python implements property-based testing, but cannot do automated
  type-directed input value generation due to Python's dynamic nature.
\item NUnit, the common .NET unit testing library, allows tests to be
  parameterised with random numeric
  values\footnote{\url{https://github.com/nunit/docs/wiki/Random-Attribute}},
  and with combinations of values of arbitrary
  types\footnote{\url{https://github.com/nunit/docs/wiki/Values-Attribute}}.
\end{itemize}

Although QuickCheck was arguably the first tool to popularise this style of
testing, and did so in Haskell, it is increasingly gaining recognition by
programmers of other languages as a good way to overcome the pitfalls and
difficulties of traditional unit testing techniques.
