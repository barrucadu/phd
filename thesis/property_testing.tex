A common approach to testing in Haskell is to give properties about
the code.  Properties are functions with boolean results, expected to
be true for all argument values.  Property testing tools are used to
generate input values, and check that these properties hold, or
display a counterexample if they do not.  The popularity of property
testing stems from the difficulty of writing good tests.  In this
chapter we give an overview of using property testing tools.  We build
on this background in \Cref{chp:coco}, where we discuss a tool to
\emph{generate} such properties for concurrency functions operating
over shared state.  We first give a general introduction to specifying
and using properties as tests~\sref{property_testing-intro}, then
discuss specific tools~\sref{property_testing-tools}.  Finally, we
summarise tools for \emph{discovering}
properties~\sref{property_testing-gen}.

\section{Properties as Tests}
\label{sec:property_testing-intro}

Property-based testing\cite{claessen2000}, sometimes called
\emph{parameterised unit testing}, is an approach to testing where the
programmer gives general laws (or properties) which should hold for
all input values.  For example, \Cref{lst:prop_sort_len} says that a
sort function should preserve length.  This approach is unlike typical
unit testing, which can be thought of as checking specific pairs of
input--output values.

\begin{listing}
\centering
\begin{cminted}{haskell}
prop_sort_len xs = length xs == length (sort xs)
\end{cminted}
\caption{A property asserting that sorting preserves length.}\label{lst:prop_sort_len}
\end{listing}

Often we do not want to check a property for arbitrary input values.
Perhaps we know something about how the functions we are testing are
expected to be used, or we are only interested in how they behave in a
certain case.  A simple way to enforce a precondition is to use
logical implication.  Implication is typically provided as part of the
property DSL of a tool, rather than being a normal boolean function.
This allows the tool to ensure that a desired number of generated
inputs pass the precondition.

\begin{listing}
\centering
\begin{cminted}{haskell}
prop_ord_insert1 x xs = ordered xs ==> ordered (insert x xs)
prop_ord_insert2 x    = forAll orderedList (\xs -> ordered (insert x xs))
\end{cminted}
\caption{Enforcing a precondition for a property.}\label{lst:prop_ord_insert}
\end{listing}

While implication is a useful technique, it can skew the input
distribution.  For example, the empty lists and lists of length one
are ordered, but only 50\% of lists of length two are.  An alternative
approach is to use a custom value generator.  By only generating input
values which satisfy the precondition, we can avoid skewing the
distribution, and improve our confidence that the property does hold
in general.  \Cref{lst:prop_ord_insert} shows the implication and
generator function approaches.

In the absence of a programmer-supplied generator function, input
values are generated in a type-directed process.  A tool will provide
a typeclass, typically called something like \verb|Arbitrary| or
\verb|Enumerable| or \verb|Listable|, which has functions to generate
values.  This typeclass will typically have instances for most common
types, but if a programmer wishes to have input values of other types,
they will need to supply a suitable instance.

\section{Property Testing Tools}
\label{sec:property_testing-tools}

Property-based testing tools mainly differ along two axes: the
expressiveness of the property DSL, and the strategy for generating
input values.  \Cref{tbl:proptools} summarises the differences between
several tools for Haskell.

\begingroup
\newcommand{\YY}{\CIRCLE}
\newcommand{\NN}{\Circle}
\newcommand{\YN}{\LEFTcircle}
\newcommand{\QQ}{\NN$^p$}

\begin{table}
\centering
\begin{tabular}{lcccccc}
&\rotatebox{90}{QuickCheck\hphantom{a}}
&\rotatebox{90}{SmartCheck}
&\rotatebox{90}{SmallCheck}
&\rotatebox{90}{Lazy SmallCheck}
&\rotatebox{90}{LeanCheck}
\\ \toprule
\textbf{Input value generation}          &&&&&\\
~~random                              &\YY&\YY&\NN&\NN&\NN\\
~~enumerative                         &\NN&\NN&\YY&\YY&\YY\\
\textbf{Property DSL}          &&&&&\\
~~existential                 &\NN&\NN&\YY&\YY&\YY\\
~~higher order                &\YY&\YY&\YY&\YY&\YY\\ \midrule
\textbf{Output}          &&&&&\\
~~generalised counterexamples            &\NN&\YY&\NN&\YN&\YN\\
\bottomrule
\multicolumn{6}{l}{
\footnotesize
Legend:\hspace{1em}
\YY{} Yes/Good.\hspace{1em}
\NN{} No/Poor.\hspace{1em}
\YN{} Partial/Median.} \\
\end{tabular}
\caption[Summary of differences in Haskell property-testing tools.]{Summary of differences between property-based testing tools for Haskell, adapted from \cite{braquehais2017phd}.}\label{tbl:proptools}
\end{table}
\endgroup

\paragraph{Input value generation}
Inputs can either be generated randomly or enumerated.  Randomisation
is a simple technique which tends to work well in practice,
QuickCheck\cite{claessen2000} is an example of a randomised property
testing tool.  Alternatively, we may assume that there is some
enumeration likely to expose useful counterexamples.
SmallCheck\cite{runciman2008} and LeanCheck\cite{braquehais2017lean}
enumerate values in size order, on the assumption that most bugs are
exhibited by simple counterexamples.  As simple counterexamples are
more useful to the programmer than large ones, random approaches must
have an additional \emph{shrinking} step, to try and remove
unnecessary complexity from counterexamples, which enumerative
approaches may not need.

\paragraph{Property DSL}
A more expressive property language complicates implementation, but
allows the programmer to say more about their tests.  Two important
types of property are \emph{existential} properties and
\emph{higher-order} properties.

Existential properties, such as in \Cref{lst:prop_gt_5}, allow the
programmer to assert that some input exists for which the property
holds.  Existential properties are apparently incompatible with a
randomised tool, such as QuickCheck, because a random test value is
unlikely to be a witness for a specific existential property.
Existential properties are more commonly supported by enumerative
tools.

\begin{listing}
\centering
\begin{cminted}{haskell}
prop_gt_5 = exists (\x -> x > 5)
\end{cminted}
\caption{Using existential quantification in a property.}\label{lst:prop_gt_5}
\end{listing}

Higher-order properties, such as in \Cref{lst:prop_map_fuse}, are
properties where some of the inputs are, themselves, functions.  To
test such a property requires the tool to be able to generate
functions.  Higher-order properties are invaluable in the testing of
higher-order functions.

\begin{listing}
\centering
\begin{cminted}{haskell}
prop_map_fuse xs f g = map g (map f xs) == map (g . f) xs
\end{cminted}
\caption{Using higher-order functions in a property.}\label{lst:prop_map_fuse}
\end{listing}

In the context of concurrency, another way in which a property could
be \emph{higher-order} is in taking an explicit schedule as an
argument.  A list of scheduling decisions cannot be generated up
front, as the property-testing tool cannot know which threads are
runnable.  So instead we can generate a scheduler
function\cite{ankuzik2014}.  This is one possible way to implement
random testing of concurrent programs.

\paragraph{Output}
How a tool presents its output is of great importance.  Randomly
generated counterexamples, such as those found by
QuickCheck\cite{claessen2000}, are often not minimal.  Searching for a
local minimum by shrinking randomly generated counterexamples before
displaying them is a common approach.  However, shrinking and
enumeration are not the only ways to produce small counterexamples.
Both SmartCheck\cite{pike2014} and Lazy SmallCheck\cite{runciman2008}
can generalise counter\-examples.  LeanCheck\cite{braquehais2017lean}
can generalise counterexamples when used with the
Extrapolate\cite{braquehais2017ifl} tool.  Generalising
counterexamples directly can be more efficient than a shrinking
process as in QuickCheck\cite{pike2014}.  Furthermore, it is often
possible to produce a generalisation which is simpler than any
concrete counterexample.  \Cref{lst:gencntr} shows such a generalised
counterexample.  The property here fails for lists which contain
duplicates, the concrete value is unimportant.

\begin{listing}
\centering
\begin{cminted}{text}
> check $ \xs -> nub xs == (xs::[Int])
*** Failed! Falsifiable (after 3 tests):
[0,0]

Generalization:
x:x:_
\end{cminted}
%$
\caption{A generalised counterexample of an incorrect property.}\label{lst:gencntr}
\end{listing}

\paragraph{Beyond Haskell}
Although this is a thesis using Haskell, a language particularly
suited for property-based testing, the interest in property-based
testing is wider than that.

\begin{itemize}
\item QuviQ provide a commercial version of QuickCheck for
  Erlang\cite{arts2006}.
\item The popular JUnit library for Java provides built-in support for
  parameterised tests\cite{junit2017}, whereas the
  junit-quickcheck\cite{holser2018} library provides a more
  traditional property testing experience.
\item The Go standard library provides a
  testing/quick\cite{golang2017tq} module.
\item The Hypothesis\cite{hypothesis2018} tool for Python implements
  property-based testing, but cannot do automated type-directed input
  value generation due to Python's dynamic nature.
\item NUnit, the common .NET unit testing library, allows tests to be
  parameterised with random numeric values\cite{nunit2017ra}, and with
  combinations of values of arbitrary types\cite{nunit2017va}.
\end{itemize}

Although QuickCheck was arguably the first tool to popularise this style of
testing, and did so in Haskell, it is increasingly gaining recognition by
programmers of other languages as a good way to overcome the pitfalls and
difficulties of traditional unit testing techniques.

\section{Searching for Properties}
\label{sec:property_testing-gen}

As we have seen, properties can be used as expressive and declarative
test cases.  However, coming up with properties can be difficult.  To
help the programmer, tools exist to discover properties.  These tools
are based on testing or examples, and so any properties found are
merely conjectures supported by a finite amount of evidence.  Despite
that, such properties are surprisingly accurate in practice, and often
lead to a deeper understanding of the program under test.

\paragraph{Testing}
QuickSpec\cite{claessen2010,smallbone2017} and
Speculate\cite{braquehais2017} are tools for Haskell which
automatically discover equational laws of pure functions.  Both are
based on generating and testing candidate expressions.  Speculate,
unlike QuickSpec, can discover inequalities and conditional equations.
Neither supports functions with effects or generating lambda-terms.

When provided with the integers \verb|0| and \verb|1| and the
functions \verb|id|, \verb|abs|, and \verb|(+)|, Speculate prints the
properties in \Cref{lst:arith_props}.  QuickSpec discovers similar
properties to \Cref{lst:arith_props0}, but not the inequalities and
conditional equations in \Cref{lst:arith_props1}.

\begin{listing}
\begin{sublisting}{\textwidth}
\centering
\begin{cminted}{text}
           id x == x
          x + 0 == x
    abs (abs x) == abs x
          x + y == y + x
    abs (x + x) == abs x + abs x
abs (x + abs x) == x + abs x
abs (1 + abs x) == 1 + abs x
    (x + y) + z == x + (y + z)
\end{cminted}
\caption{Equational laws.}\label{lst:arith_props0}
\end{sublisting}

% [layout hack]: no gap between the listings otherwise
\vspace{2.5em}

\begin{sublisting}{\textwidth}
\begin{minipage}[t]{0.45\textwidth}
\begin{minted}{text}
          x <= abs x
          0 <= abs x
          x <= x + 1
          x <= x + abs y
          x <= abs (x + x)
          x <= 1 + abs x
          0 <= x + abs x
      x + y <= x + abs y
abs (x + 1) <= 1 + abs x
\end{minted}
\end{minipage}
\begin{minipage}[t]{0.55\textwidth}
\begin{minted}{text}
    x <= y ==> x <= abs y
abs x <= y ==> x <= y
 abs x < y ==> x <  y
    x <= 0 ==> x <= abs y
abs x <= y ==> 0 <= y
 abs x < y ==> 1 <= y
    x == 1 ==> 1 == abs x
     x < 0 ==> 1 <= abs x
    y <= x ==> abs (x + abs y) == x + abs y
    x <= 0 ==>       x + abs x == 0
abs x <= y ==>     abs (x + y) == x + y
abs y <= x ==>     abs (x + y) == x + y
\end{minted}
\end{minipage}
\caption{Inequalities and conditional equations.}\label{lst:arith_props1}
\end{sublisting}
\caption{Properties of arithmetic, discovered by Speculate.}\label{lst:arith_props}
\end{listing}

\paragraph{Machine learning}
The Daikon\cite{ernst2007} tool discovers \emph{likely invariants} of
C, C++, Java, and Perl programs.  It observes variables in memory
during the execution of a program, and applies machine learning
techniques to discover properties that seem to hold.  These properties
may include: pre- and post-conditions of statements, and equational
relationships between variables at a given program point and functions
from a library.  Daikon does not synthesise and test program terms,
however.  Daikon is provided with a grammar describing patterns of
invariants, and reports which are observed to hold as the program
executes.  Properties found by Daikon correspond to assertions which
could be inserted into the program, whereas the other tools described
here discover properties based on the program API.

\paragraph{Concurrency testing}
A variant of the Daikon tool discovers likely invariants of concurrent
C and C++ programs using code instrumentation and systematic
concurrency testing techniques\cite{kusano2015}.  The invariants it
finds are so-called \emph{transition invariants} that capture the
relations amongst mutable state shared between threads.

\begin{listing}
\centering
\begin{minipage}[t]{0.3\textwidth}
\begin{minted}{c}
/* Thread 1 */
p = &A
if (p != NULL) {
  p->x += 10;
}
\end{minted}
\end{minipage}
\begin{minipage}[t]{0.3\textwidth}
\begin{minted}{c}
/* Thread 2 */
p = NULL;
\end{minted}
\end{minipage}
\caption{Two threads accessing a shared pointer.}\label{lst:cthreads}
\end{listing}

\Cref{lst:cthreads} shows two threads accessing a shared pointer.  If
Thread 2 executes \verb|p = NULL| after Thread 1 checks that
\verb|p != NULL| but before it executes the assigment
\verb|p->x += 10|, then an error will occur.  Correct executions of
the program will produce the invariant \verb|p == orig(p)| for that
if-statement, meaning that \verb|p| is unchanged.  Buggy executions
will not.  The authors argue that examining discrepancies between
invariants can lead to greater understanding of the software under
test and diagnosis of errors.

The \textsc{Determin} tool\cite{burnim2010} infers deterministic
specifications for procedures which make use of internal parallelism.
A program may have many such procedures.  These specifications are in
the form of a precondition and a postcondition over program states.
If we use $P(s, \sigma)$ to denote the resulting program state after
executing procedure $P$ in an initial state $s$ with a schedule
$\sigma$, then specifications are of the form,

\[
\forall s, s', \sigma, \sigma'.~~
\mathrm{Pre}\left(s, s'\right) \implies
\mathrm{Post}\left(P\left(s,\sigma\right),P\left(s', \sigma'\right)\right)
\]

\noindent
For example, if the precondition is $s = s'$ and the postcondition is
$v = v'$, where $v$ is some variable assigned to by $P$, then the
overall specification can be read as ``for all schedules $\sigma$ from
state $s$, the variable $v$ gets the same value (if execution
terminates).''

\paragraph{Example-driven property discovery}
The Bach\cite{smith2017} tool uses a database of examples of
input/output values from functions to synthesise properties using a
Datalog-based oracle.  As it is based on examples, it is not tied to
any particular programming language.  Bach could even be used to
discover properties of hardware components!  Properties are of the
form $G \implies P$, where both $G$ and $P$ are conjunctions of
equalities $f(x) = y$, where $f$ is some function in the database, and
$x$ and $y$ may be constants or variables.  It uses a notion of
\emph{evidence} to decide whether an inferred property holds: negative
evidence consists of counterexamples; positive evidence consists of
witnesses.  Bach crucially requires functions to have at most one
output for each distinct input, to construct negative evidence.

\section{Summary}

Going forward, the reader should keep in mind:

\begin{itemize}
\item Properties are universally quantified boolean functions which
  are expected to hold for all input values.
\item Property discovery tools, typically, take a description of a
  program API and return properties which appear to hold, based on
  testing.  In \Cref{chp:coco} we describe a property discovery tool
  where the API may involve concurrency.
\end{itemize}
