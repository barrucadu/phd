There is a tension between theory and practice in software
engineering.  Whenever one programmer suggests some sophisticated
technique or formalism, another will question its applicability to the
\emph{real world}, a nebulous and under-specified place.  For example,
a common argument against techniques such as model checking is that
they do not scale to real-world problems, whatever those are, and that
stress testing is good enough.  Stress testing, aided by dynamic
analyses like Clang's ThreadSanitizer \parencite{serebryany2009} or
Go's data race detector \parencite{golang2017}, is often effective at
finding flaws, and is less alien to most programmers.  However, unlike
stress testing, model checking can prove the absence of bugs, though
it can be difficult or resource-intensive to use.

In this thesis, we are interested in the testing of concurrent
programs.  In this setting, stress testing falls down when considering
questions like:

\begin{itemize}
\item How do we ensure that we're covering a variety of schedules?
\item How do we know a that bug has been fixed?
\item How much testing is enough?
\end{itemize}

We try to follow a middle path between the familiarity of testing and
the power of model checking.  By using \emph{systematic concurrency
  testing}, we enable programmers to test their concurrent programs
deterministically, and confidently.

A concern in academia is the tension between theoretical novelty and
practical utility.  Too often programs written during research are
abandoned as unpolished prototypes.  This practice harms the spread of
ideas from theory into practice, which is particularly regrettable
when the ideas are intended to make programmers' jobs easier.  We
adopt a stance more in favour of practical utility than is perhaps
typical.  By producing polished and featureful tools, we enable
programmers to benefit immediately from our work.

We use Haskell as the implementation language, and the source of the
concurrency abstraction we target, in this thesis.  However, our work
is not tied to Haskell.  \Cref{app:haskell} gives a brief introduction
to Haskell for the reader already somewhat familiar with functional
programming in other languages.

\section{Parallelism vs Concurrency}
\label{sec:intro-parconc}

The terms \emph{parallelism} and \emph{concurrency} are broadly, but
not quite, synonymous.  Following the lead of \cite{peytonjones1996},
we use them to refer to different but related concepts:

\paragraph{Parallelism}
A parallel program uses a multiplicity of hardware to compute
different aspects of a result simultaneously.  The goal is to arrive
at the overall result more quickly.  For example, the x86 assembly
instruction \verb#PMULHUW# computes the element-wise multiplication of
two vectors, performing each multiplication simultaneously: it enables
parallelism.

\paragraph{Concurrency}
A concurrent program uses multiple threads of control to structure the
program.  These threads conceptually execute independently and at the
same time.  But whether threads do execute simultaneously is an
implementation detail.  A concurrent program can execute on a
single-core machine through interleaved sequential execution just as
it can execute on a multi-core machine in parallel.  A concurrency
abstraction can guarantee parallelism (given suitable hardware), for
example by having the ability to restrict the execution of individual
threads to given processor cores.

It is tempting to think of parallelism as being \emph{semantically
  invisible}: not changing the result of a program, merely making it
faster.  However, on modern processors, parallelism is semantically
\emph{visible}.  This thesis is mostly concerned with concurrency, but
the \emph{relaxed memory} behaviour of modern processors, an artefact
of parallelism, appears in \Cref{chp:dejafu}.

\section{Goals and Contributions of this Thesis}
\label{sec:intro-contributions}

The overall motivation of this research has been to develop tools
which make it easier for programmers to write correct concurrent
programs.

The primary goal of this thesis is to demonstrate that concurrency
testing techniques, typically described in the context of a simple
core language, can be successfully applied to languages with rich
concurrency abstractions.  Specifically, we want:

\begin{itemize}
\item A concurrency testing tool which simultaneously supports
  multiple aspects of concurrency which are traditionally considered
  difficult: such as relaxed memory, software transactional memory,
  and inter-thread signals.
\item To show that supporting these features does not render a tool
  too expensive, in time or space, to be of practical value.
\end{itemize}

To meet and demonstrate our objectives, we develop libraries and tools
for testing Concurrent Haskell programs, and evaluate their
effectiveness on sample applications.

Our major contributions are:

\begin{itemize}
\item A library for effectively testing Concurrent Haskell programs,
  in \Cref{chp:dejafu}.  We demonstrate its effectiveness with case
  studies of three concurrency-using Haskell libraries.

  There are no sound and complete concurrency testing tools for
  Haskell.  Our tool fills this niche.
\item A new scheduling algorithm for randomised testing to allow
  testing programs where complete testing does not scale, in
  \Cref{chp:algorithms}.  We evaluate its bug-finding ability on a
  standard set of benchmarks.

  Randomised testing can be both fast and effective, but existing
  algorithms such as PCT \parencite{burckhardt2010} require
  difficult-to-obtain information about the program under test.  Ours
  does not.
\item A tool for discovering properties of Haskell functions operating
  on shared mutable state in the presence of concurrent interference,
  in \Cref{chp:coco}.  We give case studies of three concurrent data
  structures.

  Tools such as QuickSpec \parencite{smallbone2017} and Speculate
  \parencite{braquehais2017} can discover properties of pure
  functions.  Our approach extends this to nondeterministic and
  concurrent program fragments.
\end{itemize}

In achieving these major contributions, we also achieve several
smaller ones along the way:

\begin{itemize}
\item An abstraction over the GHC Haskell concurrency API, in
  \Cref{sec:dejafu-monadconc}.

  We allow the user to give names to threads and shared variables to
  make debugging easier.
\item An operational semantics for Concurrent Haskell, in
  \Cref{sec:dejafu-semantics}.

  Our semantics is similar in spirit to \cite{vollmer2017}, however we
  model a much larger set of operations, and support more memory
  models.
\item A method for soundly incorporating daemon threads into a
  concurrency testing setting, in \Cref{sec:dejafu-testing}.

  Daemon threads are threads which automatically terminate when the
  main thread terminates.  They are typically omitted from concurrency
  testing algorithms, and a straightforward implementation is too
  inefficient to be practical.
\item An algorithm for semantics-preserving simplification of
  execution traces, in \Cref{sec:dejafu-traces}.

  These simplified traces are easier for a programmer to follow than a
  trace which may contain unnecessary details and scheduling
  decisions.
\item Case studies of applying concurrency testing to three Haskell
  libraries, in \Cref{sec:dejafu-casestudies}.

  Through these case studies we demonstrate that our tools are of
  practical use to programmers.
\item A method for generating program fragments containing lambda
  terms, in a restricted setting, in \Cref{sec:coco-hiw-gen}.

  Lambda terms are essential in generating complex monadic
  expressions.
\end{itemize}

\section{Roadmap}
\label{sec:intro-roadmap}

This thesis is divided into three parts:

\paragraph{\Cref{part:review}}
We present the context and background of the work.
\Cref{chp:concurrent_haskell} gives an introduction to concurrency in
Haskell.  \Cref{chp:sct} discusses the theory behind \emph{testing}
concurrent programs.  Finally, \Cref{chp:property_testing} gives an
introduction to property testing in Haskell.

\paragraph{\Cref{part:testing}}
We present our contributions.  \Cref{chp:dejafu} gives an account of
the \dejafu{} tool for testing concurrent Haskell programs, discussing
the scope, implementation, and some case studies.
\Cref{chp:algorithms} discusses an alternative scheduling algorithm
for testing concurrent programs.  Finally, \Cref{chp:coco} gives an
account of the CoCo tool for discovering properties of concurrent
Haskell programs, discussing the scope, implementation, and some case
studies, and finally shows how it connects to \dejafu{}.

These chapters build on each other:

\begin{itemize}
\item \dejafu{}, in \Cref{chp:dejafu}, allows exploring the behaviours
  of a concurrent program, and on top of this foundation we build an
  equivalence checker.
\item Swarm scheduling, in \Cref{chp:algorithms}, is an alternative
  scheduling algorithm which can be plugged into \dejafu{}.
\item CoCo, in \Cref{chp:coco}, builds on the equivalence checker in
  \dejafu{}, by generating and testing candidate equivalences from
  programmer-supplied primitives.
\end{itemize}

\paragraph{\Cref{part:end}}
We present our overall conclusions in \Cref{chp:conclusions} and
suggest possible future work in \Cref{chp:future_work}.

\paragraph{Source availability}
The \dejafu{} and CoCo tools we develp in \Cref{part:testing} are
available on GitHub:

\begin{itemize}
\item \url{https://github.com/barrucadu/dejafu}
\item \url{https://github.com/barrucadu/coco}
\end{itemize}

% [layout hack]: there's so little on this page the indent looks
% really weird
\noindent
\dejafu{} and its related libraries are also available on Hackage:

\begin{itemize}
\item \url{https://hackage.haskell.org/package/concurrency}
\item \url{https://hackage.haskell.org/package/dejafu}
\item \url{https://hackage.haskell.org/package/hunit-dejafu}
\item \url{https://hackage.haskell.org/package/tasty-dejafu}
\end{itemize}
