When writing concurrent programs, programmers tend to fall into one of two
camps:

\begin{itemize}
\item ``If you want a reliable concurrent program, you have to model-check your
core algorithms.''
\item ``Model checking is too expensive, if you want a reliable real-world
concurrent program, you have to stress test it.''
\end{itemize}

Model checking can prove the absence of bugs, but can be difficult to use, and
requires programmers to learn yet another tool.  Stress testing, aided by
dynamic analyses like Clang's ThreadSanitizer or Go's data race detector, often
works but suffers from the problem of knowing when the testing is \emph{enough}.

We follow a middle path: between the familiarity of writing tests and the power
of model checking.  \emph{Systematic concurrency testing} lets us answer with
certainty questions like:

\begin{itemize}
\item How do we ensure that we're covering a lot of different schedules?
\item How do we know a that bug has really been fixed?
\item \emph{How much testing is enough?}
\end{itemize}

Haskell is used in this thesis.  \Cref{app:haskell} gives a brief introduction to
Haskell, along with a syntax reference and list of common library functions.

\section{Parallelism vs Concurrency}
\label{sec:intro-parconc}

In many fields, \emph{parallel} and \emph{concurrent} are synonyms, both
referring to the act of doing multiple things at once.  In programming, they
refer to different but related concepts.

A \emph{parallel} program uses a multiplicity of hardware to compute different
aspects of a result simultaneously.  The goal is to arrive at the overall result
more quickly.  For example, the x86 assembly instruction \verb#PMULHUW# computes
the element-wise multiplication of two vectors, each multiplication
simultaneously: it enables parallelism.

A \emph{concurrent} program uses a notion of multiple independently executing
threads of control to structure the program.  These threads conceptually execute
at the same time, their effects may be interleaved.  Whether threads actually do
execute simultaneously is an implementation detail.  A concurrent program can
execute on a single-core machine through interleaved sequential execution just
as it can execute on a multi-core machine in parallel.

Concurrency is often implemented using parallelism.  A concurrency abstraction
can even guarantee parallelism (given suitable hardware), for example by having
the ability to restrict the execution of individual threads to given processor
cores.

This thesis is mostly concerned with concurrency, but parallelism does rear its
head in the discussion of \emph{relaxed memory} in \cref{chp:dejafu}.

\section{Goals and Contributions of this Thesis}
\label{sec:intro-contributions}

The primary contribution of this thesis is a library for testing concurrent
Haskell programs.  This library demonstrates that concurrency testing
techniques, typically described in the context of a very simple core language,
can be successfully applied to a language with a very rich concurrency
abstraction such as Haskell.

Our contributions can be seen as follows:

\begin{itemize}
\item A method for the systematic testing of concurrent Haskell programs based
on existing work from the imperative and object-oriented worlds.
\item A new scheduling algorithm for probabilistically finding faults in concurrent
programs based on random search.
\item A generalisation of property discovery techniques to find properties of
stateful, impure, operations.
\end{itemize}

The overall goal of this research has been to contribute tools which make it
easier for programmers to write correct concurrent programs.

\section{Roadmap}
\label{sec:intro-roadmap}

This thesis is divided into three parts:

\begin{itemize}
\item \Cref{part:review} discusses the context of the work.
  \Cref{chp:concurrent_haskell} gives an introduction to concurrency
  in Haskell.  \Cref{chp:sct} discusses the theory behind
  \emph{testing} concurrent programs.  Finally,
  \cref{chp:property_testing} gives an introduction to property
  testing in Haskell.

\item \Cref{part:testing} is about \dejafu{}, our tool for testing
  concurrent Haskell programs, and CoCo, our tool for discovering
  properties of concurrent programs.  \Cref{chp:dejafu} gives an
  account of the \dejafu{} tool, covering the scope, implementation,
  and some case studies.  \Cref{chp:algorithms} discusses an
  alternative scheduling algorithm for testing concurrent programs.
  Finally, \cref{chp:coco} gives an account of the CoCo tool, covering
  the scope, implementation, some case studies, and shows how it
  connects to \dejafu{}.

\item \Cref{part:end} evaluates the work and discusses our conclusions
  in \cref{chp:conclusions} and suggests possible future work in
  \cref{chp:future_work}.
\end{itemize}

\Cref{chp:concurrent_haskell,chp:property_testing} should be
accessible to anyone with the assumed knowledge detailed in
\cref{app:haskell}\footnote{Or to the particularly motivated beginner
  who also has an introductory Haskell tutorial open.}.
