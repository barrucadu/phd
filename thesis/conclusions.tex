\blindtext

\section{Evaluation}
\label{sec:conclusions-evaluation}

\paragraph{Quality of a concurrency test set}
Being able to deterministically test concurrent programs is a
necessary step towards finding concurrency bugs more easily than
verification and with less disruption to users than waiting for bug
reports.  However, any testing regimen is only as good as the tests
that get written.  How do we come to believe that a testsuite is
strong evidence for the correctness of some code?

For sequential programs, we can use the traditional metric of code
coverage.  The continuous measure of coverage is difficult to
interpret, but the binary measure is not: code which is not covered at
all usually has more bugs than code which is covered by even
low-quality tests\cite{ahmed2016}.

For concurrent programs, what metric do we use?  If it is some notion
of coverage, what even is the space being covered?  There are a few
candidates:

\begin{itemize}
\item Schedule-sensitive branches are often unintentional and
  erroneous points of synchronisation between concurrent
  threads\cite{huang2015ssb}.  A good concurrency testsuite should try
  all cases in a schedule-sensitive branch.

\item Shared state which is not guarded by appropriate synchronisation
  can lead to invalid or corrupt data.  If we have functions which
  operate on some mutable state of the same type, then a good
  concurrency testsuite will check what happens when those states are
  shared and the functions executed concurrently.  If the state is a
  product type, a good concurrency testsuite will also check what
  happens when the states are only partially shared.
\end{itemize}

However, correct usage of a concurrent API often relies on assumptions
stated in documentation, cautioning users where they must incorporate
additional synchronisation.  An automated metric cannot read
documentation, and so may highlight forbidden interactions as needing
tests.

Finding an easily understood metric, like code coverage, which is both
automatic and useful in revealing legal but untested interactions, is
an open question.
