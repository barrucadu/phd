\section{MonadConc}
\label{app:primops-conc}

\newcommand{\defineprim}[3]{\item[#1 \textit{#2}] \hfill (#3)\\}

There are 33 primitive actions used to construct the testing instance of
\verb|MonadConc|\footnote{\url{https://github.com/barrucadu/dejafu/blob/master/dejafu/Test/DejaFu/Conc/Internal/Common.hs}, accurate at commit \texttt{c6af27f} (dejafu-0.9.0.0).}.
These primitive actions contain a continuation, allowing individual actions to
be composed into larger execution sequences.  Each thread of execution consists
of such a sequence, terminated by the \verb|AStop| primitive, which has no
continuation and signals the successful termination of the thread.

\subsection{Threading}

\begin{description}
\defineprim{AFork}{name (umask \arr action) (thread\_id \arr action)}{unsynchronised}
  Create a new thread from the first action, and continue executing the
  current thread with the second.

\defineprim{AMyTId}{(thread\_id -> action)}{unsynchronised}
  Get the ID of the current thread.

\defineprim{AYield}{action}{unsynchronised}
  Execute the action, but also signal to the scheduler that it may be worth
  running another thread next.

\defineprim{ADelay}{delay action}{unsynchronised}
  The same as \texttt{AYield}, but include the delay time in the execution
  trace.

\defineprim{AStop}{effect}{unsynchronised}
  Run the effect and terminate the current thread.
\end{description}

\subsection{\texttt{CRef}}

\begin{description}
\defineprim{ANewCRef}{name a (cref a \arr action)}{unsynchronised}
  Construct a new \verb|CRef| and give it to the continuation.

\defineprim{AReadCRef}{(cref a) (a \arr action)}{unsynchronised}
  Read the currently visible value of a \verb|CRef|.

\defineprim{AReadCRefCas}{(cref a) (ticket a \arr action)}{unsynchronised}
  Produce a \verb|Ticket| from the currently visible state of a
  \verb|CRef|.

\defineprim{AModCRef}{(cref a) (a \arr (a, b)) (b \arr action)}{synchronised}
  Commit all pending writes and atomically modify the value within a
  \verb|CRef|.

\defineprim{AModCRefCas}{(cref a) (a \arr (a, b)) (b \arr action)}{partially}
  Commit all pending writes and atomically modify the value within a
  \verb|CRef| using a compare-and-swap.

\defineprim{AWriteCRef}{(cref a) a action}{unsynchronised}
  Update the value of a \verb|CRef|.  The updated value is visible to the
  current thread immediately.

\defineprim{ACasCRef}{(cref a) (ticket a) a ((succeeded?, ticket a) \arr action)}{partially}
  Update the value of a \verb|CRef| if it hasn't changed since the ticket
  was produced.

\defineprim{ACommit}{thread\_id cref\_id}{partially}
  Make the last write to the given \verb|CRef| by that thread visible to all
  threads.
\end{description}

\subsection{\texttt{MVar}}

\begin{description}
\defineprim{ANewMVar}{name (mvar a \arr action)}{unsynchronised}
  Construct a new \verb|MVar| and give it to the continuation.

\defineprim{APutMVar}{(mvar a) a action}{synchronised}
  Block until the \verb|MVar| is empty and put a value into it.

\defineprim{ATryPutMVar}{(mvar a) a (succeeded? \arr action)}{synchronised}
  Try to put a value into the \verb|MVar| without blocking.

\defineprim{AReadMVar}{(mvar a) (a \arr action)}{synchronised}
  Block until the \verb|MVar| is full and read its value.

\defineprim{ATryTakeMVar}{(mvar a) (Maybe a \arr action)}{synchronised}
  Try to read the value from a \verb|MVar| without blocking.

\defineprim{ATakeMVar}{(mvar a) (a \arr action)}{synchronised}
  Block until the \verb|MVar| is full and take its value.

\defineprim{ATryTakeMVar}{(mvar a) (Maybe a \arr action)}{synchronised}
  Try to take the value from a \verb|MVar| without blocking.
\end{description}

\subsection{Exceptions}

\begin{description}
\defineprim{AThrow}{exception}{unsynchronised}
  Raises an exception in the current thread.

\defineprim{AThrowTo}{thread\_id exception action}{synchronised}
  Raises an exception in the other thread, blocking if the other thread has
  exceptions masked.

\defineprim{ACatching}{(exception \arr (handler a)) (inner a) (a \arr action)}{unsynchronised}
  Registers a new exception handler for the duration of the inner action.

\defineprim{APopCatching}{action}{unsynchronised}
  Remove the exception handler from the top of the stack.

\defineprim{AMasking}{masking\_state (unmask \arr inner a) (a \arr action)}{unsynchronised}
  Executes the inner action under a new masking state, and also gives
  it a function to reset the masking state.

\defineprim{AResetMask}{set? inner? masking\_state action}{unsynchronised}
  Sets the masking state.
\end{description}

\subsection{Capabilities}

\begin{description}
\defineprim{AGetNumCapabilities}{(num \arr action)}{unsynchronised}
  Get the number of capabilities (returns 2 if not set).

\defineprim{ASetNumCapabilities}{num action}{unsynchronised}
  Set the number of capabilities.
\end{description}

\subsection{Subconcurrency}

\begin{description}
\defineprim{ASub}{(inner a) (result a \arr action)}{unsynchronised}
  Execute the inner action, then call the continuation with the result.
  The continuation is always executed, even if the inner action fails.

\defineprim{AStopSub}{action}{unsynchronised}
  Signals the end of the \textbf{ASub} inner action.
\end{description}

\subsection{Miscellaneous}

\begin{description}
\defineprim{AAtom}{(stm a) (a \arr action)}{synchronised}
  Atomically execute an STM transaction.

\defineprim{ALift}{(n action)}{unsynchronised}
  Execute an action in the underlying monad.

\defineprim{AReturn}{action}{unsynchronised}
  Execute the given action.
\end{description}

\section{MonadSTM}
\label{app:primops-stm}

\renewcommand{\defineprim}[2]{\item[#1 \textit{#2}]}

There are 8 primitive actions used to construct the testing instance of
\verb|MonadSTM|\footnote{\url{https://github.com/barrucadu/dejafu/blob/master/dejafu/Test/DejaFu/STM/Internal.hs},
  accurate at commit \texttt{c6af27f} (dejafu-0.9.0.0).}.  These primitive
actions contain a continuation, allowing individual actions to be composed into
larger execution sequences.  A transaction consists of such a sequence,
terminated by the \verb|SStop| primitive, which has no continuation and signals
the successful completion of the transaction.

\begin{description}
\defineprim{SNew}{name a (tvar a \arr action)}
  Create a new \verb|TVar| containing the given value.

\defineprim{SRead}{(tvar a) (a \arr action)}
  Read the current value of a \verb|TVar|.

\defineprim{SWrite}{(tvar a) a action}
  Update a \verb|TVar|.

\defineprim{SThrow}{exception}
  Throw an exception, aborting the current execution flow.

\defineprim{SCatch}{(exception \arr (handler a)) (inner a) (a \arr action)}
  Registers a new exception handler for the duration of the action.

\defineprim{SRetry}{}
  Abort the current transaction.

\defineprim{SOrElse}{(inner a) (inner a) (a \arr action)}
  Try executing the first transaction, if it fails, execute the
  second.
\end{description}
