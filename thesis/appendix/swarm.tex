\begin{listing}
\centering
\begin{cminted}{c++}
void SwarmScheduler::Explore(State *init_state) {
  State *state = init_state;
  unsigned int steps = 0;

  while (!state->IsTerminal()) {
    AssignWeightsToNewThreads(state);

    auto it = PickNextRandom(state);
    for(unsigned int cpoint : changePoints) {
      if(steps == cpoint) {
        AssignWeightTo(it.first->uid());
      }
    }

    state = Execute(state, it.second);
    steps++;
  }
}

void SwarmScheduler::AssignWeightsToNewThreads(State *state) {
  auto enabled = state->enabled();
  for(auto it = enabled->begin(); it != enabled->end(); it++) {
    auto threadId = it->first->uid();
    if(weights.find(threadId) == weights.end()) {
      AssignWeightTo(threadId);
    }
  }
}

void SwarmScheduler::AssignWeightTo(uint32 threadId) {
  std::uniform_int_distribution<uint8> weightDist(1, 50);
  weights[threadId] = weightDist(random);
}

std::pair<systematic::Thread* const, systematic::Action*>
SwarmScheduler::PickNextRandom(State *state) {
  auto enabled = state->enabled();

  // get the enabled threads and their weights
  std::list<uint8> e_ws;
  for(auto it = enabled->begin(); it != enabled->end(); it++) {
    e_ws.push_back(weights[it->first->uid()]);
  }

  // pick an enabled thread by a weighted random choice
  auto it = enabled->begin();
  std::discrete_distribution<int> dist(e_ws.begin(), e_ws.end());
  std::advance(it, dist(random));
  return *it;
}
\end{cminted}
\caption{The core of the C++ swarm scheduling algorithm, implemented in Maple.}\label{lst:swarm}
\end{listing}

\Cref{lst:swarm} shows the core of the swarm scheduling algorithm from
\Cref{chp:algorithms}, as implemented in C++ in Maple.  There are four
methods:

\begin{itemize}
\item \verb|Explore| is called by Maple, and is what drives the
  execution of the program.  It picks a thread to run and calls
  \verb|Execute|, which causes Maple to advance that thread to the
  next scheduling point.

\item \verb|AssignWeightsToNewThreads| is called by \verb|Explore|.
  It checks that every enabled thread has a weight in the
  \verb|weights| map, assigning a weight if not.  We only consider
  enabled threads, so if a thread is blocked when it is created and
  never becomes enabled, it will never receive a weight.

\item \verb|AssignWeightTo| is called by
  \verb|AssignWeightsToNewThreads| for each new thread, and by
  \verb|Explore| when a weight change point is encountered.  It
  assigns a new weight in the range $[w_{min}, w_{max}]$, which we
  have hard-coded here to be 1 and 50, respectively.

\item \verb|PickNextRandom| is called by \verb|Explore|, and is what
  chooses the thread.  It produces a list of the weights of all
  enabled threads, and uses that list as a discrete distribution.  As
  this method is called after \verb|AssignWeightsToNewThreads|, every
  enabled thread will have a weight.
\end{itemize}
