\begin{itemize}
\item \verb|foo = bar| binds a name to a value
\begin{minted}{haskell}
five = 5
\end{minted}

\item Function definitions add argument names after the function name
\begin{minted}{haskell}
increment n = n + 1
\end{minted}

\item Function calls have no parentheses
\begin{minted}{haskell}
six = increment five
\end{minted}

\item Function calls are left associative
\begin{minted}{haskell}
seven = increment (increment five)
\end{minted}

\item Function calls take precendence over operators
\begin{minted}{haskell}
incAndAdd x y = increment x + increment y
\end{minted}

\item Anonymous functions are declared with \verb|\args -> expr|
\begin{minted}{haskell}
double = \x -> incAndAdd x x - 2
\end{minted}

\item Functions are curried, `multi-argument functions' are syntactic
  sugar
\begin{minted}{haskell}
incAndAdd x = \y -> increment x + increment y
\end{minted}

\item Functions are first-class values
\begin{minted}{haskell}
compose f g x = f (g x)
\end{minted}

\item Operators are just functions with symbolic names
\begin{minted}{haskell}
f . g = compose f g
\end{minted}

% [layout hack]: ensure the text and the code are on the same page
\pagebreak
\item The \verb|($)| operator is right-associative function application
\begin{minted}{haskell}
compose f g x = f $ g x
\end{minted}
%$

\item We can use a \verb|let| expression to introduce local definitions
\begin{minted}{haskell}
sumOf3 x y z = let temp = x + y in temp + z
\end{minted}

\item Or a \verb|where| clause
\begin{minted}{haskell}
sumOf3 x y z = temp + z where
  temp = x + y
\end{minted}

\item We can annotate a definition or a value with a type using \verb|::|
\begin{minted}{haskell}
name :: Name
name = "Michael Walker"
\end{minted}

\item \verb|type| gives an existing type a new alias
\begin{minted}{haskell}
type Name = String
\end{minted}

\item \verb|data| defines a new data type
\begin{minted}{haskell}
data Colour
  = RGB Int Int Int
  | Grey Double

magenta :: Colour
magenta = RGB 255 0 255
\end{minted}

\item Records allow us to name the fields in a type
\begin{minted}{haskell}
data RGB = RGB
  { red   :: Int
  , green :: Int
  , blue  :: Int
  }

red :: RGB
red = RGB { red = 255, green = 0, blue = 0 }
\end{minted}

\item \verb|->| denotes the type of a function
\begin{minted}{haskell}
sum3 :: Int -> Int -> Int -> Int
sum3 x y z = x + y + z
\end{minted}

% [layout hack]: ensure the text and the code are on the same page
\pagebreak
\item \verb|->| is right associative, these types are all equivalent
\begin{minted}{haskell}
sum3 :: Int -> Int -> Int -> Int
sum3 :: Int -> Int -> (Int -> Int)
sum3 :: Int -> (Int -> Int -> Int)
sum3 :: Int -> (Int -> (Int -> Int))
\end{minted}

\item Names that start with an upper case letter in types are concrete
  types
\begin{minted}{haskell}
intId :: Int -> Int
intId x = x
\end{minted}

\item Names that start with a lower case letter in types are type
  variables
\begin{minted}{haskell}
id :: a -> a
id x = x
\end{minted}

\item We can use type variables when defining types
\begin{minted}{haskell}
data Maybe  a   = Just a | Nothing
data Either a b = Left a | Right b
\end{minted}

\item The list type constructors are \verb|:| and \verb|[]|
\begin{minted}{haskell}
-- sugar: not really a legal data definition!
data [a] = a : [a] | []
\end{minted}

\item There are also tuples, each size a different type
\begin{minted}{haskell}
-- sugar: not really legal data definitions!
data (a, b) = (a, b)
data (a, b, c) = (a, b, c)
-- ...
\end{minted}

\item The `unit' type looks like an empty tuple
\begin{minted}{haskell}
-- sugar: not really a legal data definition!
data () = ()
\end{minted}

\item \verb|case| expressions let us pattern match on data
  constructors
\begin{minted}{haskell}
fromMaybe :: a -> Maybe a -> a
fromMaybe def m = case m of
  Just a  -> a
  Nothing -> def
\end{minted}

\item Function definitions can also pattern match on their arguments
\begin{minted}{haskell}
maybe :: a -> (a -> b) -> Maybe a -> b
maybe def f (Just a) = f a
maybe def f Nothing  = def
\end{minted}

\item The pattern \verb|_| matches anything
\begin{minted}{haskell}
ifThenElse :: Bool -> a -> a -> a
ifThenElse True t _ = t
ifThenElse _    _ f = f
\end{minted}

\item Typeclasses are used to group common behaviour, such as equality
\begin{minted}{haskell}
class Eq a where
  (==) :: a -> a -> Bool

instance Eq Bool where
  True  == True  = True
  False == False = True
  _ == _ = False
\end{minted}

\item Typeclass constraints can appear in type signatures
\begin{minted}{haskell}
eq3 :: Eq a => a -> a -> a -> Bool
eq3 a b c = a == b && b == c
\end{minted}

\item The \verb|Functor| typeclass represents `contexts' which can be
  mapped over
\begin{minted}{haskell}
class Functor f where
  fmap :: (a -> b) -> f a -> f b
\end{minted}

\item \verb|fmap| is sometimes written \verb|<$>|
\begin{minted}{haskell}
(<$>) :: Functor f => (a -> b) -> f a -> f b
(<$>) = fmap
\end{minted}

\item The \verb|Applicative| typeclass extends \verb|Functor| with the
  ability to wrap up a value
\begin{minted}{haskell}
class Functor f => Applicative f where
  pure :: a -> f a
\end{minted}

\item \ldots{}and to apply a value-in-a-context to a
  function-in-a-context
\begin{minted}{haskell}
class Functor f => Applicative f where
  (<*>) :: f (a -> b) -> f a -> f b
  pure  :: a -> f a
\end{minted}

\item The \verb|Monad| typeclass extends \verb|Applicative| with the
  ability to compose contexts
\begin{minted}{haskell}
class Applicative m => Monad m where
  (>>=) :: m a -> (a -> m b) -> m b

(>>) :: Monad m => m a -> m b -> m b
ma >> mb = ma >>= \_ -> mb
\end{minted}

\item do-notation is a syntactic sugar for sequencing monadic actions
\begin{minted}{haskell}
main :: IO ()
main = do
  putStrLn "hello"
  putStrLn "world"
\end{minted}

\item \verb|<-| is used to bind the result of a monadic action to a
  name in do-notation
\begin{minted}{haskell}
main :: IO ()
main = do
  putStrLn "What's your name?"
  name <- getLine
  putStrLn ("Hello " ++ name)
\end{minted}

\item \verb|let| is used to bind an expression to a name in
  do-notation
\begin{minted}{haskell}
main :: IO ()
main = do
  let sum = sum3 1 10 100
  print sum
\end{minted}

\item The \verb|RankNTypes| language extension allows universally
  quantified type variables
\begin{minted}{haskell}
pair :: (forall x. x -> x) -> a -> b -> (a, b)
pair f a b = (f a, f b)
\end{minted}

\item The \verb|TypeFamilies| language extension allows types to be
  associated with a typeclass
\begin{minted}{haskell}
class Tower a where
  type Next a :: *
  promote :: a -> Next a

instance Tower Word8 where
  type Next Word8 = Word16
  promote = fromIntegral
\end{minted}
\end{itemize}
