Tools are necessary to test concurrency, the standard approach does not suffice.
In this chapter we present and evaluate \dejafu{}, our library for testing
concurrency in Haskell.  We discuss the scope of the tool~\sref{dejafu-scope}
and present our abstraction over the GHC Haskell concurrency
functionality~\sref{dejafu-monadconc}.  We explain how programs using our
abstraction are executed~\sref{dejafu-execution} and
tested~\sref{dejafu-testing}, and argue the correctness of the testing
approach~\sref{dejafu-correctness}.  We present three case
studies~\sref{dejafu-casestudies}, evaluate the usefulness of \dejafu{} for
testing pre-existing code~\sref{dejafu-evaluation}, and finally draw conclusions
and present further work~\sref{dejafu-conclusions}.

This chapter is derived from our previous work \cite{walker2015} and
\cite{YCS-2016-503}.

\todo{Note: discussion of random scheduling is in chapter 6 (swarm) and chapter 9 (eval \& conclusions), so just have brief forward refs in this chapter}

\section{Scope}
\label{sec:dejafu-scope}

We aim to support most of the functionality of GHC’s concurrency API, as made
available through the Control.Concurrent and Control.Exception module
hierarchies, which does not unavoidably require support from the runtime system.

In particular, we do not support:

\begin{itemize}
\item Blocking a thread until a file descriptor becomes available, as this
  introduces an additional source of nondeterminism.
\item Throwing an exception to a thread if it becomes deadlocked, as we cannot
  reliably detect deadlock involving only a subset of threads without support
  from the garbage collector.
\item Querying which capability (OS thread) a Haskell thread is running on, as
  this introduces an additional source of nondeterminism.
\end{itemize}

We also do not yet support \emph{bound threads}: a Haskell thread which will
always run on the same, unique, OS thread.  Bound threads are essential for
using the FFI to call libraries which use thread-local state, to ensure the
Haskell thread always sees its state and never the state of another thread.  We
have a prototype implementation, which is not yet present in a released version
of \dejafu{}\footnote{\url{https://github.com/barrucadu/dejafu/issues/126}}.

\section{Abstracting over \texttt{IO}}
\label{sec:dejafu-monadconc}

Recall from \secref{sct-fundamentals} that there are three ways of implementing
a concurrency testing tool:

\begin{itemize}
\item Override the concurrency primitives of the programming language.
\item Instrument the source program.
\item Instrument the compiled program.
\end{itemize}

We adopt the first approach in \dejafu{}.  Haskell's typeclass machinery lets us
specify an interface for concurrency, and to provide different concrete
implementations.  There is one implementation using the \verb|IO| type and the
standard functions; there is another using our own type, based on continuations
which we can inspect.

\begin{figure}[t]
  \centering
  \begin{lstlisting}
class (Monad m, {- other constraints omitted -}) => MonadConc m where
  type MVar m :: * -> *
  -- other types omitted

  newEmptyMVar :: m (MVar m a)
  newEmptyMVar = newEmptyMVarN ""

  newEmptyMVarN :: String -> m (MVar m a)
  newEmptyMVarN _ = newEmptyMVar

  putMVar  :: MVar m a -> a -> m ()
  readMVar :: MVar m a -> m a
  takeMVar :: MVar m a -> m a
  -- other operations omitted
  \end{lstlisting}
  \caption{A fragment of the \texttt{MonadConc} typeclass.}
  \label{fig:monadconc}
\end{figure}

We call our typeclass \verb|MonadConc| for monads which can do concurrency,
\figref{monadconc} shows a fragment.  When defining an instance of this class,
the programmer must supply concrete types for the abstract types in the
interface.  They must also supply implementations of, at least, all undefined
operations.  Some operations have default definitions: for example, there are
two ways of constructing an empty \verb|MVar|.  One way takes a name, which is
displayed in debugging information, the other does not.  Each is defined in
terms of the other, and so the programmer must supply at least one.

\begin{figure}[t]
  \centering
  \begin{lstlisting}
instance Monad n => MonadConc (ConcT r n) where
  type MVar (ConcT r n) = MVar r
  -- other types omitted

  newEmptyMVarN n = toConc (ANewMVar n)

  putMVar  var a = toConc (\c -> APutMVar var a (c ()))
  readMVar var   = toConc (AReadMVar var)
  takeMVar var   = toConc (ATakeMVar var)
  -- other operations omitted
  \end{lstlisting}
  \caption{The implementation of \figref{monadconc} we use for testing.}
  \label{fig:mvarops}
\end{figure}

The type for our testing implementation is called \verb|ConcT r n|, which is a
monad that has access to \emph{references} of type \verb|r| in a monad of type
\verb|n|.  \figref{mvarops} shows the implementation of \figref{monadconc} for
this type.  Each concurrency operation is of the same form: we take the
arguments and wrap them up inside a data structure whose final argument is a
continuation, which is then converted into a \verb|ConcT| value.

A concurrent computation is just a large value, where we can inspect each
``step'' of the computation by looking at the data constructor used.
Constructors mostly correspond to operations in the \verb|MonadConc| class
(there are also a few extra).  We call these constructors \emph{primitive
  actions}.  A full listing is available in \appref{primops}.

We make a few departures from the semantics of the \verb|IO| monad where
necessary:

\begin{itemize}
\item The \verb|getNumCapabilities| operation allows the programmer to query the
  number of capabilities.  During testing, we return ``2'', despite executing
  everything in the same OS thread.  This is to avoid special-case behaviour for
  one capability, which may reduce concurrency.
\item Runtime errors, such as pattern match failures, can be caught as
  exceptions inside \verb|IO|.  As there is no non-\verb|IO| way to do the same,
  \dejafu{} cannot.
\item The \verb|threadDelay| operation is required to yield the thread, but not
  necessarily to delay it.  This is because it is not clear how to incorporate
  time into the systematic concurrency testing model.
\end{itemize}

There is more to \verb|IO| than concurrency and exceptions.  \dejafu{} supports
testing computations with embedded \verb|IO| actions provided that the
programmer ensures that the \verb|IO| action is atomic; that it is deterministic
when executed with a fixed schedule; and that it does not block on the action of
another thread.  Failing to meet any of these conditions may lead to incomplete
testing.

\section{Executing Concurrent Programs}
\label{sec:dejafu-execution}

\blindtext

\section{Testing Concurrent Programs}
\label{sec:dejafu-testing}

\blindtext

\section{Soundness and Completeness}
\label{sec:dejafu-correctness}

\blindtext

\section{Case Studies}
\label{sec:dejafu-casestudies}

\blindtext

\subsection{The auto-update Package}
\subsection{Search Party}
\subsection{The Par Monad}

\section{\dejafu{} in the Wild}
\label{sec:dejafu-evaluation}

\blindtext

\subsection{Richness of the Abstraction}
\subsection{Porting Code}
\subsection{Integration with Existing Tools}

\section{Conclusions and Future Work}
\label{sec:dejafu-conclusions}

\blindtext
