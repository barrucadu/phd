Tools are necessary to test concurrency, the standard approach does not suffice.
In this chapter we present and evaluate \dejafu{}, our library for testing
concurrency in Haskell.  We discuss the scope of the tool~\sref{dejafu-scope}
and present our abstraction over the GHC Haskell concurrency
functionality~\sref{dejafu-monadconc}.  We explain how programs using our
abstraction are executed~\sref{dejafu-execution} and
tested~\sref{dejafu-testing}, and argue the correctness of the testing
approach~\sref{dejafu-correctness}.  We present three case
studies~\sref{dejafu-casestudies}, evaluate the usefulness of \dejafu{} for
testing pre-existing code~\sref{dejafu-evaluation}, and finally draw conclusions
and present further work~\sref{dejafu-conclusions}.

This chapter is derived from our previous work \cite{walker2015} and
\cite{YCS-2016-503}.

\todo{Note: discussion of random scheduling is in chapter 6 (swarm) and chapter 9 (eval \& conclusions), so just have brief forward refs in this chapter}

\section{Scope}
\label{sec:dejafu-scope}

We aim to support most of the functionality of GHC’s concurrency API, as made
available through the Control.Concurrent and Control.Exception module
hierarchies, which does not unavoidably require support from the runtime system.

In particular, we do not support:

\begin{itemize}
\item Blocking a thread until a file descriptor becomes available, as this
  introduces an additional source of nondeterminism.
\item Throwing an exception to a thread if it becomes deadlocked, as we cannot
  reliably detect deadlock involving only a subset of threads without support
  from the garbage collector.
\item Querying which capability (OS thread) a Haskell thread is running on, as
  this introduces an additional source of nondeterminism.
\end{itemize}

We also do not yet support \emph{bound threads}: a Haskell thread which will
always run on the same, unique, OS thread.  Bound threads are essential for
using the FFI to call libraries which use thread-local state, to ensure the
Haskell thread always sees its state and never the state of another thread.  We
have a prototype implementation, which is not yet present in a released version
of \dejafu{}\footnote{\url{https://github.com/barrucadu/dejafu/issues/126}}.

\section{Abstracting over I/O}
\label{sec:dejafu-monadconc}

\blindtext

\section{Executing Concurrent Programs}
\label{sec:dejafu-execution}

\blindtext

\section{Testing Concurrent Programs}
\label{sec:dejafu-testing}

\blindtext

\section{Soundness and Completeness}
\label{sec:dejafu-correctness}

\blindtext

\section{Case Studies}
\label{sec:dejafu-casestudies}

\blindtext

\subsection{The auto-update Package}
\subsection{Search Party}
\subsection{The Par Monad}

\section{\dejafu{} in the Wild}
\label{sec:dejafu-evaluation}

\blindtext

\subsection{Richness of the Abstraction}
\subsection{Porting Code}
\subsection{Integration with Existing Tools}

\section{Conclusions and Future Work}
\label{sec:dejafu-conclusions}

\blindtext
