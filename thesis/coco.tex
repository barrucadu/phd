Properties are given by the programmer in a conventional property-testing tool.
In this chapter we present and evaluate CoCo, our tool for \emph{automatically}
discovering properties of concurrent Haskell functions.  CoCo is built on top
of \dejafu{}.  We discuss the sorts of properties CoCo is concerned
with~\sref{coco-scope} and illustrate the use of the tool with a short
example~\sref{coco-example}.  We explain how properties are
discoverd~\sref{coco-hiw} and argue the correctness of our
approach~\sref{coco-correctness}.  We present two case
studies~\sref{coco-casestudies}, evaluate the usefulness of CoCo for discovering
properties of pre-existing code~\sref{coco-evaluation}, and finally draw
conclusions and present further work~\sref{coco-conclusions}.

This chapter is derived from our previous work \todo{cite when formally
  published}.

\section{What We Can Discover}
\label{sec:coco-scope}

\blindtext

\section{An Illustrative Example}
\label{sec:coco-example}

\blindtext

\section{Generating Expressions and Finding Properties}
\label{sec:coco-hiw}

\blindtext

\subsection{Representing and Generating Expressions}
\subsection{Evaluating Most General Terms}
\subsection{Property Discovery and Schema Pruning}

\section{Soundness and Completeness}
\label{sec:coco-correctness}

\blindtext

\section{Case Studies}
\label{sec:coco-casestudies}

\blindtext

\subsection{Mutable Stacks}
\subsection{Counting Semaphores}

\section{Using CoCo For Real}
\label{sec:coco-evaluation}

\blindtext

\section{Conclusions and Future Work}
\label{sec:coco-conclusions}

\blindtext
