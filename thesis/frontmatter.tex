\begin{titlepage}
  \begin{center}
    \makeatletter
    {\LARGE \@title}\\[1.5cm]
    {\large \@author}

    \vfill

    Submitted for the degree of\par
    Doctor of Philosophy\\[1.3cm]

    University of York\par
    Computer Science\\[1.3cm]

    \@date
    \makeatother
  \end{center}
\end{titlepage}

\chapter*{Abstract}
%\addcontentsline{toc}{chapter}{Abstract}
\blindtext

% octavo disables the "ugly" table-of-contents dots, but I find they
% improve readability.  The default dot separation is 4.5, but I like
% 7:
\makeatletter\renewcommand\@dotsep{7}\makeatother
\tableofcontents
%\listoffigures
%\listoftables

\chapter*{Acknowledgements}
%\addcontentsline{toc}{chapter}{Acknowledgements}
\blindtext

\chapter*{Declaration}
%\addcontentsline{toc}{chapter}{Declaration}

Earlier versions of parts of this thesis were published in the following papers:

\begin{enumerate}
\item Michael Walker and Colin Runciman. \dejafu{}: A Concurrency Testing
  Library for Haskell.  In \emph{Proceedings of the 8th ACM SIGPLAN Symposium on
    Haskell}, Haskell 2015, pages 141--152.  ACM, 2015.\nocite{walker2015}
\item Michael Walker. \dejafu{}: A Concurrency Testing Library for Haskell.
  Technical report, University of York, Department of Computer Science,
  2016.\nocite{YCS-2016-503}
\item \todo{CoCo paper, when formally published}
\end{enumerate}

These papers were conceived, implemented, and written by myself with significant
guidance and input from Prof.~Runciman\footnote{University of York}.  The first
and second papers contribute to \chpref{dejafu}, the third to \chpref{coco}.

Furthermore, the inspiration for the investigation in \chpref{algorithms} arose
from discussions with Dr.~Alistair Donaldson and Dr.~Paul
Thomson\footnote{Imperial College London}.

This work has not previously been presented for an award at this, or any other,
university.  All sources are cited in the main text and listed in the
bibliography.
