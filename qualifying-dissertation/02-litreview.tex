\chapter{Literature Review}
\label{sec:litrev}

% "give a thorough account of previous and current work in the field,
% with ample citations of relevant literature; assess the current
% state of the field, for example, discuss assumptions generally made
% and their validity, limitations generally accepted and their
% necessity, major open problems and prospects for their solution and
% the relative strengths and weaknesses of the major lines of work
% pursued to date."

This chapter motivates the field of runtime verification by first
discussing some of the techniques and difficulties of static analysis
and verification in \S\ref{sec:litrev-stat}. Runtime Verification is
then explored in \S\ref{sec:litrev-run}, focusing on the Design by
Contract and trace analysis approaches. Runtime Enforcement, an
extension of Runtime Verification which aims to guarantee correct
functioning by correcting deviations, is discussed in
\S\ref{sec:litrev-run-enforce}. Finally, the section is concluded by a
discussion of performing offline analysis of runtime-gathered data in
\S\ref{sec:litrev-run-offline}, and some of the major open problems in
the field in \S\ref{sec:litrev-run-probs}. Case studies of particular
approaches are given throughout.

The chapter closes with a discussion of the combination of both
runtime and static verification in \S\ref{sec:litrev-both}, with some
open problems in this area in \S\ref{sec:litrev-both-probs}.

\section{Static Analysis \& Verification}
\label{sec:litrev-stat}

\todo{Intro to static analysis/verification, why it's very good when
  feasible, what makes it difficult, undecidability issues.}

\subsection{Proofs of Software Correctness}
\label{sec:litrev-stat-proof}

\todo{Outline of the process of proving software correct, why this is
  good if possible, where this sort of thing is done.}

\subsection{Case Study: Type Systems}
\label{sec:litrev-stat-types}

\todo{Quick history of static type systems, compare primitive ones
  (eg, ALGOL) and modern dependent type systems (eg, Agda), taking
  care to highlight decidability issues that arise with increased
  complexity.}

\section{Runtime Verification}
\label{sec:litrev-run}

\todo{Intro to RV, why it's often acceptable, split focus on just
  monitoring and also enforcement.}

\subsection{Design by Contract}
\label{sec:litrev-run-dbc}

\todo{Intro to DbC, how we can check contracts at runtime, dsl/host
  language choice.}

\subsubsection{Case Study: Java Modelling Language}
\label{sec:litrev-run-dbc-jml}

\todo{Critical evaluation of JML, talk a little about BML as well.}

\subsubsection{Case Study: jContractor}
\label{sec:litrev-run-dbc-jcon}

\todo{Critical evaluation of jContractor.}

\subsection{Trace Analysis}
\label{sec:litrev-run-trace}

\todo{Intro to trace analysis, difference in sorts of properties we
  can talk about with contracts and with traces, how we can monitor
  traces at runtime.}

\subsubsection{Case Study: Regular Expressions}
\label{sec:litrev-run-trace-regex}

\todo{Critical evaluation of regex, in terms of what sorts of
  properties we can state and how used it is.}

\subsubsection{Case Study: Linear Temporal Logic}
\label{sec:litrev-run-trace-ltl}

\todo{Critical evaluation of ltl, in terms of what sorts of properties
  we can state (compare with regex), and how used it is.}

\subsection{Combined Contract/Trace Monitors}
\label{sec:litrev-run-trcon}

\todo{How we can combine the two.}

\subsection{Runtime Enforcement}
\label{sec:litrev-run-enforce}

\todo{Overview of enforcement, potential pros and cons.}

\subsection{Offline Analysis}
\label{sec:litrev-run-offline}

\todo{Overview of offline analysis, potential pros and cons.}

\subsection{Open Problems}
\label{sec:litrev-run-probs}

\todo{Big open problems in RV.}

\section{Combined Static/Runtime Verification}
\label{sec:litrev-both}

\todo{Why combining both might be good, outline of how we can do that.}

\subsection{Static-informed Runtime Verification}
\label{sec:litrev-both-sir}

\todo{Using static analysis to inform or optimise runtime
  verification.}

\subsection{Runtime-informed Static Verification}
\label{sec:litrev-both-ris}

\todo{Using runtime verification to provide evidence for or discharge
  static obligations.}

\subsection{Open Problems}
\label{sec:litrev-both-probs}

\todo{Big open problems in S+R V.}
