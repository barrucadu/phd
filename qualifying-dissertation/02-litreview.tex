\part{Literature Review}

% "give a thorough account of previous and current work in the field,
% with ample citations of relevant literature; assess the current
% state of the field, for example, discuss assumptions generally made
% and their validity, limitations generally accepted and their
% necessity, major open problems and prospects for their solution and
% the relative strengths and weaknesses of the major lines of work
% pursued to date."

\chapter{Nondeterministic Concurrency in Lazy Functional Languages}
\label{chp:litrev}

\textit{This chapter surveys some of the literature on
  nondeterministic concurrency in lazy functional languages, of which
  Haskell is an example.}

\section{Avoiding Concurrency}
\label{sec:litrev-strategies}

As unrestricted concurrency leads to nondeterminism, there has been
much work on \textit{avoiding} it whilst still reaping the benefits of
parallel execution of code. Some of this work is in the guise of a
concurrency abstraction with guaranteed determinism, but others have
avoided concurrency entirely.

Perhaps the earliest such approach, \textit{evaluation strategies},
introduced by Trinder et al. (1993)\nocite{trinder}, make use of the
basic primitives for controlling evaluation order in Haskell,
\verb|par| and \verb|seq|. The semantics of \verb|seq| are to evaluate
its first argument and then to return its second, whereas \verb|par|
starts evaluating its first argument in parallel and immediately
returns its second. In both cases, evaluation is to weak-head normal
form (WHNF).

Strategies were intended to be powerful and extensible methods for
controlling how data structures are evaluated. A value of type
\verb|Strategy a| defines how to evaluate a value of type
\verb|a|.

\begin{minted}{haskell}
type Strategy a = a -> ()
\end{minted}

Strategies are composable, allowing constructs like so to be easily
defined:

\begin{minted}{haskell}
parList :: Strategy a -> Strategy [a]
parList _ [] = ()
parList strat (x:xs) = strat x `par` parList strat xs
\end{minted}

Hence, \verb|parList| takes a strategy to evaluate a value, and
produces a strategy to evaluate a list of such values in
parallel.

Strategies have evolved since their inception, with Marlow et
al. (2010)\nocite{strategies} redefining strategies to operate in
terms of an \textit{evaluation order monad}, called \verb|Eval|.

\begin{minted}{haskell}
type Strategy a = a -> Eval a
runEval :: Eval a -> a
\end{minted}

The monadic bind of \verb|Eval| is defined to be strict, giving a
flexible notation for expressing evaluation order without the need to
pepper code with additional applications of \verb|rseq| (the
\verb|Eval| analogue of \verb|seq|).

The existence of strategies is made possible by features of Haskell
not commonly found in other languages: without laziness, the
abstraction would have to be built around evaluating
\textit{functions} rather then \textit{data}; and without purity,
parallel side-effectful evaluation can alter the result of a
program. Despite more powerful tools being available, strategies are
still used today, provided by the
\textit{parallel}\footnote{\url{https://hackage.haskell.org/package/parallel}}
library, suggesting that the conflation of parallel evaluation and
concurrent execution found in many languages is a hindrance imposed by
the semantics of the language.

\section{Deterministic Parallel Concurrency}
\label{sec:litrev-det}

Strategies do not provide anything looking like typical concurrency,
however. They are only used to express evaluation order, and have no
explicit notion of threads. The \verb|Par| monad, by Marlow et
al. (2011)\nocite{parmonad}, is used for expressing \textit{dataflow
  parallelism}.

The \verb|Par| monad provides threads and mutable
variables. Determinism is enforced by having these mutable variables
only allow a single write, and reads block until a value has been
written. This prevents race conditions and, as no locks are provided,
deadlock is impossible.

Parallelism is implemented by running one worker thread on each
physical processor or core available. These worker threads are unable
to move between processors, and so can execute truly in parallel. This
does not mean that the \textit{entire} \verb|Par| computation will run
in parallel (if there are more \verb|Par| threads created than there
are processors, for example) but as much will execute in parallel as
possible.

Single-write shared variables are rather limiting, however, and
Lindsey et al (2014)\nocite{lvish}, noted that all that is required
for determinism is that the value a read returns must be
constant. This lead to their \textit{LVish}
library\footnote{\url{https://hackage.haskell.org/package/lvish}}
which uses lattice-based shared variables. A read corresponds to
glimpsing some part of the shared mutable structure. Writes cannot
destructively update a shared structure.

LVish would allow structures such as a shared mutable list. Writes
could append to the list, and reads could return the first, second,
and $n$th items (blocking until such a value has been written), but
not to read the length, or the last item added.

A rather different approach to that of \verb|Par| and LVish is taken
by Leijen et al. (2011)\nocite{revisions} in their \textit{concurrent
  revisions} library. This is strictly speaking not an example of
deterministic parallelism as it allows arbitrary I/O to occur in a
concurrent environment, but it does deterministically deal with shared
state.

Here, shared variables have a copy-on-write semantics. This makes the
\verb|fork| of concurrent revisions much more like the \verb|fork| of
the C standard library, than the \verb|fork| of the \verb|Par|
monad. When a thread is \textit{joined}, it terminates and any
modification made to shared state is integrated with the shared state
of the joining thread, by applying functions supplied when the state
was created. For example,

\begin{minted}{haskell}
counter = revisioned $ do
  c <- vcreateM merge 0

  x <- fork $ do
    vmodify c (+1)
    y <- fork (vmodify c (+3))
    vmodify c (+2)
    return y

  vmodify c (+4)
  y <- join x
  join y
  vread c

  where
  merge main joinee original = main + joinee - original
\end{minted}

The final result here is 10, the sum of all the modification
operations. This is because the merge behaviour for \verb|c| is to add
any difference to its own value. Any merge behaviour at all is
possible, however as merge functions may not perform any I/O, and
threads cannot communicate other than joining each other (and a thread
cannot be joined multiple times), the whole process is deterministic.

These approaches to deterministic parallelism aim to approximate a
typical concurrency abstraction to some extent. This clearly indicates
that concurrency is a powerful and useful software structuring
technique, and mere parallel evaluation isn't enough. Sadly, any such
approximations must remain approximations, as unbridled concurrency
\textit{is} nondeterministic.

\section{(Nondeterministic) Concurrency}
\label{sec:litrev-conc}

\textit{Concurrent Haskell}, by Peyton Jones at
al. (1996)\nocite{concurrent}, embellishes Haskell with a means to
start new threads, and for threads to communicate. As noted in the
introduction, the aim of the work is to do with structuring programs,
not with performance,

\begin{quote}
  This paper is not at all about concurrency as a means of increasing
  performance by exploiting multiprocessors. Our approach to that goal
  uses \textit{implicit}, semantically transparent, parallelism; but
  that is another story. Rather, this paper concerns the use of
  \textit{explicit}, semantically visible, concurrent I/O-performing
  processes. Our goal is to extend Haskell's usefulness into a new
  class of applications.\cite{concurrent}
\end{quote}

A new thread is started with the \verb|forkIO| function, which takes
an I/O action and begins executing it concurrently with the forking
action. Execution may not be parallel, in fact the original
implementation of Concurrent Haskell performed all execution in a
single operating system thread, scheduling internally.

The second new concept was the \verb|MVar|. An \verb|MVar| is a
mutable variable, which can either be \textit{full} or
\textit{empty}. Attempting to write to a full \verb|MVar| blocks until
it is empty, and attempting to read from an empty \verb|MVar| blocks
until it is full (and then empties it). More recent versions of the
standard library have introduced functionality to attempt
(non-blockingly) to read from or write to an \verb|MVar|, and to read
without emptying.

Despite Haskell being a purely functional language, the introduction
of multiple distinct threads of control immediately introduces the
need for inter-thread synchronisation whilst evaluating
expressions\cite{concurrent}! This is because of laziness, two threads
may try to evaluate the same thunk at the ``same time'', and
evaluating a thunk mutates the heap. Thus, if one thread starts to
evaluate a thunk which is already being evaluated by another, the
former needs to be paused until the latter finishes.

Threading and \verb|MVar|s are sufficient for nondeterminism, as
\verb|MVar|s can be written to multiple times and can give rise to
deadlock. Nondeterministic I/O greatly damages the ability to test
code, in Haskell as in any other language, however here it cannot be
avoided. The order of interleaving of threads is not specified as a
part of the language, it is an implementation detail, and out of reach
of the programmer.

\section{Software Transactional Memory}
\label{sec:litrev-stm}

Concurrent algorithms based on locks and mutual exclusion do not
compose. Two individually correct fragments of code may become
incorrect when composed, without knowledge of their
implementations. Consider a data structure with thread-safe insert and
remove operations. Suppose we want to move an item from one structure
to another, but without the intervening state of it not being present
in either visible. This cannot be done without additional
functionality to lock the structures preventing access by another
thread! This breaks abstraction, and also opens the door to potential
deadlock.

Transactional memory, well known in databases, was originally proposed
by Herlihey and Moss (1993)\nocite{hardwaretm} as an architectural
technique supporting lock-free data structures. A \textit{transaction}
consists of a sequence of \textit{tentative} modifications to global
state, which can be applied atomically, or aborted. Transactions are
therefore linearisable: they appear to take effect in a sequential
ordering as if one thread were orchestrating the process.

Despite the age of transactional memory (STM), implementing it for
programming languages has proved challenging, with Herlihey et al
(2003)\nocite{dstm} providing the first implementation of STM where
all of the transactions do not have to be known statically, making it
suitable for dynamically-sized data structures.

A thorn in the side of many STM implementations is the question of
safety: an STM transaction must be able to be aborted or executed
multiple times with no observable side-effects until it is committed,
but most programming languages cannot guarantee this statically. The
Haskell STM implementation, by Harris et al (2006)\nocite{haskellstm}
achieves this by defining an STM monad, and so the type system
prevents IO actions from being performed in the same transaction as
STM actions.

\section{Systematic Concurrency Testing}
\label{sec:litrev-sct}

Systematic concurrency testing grew from efforts in
verification\cite{pbound}, where traditional model-checking techniques
are unsuitable for concurrent programs. Older techniques like
execution depth bounding are less suitable for concurrent programs, as
the depth is less predictable.

Musuvathi and Qadeer (2007) were the first to propose a heuristic
specifically for \textit{concurrent} programs: \textit{context-switch
  bounding}. Furthermore, they identified that merely bounding
\textit{pre-emptive} context switches is very powerful, as any state
(in a terminating program) can be driven to termination (or deadlock)
without incurring any pre-emptions, allowing the model checker to
explore interesting behaviours with only low bounds. Furthermore, for
a fixed number of pre-emptions, the number of possible schedules (in a
terminating program) is polynomial in the number of threads and the
execution length of the program.

The work on iterative pre-emption bounding was first implemented for
run-time testing purposes in the \textsc{Chess}\cite{heisenbugs} tool,
also by Musuvathi et al (2008).

Qadeer et al (2011) furthered this work with \textit{delay bounding},
where the number of deviations from an otherwise deterministic
scheduler (such as round-robin in order of thread creation) is
bounded. This has the advantage that the number of schedules grows
much more slowly than with pre-emption bounding as, for example, there
is only one schedule with zero delays, but potentially many with zero
pre-emptions, allowing for more rapid testing.

One concern with SCT is that as concurrent programs grow, the number
of possible interleavings of operations grows much faster, resulting
in an explosion of possible schedules. There is evidence that many
concurrency bugs can be found with small test cases with few threads
and context switches\cite{pbound, dbound, empirical}, providing hope
that unit-test-like test cases can be produced and known-good
components composed efficiently.

Another is that test cases suitable for SCT must be deterministic when
the schedule is fixed, which often isn't easy to achieve if external
processes or network communication is involved. Producing small
suitable tests may be difficult.

\section{SCT in Functional Languages}
\label{sec:litrev-sctfunc}

Unsurprisingly, much of the work regarding concurrency testing in a
functional setting has happened in the context of Erlang. Erlang is
somewhat different to most languages, as it doesn't have shared-memory
concurrency (beyond an atomic key/value store provided by the virtual
machine) and so all communication is in the form of
message-passing. Claessen et al (2009)\nocite{pulse} developed PULSE,
a user-level scheduler for Erlang, and an automatic instrumentation
tool to convert existing programs to use PULSE. Random scheduling was
then implemented using QuickCheck, a tool for randomised testing.

Random scheduling proved to be effective in the industrial case study
used in the paper, however it still leaves much to be desired. PULSE
was augmented with \textit{procrastination} by Arts et al (2011). The
procrastination technique looks for two possibly racey actions in a
schedule, for example two threads sending a message to a third, and
swapping the order of the actions. This proved to be even more
effective than random scheduling, and also led to easier production of
minimal failing examples.

The only Haskell application of SCT I am aware of is a blog
post\cite{typeclass} showing how some of the concurrency primitives
can be abstracted using a typeclass, and then a testing implementation
use QuickCheck in order to try to find bugs. However Stolz and Huch
(2005)\nocite{rvhaskell} do show an implementation of runtime
verification of temporal logic specifications of Haskell
programs. This, combined with an SCT scheduling technique, could
possibly be used to produce very expressive tests.

\section{Verified Concurrency}
\label{sec:litref-verify}

\todo{Logics and tools for automated verification or analysis of
  concurrency, highlight lack of work on lazy languages.}

\section{Review}
\label{sec:litrev-review}

\todo{Summarise chapter and highlight open problems.}
