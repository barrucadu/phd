\part{Research Proposal}

% "define carefully the aim and path of your proposed research; state
% specific objectives, give criteria by which you will judge success,
% set out plans of attack, identify the most important techniques and
% methods to be used.%

\chapter{Research Proposal}
\label{chp:proposal}

\textit{This chapter lays out goals for the next stage of
  research. The overall aims of the work are detailed, and a list of
  tasks for achieving them is formulated. To finish, in reference to
  this strategy, a plan for the next year is presented.}

\section{Overall Aims}
\label{sec:proposal-goals}

The goal of my research is to implement libraries and tools to enable
programmers to make use of potentially nondeterministic concurrency
whilst being confident of the correctness of their programs.

As many concurrent programs are deterministic, but impossible or
awkward to express with current libraries, common patterns should be
identified and generalised into guaranteed-deterministic
libraries. For the remaining programs which are nondeterministic,
powerful testing and verification tools should be developed.

\section{Tasks}
\label{sec:proposal-tasks}

The planned work can be broken down into a number of sequential
tasks. To summarise: firstly, the work on concurrency testing will be
completed by implementing DPOR techniques in \dejafu{}, and then the
verification of concurrency will be approached gradually, by first
selecting an appropriate tool, and then developing a core language.

\subsection*{Dynamic Partial-Order Reduction in \dejafu{}}
\label{sec:proposal-tasks-dpor}

\dejafu{} is a library implementing SCT for Haskell. However, without
a reduction of schedules to try, even simple programs experience a
combinatorial explosion of possible schedules as they grow. The
na\"{\i}ve approach currently used is simply to eliminate scheduling
decisions which could not possibly change the result, such as context
switches around reading the thread's own ID.

Dynamic partial-order reduction is a much more robust set of
techniques to eliminate redundant work by determining during the
testing process, from the program under test, whether alternative
decisions could affect the result. Schedules can be grouped into
equivalence classes, and only one schedule from each class must be
tested. The testing library, then, should implement some known DPOR
techniques to reduce work.

By the end of this task, I should have implemented existing DPOR
techniques (such as ignoring commuting operations) in \dejafu{},
making the search for errors more time- and space-efficient.

\subsection*{Review \& Choice of Verification Tools}
\label{sec:proposal-tasks-tools}

A number of logics, theorem provers, and proof assistants are
available. Some are already used for the automated verification of
concurrent algorithms. I feel I need to explore the options available
as this will inform the core language developed later.

A significant consideration will be support of first-class and
higher-order functions and ``higher-order'' concurrency operations
such as passing an \verb|MVar| through another \verb|MVar|. The former
makes it possible to model a functional language without an arduous
compilation process. The latter is absolutely essential in the
implementation of higher-level abstractions in a pure context, where
the same effect cannot be achieved more simply with true mutable
variables.

By the end of this task, I should have selected an environment for
verifying concurrent programs, and have some proofs of simple
programs.

\subsection*{A Core Language for Concurrency}
\label{sec:proposal-tasks-core}

Haskell is a big language, and so for research and tooling purposes a
core language is usually defined, into which the rest of Haskell can
be transformed mechanically. However, these core languages often
neglect parallelism and concurrency.

The primitives available in such a language must be suitable for
transforming more complex programs to, but must also admit a simple
translation into the logic selected previously. It is possible that
something based upon GHC Core might work, as that necessarily covers
everything allows by GHC Haskell. Further investigation is required.

By the end of this task, I should have developed a core language and
have a simple translator into the logic selected previously. This
necessitates also developing a formal semantics of the core language.

\subsection*{A Verifier for Core Concurrent Programs}
\label{sec:proposal-tasks-verifier}

Bugs found by a static analysis must be understood in terms of the
original program, not the logic of the tool. Hence, results relating
to programs written in the core language in the chosen logic will be
related back to the original programs. This may require the
core--logic translation process to decompose the original program into
smaller pieces with unique names, or by being whitespace-aware (if the
tool reports line numbers with errors).

Furthermore, a core language isn't very useful in itself. At least the
common concurrency functions must be available to programs written in
this language, even if they are not themselves primitive.

By the end of this task, I should have produced a tool capable of
taking programs written in the concurrent core language, applying
automated verification techniques, and reporting any errors found in
terms of the original program. Furthermore, a library of standard
concurrency functions taken from the Haskell standard library will be
reimplemented using the core language to ease program transformation.

\section{Current Progress}
\label{sec:proposal-progress}

A new concurrency abstraction has been developed, incorporating most
of the standard Haskell concurrency primitives, other than things
which cannot be used for testing, such as \verb|threadDelay|, and
primitives which remain out of reach, such as \verb|threadWaitRead|.

The \verb|threadDelay| function is omitted, as its behaviour is
notoriously nondeterministic, depending in part on the operating
system scheduler. The \verb|threadWaitRead| (and similar) functions
are omitted as the readability of a file descriptor cannot be checked
without blocking.

\dejafu{}, a library for SCT, has been implemented detailed in
\chap{dejafu}, although only using na\"{\i}ve elimination of duplicate
schedules. DPOR techniques need to be investigated and integrated into
the schedule generation process.

\section{Immediate Plan}
\label{sec:proposal-plan}

Over the next year, I intend to:

\begin{itemize}
  \item Implement DPOR in \dejafu{}, by investigating current
    approaches and implementing suitable methods in the tool. There
    has been recent work in integrating DPOR with schedule
    bounding\cite{bpor} which is a natural starting point for this
    investigation. It may also be the case that the pure functional
    setting gives rise to methods not so suitable to the imperative
    setting.

    I anticipate this to take no more than 3 months.

  \item Develop (or find) a core concurrent pure functional language
    and a compiler into the logic of some existing verification
    tool. The main considerations are, as mentioned, first-class and
    higher-order functions and ``higher-order''
    concurrency. Non-strictness can of course be modelled in a strict
    language with such functions by wrapping all values in a function
    of type \verb|() -> a|.

    I anticipate the investigation and selection of a logic and tool
    to take 2 months, and the development of a core language and
    compiler 1 month. The latter will hopefully not take a long time,
    as the core language will be strongly informed by the chosen
    logic.

  \item Develop a more functional verifier for concurrent
    programs. This consists of a number of subtasks:

    \begin{enumerate}
      \item Allow relating errors found by the tool back to the
        original program.
      \item Implement verification tactics for desirable correctness
        criteria (such as deadlock freedom) if not already available.
      \item Implement a library of concurrency functions in the core
        language.
    \end{enumerate}

    I anticipate the whole to take 6 months.
\end{itemize}

