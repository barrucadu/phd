Something I care about in my research is that any tools or theoretical
work I produce be applicable to real-world problems. If a research
artefact has beautiful properties in principle, but takes a day to run
on anything but tiny problems, that is not good enough. This is the
primary metric by which things will be evaluated.

Furthermore, these should be tools which a programmer could reasonably
use. No matter how good a tool is, if it gives the programmer
significantly more work to make their code compatible with it, it will
not be used.

To summarise, research artefacts should:

\begin{itemize}
\item \textbf{scale to real-world problems:} there exist standard
  concurrency benchmark programs which could be used for this.

\item \textbf{be convenient to use:} ultimately subjective, but
  metrics such as amount of code change needed before use is possible
  can provide insight.
\end{itemize}
