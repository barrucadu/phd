My research method is based strongly on analysis of prior work, with
insight and refinement driven by experimental implementation. So far,
most of the work has been understanding and applying existing results
to the purely functional setting. This has been challenging, as
algorithms for concurrency testing are typically formulated in terms
of imperative languages with mutable state. Now that this has been
realised in the \dejafu{} tool, further original work has been
possible.

As it is impossible to explore the entire state-space of a nontrivial
concurrent program, trade-offs have to be made and justified. Some of
these can retain completeness, whereas others sacrifice it.

If an analysis appears to preserve completeness and avoid redundant
work, it must first be carefully examined, to determine if this is in
fact the case. Once I am confident of its correctness (or lack
thereof), testing can be done with real programs to increase
confidence further.

If an analysis sacrifices completeness, then how exactly this is done
must be documented. It must be sound to combine it with existing
completeness-sacrificing analyses. This can also be determined with
confidence through careful thought and experimentation.
