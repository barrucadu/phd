\dejafu{} allows the testing of concurrent programs, but these test
cases still need to be written. It can be hard to think of good test
cases for concurrent systems, as bugs are typically caused by the
pre-emption of a thread where this is not expected. As programmers are
bad at recognising this situation, it is also difficult to
deliberately try to provoke it.

QuickSpec \citep{quickspec} is a tool for generating laws that a
collection of functions appear to hold based on random testing. It can
be used as a way to obtain suitable test cases, if the user selects
the generated laws that \emph{should} hold, rather than those which
merely \emph{accidentally} hold. Furthermore, the \emph{absence} of
expected laws can often hint at the presence of a bug which further
testing can reveal.

A QuickSpec-like tool which can generate laws for a concurrency-using
API would be interesting and useful. Ideally the programmer would
supply a description of some collection of functions, and the tool
would generate small concurrent programs and find properties of their
results.

For example, given a concurrent stack type, it would be desirable to
be able to generate both of these properties automatically:

\begin{haskellcode}
do { s <- newStack; push s 1; push s 2; a <- pop s; b <- pop s; return (a, b) }
  `gives` {(2,1)}

do { s <- newStack; push s 1 <|> push s 2; a <- pop s; b <- pop s; return (a, b) }
  `gives` {(2,1), (1,2)}
\end{haskellcode}

Where \verb|gives| produces the set of results of a concurrent
computation and \verb+<|>+ is parallel composition. Such properties
can serve as small simple test cases, and the basis of a larger test
suite.

There are a few problems with QuickSpec which make it unsuitable for
this task currently:

\begin{itemize}
\item QuickSpec cannot generate $\lambda$-functions directly, which
  makes it of limited use when generating properties involving monadic
  bind, \verb|(>>=)|.

\item QuickSpec generates ``small'' functions, and slows down
  significantly as the size bound is increased. However, interesting
  properties of concurrent data structures are likely to involve
  several function calls monadically composed.
\end{itemize}

One possible approach may be from the angle of generating do-notation
code one line at a time, rather than generating expressions in
generality. As we are interested in properties of monadic things only,
this alternative view may provide new insights, and allow restricting
the search space.

\paragraph{Timeline:}

\begin{description}
\item[End of May 2016] Have a prototype tool which can generate
  type-correct monadic expressions in do-notation style from a
  collection of provided functions.

\item[End of Jul 2016] Support concurrency and relate generated
  expressions to properties over sets of possible results.
\end{description}

\paragraph{Success Criteria:}

The tool is capable of generating expected properties of concurrent
implementations of stacks, queues, binary heaps, maps, and any other
data structures of interest. Properties exhibiting race conditions and
deadlocks should be generated where they exist.

\paragraph{Publications:}

I believe that this would be worthy of a paper. However, the proximity
of the submission deadlines for the Haskell Symposium and the
International Conference on Runtime Verification is problematic. It is
unlikely to be possible to produce a paper in time. There are possibly
other suitable venues, however.

Unfortunately the ICFP deadline is in the middle of March, which is
unlikely to be feasible.

\paragraph{Future Research:}

Most of the difficulty in implementing this tool will be the
combinatorial explosion of possible type-correct expressions. Trying
to cut down on this should be a rich area to explore. One possible
avenue is attempting to find heuristics to bias the search towards
more ``interesting'' properties, sacrificing completeness for speed.
