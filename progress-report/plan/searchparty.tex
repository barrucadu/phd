The Search Party library supports speculative parallelism in
generate-and-test search problems. It is motivated by the
consideration that if multiple acceptable solutions exist, it may not
matter which one is returned. It was originally developed as a case
study for \dejafu{}, but was also presented as a poster at the 2015
Haskell Symposium.

The library provides a collection of combinators used to express a
generate-and-test problem, which are executed in using a concurrent
producer/consumer pattern. Unfortunately, efficiently parallelising a
list comprehension is less than obvious to a new user of the library.

Whilst the efficient and automatic parallelisation of functional
programs based purely on static analysis is not feasible
\citep{autopar}, it may be possible to construct a ``good enough''
analysis for list comprehensions. This would be realised in the form
of Template Haskell ``search comprehensions'', which would look like
list comprehensions but expand to some appropriate Search Party
formulation of the computation. If reasonably-successful heuristics
could be developed to determine how to break up the problem, then this
could be a cheap way for a programmer to speed up their code.

\paragraph{Timeline:}

\begin{description}
\item[End of Sep 2016] Implement search comprehensions.

\item[End of Dec 2016] Develop heuristics for splitting up list
  comprehensions into Search Party-using parallel computations.
\end{description}

\paragraph{Success Criteria:}

The tool produces as-good-as-human results in some nontrivial
cases. As the programmer will need to manually introduce a search
comprehension, occasionally producing worse results is not such a
problem, as bad comprehensions can be disabled.

\paragraph{Publications:}

This would significantly expand the current Search Party paper, which
was rejected from the 2015 Haskell Symposium, and make it worth
resubmitting.
