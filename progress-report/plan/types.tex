Static type systems can be thought of as a kind of lightweight formal
method for proving properties of programs. The type system of GHC
Haskell is significantly more powerful than that of standard Haskell,
augmented with features such as type-level functions, generalised
algebraic data types, and higher-ranked polymorphism. More recently,
work has been ongoing to introduce full dependent typing. Given all of
this functionality, and the long-standing tradition in Haskell of
making illegal states unrepresentable in a type-correct way, it is
natural to wonder what can be done to enforce concurrency best
practices statically.

At one extreme, types can enforce that a concurrent computation is
totally deterministic, as in the Par monad \citep{parmonad}. However,
surely there are less strict guarantees that can be expressed and
statically checked.

Monads are a useful abstraction to structure computations with an
inherent notion of sequence and potential side effects in a pure
functional language. Indexed monads allow additional information to be
carried around at the type level. \citep{typefun} gives an example of
using an indexed monad to ensure that a computation releases every
lock it acquires. A similar idea can be used to, in addition, enforce
that locks are acquired in a fixed order. This approach is desirable
because the checking is done statically, by the type checker, rather
than at runtime, by code the programmer has written.

Another possible avenue is session types. Session types are used to
ensure that some communication protocol is adhered to, and so applies
to both networked systems, and programs with multiple threads sharing
state.

\paragraph{Timeline:}

\begin{description}
\item[End of Mar 2017] Discover some interesting new ways of
  expressing (and enforcing) properties of concurrent programs at the
  type level.
\item[Mid May 2017] Submit a paper to the Haskell Symposium.
\end{description}

\paragraph{Success Criteria:}

This is very speculative, so it's difficult to name some concrete
success criteria. Some sort of enforcement of properties beyond a
simple static prevention of deadlock is desirable.

\paragraph{Publications:}

This should lead to some new examples of using the type system to
restrict concurrency, and so give rise to a paper to submit to the
2017 Haskell Symposium.
