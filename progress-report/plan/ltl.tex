Runtime Verification (RV) is a broad field concerned with integrating
techniques from program verification with analysis of data gathered
from real-world executions of programs. The advantage of runtime
verification is that many static analyses are necessarily
\emph{undecidable} or \emph{inefficient} in general, however
restricting interest to a particular execution often allows further
optimisation and analysis to be done.

An example of this is DPOR. Partial-order reduction is a purely
static-analysis based method. DPOR is considered to be an
\emph{optimisation} as it can often use the data gathered at run-time
about dependencies between actions to do less overall work
\citep{dpor}.

\emph{Linear Temporal Logic} (LTL) is a popular logic in the field of
runtime verification for formulating and checking properties of
programs \citep{ltl}. There has been prior work on checking LTL
properties of concurrent Haskell programs \citep{hsrv}, but this work
appears not to have been developed further. In particular, the authors
do not integrate their work with any sort of systematic testing
framework. This allows violations of properties to be detected in
individual real-world executions, but does not allow for searching for
violations.

LTL is an extension of propositional logic with the ability to reason
about \emph{time}. For checking properties of programs, this means
that the formula-verifier has access to an execution trace. There are
four additional operators:

\begin{description}
\item[$p~\mathbf{U}~q$ (``$p$ until $q$'')] $p$ must hold (at least)
  until $q$ does. $q$ must become true eventually.

\item[$\mathbf{X}~p$ (``next $p$'')] $p$ holds in the next time step.

\item[$\mathbf{F}~p$ (``in the future $p$'')] $p$ holds at some future
  point.

\item[$\mathbf{G}~p$ (``globally $p$'')] $p$ always holds.
\end{description}

Writing a tool which uses \dejafu{} to check LTL properties of
concurrent programs would allow the systematic search for violations
of complex run-time invariants. An example given in \citep{hsrv} is
detecting the presence of lock reversal (acquiring a pair of locks in
one order and then using them in the other order), which \emph{can}
cause deadlock. It is nice to be able to detect that an invariant
violation occurs at run-time, it would be nicer to be able to detect
it during testing.

There are two ways to approach this. Either this property checking
could be performed as the computation is executed, aborting execution
when a violation occurs; or, the execution trace could be examined
after-the-fact for property violations. It is not immediately obvious
which is better.

\paragraph{Timeline:}

\begin{description}
\item[End of Apr 2016] Have a tool capable of checking LTL properties
  against all executions of a concurrent program under \dejafu{},
  showing failing schedules.
\end{description}

\paragraph{Success Criteria:}

The tool is able to, for each execution, prove or disprove a LTL
formula where possible; to return a new formula if complete evaluation
was not possible; and to report failing executions.

\paragraph{Publications:}

If successful, this should result in a paper being submitted to the
2016 International Conference on Runtime Verification.

\paragraph{Further Research:}

There are other temporal logics, like computational tree logic (CTL),
which models the future of a computation as a tree. This corresponds
very closely with how \dejafu{} actually explores computations, and so
it might be an interesting further development.
