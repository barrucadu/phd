Concurrency is notoriously difficult to get right \citep{overrated},
and can have dire consequences \citep{therac25}. The problem stems
from the nondeterminism of scheduling: the same program with the same
inputs may produce different results depending on the schedules chosen
at execution time. This makes it difficult to use traditional testing
techniques with concurrent programs, which rely on the result of
executing a test to be deterministic. So-called ``Heisenbugs'' make it
difficult to be confident of the correctness of concurrent programs:
no bug has been observed during the testing process, but how do we
\emph{know} that there aren't any?

Despite the difficulty, concurrency is important for producing many
real-world applications. For example, applications with a lot of input
and output can be more responsive by executing I/O asynchronously.
Concurrency is a useful program structuring technique, and it is here
to stay.

My research is concerned with the problems of nondeterministic
concurrency in pure functional programming languages. My goal is to
implement libraries and tools to enable programmers to make use of
concurrency whilst being confident of the correctness of their
programs, without needing to formally verify everything.

The questions I want to answer are:

\begin{itemize}
\item Can existing results from concurrency testing in other languages
  be reapplied to the pure functional setting?

\item Can existing techniques like automated test-case generation be
  applied to uses of concurrency?

\item What common usages of concurrency can be abstracted away in a
  safe and deterministic fashion?

\item Can types be used to statically enforce interesting concurrency
  safety properties?
\end{itemize}

The first question has been answered to some degree with \dejafu{}
\citep{dejafu}, a library and tool developed to test concurrent
Haskell programs. The further development of this since the submission
of the qualifying dissertation is discussed in \chap{progress}.
