In the qualifying dissertation, the plan was to spend the months
October 2015 to July 2016 producing a tool for verifying concurrent
programs written in a core functional language.

Formal verification was investigated, starting with the formalisation
of the core data structures and functions of \dejafu{}. This was
attempted in Isabelle. Unfortunately, due to the nature of the core
data structure, this seemed to have limited prospects of success,
without significantly restructuring the implementation. The testing
framework in \dejafu{} is built around an $n$-ary prefix
tree. Formulations of recursive data structures in Isabelle tend to
have a fixed number of branches as a part of their definition, whereas
the \dejafu{} tree structure has a variable branching factor, based on
which threads are runnable at each point. This led to a lot of
difficulty in expressing the structure in a way amenable to automated
reasoning, and was abandoned as being too complicated for the return.

Discussions following the presentation of \dejafu{} at the Haskell
Symposium suggested further work. This resulted in support for
modelling computations with relaxed memory effects, and new basic
operations (see \sect{progress-dejafu}).

Now the intent is to return the focus to the area of testing, rather
than verification, see \chap{plan}.
