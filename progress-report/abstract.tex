Writing correct concurrent programs is very difficult, even for
experienced programmers. Traditional testing methodologies fail to
produce reliable results due to the nondeterminism inherent in the
scheduling of multiple threads. Therefore, different testing and
verification techniques are needed to explicitly account for this new
source of nondeterminism.

This report discusses a tool for testing concurrent Haskell programs,
called \dejafu{}. Research made possible by the existence of this tool
is discussed, and a plan for the next several months
formulated. Further research directions that are likely to arise from
this work are speculated upon.

\vfill

\paragraph{Keywords}

\textit{functional programming, concurrency,
  testing, formal methods}