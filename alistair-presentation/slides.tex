\documentclass[12pt]{beamer}
\usetheme{default}
\usefonttheme{serif}

\usepackage{attrib}

\author{Michael Walker}
\title{Static Analysis is Hard}
\subtitle{Hopefulness? Hopelessness? The Rise of Darkness!}
\institute{Department of Computer Science\\
  University of York\\
  \texttt{msw504@york.ac.uk}
}

\begin{document}

\begin{frame}[plain]
  \titlepage
\end{frame}

\begin{frame}{Static Analysis}
  \textbf{Static Analysis:} The (fully or partially) automatic process
  of proving properties of programs prior to executing them.

  \visible<2->{
    \begin{itemize}
      \item Type checking
      \item Memory safety
      \item Deadlock/livelock freedom
      \item Schedulability
    \end{itemize}
  }
\end{frame}

\begin{frame}{Static Analysis: \small A Nemesis Arises!}
  \only<1>{
    \begin{quote}
      If $\mathcal P$ is any property possessed by some, but not all,
      recursively enumerable sets, then there exists no effective
      general method for deciding, given a set $\alpha$ by means of a
      partial recursive function enumerating it, whether or not
      $\alpha$ has the property $\mathcal P$.  [\ldots] Of course,
      there will exist special methods for particular functions.

      \attrib{Rice's Theorem, Corollary B, 1953}
    \end{quote}

    There exists no algorithm which can, for all input programs,
    either prove or disprove some nontrivial property $\mathcal P$.
  }

  \visible<2->{
    Suppose we had a function \texttt{identity?} :: $(a \rightarrow a)
    \rightarrow Bool$, which returns ``yes'' or ``no'' depending on
    whether its argument is the identity function.

    \vspace{0.25cm}
  }

  \visible<3->{
    Or even \texttt{identity?} :: $(Bool \rightarrow Bool) \rightarrow
    Bool$

    \vspace{0.25cm}
  }

  \visible<4->{
    We can use this to solve the halting problem!
  }

  \visible<5->{
    \begin{center}
      \texttt{halts?(P, I) = identity?($\lambda$x $\mapsto$ P(I); return x)}
    \end{center}

    Undecidable in general!
  }
\end{frame}

\begin{frame}{Static Analysis: \small Defeating Rice, the Evil Sorcerer}
  Rice's theorem:
  \begin{itemize}
    \item applies to Turing machines,
    \item refers to exact results,
    \item refers to a fully automatic analysis.
  \end{itemize}

  \visible<2->{
    We can:
    \begin{itemize}
      \item weaken our systems,
      \item use heuristics,
      \item introduce human involvement.
    \end{itemize}
  }
\end{frame}

\begin{frame}
  Rice, H. G. 1953 ``Classes of Recursively Enumerable Sets and Their
  Decision Problems.''  \textit{Transactions of the American
    Mathematical Society} 74 (2)

  \vspace{0.25cm}

  \url{http://www.jstor.org/stable/1990888}
\end{frame}

\end{document}