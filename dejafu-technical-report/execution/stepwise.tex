Each thread is represented as a sequence of primitive actions, where
the continuation of one action is the next action that a thread will
take.

Execution of an entire computation proceeds in a stepwise manner: a
thread is chosen by the scheduler, its primitive action is executed,
and a new action is returned to be executed by that thread in the next
step. In order to model all the possible effects, this step function
returns a completely new set of threads and identifier source.

Normally only the chosen thread will be modified, but this allows, for
example, \verb|AFork| to create new threads, and \verb|AThrowTo| to
terminate other threads. The identifier source is used to produce
fresh thread, \verb|CRef|, \verb|CVar|, and \verb|CTVar| identifiers.

After the scheduler is consulted to choose a thread, a function is
called to evaluate the next step of that thread. The simplest thing
that a thread can do is to stop, which will serve as a useful
example:

\begin{haskellcode}
stepStop = simple (kill tid threads) Stop
\end{haskellcode}

The \verb|simple| function is defined as follows:

\begin{haskellcode}
simple threads' act = return (Right (threads', idSource, act, wb))
\end{haskellcode}

The effect of \verb|stepStop| can be read as: remove the current
thread from the map of live threads (the \verb|kill tid threads| bit);
and then return the new thread map and the name of the action to
appear in the trace (\verb|Stop|, here). \verb|simple| is a helper
function for actions which don't modify the identifier source or have
any relaxed-memory effects.

The \verb|Right|\footnote{The type \texttt{Either a b} type is
  commonly used to represent computations that might fail with an
  error value. By convention \texttt{Left err} means that the
  computation failed with reason \texttt{err}, and \texttt{Right x}
  means that the computation succeeded, producing \texttt{x}.}
indicates that the action completed successfully. There are a few
different possible failures, such as an uncaught exception, which will
terminate the current thread. If the main thread is terminated like
so, the entire computation terminates with that failure.

Another simple action that a thread can perform is \verb|AReturn|:

\begin{haskellcode}
stepReturn c = simple (goto c tid threads) Return
\end{haskellcode}

The effect of \verb|stepReturn| can be read as: extract the
continuation of the action and replace the continuation of the current
thread with it.

\subsection{Threading}
\label{sec:execution-stepwise-threading}

Threads are represented as a record type:

\begin{haskellcode}
data Thread n r s = Thread
  { continuation :: Action n r s
  , blocking     :: Maybe BlockedOn
  , handlers     :: [Handler n r s]
  , masking      :: MaskingState
  , known        :: [Either CVarId CTVarId]
  , fullknown    :: Bool
  }
\end{haskellcode}

The \verb|continuation| field contains the action to execute in the
next step. \verb|blocking| records whether the thread is blocked and,
if so, what it is waiting for. \verb|handlers| is the stack of
exception handlers. \verb|masking| is the masking state. \verb|known|
is the collection of \verb|CVar|s and \verb|CTVar|s the thread is
known to have access to. \verb|allknown| indicates whether
\verb|_concAllKnown| has been called in this thread. If
\verb|allknown| is \verb|True| for all threads, then detection of
deadlocks only involving a subset of the threads is possible.

Furthermore, a simple map is used to keep track of all the threads
currently extant:

\begin{haskellcode}
type Threads n r s = Map ThreadId (Thread n r s)
\end{haskellcode}

There are a number of functions to ease manipulating this structure,
\verb|kill| and \verb|goto| are two. Another is \verb|launch|, used to
create a new thread:

\begin{haskellcode}
stepFork a b = return (Right (threads', idSource', Fork newtid, wb)) where
  threads' = goto (b newtid) tid (launch tid newtid a threads)
  (idSource', newtid) = nextTId idSource
\end{haskellcode}

This is somewhat more complex than the two examples seen before, as it
involves two modifications to the thread map: firstly, a new thread is
created (and inherits the masking state of its parent), secondly the
continuation of the current thread is updated. Here \verb|simple|
cannot be used, as the identifier source is being modified.

Both \verb|AMyTId| and \verb|AYield| follow the \verb|simple| pattern:

\begin{haskellcode}
stepMyTId c = simple (goto (c tid) tid threads) MyThreadId
stepYield c = simple (goto c tid threads) Yield
\end{haskellcode}

Note that \verb|AYield| does not have any special implementation
here. Its effect is purely a scheduling concern; from the point of
view of updating the state of the system, it is no different to
\verb|AReturn|.

\subsection{\texttt{CRef}s and Relaxed Memory}
\label{sec:execution-stepwise-cref}

\begin{haskellcode}
newtype CRef r a = CRef (CRefId, r (Map ThreadId a, Integer, a))
\end{haskellcode}

A \verb|CRef| is implemented as a mutable reference containing a
\emph{globally visible} value, a counter of how many write commits
there have been, and a number of \emph{thread-specific} values. These
thread-specific values correspond to uncommitted writes, and so only
show up when using relaxed memory.\footnote{The
  \texttt{name@(pattern)} syntax is called an \emph{as-pattern}. The
  name before the \texttt{@} can refers to the entire value.}

\begin{haskellcode}
newtype Ticket a = Ticket (CRefId, Integer, a)
\end{haskellcode}

A \verb|Ticket| just keeps track of the \verb|CRef| it was produced
for, what the write count was when it was produced, and the
thread-specific value seen by its creating thread. Checking if a
\verb|CRef| has been modified since the creation of a \verb|Ticket|
becomes very simple with this implementation: the write counts are
compared.

\begin{haskellcode}
stepWriteRef cref@(CRef (crid, _)) a c = case memtype of
\end{haskellcode}

There are three memory models supported by \dejafu{}, each of which
has a different implementation for writing to a \verb|CRef|. Firstly,
sequential consistency. This does not have any relaxed memory
effects:\footnote{This is an emple of \emph{do notation}, which is a
  convenient synctatic sugar for composition of monadic functions.}

\begin{haskellcode}
  SequentialConsistency -> do
    writeImmediate cref a
    simple (goto c tid threads) (WriteRef crid)
\end{haskellcode}

The \verb|writeImmediate| function writes to the globally visible
value, and clears the thread-specific values.

Total store order (TSO) corresponds to an architecture where each
thread has its own cache: writes made by a thread will be cached, but
they will be committed in that same order to main memory:

\begin{haskellcode}
  TotalStoreOrder -> do
    wb' <- bufferWrite wb tid cref a tid
    return (Right (goto c tid threads, idSource, WriteRef crid, wb'))
\end{haskellcode}

The \verb|bufferWrite| function appends a write to the relevant write
buffer, in this case the one corresponding to that thread. Total store
order corresponds to modern x86 and x86\_64 processors.

Partial store order (PSO) is a more relaxed version of total store order,
where the writes a thread makes may not necessarily be committed in
order. It can be modelled by giving each \verb|CRef| a write buffer,
rather than each thread:

\begin{haskellcode}
  PartialStoreOrder -> do
    wb' <- bufferWrite wb crid cref a tid
    return (Right (goto c tid threads, idSource, WriteRef crid, wb'))
\end{haskellcode}

Both the TSO and PSO cases update the thread-specific map. A thread
will always see the writes it has made, but other threads may not.

The compare-and-swap write is a little different, as this has the
effect of being a memory barrier: any uncommitted writes to any
\verb|CRef| are committed before the CAS is done, and the result is
immediately globally visible. There is a \verb|synchronised| function
for actions which have this barrier property:\footnote{The \texttt{\$}
  operator is function application, but with a very low
  precedence. This makes it convenient for avoiding parentheses, which
  can be more readable when multi-line expressions are involved.}

\begin{haskellcode}
stepCasRef cref@(CRef (crid, _)) tick a c = synchronised $ do
  (suc, tick') <- casCRef cref tid tick a
  simple (goto (c (suc, tick')) tid threads) (CasRef crid suc)
\end{haskellcode}
%$

The \verb|casCRef| function here generates a new \verb|Ticket|,
compares the write counts, and then swaps the value. It is provided,
rather than the logic be included verbatim, as it is used again in the
implementation of \verb|stepModRefCas|.

The implementation of \verb|synchronised| is as follows:

\begin{haskellcode}
synchronised ma = do
  writeBarrier wb
  res <- ma

  case res of
    Right (threads', idSource', act', _) -> return
      (Right (threads', idSource', act', emptyBuffer))
    _ -> return res
\end{haskellcode}

Here \verb|writeBarrier| commits all cached writes. The action is then
executed, and an empty write buffer returned. This is why
\verb|simple| can be used in the implementation of \verb|stepModRef|
despite the write buffer being changed.

Cached writes can be committed to the globally visible value (at which
point the thread-specific values disappear) by executing an
\verb|ACommit| action:

\begin{haskellcode}
stepCommit t c = do
  wb' <- case memtype of
    TotalStoreOrder   -> commitWrite wb t
    PartialStoreOrder -> commitWrite wb c

  return (Right (threads, idSource, CommitRef t c, wb'))
\end{haskellcode}

The \verb|commitWrite| function is used here. Note how the invocation
differs between the cases: under TSO, the cache corresponding to the
thread is used; whereas under PSO, the cache corresponding to the
\verb|CRef| is used. There is no case for sequential consistency here,
as commit actions are not explicitly introduced by the program under
test; they are introduced by the test runner when executing under a
relaxed memory model. This is expanded upon in
\sect{execution}{scheduling}.

\begin{haskellcode}
stepModRef cref@(CRef (crid, _)) f c = synchronised $ do
  val <- readCRef cref tid
  let (new, ret) = f val
  writeImmediate cref new
  simple (goto (c ret) tid threads) (ModRef crid)

stepModRefCas cref@(CRef (crid, _)) f c = synchronised $ do
  tick@(Ticket (_, _, val)) <- readForTicket cref tid
  let (new, ret) = f val
  casCRef cref tid tick new
  simple (goto (c ret) tid threads) (ModRefCas crid)
\end{haskellcode}

The modification is made globally visible by the use of
\verb|writeImmediate|. Here \verb|casCRef| is used in the
implementation \verb|stepModRefCas|, because it is strict in the value
written, which \verb|writeImmediate| is not.

Reading a reference is quite simple:

\begin{haskellcode}
stepReadRef cref@(CRef (crid, _)) c = do
  val <- readCRef cref tid
  simple (goto (c val) tid threads) (ReadRef crid)

stepReadRefCas cref@(CRef (crid, _)) c = do
  tick <- readForTicket cref tid
  simple (goto (c tick) tid threads) (ReadRefCas crid)
\end{haskellcode}

The \verb|readCRef| function checks if there is a cached value for
that thread and, if so, returns it. Otherwise it returns the globally
visible value. The \verb|readForTicket| function behaves similarly,
but returns a \verb|Ticket| rather than the current value.

Creating a new \verb|CRef| looks a little more involved, but it is
really quite simple. Firstly, a new mutable reference containing the
given value and no thread-specific values is created; then this is
packaged up into a \verb|CRef| by giving it a unique identifier;
finally the thread is given it:
\pagebreak
\begin{haskellcode}
stepNewRef a c = do
  ref <- newRef (empty, 0, a)

  let (idSource', newcrid) = nextCRId idSource
  let threads' = goto (c (CRef (newcrid, ref))) tid threads

  return (Right (threads', idSource', NewRef newcrid, wb))
\end{haskellcode}

\subsection{\texttt{CVar}s}
\label{sec:execution-stepwise-cvar}

As there are no relaxed memory issues to worry about, the \verb|CVar|
implementation is in many respects simpler than that for
\verb|CRef|. However, \verb|CVar|s have their own unique features:
specifically, blocking. Attempting to read or take from an empty
\verb|CVar| blocks the thread, and attempting to put into a full
\verb|CVar| does the same.

Firstly, creating a new \verb|CVar| is almost identical to creating a
new \verb|CRef|:

\begin{haskellcode}
stepNewVar c = do
  ref <- newRef Nothing

  let (idSource', newcvid) = nextCVId idSource
  let threads' = knows [Left newcvid] tid
                 (goto (c (CVar (newcvid, ref))) tid threads)

  return (Right (threads', idSource', New newcvid, wb))
\end{haskellcode}

There is a difference, however. The \verb|knows| function is used to
record that a thread has a reference to a \verb|CVar| or \verb|CTVar|,
which can be used to improve deadlock detection.

Fortunately there is a lot of similarity between the \verb|CVar|
functions, which makes them easy to follow.

\begin{haskellcode}
stepPutVar cvar@(CVar (cvid, _)) a c = synchronised $ do
  (success, threads', woken) <- putIntoCVar cvar a c tid threads
  simple threads' $ if success then Put cvid woken else BlockedPut cvid

stepTryPutVar cvar@(CVar (cvid, _)) a c = synchronised $ do
  (success, threads', woken) <- tryPutIntoCVar cvar a c tid threads
  simple threads' (TryPut cvid success woken)
\end{haskellcode}
%$

Note that these functions are all \verb|synchronised|, and so commit
\verb|CRef| writes. All \verb|CVar| actions (other than
\verb|ANewVar|) are.

\pagebreak
\begin{haskellcode}
stepReadVar cvar@(CVar (cvid, _)) c = synchronised $ do
  (success, threads', _) <- readFromCVar cvar c tid threads
  simple threads' $ if success then Read cvid else BlockedRead cvid

stepTakeVar cvar@(CVar (cvid, _)) c = synchronised $ do
  (success, threads', woken) <- takeFromCVar cvar c tid threads
  simple threads' $ if success then Take cvid woken else BlockedTake cvid

stepTryTakeVar cvar@(CVar (cvid, _)) c = synchronised $ do
  (success, threads', woken) <- tryTakeFromCVar cvar c tid threads
  simple threads' (TryTake cvid success woken)
\end{haskellcode}
%$

The \verb|putInto|/\verb|readFrom|/\verb|takeFromCVar| functions, and
their \verb|try| variants, handle waking threads which are blocked in
the appropriate way on that \verb|CVar|. All such threads get woken at
once, and a list of them is returned to be included in the execution
trace. This is somewhat different to how GHC does things, where
threads blocked on an \verb|MVar| are woken up in a FIFO order to
guarantee fairness. The current behaviour was chosen because there is
no standard for Haskell concurrency, and so that ordering is only an
implementation detail which could, conceivably, be changed in the
future if another were judged more desirable.

\subsection{Exceptions}
\label{sec:execution-stepwise-exception}

A thread has both a stack of exception handlers, and a masking
state. The handler stack affects all exceptions raised in the thread,
whereas the masking state only affects exceptions raised by
\verb|AThrowTo|.

\begin{haskellcode}
stepCatching h ma c = simple threads' Catching where
  a     = runCont ma      (APopCatching . c)
  e exc = runCont (h exc) (APopCatching . c)

  threads' = goto a tid (catching e tid threads)
\end{haskellcode}

This introduces the \verb|APopCatching| action, at the end of both the
enclosed action, and at the end of the handler. This is necessary
because actions are executed one at a time, and so we cannot just run
the entire inner computation in one go and then check the result for
an uncaught exception.

\begin{haskellcode}
stepPopCatching a = simple threads' PopCatching where
  threads' = goto a tid (uncatching tid threads)
\end{haskellcode}

The \verb|catching| and \verb|uncatching| functions are used to modify
the handler stack, corresponding to push and pop operations.

When an exception is thrown, it may not be able to be handled by the
topmost handler, as there are exceptions of many types:

\begin{haskellcode}
stepThrow e = case propagate e tid threads of
    Just threads' -> simple threads' Throw
    Nothing -> return (Left UncaughtException)
\end{haskellcode}

The \verb|propagate| function pops from the stack of exception
handlers until one is found capable of handling that type of
exception. It then jumps to the handler, and returns the new thread
map. If no handler was found, the thread is killed by the uncaught
exception.

Throwing an exception to another thread is significantly more
complicated, and is also a \verb|synchronised| operation:

\begin{haskellcode}
stepThrowTo t e c = synchronised $
  let threads' = goto c tid threads
      blocked  = block (OnMask t) tid threads
  in if interruptible (lookup t threads)
     then case propagate e t threads' of
            Just threads'' -> simple threads'' (ThrowTo t)
            Nothing
              | t == 0    -> return (Left UncaughtException)
              | otherwise -> simple (kill t threads') (ThrowTo t)
     else simple blocked (BlockedThrowTo t)
\end{haskellcode}
%$

Firstly, whether the thread is interruptible is checked. If it's not,
the current thread is blocked. If it is interruptible, then the
exception is propagated through its handler stack. If a handler is
found, the thread jumps to it, throwing away whatever it was going to
do next. If a handler is not found, the thread is killed. If the
thread is killed and is the main thread, the entire computation
terminates.

\begin{haskellcode}
stepMasking m ma c = simple threads' (SetMasking False m) where
  a = runCont (ma umask) (AResetMask False False m' . c)

  m' = masking (lookup tid threads)
  umask mb = do
    resetMask True m'
    b <- mb
    resetMask False m
    return b
  resetMask typ ms = cont (\k -> AResetMask typ True ms (k ()))

  threads' = goto a tid (mask m tid threads)
\end{haskellcode}

Similarly to \verb|ACatching|, \verb|AMasking| introduces an
additional action into its continuation: \verb|AResetMask|, which
returns the masking state to what it originally was. It also
constructs a function to execute an action with the original masking
state, the \verb|umask| function.

\begin{haskellcode}
stepResetMask b1 b2 m c = simple threads' action where
  action   = (if b1 then SetMasking else ResetMasking) b2 m
  threads' = goto c tid (mask m tid threads)
\end{haskellcode}

\subsection{Software Transactional Memory}
\label{sec:execution-stepwise-stm}

As STM transactions are atomic, the implementation is vastly
simplified. They are still implemented in terms of a step function,
but it is just iterated until termination.

Firstly, the transaction is executed:

\begin{haskellcode}
stepAtom stm c = synchronised $ do
  (res, newctvid) <- runstm stm (nextCTVId idSource)
  let idSource'   = idSource { nextCTVId = newctvid }
  case res of
\end{haskellcode}
%$

There are now three possible results: the transaction succeeded; the
transaction aborted due to calling \verb|retry|; or the transaction
aborted due to an uncaught exception.

If the transaction succeeds, all threads blocked on \verb|CTVar|s
which were modified are woken:

\begin{haskellcode}
    Success readen written val
      let (threads', woken) = wake (OnCTVar written) threads
      in return (Right (goto (c val) tid threads', idSource', STM woken, wb))
\end{haskellcode}

If the transaction aborts due to \verb|retry|, the thread is blocked
until any of the read \verb|CTVar|s are modified:

\begin{haskellcode}
    Retry touched ->
      let threads' = block (OnCTVar touched) tid threads
      in return (Right (threads', idSource, BlockedSTM, wb))
\end{haskellcode}

If the transaction aborts due to an uncaught exception, the exception
is thrown in the thread:

\begin{haskellcode}
    Exception e -> stepThrow e
\end{haskellcode}

There are 9 primitive actions used to implement STM transactions,
discussed in \sect{execution}{primops}. The implementation of two of
them, \verb|SNew| and \verb|SLift|, are virtually identical to similar
primitives discussed elsewhere, and so will not be discussed here. The
remaining primitives are \verb|SRead|, \verb|SWrite|, \verb|SCatch|,
\verb|SOrElse|, \verb|SRetry|, \verb|SThrow|, and \verb|SStop|.

\begin{haskellcode}
stepRead (CTVar (ctvid, ref)) c = do
  val <- readRef ref
  return (c val, nothing, [ctvid], [])
\end{haskellcode}

The most obvious difference is that there is only one thread of
control. Moreover, a new transaction is built up to \emph{undo} what
has already been done (\verb|nothing| in this case), so that a
transaction can be reverted. Furthermore, lists of the \verb|CTVar|s
read from and written to are constructed.

There are no relaxed memory effects in STM transactions, so reading
and writing is incredibly simple.

\begin{haskellcode}
stepWrite (CTVar (ctvid, ref)) a c = do
  old <- readRef ref
  writeRef ref a
  return (c, writeRef ref old, [], [ctvid])
\end{haskellcode}

Here we see the inverse transaction being built up: to undo a write,
write the old value.

The handling of exceptions is vastly simplified, as \verb|SCatch| can
just execute the entire inner transaction and examine the result:

\begin{haskellcode}
stepCatch h stm c = onFailure stm c
  (\readen -> return (SRetry, nothing, readen, []))
  (\exc    -> case fromException exc of
    Just exc' -> transaction (h exc') c
    Nothing   -> return (SThrow exc, nothing, [], []))
\end{haskellcode}

Here \verb|onFailure| and \verb|transaction| are functions to do
different things for the three different possible results of a
transaction:

\begin{haskellcode}
transaction stm onSuccess = onFailure stm onSuccess
  (\readen -> return (SRetry, nothing, readen, []))
  (\exc    -> return (SThrow exc, nothing, [], []))

onFailure stm onSuccess onRetry onException = do
  (res, undo) <- doTransaction stm
  case res of
    Success readen written val -> return (onSuccess val, undo, readen, written)

    Retry readen  -> onRetry readen
    Exception exc -> onException exc
\end{haskellcode}

The effect of \verb|stepCatch|, then, is to run the entire inner
transaction. If it throws an exception, and the exception is of the
right type for the handler, the handler is executed. If the handler
throws an exception, it is not dealt with. If the exception is not of
the right type for the handler, it propagates upwards.

The implementation of \verb|SOrElse| is actually quite similar to
\verb|SCatch|: it runs the entire first transaction and, if it aborts
due to a \verb|retry|, runs the second:

\begin{haskellcode}
stepOrElse a b c = onFailure a c
  (\_   -> transaction b c)
  (\exc -> return (SThrow exc, nothing, [], []))
\end{haskellcode}

The terminating cases are implemented very simply:

\begin{haskellcode}
stepRetry   = return (SRetry,   nothing, [], [])
stepThrow e = return (SThrow e, nothing, [], [])
stepStop    = return (SStop,    nothing, [], [])
\end{haskellcode}

Termination is achieved by checking if the next action a thread will
perform is one of these three.

\subsection{Testing Annotations}
\label{sec:execution-stepwise-annotations}

Normally \dejafu{} can only detect a deadlock when \emph{every} thread
is blocked. However, it may be the case that a smaller collection of
threads are deadlocked, if they are all blocked on a \verb|CVar| which
no thread outside the collection has a reference to, for example. GHC
can do this sort of deadlock detection using its garbage collector,
and can signal to threads when they are blocked.

\dejafu{} does not have access to the garbage collector, and so relies
on programmer-provided hints about which \verb|CVar|s and
\verb|CTVar|s are known about by which threads.

\begin{haskellcode}
stepKnowsAbout v c = simple (knows   [v] tid (goto c tid threads)) KnowsAbout
stepForgets    v c = simple (forgets [v] tid (goto c tid threads)) Forgets
stepAllKnown     c = simple (fullknown   tid (goto c tid threads)) AllKnown
\end{haskellcode}

The \verb|knows| and \verb|forgets| functions are used to modify the
set of variables that a thread is known to have a reference to. The
\verb|fullknown| function indicates that this set is complete.

If every thread is in a fully-known state, then the deadlock detection
algorithm is enhanced to: for a given thread blocked on a \verb|CVar|
or \verb|CTVar|, if no other thread \emph{which is not also blocked on
  the same thing} has a reference to that variable, then the thread is
deadlocked.

\subsection{Lifting from the Underlying Monad}
\label{sec:execution-stepwise-lift}

Because all of the step functions are defined in terms of the
underlying monad, lifting an action is incredibly simple:

\begin{haskellcode}
stepLift na = do
  a <- na
  simple (goto a tid threads) Lift
\end{haskellcode}
