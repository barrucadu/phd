When there are multiple, non-blocked, threads available, the choice of
which one to execute next is made by the scheduler.

A scheduler is represented as a pure function, and is supplied as a
parameter when testing. This allows for deterministic results and,
just as importantly, allows for computing a list of scheduling
decisions in advance, designed to try to provoke the system into a new
state. This is the basis for the systematic concurrency testing
implementation.

\begin{haskellcode}
type Scheduler s = s
  -> Maybe (ThreadId, ThreadAction)
  -> NonEmpty (ThreadId, NonEmpty Lookahead)
  -> (ThreadId, s)
\end{haskellcode}

In order to make nontrivial decisions, a scheduler maintains some
state, of type \verb|s|. This could be, for example, a random number
generator:

\begin{haskellcode}
randomSched :: RandomGen g => Scheduler g
randomSched g _ threads = (threads' !! choice, g') where
  (choice, g') = randomR (0, length threads' - 1) g
  threads'     = map fst (toList threads)
\end{haskellcode}

The initial state is supplied when the execution begins, and the final
state is returned when it terminates. Use of this state is, of course,
not mandatory, as a simple round-robin scheduler illustrates:

\begin{haskellcode}
roundRobinSched :: Scheduler ()
roundRobinSched _ Nothing _ = (0, ())
roundRobinSched _ (Just (prior, _)) threads
  | prior >= maximum threads' = (minimum threads', ())
  | otherwise = (minimum (filter (>prior) threads'), ())

  where threads' = map fst (toList threads)
\end{haskellcode}

A scheduler is also given information about the state of the system:
what the last thread it scheduled did (this is \verb|Nothing| if this
is the first step of the computation), and what every runnable thread
in the system will do in the next few steps. Here \verb|NonEmpty| is
the type of non-empty lists,\footnote{And \texttt{toList} converts a
  \texttt{NonEmpty a} to a \texttt{[a]}.} to give a type-level
guarantee that there \emph{are} threads to run: if there are no
runnable threads, the execution terminates, signalling a deadlock
condition.

The \verb|ThreadAction| type is a record of what has been done, and
the \verb|Lookahead| type is a slightly simpler view of what will
happen. The two types cannot be the same, because in general the
effect of performing a primitive action at some point in the future
cannot be determined, due to interactions between threads.

\subsection{Phantom Threads}
\label{sec:execution-scheduling-phantom}

In a sequentially consistent memory model, the set of runnable threads
is exactly the set of threads created by \verb|AFork| which are not
blocked.

Under relaxed memory, however, this is not the case. In order to model
the nondeterministic committing of \verb|CRef| writes, for every
buffer with an uncommitted write (threads, under TSO; \verb|CRef|s,
under PSO), a \emph{phantom thread} is created, and added to the
runnable set. A phantom thread is a thread with only one action:
\verb|ACommit|. These threads do not exist in the same way that other
threads do, they are never added to the thread map, they only exist in
order for the scheduler to determine when commits happen.

This may seem like an odd approach: why create new not-quite-threads
in order to model relaxed memory? The advantage is that systematic
concurrency testing techniques assume there is only one source of
nondeterminism: the scheduler. If a second source is added, such as
when writes are committed, it is difficult to integrate this with
existing algorithms. By using phantom threads, the two sources of
nondeterminism are unified, and existing algorithms just work. This
approach was suggested by \citep{rdpor}.

This approach also ensures that \verb|ACommit| actions are never
introduced under a sequentially-consistent memory model.
