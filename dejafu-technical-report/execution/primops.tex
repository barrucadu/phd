There are currently 31 primitive actions used to construct the testing
instances of \verb|MonadConc|, one of which only arises when testing
under relaxed memory. These primitive actions contain a continuation,
allowing individual actions to be composed into larger execution
sequences. Each thread of execution consists of such a sequence,
terminated by the \verb|AStop| primitive, which has no continuation
and signals the termination of the thread.

\defineprim{AStop}{}{%
  Terminate the current thread.}

Computations are composed out of a continuation monad, defined as
follows:

\begin{haskellcode}
newtype M n r s a = M { runM :: (a -> Action n r s) -> Action n r s }
\end{haskellcode}

The \verb|Action| type is the type of primitive actions. The type
variables \verb|n|, \verb|r|, \verb|s| and \verb|a| are the underlying
monad (this must be something allowing mutable state, like \verb|IO|
or \verb|ST|); the mutable reference type of that monad; the
corresponding \verb|MonadSTM|; and the input type.

That is, the \verb|M n r s a| type is a wrapper around a function
which, given a continuation, produces a new primitive action.

The \verb|Functor| instance allows applying a function to the input
value of the continuation:

\begin{haskellcode}
instance Functor (M n r s) where
  fmap :: (a -> b) -> M n r s a -> M n r s b
  fmap f (M m) = M (\c -> m (c . f))
\end{haskellcode}

The \verb|Applicative| instance allows injecting a pure value into the
\verb|M| type, by constructing a continuation which consumes this
value. \verb|AReturn| here is a primitive action, a constructor of the
\verb|Action| data type. It also allows extracting a function from one
computation, a value from another, and applying them.

\begin{haskellcode}
instance Applicative (M n r s) where
  pure :: a -> M n r s a
  pure x = M (\c -> AReturn (c x))

  (<*>) :: M n r s (a -> b) -> M n r s a -> M n r s b
  (M f) <*> (M v) = M (\c -> f (\g -> v (c . g)))
\end{haskellcode}

\defineprim{AReturn}{action}{%
  Execute the given action.}

It's not immediately apparent why \verb|AReturn| is necessary, and why
\verb|\c -> c x| can't be used instead. This is because the scheduler
in the testing implementation cannot work at a finer granularity than
individual primitive actions, whereas in the real implementation,
things like exceptions can pre-empt a return. In order to model this
possibility, then, \verb|pure| and \verb|return| must have a
corresponding primitive.

Finally, the \verb|Monad| instance allows sequencing of primitive actions.

\begin{haskellcode}
instance Monad (M n r s) where
  return :: a -> M n r s a
  return = pure

  (>>=) :: M n r s a -> (a -> M n r s b) -> M n r s b
  (M m) >>= (M k) = M (\c -> m (\x -> k x c))
\end{haskellcode}

Each of the operations in \verb|MonadConc| will now be mapped down to
the primitive actions which comprise them.

\subsubsection{Threading}
\label{sec:execution-primops-threading}

All of the forking functions are implemented with the same primitive
action, which corresponds most closely to
\verb|forkWithUnmask|:\footnote{Type signatures have been omitted here
  as they were provided in \sect{abstraction}{typeclass}.}

\begin{haskellcode}
fork     ma = forkWithUnmask     (\_ -> ma)
forkOn c ma = forkOnWithUnmask c (\_ -> ma)
\end{haskellcode}

The \verb|forkWithUnmask| function uses the \verb|AFork| primitive,
which creates a new thread:

\begin{haskellcode}
forkWithUnmask (M ma) = M (AFork action) where
  action unmask = runM (ma unmask) (\_ -> AStop)
\end{haskellcode}

\defineprim{AFork}{(unmask \arr action) (thread\_id \arr action)}{%
  Create a new thread from the first action, and continue executing
  the current thread with the second.}

As the testing implementation does not run things in parallel, the
\verb|forkOn| variant simply ignores its argument:

\begin{haskellcode}
forkOnWithUnmask _ = forkWithUnmask
\end{haskellcode}

Similarly, the \verb|getNumCapabilities| function just lies:

\begin{haskellcode}
getNumCapabilities = return 2
\end{haskellcode}

In the \verb|IO| implementation, these functions behave as expected.

A thread can query its own identifier, by means of the \verb|AMyTId|
primitive:

\begin{haskellcode}
myThreadId = M AMyTId
\end{haskellcode}

\defineprim{AMyTId}{(thread\_id \arr action)}{%
  Continue execution of the current thread by querying the thread
  identifier.}

A thread can signal that another should run with the \verb|AYield|
primitive:

\begin{haskellcode}
yield = M (\c -> AYield (c ()))
\end{haskellcode}

\defineprim{AYield}{action}{%
  Execute the given action, but also signify to the scheduler that it
  may be worth running a different thread now.}

\subsubsection{\texttt{CRef}s}
\label{sec:execution-primops-cref}

There are two testing implementations of \verb|MonadConc|, which
differ in how they implement mutable state. One uses the \verb|ST|
monad, which allows extracting a pure result at the end, but does not
allow any interaction with \verb|IO|. The other uses \verb|IO|. This
choice of \verb|ST| or \verb|IO| is what determines the \verb|n|,
\verb|r|, and \verb|s| parameters of the \verb|M| type.

\begin{haskellcode}
newCRef a = M (ANewRef a)
\end{haskellcode}

\defineprim{ANewRef}{a (cref a \arr action)}{%
  Construct a new \texttt{CRef} and give it to the continuation.}

Internally, \verb|CRef|s have a unique identifier, which is used to
determine if two operations on \verb|CRef|s could interfere with each
other: two writes to the same \verb|CRef| can, but writes to two
different \verb|CRef|s cannot, for example. Each \verb|CRef| also
contains a map from thread identifiers to values, which is used for
implementing relaxed memory, allowing different threads to see values
more recent than the \emph{canonical} value inside the \verb|CRef|.

\begin{haskellcode}
readCRef   ref = M (AReadRef    ref)
readForCAS ref = M (AReadRefCas ref)
\end{haskellcode}

\defineprim{AReadRef}{(cref a) (a \arr action)}{%
  Read the currently visible value of a \texttt{CRef}.}

\defineprim{AReadRefCas}{(cref a) (ticket a \arr action)}{%
  Produce a \texttt{Ticket} from the currently visible state of a
  \texttt{CRef}.}

Note that when testing under relaxed memory, \verb|AReadRef| may not
read the \emph{latest} value, as a write by another thread may have
not yet been committed; similarly \verb|AReadRefCas| might not produce
a ticket for the latest value either.

\begin{haskellcode}
peekTicket tick = M (APeekTicket tick)
\end{haskellcode}

\defineprim{APeekTicket}{(ticket a) (a \arr action)}{%
  Get the value out of a \texttt{Ticket}.}

This will always return the value seen when the \verb|Ticket| was
produced, regardless of whether it has been updated since then.

\begin{haskellcode}
modifyCRef    ref f = M (AModRef    ref f)
modifyCRefCAS ref f = M (AModRefCas ref f)
\end{haskellcode}

\defineprim{AModRef}{(cref a) (a \arr (a, b)) (b \arr action)}{%
  Commit all pending writes and atomically modify the value within a
  \texttt{CRef}.}

\defineprim{AModRefCas}{(cref a) (a \arr (a, b)) (b \arr action)}{%
  Commit all pending writes and atomically modify the value within a
  \texttt{CRef} using a compare-and-swap.}

\begin{haskellcode}
writeCRef ref      a = M (\c -> AWriteRef ref a (c ()))
casCRef   ref tick a = M (ACasRef ref tick a)
\end{haskellcode}

\defineprim{AWriteRef}{(cref a) a action}{%
  Update the value of a \texttt{CRef}. This is visible to the current
  thread immediately.}

\defineprim{ACasRef}{(cref a) (ticket a) a ((succeeded?, ticket a) \arr action}{%
  Update the value of a \texttt{CRef} if it hasn't changed since the
  ticket was produced..}

When testing under a relaxed memory model, the \verb|AWriteRef|
primitive does not cause the write to be visible to threads other than
the one which performed the write. An \verb|ACommit| action is
performed at some point later to do that.

\defineprim{ACommit}{thread\_id cref\_id}{%
  Make the last write to the given \texttt{CRef} by that thread
  visible to all threads.}

The \verb|atomicWrite| operation is implemented in terms of
\verb|modify|, rather than being given its own primitive.

\begin{haskellcode}
atomicWriteCRef ref a = modifyCRef ref (const (a, ()))
\end{haskellcode}

\subsubsection{\texttt{CVar}s}
\label{sec:execution-primops-cvar}

The implementation of \verb|CVar|s is very similar to \verb|CRef|s,
the differences are that there are no relaxed memory issues to worry
about, and that a \verb|CVar| may be empty.

\begin{haskellcode}
newEmptyCVar = M ANewVar
\end{haskellcode}

\defineprim{ANewVar}{cvar a \arr action}{%
  Construct a new \texttt{CVar} and give it to the continuation.}

\begin{haskellcode}
putCVar    cvar a = M (\c -> APutVar cvar a (c ()))
tryPutCVar cvar a = M (ATryPutVar cvar a)
\end{haskellcode}

\defineprim{APutVar}{(cvar a) a action}{%
  Block until the \texttt{CVar} is empty and put a value into it.}

\defineprim{ATryPut}{(cvar a) a (succeeded? \arr action)}{%
  Try to put a value into the \texttt{CVar} without blocking.}

\begin{haskellcode}
readCVar cvar = M (AReadVar cvar)
\end{haskellcode}

\defineprim{AReadVar}{(cvar a) (a \arr action)}{%
  Block until the \texttt{CVar} is full and read its value.}

\begin{haskellcode}
takeCVar    cvar = M (ATakeVar    cvar)
tryTakeCVar cvar = M (ATryTakeVar cvar)
\end{haskellcode}

\defineprim{ATakeVar}{(cvar a) (a \arr action)}{%
  Block until the \texttt{CVar} is full and take its value.}

\defineprim{ATryTakeVar}{(cvar a) (Maybe a \arr action)}{%
  Try to take the value from a \texttt{CVar} without blocking.}

Furthermore, all the \verb|CVar| operations (other than creating a new
one) cause all uncommitted \verb|CRef| writes to be committed.

\subsubsection{Exceptions}
\label{sec:execution-primops-exceptions}

Exceptions can be used to terminate a computation, either in the
current thread or a different one. They can also be caught. The
presentation in terms of primitive actions is deceptively simple, the
discussion in \sect{execution}{stepwise} will elaborate.

\begin{haskellcode}
throw       e = M (\_ -> AThrow e)
throwTo tid e = M (\c -> AThrowTo tid e (c ()))
\end{haskellcode}

\defineprim{AThrow}{exception}{%
  Raises an exception in the current thread, terminating the current
  execution.}

Consider the continuation produced within \verb|throw|. It throws away
its argument, there is no further action to perform, hence the only
possible thing that the execution can do is to terminate the thread.

\defineprim{AThrowTo}{exception action}{%
  Raises an exception in the other thread, blocking if the other
  thread has exceptions masked.}

When an exception is raised, the thread it is raised within terminates
whatever it is currently doing, and backtracks to the closest
exception handler capable of dealing with that type of exception. If
there is no capable handler, the thread is terminated.

\begin{haskellcode}
catch (M ma) h = M (ACatching (runM . h) ma)
\end{haskellcode}

\defineprim{ACatching}{(exception \arr handler) action continuation}{%
  Registers a new exception handler for the duration of the inner
  action.}

Each thread has a stack of exception handlers, which is appended to
with the \verb|ACatching| primitive, and removed from with the
\verb|APopCatching| primitive, which is added automatically when an
\verb|ACatching| primitive is evaluated.

\defineprim{APopCatching}{action}{%
  Remove the exception handler from the top of the stack.}

There is one primitive action when entering a new masking state:

\begin{haskellcode}
mask (M mb) = M (AMasking MaskedInterruptible (\f -> mb f))

uninterruptibleMask (M mb) = M (AMasking MaskedUninterruptible (\f -> mb f))
\end{haskellcode}

\defineprim{AMasking}{masking\_state (unmask \arr action) continuation}{%
  Executes the inner action under a new masking state, and also gives
  it a function to reset the masking state.}

Similarly to \verb|ACatching|, this also has a counterpart primitive
to undo its effect, called \verb|AResetMask|. One might wonder why
this is necessary, and why \verb|AMasking| can't just be used to both
set and reset the masking state, but this is so that more informative
execution traces can be generated for the user.

\defineprim{AResetMask}{set? inner? masking\_state action}{%
  Sets the masking state.}

The \verb|set?| and \verb|inner?| flags are used so that generated
traces can helpfully indicate when a \verb|mask| or
\verb|uninterruptibleMask| function started and stopped executing, and
when an unmasking function passed in to a continuation was used. This
is much more helpful for debugging purposes than just seeing that the
masking state had been changed.

\subsubsection{Software Transactional Memory}
\label{sec:execution-primops-stm}

STM is implemented with its own set of primitive actions which operate
in much the same way as the concurrency primitives. The major
difference is that a transaction is executed atomically, whereas the
concurrency actions are executed one action at a time, allowing for
threads to interfere with each other.

\begin{haskellcode}
atomically stm = M (AAtom stm)
\end{haskellcode}

\defineprim{AAtom}{transaction continuation}{%
  Execute an STM transaction atomically.}

When a transaction is executed by \verb|AAtom|, one of three possible
conditions are differentiated between: the transaction completed
successfully, and returned a value; the transaction aborted due to an
uncaught exception; and the transaction aborted due to a call to
\verb|retry|. In this third case, the thread is blocked until any of
the \verb|CTVar|s referenced in the transaction are mutated, after
which it can be tried again. There are far fewer \verb|MonadSTM|
operations than \verb|MonadConc| ones, so the entirety of the STM
implementation is presented here.

\begin{haskellcode}
retry = S (\_ -> SRetry)
\end{haskellcode}

\defineprim{SRetry}{}{%
  Abort the current transaction.}

the \verb|check| function is provided as a simple wrapper around
\verb|retry|:

\begin{haskellcode}
check b = if b then return () else retry
\end{haskellcode}

Transactions can be composed into larger atomic transactions, and so
it is necessary to be able to handle an aborting transaction without
needing to break atomicity. The \verb|orElse| function allows
combining transactions in this way:

\begin{haskellcode}
orElse a b = S (SOrElse (runSTM a) (runSTM b))
\end{haskellcode}

\defineprim{SOrElse}{transaction transaction continuation}{%
  Try executing the first transaction, if it fails, execute the
  second.}

\begin{haskellcode}
throwSTM e = S (\_ -> SThrow e)

catchSTM stm handler = S (SCatch (runSTM stm) (runSTM . handler))
\end{haskellcode}

\defineprim{SThrow}{exception}{%
  Throw an exception, aborting the current execution flow.}

\defineprim{SCatch}{(exception \arr handler) action continuation}{%
  Registers a new exception handler for the duration of the action.}

As transactions are atomic, the handling of exceptions can be vastly
simplified, where an \verb|SCatch| action is performed by executing
the entire inner action in one step and inspecting the result. No
explicit stack of exception handlers needs to be maintained, the
function call stack suffices. There are no more primitives related to
exception handling than these.

\begin{haskellcode}
newCTVar a = toSTM (SNew a)
\end{haskellcode}

\defineprim{SNew}{a (ctvar a \arr action)}{%
  Create a new \texttt{CTVar} containing the given value.}

\begin{haskellcode}
readCTVar ctvar = S (SRead ctvar)
\end{haskellcode}

\defineprim{SRead}{(ctvar a) (a \arr action)}{%
  Read the current value of a \texttt{CTVar}.}

\begin{haskellcode}
writeCTVar ctvar a = S (\c -> SWrite ctvar a (c ()))
\end{haskellcode}

\defineprim{SWrite}{(ctvar a) a action}{%
  Update a \texttt{CTVar}.}

Remember that the effects of a transaction only take place when it
completes successfully, so \verb|SWrite| will not have any
externally-visible effects until the whole thing
finishes. Furthermore, executing a transaction with \verb|AAtom|
enforces a write barrier.

\subsubsection{Testing Annotations}
\label{sec:execution-primops-annotations}

In order for \dejafu{} to perform limited detection of \verb|CVar|-
and \verb|CTVar|-based deadlocks, something which GHC can use the
garbage collector for, there are the optional testing annotations:

\begin{haskellcode}
_concKnowsAbout var = M (\c -> AKnowsAbout var (c ()))
_concForgets    var = M (\c -> AForgets    var (c ()))
_concAllKnown       = M (\c -> AAllKnown       (c ()))
\end{haskellcode}

\defineprim{AKnowsAbout}{(Either cvar ctvar) action}{%
  Record that the thread has access to the given variable.}

\defineprim{AForgets}{(Either cvar ctvar) action}{%
  Record that the thread no longer has access to the given variable.}

\defineprim{AAllKnown}{action}{%
  Record that all variables the thread knows about have been
  reported.}

The deadlock detection process works like so: if a thread is blocked
on a \verb|CVar| or \verb|CTVar| (an STM transaction referencing it
has aborted), \emph{and} it is known what variables all threads have
access to, \emph{and} all other threads with a reference to that
variable are also blocked on it, \emph{then} the thread is
deadlocked. This can easily be extended to collections of threads
which are all blocked on the same variable.

\subsubsection{Lifting from the Underlying Monad}
\label{sec:execution-primops-lift}

Finally, there is the \verb|ALift| primitive, for lifting an action
from the underlying monad. This is used in the implementation of
\verb|liftIO|:

\begin{haskellcode}
liftIO = M (\c -> ALift (fmap c ma))
\end{haskellcode}

\defineprim{ALift}{(n action)}{%
  Execute an action from the underlying monad.}
