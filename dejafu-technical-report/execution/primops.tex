Computations are composed out of a continuation monad, defined as
follows:

\begin{haskellcode}
newtype M n r s a = M { runM :: (a -> Action n r s) -> Action n r s }
\end{haskellcode}

The \verb|Action| type is the type of primitive actions. There are a
number of primitive actions used to construct the testing instances of
\verb|MonadConc|. Each thread of execution consists of a sequence of
continuations, terminated by the \verb|AStop| primitive, which has no
continuation and signals the termination of the thread.

The type variables \verb|n|, \verb|r|, \verb|s| and \verb|a| are the
underlying monad (this must be something allowing mutable state, like
\verb|IO| or \verb|ST|); the mutable reference type of that monad; the
corresponding \verb|MonadSTM|; and the input type. That is, the
\verb|M n r s a| type is a wrapper around a function which, given a
continuation, produces a new primitive action.

The \verb|Functor| instance allows applying a function to the input
value of the continuation:

\begin{haskellcode}
instance Functor (M n r s) where
  fmap f (M m) = M (\c -> m (c . f))
\end{haskellcode}

The \verb|Applicative| instance allows injecting a pure value into the
\verb|M| type, by constructing a continuation which consumes this
value. It also allows extracting a function from one computation, a
value from another, and applying them.

\begin{haskellcode}
instance Applicative (M n r s) where
  pure x = M (\c -> AReturn (c x))

  (M f) <*> (M v) = M (\c -> f (\g -> v (c . g)))
\end{haskellcode}

\defineprim{AReturn}{action}{Execute the given action.}

Why is \verb|AReturn| necessary? Why not \verb|\c -> c x| instead? It
is because the scheduler in the testing implementation cannot work at
a finer granularity than individual primitive actions, whereas in the
real implementation, things like exceptions can pre-empt a return. In
order to model this possibility, then, \verb|pure| must have a
corresponding primitive.

Finally, the \verb|Monad| instance allows sequencing.

\begin{haskellcode}
instance Monad (M n r s) where
  return = pure

  (M m) >>= (M k) = M (\c -> m (\x -> k x c))
\end{haskellcode}

The full listing of primitive actions is available in Appendix
\ref{app:primops}. Some of the more interesting ones will be expanded
upon in this section.

\subsubsection{Threading}
\label{sec:execution-primops-threading}

All of the forking functions are implemented with the same primitive
action, which corresponds most closely to
\verb|forkWithUnmask|:\footnote{Type signatures have been omitted here
  as they were provided in \sect{abstraction}{typeclass}.}

\begin{haskellcode}
fork     ma = forkWithUnmask     (\_ -> ma)
forkOn c ma = forkOnWithUnmask c (\_ -> ma)
\end{haskellcode}

The \verb|forkWithUnmask| function uses the \verb|AFork| primitive,
which creates a new thread:

\begin{haskellcode}
forkWithUnmask (M ma) = M (AFork action) where
  action unmask = runM (ma unmask) (\_ -> AStop)
\end{haskellcode}

\defineprim{AFork}{(unmask \arr action) (thread\_id \arr action)}{%
  Create a new thread from the first action, and continue executing
  the current thread with the second.}

As the testing implementation does not run things in parallel, the
\verb|forkOn| variant simply ignores its argument:

\begin{haskellcode}
forkOnWithUnmask _ = forkWithUnmask
\end{haskellcode}

Similarly, the \verb|getNumCapabilities| function just returns a
constant value:

\begin{haskellcode}
getNumCapabilities = return 2
\end{haskellcode}

In the \verb|IO| implementation, these functions behave as expected.

\subsubsection{Exceptions}
\label{sec:execution-primops-exceptions}

Exceptions can be used to terminate a computation, either in the
current thread or a different one. They can also be caught. The
presentation in terms of primitive actions is deceptively simple, see
\sect{execution}{stepwise} for the more complex execution details.

\begin{haskellcode}
throw       e = M (\_ -> AThrow e)
throwTo tid e = M (\c -> AThrowTo tid e (c ()))
\end{haskellcode}

\defineprim{AThrow}{exception}{%
  Raises an exception in the current thread, terminating the current
  execution.}

Consider the continuation produced within \verb|throw|. It throws away
its argument, there is no further action to perform, hence the only
possible thing that the execution can do is to terminate the thread.

\defineprim{AThrowTo}{exception action}{%
  Raises an exception in the other thread, blocking if the other
  thread has exceptions masked.}

When an exception is raised, the thread it is raised within stops
whatever it is currently doing, and backtracks to the closest
exception handler capable of dealing with that type of exception. If
there is no capable handler, the thread is terminated.

\begin{haskellcode}
catch (M ma) h = M (ACatching (runM . h) ma)
\end{haskellcode}

\defineprim{ACatching}{(exception \arr handler) action continuation}{%
  Registers a new exception handler for the duration of the inner
  action.}

Each thread has a stack of exception handlers, where the
\verb|ACatching| primitive pushes a new handler, and
\verb|APopCatching| pops. \verb|APopCatching| is added automatically
when an \verb|ACatching| primitive is evaluated.

\defineprim{APopCatching}{action}{%
  Remove the exception handler from the top of the stack.}

There is one primitive action for entering a new masking state:

\begin{haskellcode}
mask (M mb) = M (AMasking MaskedInterruptible (\f -> mb f))

uninterruptibleMask (M mb) = M (AMasking MaskedUninterruptible (\f -> mb f))
\end{haskellcode}

\defineprim{AMasking}{masking\_state (unmask \arr action) continuation}{%
  Executes the inner action under a new masking state, and also gives
  it a function to reset the masking state.}

Like \verb|ACatching|, the \verb|AMasking| action also has a
counterpart primitive to undo its effect, called
\verb|AResetMask|. Why is this necessary? \verb|AMasking| can both set
and reset the masking state, but the separate function enables more
informative execution traces to be generated for the user.

\defineprim{AResetMask}{set? inner? masking\_state action}{%
  Sets the masking state.}

The \verb|set?| and \verb|inner?| flags are used so that generated
traces can helpfully indicate when a \verb|mask| or
\verb|uninterruptibleMask| function started and stopped executing, and
when an unmasking function passed in to a continuation was used. This
is much more helpful for debugging purposes than just seeing that the
masking state had been changed.

\subsubsection{Software Transactional Memory}
\label{sec:execution-primops-stm}

STM is implemented with its own set of primitive actions which operate
in much the same way as the concurrency primitives. The major
difference is that a transaction is executed atomically, whereas the
concurrency actions are executed one action at a time, allowing
threads to interfere with each other.

\begin{haskellcode}
atomically stm = M (AAtom stm)
\end{haskellcode}

\defineprim{AAtom}{transaction continuation}{%
  Execute an STM transaction atomically.}

When a transaction is executed by \verb|AAtom|, there are three
possible outcomes:

\begin{enumerate}
\item the transaction completed successfully, and returned a value;

\item the transaction aborted due to an uncaught exception; and

\item the transaction aborted due to a call to \verb|retry|. In this
  third case, the thread is blocked until any of the \verb|CTVar|s
  referenced in the transaction are mutated, after which it can be
  tried again.
\end{enumerate}

Transactions can be composed into larger atomic transactions. Aborting
a component transaction must not break atomicity. The \verb|orElse|
function combines transactions appropriately:

\begin{haskellcode}
orElse a b = S (SOrElse (runSTM a) (runSTM b))
\end{haskellcode}

\defineprim{SOrElse}{transaction transaction continuation}{%
  Try executing the first transaction, if it fails, execute the
  second.}

As transactions are atomic, the handling of exceptions can be vastly
simplified. Catching is performed by executing the entire inner action
and inspecting the result. No explicit stack of exception handlers
needs to be maintained, as the function call stack suffices.

The effects of a transaction only become visible when it completes
successfully. Furthermore, executing a transaction enforces a write
barrier.

\subsubsection{Testing Annotations}
\label{sec:execution-primops-annotations}

In order for \dejafu{} to perform limited detection of \verb|CVar|-
and \verb|CTVar|-based deadlocks, something which GHC can use the
garbage collector for, there are the optional testing annotations:

\begin{haskellcode}
_concKnowsAbout var = M (\c -> AKnowsAbout var (c ()))
_concForgets    var = M (\c -> AForgets    var (c ()))
_concAllKnown       = M (\c -> AAllKnown       (c ()))
\end{haskellcode}

\defineprim{AKnowsAbout}{(Either cvar ctvar) action}{%
  Record that the thread has access to the given variable.}

\defineprim{AForgets}{(Either cvar ctvar) action}{%
  Record that the thread no longer has access to the given variable.}

\defineprim{AAllKnown}{action}{%
  Record that all variables the thread knows about have been
  reported.}

If a thread is blocked on a \verb|CVar| or \verb|CTVar| (an STM
transaction referencing it has aborted), \emph{and} it is known what
variables all threads have access to, \emph{and} all other threads
with a reference to that variable are also blocked on it, \emph{then}
the thread is deadlocked. This can easily be extended to collections
of threads which are all blocked on the same variable.
