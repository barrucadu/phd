Firstly we explore the problem of testing concurrent programs and how
it can be done. In \textbf{\chap{abstraction}} we discuss our
typeclass abstraction for concurrency and how it relates to GHC's
standard concurrency API in terms of
functionality. \textbf{\chap{execution}} explains how, given a monadic
action polymorphic in the monad (as long as it has an instance of to
our typeclass) we can execute it with a given scheduler, and
\textbf{\chap{sct}} extends this to cover a systematic exploration of
the space of all schedules. \textbf{\chap{correctness}} discusses the
issues of correctness: how do we know if a result reported by
\dejafu{} is actually right?

Then, we move on to the real-world impact of this work, with case
studies of \dejafu{} applied to two instances of pre-existing code,
and one custom library in \textbf{\chap{casestudies}}.
\textbf{\chap{practice}} further discusses the usage of \dejafu{} in
combination with existing code. To conclude,
\textbf{\chap{conclusions}} discusses related work, and summarises the
community reception to the idea and what is still to be done.
