The report is divided into two broad parts:

Part I explores the problem of testing concurrent programs and how it
can be done, culminating in case studies of \dejafu{} applied to two
instances of pre-existing code, and one custom library in
\chap{casestudies}. In \chap{abstraction} we discuss our typeclass
abstraction for concurrency ans how it relates to GHC's standard
concurrency API in terms of functionality. \chap{execution} explains
how, given a monadic action polymorphic in the monad (as long as it
belongs to our typeclass) we can execute it with a given scheduler,
and \chap{sct} extends this to cover a systematic exploration of the
space of all schedules.

Part II discusses the real-world impact of this work with
\chap{practice} discussing the usage of \dejafu{} in combination with
existing code, and \chap{conclusions} summarising the community
reception to the idea and what is still to be done.
