We aim to support all of the functionality of GHC's concurrency API,
as made available through the Control.Concurrent and Control.Exception
module hierarchies, which (1) does not unavoidably \emph{require}
support from the run-time system, and (2) is not so nondeterministic
that there is no sensible way to model it accurately.

Some specific examples of things which are out of scope:

\begin{itemize}
\item \verb|threadDelay|, as all this guarantees is that a thread will
  not run \textit{sooner} than the delay. There is no upper bound on
  the delay, and also no guarantee that any other thread will be
  scheduled during the delay.

\item \verb|threadWaitRead|, \verb|threadWaitWrite|, and the
  \verb|STM| variants, as there is no way to tell if a file descriptor
  can be read from or written to without involving \verb|IO|, and in
  addition is influenced by other non-Haskell processes accessing the
  same file.

\item Threads bound to specific operating-system threads (``bound
  threads''), as these affect which operating system thread FFI calls
  operate on, and so alter program behaviour in a parallel setting.

\item \verb|BlockedIndefinitely| exceptions, as determining if a
  thread is blocked indefinitely on an \verb|MVar| is a garbage
  collection problem, which is out of the reach of the
  programmer. \dejafu{} does provide some annotation functions to
  record which shared state a thread knows about, and so similar
  functionality can be supported on a limited scale, even the GHC
  documentation warns programmers against relying on these exceptions
  for correct functioning of a program.\footnote{ \quot{Note that this
      feature is intended for debugging, and should not be relied on
      for the correct operation of your program. There is no guarantee
      that the garbage collector will be accurate enough to detect
      your deadlock, and no guarantee that the garbage collector will
      run in a timely enough manner.}{controlConcurrent}}
\end{itemize}
