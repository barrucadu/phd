Systematic concurrency testing (SCT) \citep{dpor, pbound, heisenbugs,
  empirical} is a way of tackling the problem of nondeterminism when
writing tests. It aims to test a large number of schedules, whilst
typically also making use of local knowledge of the program to reduce
the number of schedules needed to be confident of an accurate
result. By testing many schedules, we can increase our confidence
confident that any bugs which have not been found are unlikely to be
exhibited.

SCT overcomes the scheduling problem by forcing a concurrent program
to use a scheduler implemented as part of the testing framework:
either by overriding the concurrency primitives of the language, or by
modifying the program under test to call out to this new scheduler (as
in PULSE \citep{pulse}).

Once the scheduler is under control, schedules can be recorded and
replayed, giving reproducibility. Furthermore, by observing which
scheduling decisions are available at each decision point, possible
schedules can be systematically explored, making different decisions
on subsequent executions. Common methods of choosing schedules to take
are random \citep{empirical}, schedule bounding \citep{pbound}, and
partial-order reduction \citep{dpor}. The last of these is
\emph{complete}: partial-order reduction will find all distinct
program states given enough time, in a more intelligent way than just
trying all schedules.
