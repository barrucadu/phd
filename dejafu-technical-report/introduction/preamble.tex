Concurrency is notoriously difficult to get right \citep{overrated},
sometimes even leading to death \citep{therac25}. The problem largely
stems from the nondeterminism of scheduling: the same program with the
same inputs may produce different results depending on the schedules
chosen at runtime. This makes it difficult to use traditional testing
techniques with concurrent programs, which rely on the result of
executing a test to be deterministic. So-called ``Heisenbugs'' make it
difficult to be confident of the correctness of concurrent programs:
no bug has been observed during the testing process, but how do we
\emph{know} that there aren't any?

Despite the difficulty, concurrency is important for producing many
real-world applications. For example, applications with a lot of input
and output can be more responsive by executing I/O asynchronously.
Concurrency is a useful program structuring technique, and it is here
to stay.

There are now a few well-known techniques to avoid concurrency bugs,
such as protecting mutable state with locks, and acquiring locks in a
fixed order. Exercises like the Dining Philosophers
\citep{diningphilosophers} and the Santa Claus Problem
\citep{santaclaus} allow programmers to explore these topics in small
well-understood settings. However, as systems grow, it becomes
difficult to think about how different components interact, and it is
easy to slip up and introduce a bug.
