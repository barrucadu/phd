\verb|CVar|s are nice, until we need more than one, and find they need
to be kept synchronised. As we can only claim \emph{one} \verb|CVar|
atomically, it seems we need to introduce a \verb|CVar| to control
access to \verb|CVar|s! This is unwieldy and prone to bugs.

\emph{Software transactional memory} (STM) is the solution. STM uses
\verb|CTVar|s, or \emph{Concurrent Transactional Variables}, and is
based upon the idea of atomic \emph{transactions}. An STM transaction
consists of one or more operations over a collection of \verb|CTVar|s,
where a transaction may be aborted part-way through depending on their
values. If the transaction fails, \emph{none of its effects take
  place}, and the thread blocks until the transaction can
succeed. This means we need to limit the possible actions in an STM
transaction to those which can be safely undone and repeated, so we
have another typeclass, \verb|MonadSTM|.

\verb|CTVar|s always contain a value, as shown in the types of the
functions:

\begin{haskellcode}
newCTVar   :: MonadSTM s => a -> s (CTVar s a)
readCTVar  :: MonadSTM s => CTVar s a -> s a
writeCTVar :: MonadSTM s => CTVar s a -> a -> s ()
\end{haskellcode}

If we read a \verb|CTVar| and don't like the value it has, the
transaction can be aborted, and the thread will block until any of the
referenced \verb|CTVar|s have been mutated:

\begin{haskellcode}
retry :: MonadSTM s => s a
check :: MonadSTM s => Bool -> s ()
\end{haskellcode}

We can also try executing a transaction, and do something else if it
fails:

\begin{haskellcode}
orElse :: MonadSTM s => s a -> s a -> s a
\end{haskellcode}

The nice thing about STM transactions is that they \emph{compose}. We
can take small transactions and build bigger transactions from them,
and the whole is still executed atomically. This means we can do
complex state operations involving multiple shared variables without
worrying!

Each \verb|MonadConc| has an associated \verb|MonadSTM|, and can
execute transactions of it atomically:

\begin{haskellcode}
atomically :: MonadConc m => STMLike m a -> m a
\end{haskellcode}

The instance of \verb|MonadConc| for \verb|IO| uses \verb|STM| as its
\verb|MonadSTM|.

STM can also use exceptions:

\begin{haskellcode}
throwSTM :: (Exception e, MonadSTM s) => e -> s a
catchSTM :: (Exception e, MonadSTM s) => s a -> (e -> s a) -> s a
\end{haskellcode}

If an exception propagates uncaught to the top of a transaction, that
transaction is aborted and the exception is re-thrown in the thread.

There are utility functions for \verb|CTVar|s provided in
Control\-.Concurrent\-.STM\-.CTVar, and an STM equivalent of
\verb|CVar|s in Control\-.Concurrent\-.STM\-.CTMVar.
