There are three memory models supported in \dejafu{}:

\defineword{Sequential Consistency}{This model is the most
  intuitive. A program behaves as a simple interleaving of the actions
  in different threads. When a \texttt{CRef} is written to, that write
  is immediately visible to all threads.}

\defineword{Total Store Order (TSO)}{Each thread has a write buffer. A
  thread sees its writes immediately, but other threads will only see
  writes when they are committed, which may happen later. Writes are
  committed in the same order that they are created.}

\defineword{Partial Store Order (PSO)}{A relaxation of TSO where each
  thread has a write buffer for each \texttt{CRef}. A thread sees its
  writes immediately, but other threads will only see writes when they
  are committed, which may happen later. Writes to different
  \texttt{CRef}s are not necessarily committed in the same order that
  they are created.}

The memory model only makes a difference for unsynchronised
operations, such as \verb|readCRef|, \verb|writeCRef|, and
\verb|readForCAS|.

The default memory model for testing is TSO, as that most accurately
models the behaviour of modern x86 processors. The use of a relaxed
memory model can require a much larger number of schedules to be
tested when unsynchronised operations are used. PSO required no
additional work to support, and so was also included.
