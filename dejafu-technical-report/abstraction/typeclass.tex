Readers already familiar with GHC's concurrency primitives may find it
enough to skim this section noting the syntactic differences in the
\dejafu{} variant.

\begin{departure}
  The few departures from the semantics of the traditional concurrency
  abstraction are highlighted like this.
\end{departure}

The \verb|MonadConc| typeclass has an instance\footnote{A typeclass
  has \emph{instances}; each type may have one unique instance for a
  typeclass.} for \verb|IO|, and so existing code using no I/O other
than concurrency and exceptions can be made suitable for testing quite
simply. Existing code which makes use of more functionality may
require a light dusting of \verb|liftIO|\footnote{The \texttt{IO} type
  allows unrestricted side-effects during execution. It turns out that
  many useful types are just \texttt{IO} with some extra structure
  applied, and the \texttt{liftIO} function (which belongs to a
  typeclass called \texttt{MonadIO}) can be used to `translate' the
  effects into such a type.} where it is safe, see
\sect{abstraction}{typeclass-lifting}.

\subsubsection{Threads}
\label{sec:abstraction-typeclass-threads}

Threads let a program do multiple things at once. Every program has at
least one thread, which starts where \verb|main| does and runs until
the program terminates. A thread is the basic unit of concurrency. It
lets us pretend (with parallelism, it might even be true!) that we're
computing multiple things at once.

We can start a new thread with the function:\footnote{This is a
  function named \texttt{fork} with a \emph{type signature}. Type
  signatures may contain typeclass constraints, type variables, type
  constructors (similar to generics in other languages), and concrete
  types. Here \texttt{ThreadId} is a type constructor and is applied
  to the type variable \texttt{m}, which is constrained to be a type
  with an instance of \texttt{MonadConc}.}

\begin{haskellcode}
fork :: MonadConc m => m () -> m (ThreadId m)
\end{haskellcode}

The \verb|fork| function starts evaluating its argument in a separate
thread. It also gives us back a (monad-specific) \verb|ThreadId|
value, which we can use to kill the thread later on, if we want.

A thread can query its own \verb|ThreadId|:

\begin{haskellcode}
myThreadId :: MonadConc m => m (ThreadId m)
\end{haskellcode}

In a real machine, there are of course a number of processors and
cores. It may be that a particular application of concurrency is only
a net gain if every thread is operating on a separate core, so that
threads are not interrupting each other. The GHC run-time system
refers to the number of Haskell threads that can run truly
simultaneously as the number of \emph{capabilities}. We can query this
value, and fork threads which are bound to a particular capability:

\begin{haskellcode}
getNumCapabilities :: MonadConc m => m Int
forkOn :: MonadConc m => Int -> m () -> m (ThreadId m)
\end{haskellcode}

% Layout hack - line is too long otherwise
The \verb|forkOn| function interprets the capability number modulo the
value returned by \\\verb|getNumCapabilities|.

\begin{departure}
  \verb|getNumCapabilities| is not required to return a true
  result. The testing instances return ``2'' despite executing
  everything in the same capability, to encourage more
  concurrency. This is to preserve determinism between different
  executions, where the actual number of capabilities may differ. The
  \verb|IO| instance does return a true result.
\end{departure}

Sometimes we just want the special case of evaluating something in a
separate thread, for which we can use \verb|spawn| (implemented in
terms of \verb|fork|):

\begin{haskellcode}
spawn :: MonadConc m => m a -> m (CVar m a)
\end{haskellcode}

A \verb|spawn| application returns a \verb|CVar| (\emph{Concurrent
  Variable}), to which we can apply \verb|readCVar|, blocking until
the computation is done and the value is stored.

Threads are scheduled non-deterministically. Every time the run-time
system decides to perform a context switch, one of the runnable
threads will be executed. Sometimes, however, a thread may be runnable
but also waiting for something to happen. The programmer can provide a
clue to the scheduler that another thread should be tried instead:

\begin{haskellcode}
yield :: MonadConc m => m ()
\end{haskellcode}

Calling \verb|yield| gives any other thread the opportunity to execute
instead of the yielding one, but it is not \emph{required} to cause a
context switch except on co-operative multitasking systems.

\subsubsection{Threading and the Foreign Function Interface}
\label{sec:abstraction-typeclass-ffithreads}

In order to accommodate Foreign Function Interface (FFI) calls which
may block, GHC provides a mechanism for forking a Haskell thread to an
operating system thread. This allows FFI calls to be managed by the
operating system, unlike normal Haskell threads which are managed by
the run-time system and multiplexed onto a smaller number of operating
system threads. This means that blocking FFI calls do not necessarily
block the entire program.

There is no \verb|MonadConc| equivalent of bound threads, as there
would be no way to reliably test their behaviour. Unfortunately, if
bound threads are required, \verb|IO| will have to be used.

A few predicates are provided for compatibility:\footnote{Function
  application in Haskell uses no special syntax to denote argument
  values, only juxtaposition, so \texttt{return False} is applying the
  value \texttt{False} to the function \texttt{return}. The function
  \texttt{return} is used to inject a value into a monad, it is
  unfortunately named and has nothing to do with the return keyword in
  other languages.}

\begin{haskellcode}
rtsSupportsBoundThreads :: Bool
rtsSupportsBoundThreads = False

isCurrentThreadBound :: MonadConc m => m Bool
isCurrentThreadBound = return False
\end{haskellcode}

\subsubsection{Mutable State}
\label{sec:abstraction-typeclass-crefs}

Threading by itself is not really enough. We need to be able to
\emph{communicate} between threads: we've already seen an instance
of this with the \verb|spawn| function.

The simplest type of mutable shared state provided is the \verb|CRef|
(\emph{Concurrent Reference}). \verb|CRef|s are shared variables which
can be written to and read from:

\begin{haskellcode}
newCRef    :: MonadConc m => a -> m (CRef m a)
readCRef   :: MonadConc m => CRef m a -> m a
modifyCRef :: MonadConc m => CRef m a -> (a -> (a, b)) -> m b
writeCRef  :: MonadConc m => CRef m a -> a -> m ()
\end{haskellcode}

The \verb|modifyCRef| function is atomic. The \verb|readCRef| and
\verb|writeCRef| functions are not synchronised: it is possible for
one thread to read from a \verb|CRef| strictly after another thread
has written to it and to observe an old value! See
\sect{abstraction}{mem}. To ensure that every thread sees a value as
soon as it is written there is a synchronised write function:

\begin{haskellcode}
atomicWriteCRef :: MonadConc m => CRef m a -> a -> m ()
\end{haskellcode}

However, synchronisation can slow down execution in a parallel
environment. Note that \verb|modifyCRef| is also synchronised.

\subsubsection{Compare-and-swap (CAS)}
\label{sec:abstraction-typeclass-cas}

As \verb|CRef|s correspond very closely to mutable memory locations,
there is also a compare-and-swap interface available. Compare-and-swap
is a synchronised atomic primitive which is used to update a location
in memory if and only if it has not been changed since some witness
value was produced. This role of this witness value is called a
\verb|Ticket| here:

\begin{haskellcode}
readForCAS :: MonadConc m => CRef m a -> m (Ticket m a)
peekTicket :: MonadConc m => Ticket m a -> m a
\end{haskellcode}

A \verb|Ticket| can be used to check if a \verb|CRef| has been written
to since it was produced, and can also be used to get the value that
was seen then.

\begin{haskellcode}
casCRef :: MonadConc m => CRef m a -> Ticket m a -> a -> m (Bool, Ticket m a)
\end{haskellcode}

The \verb|casCRef| function is synchronised, and strict in the value
written. It replaces the value within a \verb|CRef| if it hasn't been
modified since the \verb|Ticket| was produced. It returns an
indication of success and a \verb|Ticket| to use in future
operations. This operation is often used in the implementation of
lock-free synchronisation primitives.

There is also an equivalent of \verb|modifyCRef| using a
compare-and-swap. This behaves almost the same as the non-CAS version
but may be more performant in some cases, and is strict in the value
being written:

\begin{haskellcode}
modifyCRefCAS :: MonadConc m => CRef m a -> (a -> (a, b)) -> m b
\end{haskellcode}

\subsubsection{Mutual Exclusion}
\label{sec:abstraction-typeclass-cvars}

A \verb|CVar| is a shared variable under \emph{mutual exclusion}. It
has two possible states: \emph{full} or \emph{empty}. Writing to a
full \verb|CVar| blocks until it is empty, and reading or taking from
an empty \verb|CVar| blocks until it is full. There are also
non-blocking functions which return an indication of success:

\begin{haskellcode}
newEmptyCVar :: MonadConc m => m (CVar m a)
putCVar      :: MonadConc m => CVar m a -> a -> m ()
readCVar     :: MonadConc m => CVar m a -> m a
takeCVar     :: MonadConc m => CVar m a -> m a
tryPutCVar   :: MonadConc m => CVar m a -> a -> m Bool
tryTakeCVar  :: MonadConc m => CVar m a -> m (Maybe a)
\end{haskellcode}

Unfortunately, the mutual exclusion behaviour of \verb|CVar|s means
that computations can become \emph{deadlocked}. For example, deadlock
occurs if every thread tries to take from the same \verb|CVar|. The
GHC run-time system can detect this in some situations (and will
complain if it does), and so can \dejafu{} in a more informative way.

\begin{departure}
  \dejafu{} can only detect deadlock to the same extent as GHC if
  every thread is annotated with a declaration of the \verb|CVar|s it
  knows about. This is because GHC can detect deadlocks during garbage
  collection, which is out of the reach of \dejafu{}.
\end{departure}

\subsubsection{Exceptions}
\label{sec:abstraction-typeclass-excs}

Exceptions are a way to bail out of a computation early. Whether
they're a good solution to that problem is a question of style, but
they can be used to jump quickly to error handling code when
necessary. The basic functions for dealing with exceptions are:

\begin{haskellcode}
catch :: (Exception e, MonadConc m) => m a -> (e -> m a) -> m a
throw :: (Exception e, MonadConc m) => e -> m a
\end{haskellcode}

Calling \verb|throw| causes the computation to jump back to the
nearest enclosing \verb|catch| capable of handling the particular
exception. As exceptions belong to a typeclass, rather than being a
concrete type, different \verb|catch| functions can be nested, to
handle different types of exceptions.

\begin{departure}
  The \verb|IO| \verb|catch| function can catch exceptions from pure
  code. This is not true in general for \verb|MonadConc| instances.
  So some things which work normally may not work in testing, and we
  risk false negatives. This is a small cost, however, as exceptions
  from pure code are things like pattern match failures and evaluating
  \verb|undefined|, which are arguably bugs.
\end{departure}

Exceptions can be used to kill a thread:

\begin{haskellcode}
throwTo :: (Exception e, MonadConc m) => ThreadId m -> e -> m ()
killThread :: MonadConc m => ThreadId m -> m ()
\end{haskellcode}

These functions block until the target thread is in an appropriate
state to receive the exception. A thread has a \emph{masking state},
which can be used to block exceptions from other threads. There are
three masking states: \emph{unmasked}, in which a thread can have
exceptions thrown to it; \emph{interruptible}, in which a thread can
only have exceptions thrown to it if it is blocked; and
\emph{uninterruptible}, in which a thread cannot have exceptions
thrown to it. When a thread is started, it inherits the masking state
of its parent. We can also execute a subcomputation with a new masking
state:\footnote{These functions have a \emph{higher-ranked}
  type. Removing the \texttt{forall} stuff, we have \texttt{(m a -> m
    a) -> m b}, which is a function which takes a function as an
  argument and returns a result. The \texttt{forall} is necessary
  because, without it, the concrete type that the variable \texttt{a}
  is unified with is fixed across \emph{all} usage sites, whereas with
  the \texttt{forall} it can be determined uniquely everywhere it is
  used.}

\begin{haskellcode}
mask :: MonadConc m => ((forall a. m a -> m a) -> m b) -> m b
uninterruptibleMask :: MonadConc m => ((forall a. m a -> m a) -> m b) -> m b
\end{haskellcode}

In both cases, the action evaluated is passed a function to reset the
masking state to the original one. A thread can be forked and given a
function to reset the masking state:

\begin{haskellcode}
forkWithUnmask :: MonadConc m => ((forall a. m a -> m a) -> m ())
  -> m (ThreadId m)
forkOnWithUnmask :: MonadConc m => Int -> ((forall a. m a -> m a) -> m ())
  -> m (ThreadId m)
\end{haskellcode}

We can also fork a thread and call a supplied function when the thread
is about to terminate, which is useful for informing the parent when a
child terminates, for example:

\begin{haskellcode}
forkFinally :: MonadConc m => m a -> (Either SomeException a -> m ())
  -> m (ThreadId m)
\end{haskellcode}

The \verb|SomeException| type is the top of the exception hierarchy,
and so can be used to catch all exceptions.

\subsubsection{Lifting Actions into \texttt{MonadConc}}
\label{sec:abstraction-typeclass-lifting}

If the programmer needs to make use of \verb|IO| actions, rather than
\verb|MonadConc| actions, then this can be achieved by adding a
\verb|MonadIO| context and using \verb|liftIO|. However, this can
easily compromise the results of testing, as the test runner cannot
peek inside \verb|IO| actions. Thus, it is only safe to life an
\verb|IO| action in this way if:

\begin{itemize}
\item \emph{The action is atomic and synchronised.}

  Otherwise the test framework will possibly miss schedules which lead
  to a bug.

\item \emph{The action is deterministic}, when executed as part of a
  computation with a deterministic schedule.

  Otherwise a fundamental assumption behind the testing methodology
  is false, and no guarantees about completeness can be made.

\item \emph{The action cannot block on the action of another thread.}

  Otherwise test execution may deadlock.
\end{itemize}

In practice these restrictions are not particularly onerous. The only
unsynchronised actions are the \verb|CRef| ones, or primitives exposed
by GHC. Testing in any language already assumes that any I/O is
deterministic. Atomicity and inter-thread blocking tend not to be
issues when communicating with the external environment.
