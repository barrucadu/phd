For readers who are unfamiliar with Haskell, syntax and terminology
will be explained as they are introduced.

In order to test concurrent programs, we must first create a
concurrency abstraction rich enough to express everything that is
commonly needed. However, care must be taken that it does not become
so rich that it becomes too difficult to implement.

The abstraction developed for \dejafu{} can be thought of as three
independent components:

\begin{enumerate}
\item There is the \verb|MonadConc| typeclass,\footnote{Typeclasses
    are similar to interfaces in object-oriented languages. The key
    difference is that they also allow polymorphism based on the
    \emph{return type} of a function as well as the \emph{argument
      types}.} which abstracts over much of the operations provided in
  the Control.Concurrent hierarchy,\footnote{Haskell modules are
    arranged into a hierarchy, corresponding to files and
    directories.} and also some other functionality like mutable
  memory cells (\verb|IORef|s) and exceptions.

\item There is the \verb|MonadSTM| typeclass, which abstracts over
  GHC's software transactional memory API, and is related to
  \verb|MonadConc| but can be used independently.

\item There is the memory model, which influences the behaviour of
  some of the \verb|MonadConc| operations and also the SCT behaviour.
\end{enumerate}

Originally only a sequentially consistent memory model was provided,
but some support for relaxed memory was added following community
feedback.
