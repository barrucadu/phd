The \emph{auto-update} library runs tasks periodically, but only if
needed. For example, a single worker thread may get the time every
second and store it to a shared \verb|IORef|, rather than have many
threads starting within a second of each other all get the time
independently \citep{autoupdate}. Despite the core functionality being
very simple, two race conditions were noticed by users inspecting the
code in October 2014.

\begin{figure}[t]
  \centering
  \begin{minted}{haskell}
data UpdateSettings a = UpdateSettings
    { updateFreq           :: Int
    , updateSpawnThreshold :: Int
    , updateAction         :: IO a
    }

defaultUpdateSettings :: UpdateSettings ()
defaultUpdateSettings = UpdateSettings
    { updateFreq           = 1000000
    , updateSpawnThreshold = 3
    , updateAction         = return ()
    }

mkAutoUpdate :: UpdateSettings a -> IO (IO a)
mkAutoUpdate us = do
    currRef      <- newIORef Nothing
    needsRunning <- newEmptyMVar
    lastValue    <- newEmptyMVar

    void $ forkIO $ forever $ do
        takeMVar needsRunning

        a <- catchSome $ updateAction us

        writeIORef currRef $ Just a
        void $ tryTakeMVar lastValue
        putMVar lastValue a

        threadDelay $ updateFreq us

        writeIORef currRef Nothing
        void $ takeMVar lastValue

    return $ do
        mval <- readIORef currRef
        case mval of
            Just val -> return val
            Nothing -> do
                void $ tryPutMVar needsRunning ()
                readMVar lastValue

catchSome :: IO a -> IO a
catchSome act = catch act $
  \e -> return $ throw (e :: SomeException)
  \end{minted}
  \caption{\emph{auto-update} implementation}
  \label{fig:example-autoupdate}
\end{figure}

The entire implementation, excluding comments and imports, is
reproduced in Figure \ref{fig:example-autoupdate}. The
\verb|mkAutoUpdate| function spawns a worker thread, which performs
the update action at the given frequency, only if the
\verb|needsRunning| flag has been set. It returns an action to attempt
to read the current result, demanding one be computed and blocking
until it has been done if there isn't one.

The simpler race condition occurs if the reading thread is pre-empted
by the worker thread after putting into \verb|needsRunning|, and does
not run again until after the delay has passed. In this case the
worker thread can become blocked on taking for a second time from
\verb|needsRunning|. The reader thread will be unable to read from
\verb|lastValue| as the worker thread emptied it as the last action it
performed. The transformation to the \verb|MonadConc| typeclass is
mostly simple, however the \verb|threadDelay| must be wrapped inside a
call to \verb|liftIO|. The first race condition can be exhibited with
the following test:

\begin{minted}{haskell}
test :: (MonadConc m, MonadIO m) => m ()
test = do
  auto <- mkAutoUpdate defaultUpdateSettings
  auto
\end{minted}

The output is as we would expect, knowing the bug is present:

\begin{verbatim}
> autocheckIO test
[fail] Never Deadlocks (checked: 1)
        [deadlock] S0--------S1-----------S0-
[pass] No Exceptions (checked: 9)
[fail] Consistent Result (checked: 8)
        [deadlock] S0-----P1-S0---S1-----------S0-
        () S0--------S1---------P0---
False
\end{verbatim}

This deadlock may arise in any use of the library, as it depends only
on the timing of the delay, and not on the computation performed.

The more complex race condition arises if \verb|readMVar| isn't
atomic, as in GHC versions before 7.8. In this case an old value can
be returned if the read of \verb|lastValue| is pre-empted between the
internal take and put operations, as shown in this test:

\begin{minted}{haskell}
test :: (MonadConc m, MonadIO m) => m Int
test = do
  var  <- newCRef 0
  auto <- mkAutoUpdate $ defaultUpdateSettings
    { updateAction = modifyCRef var (\x -> (x+1, x)) }

  auto
  auto
\end{minted}
%$

Here \verb|auto| is called twice to update the counter variable
twice. Actually reproducing this bug requires a new \verb|readCVar|
function be written, as the library does not currently provide an
option for non-atomic reads. Exhibiting this bug requires three
pre-emptions:

\begin{verbatim}
> dejafusIO' TotalStoreOrder 3 5 test [("Consistent Result", alwaysSame)]
[fail] Consistent Result (checked: 23)
        [deadlock] S0------P1-S0---S1-----------S0-
        0 S0---------S1--------P0-----
        1 S0---------S1---------P0---P1--------P0---
\end{verbatim}

Despite the bugs being rather simple, one not requiring any
pre-emptions at all to trigger, they both arose in practice. How easy
it is to make mistakes when implementing concurrent programs!
