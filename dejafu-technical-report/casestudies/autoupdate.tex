The \emph{auto-update} library runs tasks periodically, but only if
needed. For example, a single worker thread may get the time every
second and store it to a shared \verb|IORef|, rather than have many
threads starting within a second of each other all get the time
independently \citep{autoupdate}. Despite the core functionality being
very simple, two race conditions were noticed by users inspecting the
code in October 2014.

\begin{figure}[h!]
  \captionsetup{format=fnoline}
  \centering
  \begin{cminted}{haskell}
data UpdateSettings a = UpdateSettings
    { updateFreq           :: Int
    , updateSpawnThreshold :: Int
    , updateAction         :: IO a
    }

defaultUpdateSettings :: UpdateSettings ()
defaultUpdateSettings = UpdateSettings
    { updateFreq           = 1000000
    , updateSpawnThreshold = 3
    , updateAction         = return ()
    }

mkAutoUpdate :: UpdateSettings a -> IO (IO a)
mkAutoUpdate us = do
    currRef      <- newIORef Nothing
    needsRunning <- newEmptyMVar
    lastValue    <- newEmptyMVar

    void $ forkIO $ forever $ do
        takeMVar needsRunning

        a <- catchSome $ updateAction us

        writeIORef currRef $ Just a
        void $ tryTakeMVar lastValue
        putMVar lastValue a

        threadDelay $ updateFreq us

        writeIORef currRef Nothing
        void $ takeMVar lastValue

    return $ do
        mval <- readIORef currRef
        case mval of
            Just val -> return val
            Nothing -> do
                void $ tryPutMVar needsRunning ()
                readMVar lastValue

catchSome :: IO a -> IO a
catchSome act = catch act $
  \e -> return $ throw (e :: SomeException)
  \end{cminted}
  \caption{\emph{auto-update} implementation}
  \label{fig:example-autoupdate}
\end{figure}


The entire implementation, excluding comments and imports, is
reproduced in Figure \ref{fig:example-autoupdate}. The
\verb|mkAutoUpdate| function spawns a worker thread, which performs
the update action at the given frequency, only if the
\verb|needsRunning| flag has been set. It returns an action to attempt
to read the current result, if necessary blocking until there is
one. The transformation to the \verb|MonadConc| typeclass is mostly
simple, however the \verb|threadDelay| must be wrapped inside a call
to \verb|liftIO|.

The simpler race condition occurs if the reading thread is pre-empted
by the worker thread after putting into \verb|needsRunning|, and does
not run again until after the delay has passed. In this case the
worker thread can become blocked on taking for a second time from
\verb|needsRunning|. The reader thread will be unable to read from
\verb|lastValue| as the worker thread emptied it as the last action it
performed. The race condition can be exhibited with the following
test:

\begin{minted}{haskell}
test :: (MonadConc m, MonadIO m) => m ()
test = join (mkAutoUpdate defaultUpdateSettings)
\end{minted}

This test was chosen as it is one of the simplest things a user may
wish to do with the library: to create the worker, and to then read
the computed value. The output of testing shows the different results
that were found, with a sample trace leading to each one:

\begin{verbatim}
> autocheckIO test
[fail] Never Deadlocks (checked: 1)
        [deadlock] S0--------S1-----------S0-
[pass] No Exceptions (checked: 9)
[fail] Consistent Result (checked: 8)
        [deadlock] S0-----P1-S0---S1-----------S0-
        () S0--------S1---------P0---
False
\end{verbatim}

The \verb|autocheckIO| function is used to search for deadlocks,
uncaught exceptions in the main thread, and nondeterminism in
results. It allows the use of \verb|liftIO|, there is an
\verb|autocheck| function which does not.

``S$n$'' indicates that thread $n$ began executing after the prior one
blocked, ``P$n$'' indicates that thread $n$ pre-empted the prior
one. It is also possible to obtain a richer data structure with a
clearer account of what happened.

To read this trace, it is helpful to look at the ``S'' points, rather
than to count the steps. Using this tactic, we can see that a deadlock
occurs if the initial thread runs until it blocks (which is the
\verb|readMVar lastValue| call), then thread 1 runs until it blocks
(the second time \verb|takeMVar needsRunning| is reached). At this
point the initial thread is runnable, as it was unblocked by the write
to \verb|lastValue|. This is scheduled, and immediately blocks because
thread 1 took the value. Both threads are now blocked, and so the
computation is deadlocked.

This deadlock may arise in any use of the library, as it depends only
on the timing of the delay, and not on the computation performed. If
the call to \verb|threadDelay| completes before the reading thread has
resumed execution, this situation will arise.

The more complex race condition arises if \verb|readMVar| isn't
atomic, as in GHC versions before 7.8. The \verb|readMVar| function
used to be a combination of a take and a put. In this case an old
value can be returned if the read of \verb|lastValue| is pre-empted
between these two operations, as shown in this test:

\begin{minted}{haskell}
test :: (MonadConc m, MonadIO m) => m Int
test = do
  var  <- newCRef 0
  let action = modifyCRef var $ \x -> (x+1, x)
  auto <- mkAutoUpdate $ defaultUpdateSettings { updateAction = action }
  auto
  auto
\end{minted}

Here \verb|auto| is called twice to update the counter variable
twice. Actually reproducing this bug requires a new \verb|readCVar|
function be written, which has the same behaviour as the old
\verb|readMVar| function. Exhibiting this bug requires three
pre-emptions. As we need to supply our own bounds, we cannot use
\verb|autocheckIO|. The more \verb|dejafuIO'| function allows for the
memory model and bounds to be specified:

\begin{verbatim}
> let bounds = defaultBounds { preEmptionBound = Just 3 }
> dejafuIO' TotalStoreOrder bounds test ("Consistent Result", alwaysSame)
[fail] Consistent Result (checked: 23)
        [deadlock] S0------P1-S0---S1-----------S0-
        0 S0---------S1--------P0-----
        1 S0---------S1---------P0---P1--------P0---
\end{verbatim}

Here we see two different (non-deadlocking) results. The issue is the
\verb|readMVar| in the main thread and the \verb|tryTakeMVar| followed
by \verb|putMVar| in the other. If the main thread takes from the
\verb|MVar| before the \verb|tryTake|, it will retrieve the prior
value. This in itself is not a problem. However, if the main thread
then puts that value back between the \verb|tryTake| and the
\verb|put|, the \verb|put| will block, and not be able to store the
updated value.

Despite the bugs being rather simple, one not requiring any
pre-emptions at all to trigger, they both arose in practice. How easy
it is to make mistakes when implementing concurrent programs!
