The \verb|Par| monad \citep{parmonad} is a library providing a
traditional-looking concurrency abstraction, providing the programmer
with threads and mutable state, however it maintains determinism by
restricting its shared variables to one write, and operations to read
block until a value has been written. Thus, \verb|Par|'s \verb|IVar|s
are \emph{futures}, not \emph{mutable} state. \verb|Par| uses a
work-stealing scheduler running on multiple operating system threads,
fully evaluating values on their own threads before inserting them
into an \verb|IVar|. Despite its limitations, the \verb|Par| monad can
be very effective in speeding up pure code.

The following example maps a function in parallel over a list, fully
evaluating it. Of course, laziness is generally what is desired in
Haskell programs, but often it is known that an entire result will
definitely be needed:

\begin{minted}{haskell}
parMap :: NFData b => (a -> b) -> [a] -> [b]
parMap f as = runPar $ do
  bs <- mapM (spawnP . f) as
  mapM get bs
\end{minted}
%$

However, with a lack of multi-write shared variables and non-blocking
reads, \verb|Par| is unsuitable for long-lived concurrent programs
with a central shared state. It could not be used to implement a
multi-threaded work-stealing scheduler, such as the one underpinning
\verb|Par| itself. The library provides a number of different
schedulers, the default being the ``trace'' scheduler. Due to reports
of potential deadlocks with the ``direct'' scheduler from a year ago,
it was tested with \dejafu{}.

% removed: \citep{parreddit}

To reduce the effort in modifying the code, only the direct
dependencies of the ``direct'' scheduler were modified, the rest of
the library being left unchanged. This resulted in four files needing
change: two from the
\emph{abstract-deque}\footnote{\hackage{abstract-deque}} package and
two from the \emph{monad-par}\footnote{\hackage{monad-par}} package.

Converting \emph{monad-par} to use \dejafu{} was quite simple. All
relevant types were parametrised by the underlying monad, all
functions had a \verb|MonadConc| context added, functions were swapped
for their \dejafu{} alternatives, and a \verb|runPar'| function was
added:

\begin{minted}{haskell}
runPar' :: MonadConc m => Par m a -> m a
\end{minted}

Some simplifications were made in the conversion process:

\begin{itemize}
\item \verb|Par| normally uses the
  \emph{mwc-random}\footnote{\hackage{mwc-random}} package when
  performing its internal scheduling. This was replaced with a
  \verb|StdGen|, as \emph{mwc-random} does not allow the user to
  provide a seed.

\item Behaviour of the \verb|Par| scheduler can be configured using
  cpp, but only the default configuration was tested.
\end{itemize}

Figure \ref{fig:example-parmonad-sched} shows the original and
converted scheduler initialisation code. As can be seen, they are very
similar, even though this is a core component of a rather
sophisticated library, where the types have been changed.

In total, there were 127 changed lines across the two modules of
interest, out of 647 total. 47 of these were type signatures, 10 were
type and data declarations, and 4 were typeclass instance
declarations. Of the 66 remaining lines, over half were renaming
\verb|*MVar| and \verb|*IORef| functions to their \verb|*CVar| and
\verb|*CRef| equivalents. The remaining lines were a combination of
import declarations, calls to \verb|getNumCapabilities| (\dejafu{}
does not provide the \verb|numCapabilities| binding, which was used in
several places), and replacing \verb|liftIO| with \verb|lift| from the
\emph{mtl}\footnote{\hackage{mtl}} package.

The refactoring was driven entirely by resolving type errors, after
first parameterising the basic types on the underlying monad, rather
than assuming \verb|IO|. The more general types could even be hidden
from the user by providing type synonyms to supply \verb|IO| as the
monad, making the new functionality purely an internal concern, but
one which allows testing.

Converting the \emph{abstract-deque} package proved a little more
challenging, as the typeclass interface requires knowledge of both the
queue type and the monad results are produced in. This issue was
solved by use of type families:

\begin{minted}{haskell}
class MonadConc (MConc d) => DequeClass d where
  type MConc d :: * -> *

  newQ :: MConc d (d elt)
  ...
\end{minted}

This solution is not ideal as it adds explicit knowledge of
\verb|MonadConc| to the \verb|DequeClass| typeclass, but it suffices
for testing purposes.

Only 47 lines out of 342 across three modules needed change. 20 of
these were in type signatures, 8 in type and data declarations, and 6
in typeclass instances. The remaining lines were, again, renaming
\verb|*IORef| functions to \verb|*CRef| equivalents, import
statements, and the addition of a type family as displayed in the code
sample above.

With the constant value `PRNG', a deadlock was discovered. It only
arises after 200 queries. Given that the range of values is from 0 to
the number of capabilities, and the probability is uniformly
distributed, the probability of an actual deadlock is about $4 \times
10^{-121}$ on a quad-core computer. No deadlocks were discovered when
using the \verb|StdGen| generator, with a variety of initial seeds
tried. If there is still a deadlock, it may require more than 2
capabilities to manifest.

\begin{landscape}
\begin{figure*}[t]
  \captionsetup{format=fnoline}
  \centering
  \begin{minipage}[t]{0.49\linewidth}
    \begin{minted}{haskell}
makeScheds :: Int -> IO [Sched]
makeScheds main = do
  caps <- getNumCapabilities
  workpools <- replicateM caps R.newQ
  rngs <- replicateM caps
            (Random.create >>= newHotVar)
  idle <- newHotVar []

  sessionFinished <- newHotVar False
  let sess = [Session baseSessionID sessionFinished]
  sessionStacks <- mapM newHotVar
                     (replicate caps sess)
  activeSessions <- newHotVar S.empty
  sessionCounter <- newHotVar (baseSessionID + 1)
  let allscheds =
       [ Sched { no=x, idle, isMain=(x==main),
                 workpool=wp, scheds=allscheds,
                 rng=rng, sessions=stck,
                 activeSessions=activeSessions,
                 sessionCounter=sessionCounter
               }
         | x    <- [0 .. caps-1]
         | wp   <- workpools
         | rng  <- rngs
         | stck <- sessionStacks
       ]
  return allscheds
    \end{minted}
    \caption*{Original}
  \end{minipage}
  \begin{minipage}[t]{0.49\linewidth}
    \begin{minted}{haskell}
makeScheds :: MonadConc m => Int -> m [Sched m]
makeScheds main = do
  caps <- getNumCapabilities
  workpools <- replicateM caps R.newQ
  rngs <- replicateM caps
            (newHotVar (mkStdGen 0))
  idle <- newHotVar []

  sessionFinished <- newHotVar False
  let sess = [Session baseSessionID sessionFinished]
  sessionStacks <- mapM newHotVar
                     (replicate caps sess)
  activeSessions <- newHotVar S.empty
  sessionCounter <- newHotVar (baseSessionID + 1)
  let allscheds =
       [ Sched { no=x, idle, isMain=(x==main),
                 workpool=wp, scheds=allscheds,
                 rng=rng, sessions=stck,
                 activeSessions=activeSessions,
                 sessionCounter=sessionCounter
               }
         | x    <- [0 .. caps-1]
         | wp   <- workpools
         | rng  <- rngs
         | stck <- sessionStacks
       ]
  return allscheds
    \end{minted}
    \caption*{\dejafu{}}
  \end{minipage}
  \caption{Par ``direct'' scheduler initialisation}
  \label{fig:example-parmonad-sched}
\end{figure*}
\end{landscape}


Due to the very contrived nature of the one deadlock found, and the
length of the trace, it is not displayed here. If there is anything
that can be taken away from this experiment with the Par monad, it is
that it is very reliable.
