There are two popular libraries for unit testing in Haskell,
\emph{HUnit}\footnote{\hackage{HUnit}} and
\emph{tasty}\footnote{\hackage{tasty}}. From the perspective of the
user, both libraries are very similar, but from the perspective of the
implementer, they have different approaches to integration. Packages
providing integration with both, called \emph{hunit-dejafu} and
\emph{tasty-dejafu} are provided.

Both packages provide a common set of testing functions, an analogue
of Test.\-DejaFu but constructing values representing individual tests
which the frameworks can run, rather than executing and printing
results directly:

\begin{haskellcode}
testAuto    :: (Eq a, Show a) => (forall t. ConcST t a) -> Test
testDejafu  :: Show a => (forall t. ConcST t a) -> String -> Predicate a -> Test
testDejafus :: Show a => (forall t. ConcST t a) -> [(String, Predicate a)] -> Test
\end{haskellcode}

Here \verb|Test| is the type of individual tests, from
\emph{HUnit}. \emph{tasty} uses \verb|TestTree|, which has a similar
purpose; it also uses \verb|TestName| rather than \verb|String|. To
complete the set, variants of these functions for \verb|ConcIO|, and
also taking the schedule bounds and memory type as parameters, are
provided. All of the testing functions are implemented in terms of
\verb|testDejafus'| and \verb|testDejafusIO'|.

The \emph{test-framework}\footnote{\hackage{test-framework}} library
is also in common use, however it supports integration with
\emph{HUnit}, and so needs no special support.

\subsection{HUnit}
\label{sec:practice-integration-hunit}

Tests in \emph{HUnit} are just a thin wrapper around an \verb|IO ()|
action, which can be grouped together into collections and given
names. The testing model is very simple: a test fails if and only if
it produces some output. There are a number of provided testing
functions, which throw an exception if they fail, terminating the rest
of the test case.

\begin{haskellcode}
test :: Show a => MemType -> Bounds -> (forall t. ConcST t a)
  -> [(String, Predicate a)] -> Test
test memtype cb conc tests = case map toTest tests of
  [t] -> t
  ts  -> TestList ts

  where
    toTest (name, p) = TestLabel name . TestCase . assertString . showErr $ p traces

    traces = sctBound memtype cb conc
\end{haskellcode}
%$

Here, each \verb|(String, Predicate a)| pair is turned into a separate
test case. If there is only one, it is returned directly, otherwise
they are grouped together into a \verb|TestList|. A \verb|TestList|
\emph{can} consist of only one entry, but making this distinction
results in a closer correspondance between the generated \verb|Test|
and the call to the testing function which produced it.

The \verb|assertString| function is provided by \emph{HUnit}. If the
provided string is non-empty (\verb|showErr| here is a function to
pretty-print the failures, if any) the test fails.

\subsection{tasty}
\label{sec:practice-integration-tasty}

In contrast to the simple function-based method of \emph{HUnit},
\emph{tasty} has a much more complex approach based on a typeclass of
things which can be converted to a unit test:

\begin{haskellcode}
test :: Show a => MemType -> Bounds -> (forall t. ConcST t a)
  -> [(TestName, Predicate a)] -> TestTree
test memtype cb conc tests = case map toTest tests of
  [t] -> t
  ts  -> testGroup "Deja Fu Tests" ts

  where
    toTest (name, p) = singleTest name $ ConcTest traces p

    traces = sctBound memtype cb conc
\end{haskellcode}
%$

This is very similar to the \emph{HUnit} approach, however instead of
constructing a test value directly, it constructs an intermediate
\verb|ConcTest| value. Note also that \emph{tasty} does not allow
nameless test lists. The \verb|singleTest| function takes a value
which is a member of the \verb|IsTest| typeclass, and uses that to
construct a \verb|TestTree|:

\begin{haskellcode}
data ConcTest where
  ConcTest :: Show a => [(Either Failure a, Trace)] -> Predicate a -> ConcTest
  deriving Typeable

instance IsTest ConcTest where
  testOptions = return []

  run _ (ConcTest traces p) _ =
    let err = showErr $ p traces
     in return $ if null err then testPassed "" else testFailed err
\end{haskellcode}

\emph{tasty} allows for passing tests to also have output associated
with them, through the \verb|testPassed| function, which is only
displayed if the tests are executed with sufficient
verbosity. Furthermore, tests can have options associated with them,
which can be set when the test is executed. Neither of these features
are used, as this was largely just a port of
\emph{hunit-dejafu}. \emph{tasty} is definitely a more featureful
library than \emph{HUnit}, but this comes at the cost of additional
complexity for developers trying to integrate new functionality.
