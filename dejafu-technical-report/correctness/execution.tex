Correctness of execution asks whether the result of an arbitrary
execution of \dejafu{}'s testing implementation happen in reality.
Furthermore, do all real-world executions correspond to a possible
execution under \dejafu{}? To put it more simply:

\begin{itemize}
\item Is the behaviour of the primitive functions the same?

\item Is the granularity of scheduling decisions the same?
\end{itemize}

Both of these come with the caveat that the behaviour can be
different, as long as this difference can't be observed.

\subsubsection{Primitives}
\label{sec:correctness-execution-primops}

The method of implementing the members of the \verb|MonadConc|
typeclass that would be most amenable to proof would be to implement
analogues of the GHC primitives directly, and implement everything
else in terms of these. This matches how actual Haskell is
implemented, and would lend itself to establishing a formal
correspondence between the \dejafu{} primitives and the GHC
primitives, and the higher-level \verb|MonadConc| functions and the
higher-level functions in Control.\-Concurrent.

This approach was not taken, however. Firstly, it ties the
implementation and correctness of \dejafu{} very closely to the
implementation of GHC; in principle the implementation of GHC's
concurrency primitives could be completely changed but the behaviour
preserved. Secondly, this restricts \dejafu{} to a very specific type
of concurrency, supporting low-level operations, which may not map to
all interesting implementations of concurrency.

Instead, a reimplementation of GHC's concurrency based on the
documented and observable behaviour of the various functions was
done. This allows observing the behaviour of a program and
determining, intuitively, whether it is correct or not; but it's not
so good for proof. The matter is complicated by there being no
standard for concurrent Haskell, there is only what GHC provides.

The correctness of operations using \verb|CRef|s is complicated even
further, as the behaviour of these depends on the underlying memory
model. Total store order was chosen to be the default, as it is what
x86 processors do with unsynchronised memory accesses, but a
\verb|CRef| is more complicated than a simple memory cell: it is a
pointer to a cell, which can be moved around in garbage collection,
and it is accessed through primitive operations more complicated than
simple loads and stores. The lack of a standard, or even comprehensive
documentation, means that in order to formally establish the memory
model for \verb|CRef|s, the compilation of \verb|IORef| functions must
be traced through GHC from Haskell source to machine code. As GHC uses
C{-}{-} as an intermediary language, this may also require determining
a memory model for C{-}{-}.

Finally, there are some intended departures from the behaviour of
GHC's behaviour documented in \sect{abstraction}{typeclass}:

\begin{itemize}
\item \verb|getNumCapabilities| is not required to return a true
  result.

\item Deadlock detection can only function to the same extent as GHC
  if every thread is annotated with which \verb|CVar|s and
  \verb|CTVar|s it knows about, as \dejafu{} cannot use the garbage
  collector for this task.

\item \verb|catch| is not required to be able to catch exceptions from
  pure code.
\end{itemize}

\subsubsection{Scheduling}
\label{sec:correctness-execution-scheduling}

The stepwise implementation allows for a scheduling decision to be
made between each primitive action, which doesn't quite correspond to
how GHC handles scheduling:

\quot{GHC implements pre-emptive multitasking: the execution of
  threads are interleaved in a random fashion. More specifically, a
  thread may be pre-empted whenever it allocates some memory, which
  unfortunately means that tight loops which do no allocation tend to
  lock out other threads (this only seems to happen with pathological
  benchmark-style code, however).}{controlConcurrent}

That is, GHC allows for pre-emptions to occur whilst evaluating pure
code, which the stepwise executor does not. There are executions
involving the pre-emption of the evaluation of non-terminating
expressions which are possible under GHC but not under \dejafu{}, but
whether this can be used to produce different outputs is less clear.
