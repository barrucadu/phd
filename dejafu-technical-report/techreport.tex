\documentclass[openright, dottedtoc, headinclude, footinclude=true, a4paper, numbers=noenddot, fontsize=10pt]{scrreprt}

\title{D\'{e}j\`{a} Fu Technical Report}
\author{Michael Walker}

%TC:group minted      [ignore] xall
%TC:group haskellcode [ignore] xall

% Stuff cargo-culted from jmct's thesis
\usepackage[eulermath, pdfspacing, nochapters]{classicthesis}

\usepackage{setspace}
\onehalfspacing

\usepackage{graphicx}
\usepackage{caption}
\usepackage{mathtools}
\usepackage{fancyvrb}
\usepackage{hyperref}
\usepackage{subcaption}
\usepackage{alltt}
\usepackage{color}

\usepackage{microtype}
\usepackage{xargs}
\usepackage{amsmath}
\usepackage{amssymb}
\usepackage{latexsym}
\usepackage{textcomp}
\usepackage{multicol}
\usepackage{multirow}
\usepackage{url}
\usepackage{siunitx}
\usepackage{appendix}
\usepackage{nomencl}
\usepackage[top=3cm,bottom=3cm]{geometry}

\usepackage{tikz}

\usepackage{csquotes}

\usepackage{minted}
\usemintedstyle{trac}
\newminted{haskell}{}

\usepackage{haskell}

\usepackage[vlined, boxed]{algorithm2e}

\hypersetup{
 colorlinks,
 citecolor=Red,
 linkcolor=Black,
 urlcolor=Blue
}

\usepackage[square]{natbib}

% Definitions

\newcommand{\defineword}[2]{%
\begin{description}%
    \item{\textbf{#1}} \hfill \\%
        {#2}%
\end{description}%
}

\newcommand{\definewordc}[3]{%
\begin{description}%
    \item{\textbf{#1}} \citep{#2}\hfill \\%
        {#3}%
\end{description}%
}

% Front-matter

\usepackage{tocloft}
\setcounter{tocdepth}{1}

\DeclareCaptionFormat{myformat}{#1#2#3\hrulefill}
\DeclareCaptionFormat{fnoline}{#1#2#3}
\captionsetup[figure]{format=myformat}

\newcommand{\frontmatter}{\cleardoublepage\pagenumbering{roman}}

\newcommand{\mainmatter}{\cleardoublepage\pagenumbering{arabic}}

\renewcommand*\thesection{\arabic{section}}
\setcounter{tocdepth}{2}

% Chapter quotes

\makeatletter
\newenvironment{chapquote}[2][2em]
  {\setlength{\@tempdima}{#1}%
   \def\chapquote@author{#2}%
   \parshape 1 \@tempdima \dimexpr\textwidth-2\@tempdima\relax%
   \itshape}
  {\par\normalfont\hfill--\
    \chapquote@author\hspace*{\@tempdima}\par\noindent\hrulefill\\[1cm]}
\makeatother

% Todo notes
\usepackage[colorinlistoftodos,prependcaption,textsize=small]{todonotes}
\let\marginpar\oldmarginpar

\newcommandx{\tocite}[2][1=]{\todo[linecolor=red,backgroundcolor=red!25,bordercolor=red,#1]{Cite: #2}}
\newcommandx{\todoinline}[1]{\todo[inline]{#1}}
\newcommandx{\todofig}[1]{\todo[inline]{Make figure: #1}}

% Departures in \chap{abstraction}:
\usepackage{framed}
\usepackage{amsthm}

\newenvironment{justspacing}{%
\def\FrameCommand{\hspace{0pt}}%
\MakeFramed {\advance\hsize-\width \FrameRestore}}%
{\endMakeFramed}

\newenvironment{departure}%
{\begin{justspacing}%
\noindent%
\paragraph{Departure}}%
{\hfill $\qed$ \end{justspacing}}

% Landscape pages
\usepackage{pdflscape}

% Misc
\newcommand{\quot}[2]{``\textit{#1}'' \cite{#2}}
\newcommand{\sect}[2]{\S\ref{sec:#1-#2}}
\newcommand{\chap}[1]{Section \ref{chap:#1}}
\newcommand{\dejafu}{D\'{e}j\`{a} Fu}
\newcommand{\defineprim}[3]{\defineword{#1 \textit{#2}}{#3}}
\newcommand{\dependent}{\nleftrightarrow}
\newcommand{\arr}{$\rightarrow$ }
\newcommand{\hackage}[1]{\url{https://hackage.haskell.org/package/#1}}
\newcommand{\github}[2]{\url{https://github.com/#1/#2}}
\newcommand{\dom}[1]{\mathbf{dom}~#1}
\newcommand{\ran}[1]{\mathbf{ran}~#1}
\newcommand{\wellf}[1]{\mathbf{well\_formed}~#1}

\begin{document}

\frontmatter

\begin{titlepage}
\begin{center}

\textsc{\Huge \dejafu{}: A Concurrency Testing Library for Haskell}\\[0.25cm]

\large{\github{barrucadu}{dejafu}}\\[0.5cm]

\textsc{\Large Michael Walker}
\end{center}

\vspace{3.5cm}

\centerline{\rule{40pt}{1pt} \textsc{\large Abstract} \rule{40pt}{1pt}}
\vspace{0.1cm}

Out of work in the formal verification and model checking field has
grown the topic of \emph{systematic concurrency testing} (SCT), also
known as \emph{stateless model checking}. This technique allows the
reliable, deterministic, and rigorous testing of concurrent programs,
and has enjoyed some success in the imperative and object-oriented
settings. We propose that the functional world could also benefit from
systematic concurrency testing, as GHC Haskell in particular provides
a very rich set of concurrency primitives.

We have developed a library for writing testable concurrent Haskell
programs, using a typeclass-abstraction to select based on the context
of use the concrete implementation to use: the primitives provided by
the run-time system, or emulated versions provided as part of a
testing framework.

This report discusses the design and implementation of this library,
called \dejafu{}, including some case studies and the community
reception of the initial version presented at the 2015 Haskell
Symposium.

\vspace{0.1cm}
\centerline{\rule{130pt}{1pt}}

\vfill

\begin{center}
Department of Computer Science\par
University of York \\[1,3cm]

November 2015
\end{center}
\end{titlepage}

\tableofcontents

%\listoffigures
%\listoftables

\mainmatter

\section{Introduction}
\label{chap:intro}
\begin{chapquote}{Terry Pratchett, Thief of Time\nocite{pratchett2001}}
  [\dejafu{} is] A martial art in which the user's limbs move in time
  as well as space, [\ldots] It is best described as ``the feeling
  that you have been kicked in the head this way before''.
\end{chapquote}

This chapter describes the test execution of a concurrent program
written using the \verb|MonadConc| abstraction, not the execution of
programs using GHC's actual concurrency primitives. Of course, for
correctness of testing, there should be a correspondence between these
two models, see \chap{correctness}.

The execution of a concurrent program is considered to be the
sequential stepwise execution of \emph{primitive actions}, the most
basic things that a computation can do.


  \subsection{Parallelism vs Concurrency}
  \label{sec:intro-parconc}
  It is worth clarifying at this early stage some terminology which will
be frequently used throughout this report:

\defineword{Concurrency}{A programming methodology, using concepts
  such as threads, locks, and mutable variables to structure
  programs.}

\defineword{Parallelism}{An implementation detail, where a
  multiplicity of hardware components are used to execute distinct
  pieces of code simultaneously.}

Concurrency does not require parallelism, as demonstrated by the
single-core, single-processor computers of yore. Similarly,
parallelism does not require concurrency, as demonstrated by the
data-parallel x86 assembly instructions such as \verb|PMULHUW|, which
computes an element-wise multiplication of two vectors, each
multiplication in parallel.

Unrestricted concurrency is explicit and \emph{semantically visible}
\citep{concurrent}. The interleaved execution of threads, when
combined with mutable state, gives rise to nondeterminism. Semaphores
and locks give rise to termination errors in the form of deadlock and
livelock. Parallelism, in particular the parallel evaluation of
expressions, is \emph{semantically invisible} in a language without
side-effects.

Concurrency is often implemented using parallelism, and indeed a
concurrency abstraction can be used to guarantee parallelism (given
suitable hardware), for example by having the ability to restrict the
execution of individual threads to given processor cores.

Parallelism is largely outside the scope of this report, although it
does make an appearance in the discussion of relaxed memory in
\chap{abstraction}.


  \subsection{Testing Concurrent Programs}
  \label{sec:intro-testing}
  Correctness of testing asks whether the schedule prefixes generated by
the partial-order reduction mechanism are valid, and are there any
results possible in real-world execution for which no schedule will be
generated? This is definitely \emph{not} the same as asking if every
real-world schedule will be generated by the testing framework, as
that is precisely what partial-order reduction tries to avoid.

\subsection{Prefix Validity}
\label{sec:correctness-testing-validity}

Multiple executions with different schedules are stored internally as
a tree, with each path from the root to a leaf corresponding to a
complete execution:

\begin{haskellcode}
data BPOR = BPOR
  { runnable :: Set ThreadId
  , todo     :: Set Threadid
  , done     :: Map ThreadId BPOR
  , action   :: Maybe ThreadAction
  }
\end{haskellcode}

The \verb|runnable| field is the set of all threads runnable at that
point; \verb|todo| is the decisions still to try making; \verb|done|
is the decisions already made; and \verb|action| is what was done at
this step. \verb|action| is a \verb|Maybe| value, because initially no
action has been performed, as the computation hasn't yet started.

There is a unique initial state, where only the initial thread is
runnable and nothing has been done:

\begin{haskellcode}
initialState :: BPOR
initialState = BPOR (singleton 0) (singleton 0) empty Nothing
\end{haskellcode}

There are some basic well-formedness invariants associated with a
\verb|BPOR| value:

\begin{itemize}
\item Every decision in the to-do set is possible:
  $\mathrm{todo} \subseteq \mathrm{runnable}$

\item Every decision in the done map is possible:
  $\dom \mathrm{done} \subseteq \mathrm{runnable}$

\item No decision that has been done is in the to-do set:
  $\mathrm{todo} \cap \dom \mathrm{done} = \varnothing$

\item These properties hold recursively:
  $\forall p \in \ran \mathrm{done}.~\wellf{p}$
\end{itemize}

Some work has been started in the Isabelle/HOL theorem prover to
formalise part of the recursive loop in \verb|sctBounded| (see
\sect{sct}{por}) and to prove that the invariants are preserved,
assuming that the stepwise executor is correct. It is hoped that
schedule prefix validity will follow from this. Specifically, the
things to be proved are:

\begin{itemize}
\item A prefix produced by \verb|next| is valid if the \verb|BPOR|
  tree is well-formed; furthermore, it consists of a sequence of
  decisions that have already been made and is terminated by a single
  decision from a to-do set.

\item \verb|next| returns \verb|Nothing| if and only if the
  \verb|todo| field of every node in the \verb|BPOR| tree is empty.

\item \verb|grow| adds the information in a trace to the \verb|BPOR|
  tree, making no other changes.
\end{itemize}

Data structure invariants are an important property to verify, as if
they are broken any assumptions made in the rest of the code cannot be
trusted.

\subsection{Result Completeness}
\label{sec:correctness-testing-complete}

The simplest notion of completeness of interest here is that for all
results possible by executing a given program for real, the
partial-order reduction framework can give that result.

However, as schedule bounding is involved, this is clearly not the
case. Therefore, there are two different notions of completeness which
are of interest:

\begin{itemize}
\item If the bounds are all set to $\infty$, all results possible
  under real execution show up under \dejafu{} execution.

\item For all sets of bounds, all results possible under real
  execution subject to those bounds show up under \dejafu{} execution
  with the same bounds.
\end{itemize}

The former corresponds to the correctness of partial-order reduction
with no bounds, and does not imply the latter. The latter implies the
former, and is the more interesting property. The latter is what we
really want of our testing framework.

The proof would need to proceed by first showing that the dependency
relation is correct: that only actions related by the dependency
relation can influence each others results. It's not clear how to
approach this, as it relies on the implementation of the actions. Once
the correctness of the dependency relation is established, it must be
shown that the partial-order reduction only prunes schedules where
there is no dependency.


  \subsection{Scope}
  \label{sec:intro-scope}
  We aim to support all of the functionality of GHC's concurrency API
which does not unavoidably \emph{require} support from the runtime,
and which is not so nondeterministic that there is no sensible way to
model it accurately.

Some specific examples of things which are out of scope:

\begin{itemize}
\item \verb|threadDelay|, as all this guarantees is that a thread will
  not run \textit{sooner} than the delay. There is no upper bound on
  the delay, and also no guarantee that any other thread will be
  scheduled during the delay.

\item \verb|threadWaitRead|, \verb|threadWaitWrite|, and the
  \verb|STM| variants, as there is no way to tell if a file descriptor
  can be read from or written to without involving \verb|IO|, and in
  addition this is influenced by other non-Haskell processes accessing
  the same file.

\item Bound threads, as these affect which operating system thread FFI
  calls operate on, and so alters program behaviour in a parallel
  setting.

\item \verb|BlockedIndefinitely| exceptions, as this is a garbage
  collection problem, which is out of the reach of the runtime. There
  are some annotation functions to record which shared state a thread
  knows about, and so this can be supported on a limited scale, but
  not even GHC makes any guarantee of it being reliable.\footnote{
    \quot{Note that this feature is intended for debugging, and should
      not be relied on for the correct operation of your
      program. There is no guarantee that the garbage collector will
      be accurate enough to detect your deadlock, and no guarantee
      that the garbage collector will run in a timely enough
      manner.}{controlConcurrent}}
\end{itemize}


  \subsection{Contributions}
  \label{sec:intro-contribs}
  Our contributions can be seen as follows:

\begin{itemize}
\item Existing results from the concurrency testing world have been
  applied to functional programming.

\item A somewhat novel partial-order reduction algorithm for
  systematic concurrency testing based on a combination of two others:
  bounded partial-order reduction \cite{bpor} and relaxed-memory DPOR
  \cite{rdpor}.

\item The \dejafu{} library for testing concurrent Haskell programs,
  using this algorithm.
\end{itemize}


  \subsection{Report Roadmap}
  \label{sec:intro-roadmap}
  Firstly we explore the problem of testing concurrent programs and how
it can be done. In \textbf{\chap{abstraction}} we discuss our
typeclass abstraction for concurrency and how it relates to GHC's
standard concurrency API in terms of
functionality. \textbf{\chap{execution}} explains how, given a monadic
action polymorphic in the monad (as long as it has an instance of to
our typeclass) we can execute it with a given scheduler, and
\textbf{\chap{sct}} extends this to cover a systematic exploration of
the space of all schedules. \textbf{\chap{correctness}} discusses the
issues of correctness: how do we know if a result reported by
\dejafu{} is actually right?

Then, we move on to the real-world impact of this work, with case
studies of \dejafu{} applied to two instances of pre-existing code,
and one custom library in \textbf{\chap{casestudies}}.
\textbf{\chap{practice}} further discusses the usage of \dejafu{} in
combination with existing code. To conclude,
\textbf{\chap{conclusions}} discusses related work, and summarises the
community reception to the idea and what is still to be done.


\section{Concurrency Abstraction}
\label{chap:abstraction}
This chapter describes the test execution of a concurrent program
written using the \verb|MonadConc| abstraction, not the execution of
programs using GHC's actual concurrency primitives. Of course, for
correctness of testing, there should be a correspondence between these
two models, see \chap{correctness}.

The execution of a concurrent program is considered to be the
sequential stepwise execution of \emph{primitive actions}, the most
basic things that a computation can do.


  \subsection{The \texttt{MonadConc} Typeclass}
  \label{sec:abstraction-typeclass}
  Readers already familiar with GHC's concurrency primitives may find it
enough to skim this section noting the syntactic differences in the
\dejafu{} variant.

\begin{departure}
  The few departures from the semantics of the traditional concurrency
  abstraction are highlighted like this.
\end{departure}

The \verb|MonadConc| typeclass has an instance\footnote{A typeclass
  has \emph{instances}, each type may have one unique instance for a
  typeclass.} for \verb|IO|, and so existing code using only the
functions generalised over can be made suitable for testing quite
simply. Existing code which makes use of more functionality may
require a light dusting of \verb|liftIO|\footnote{The \texttt{IO} type
  allows unrestricted side-effects during execution. It turns out that
  many useful types are just \texttt{IO} with some extra structure
  applied, and the \texttt{liftIO} function (which belongs to a
  typeclass called \texttt{MonadIO}) can be used to `translate' the
  effects into such a type.} where it is safe to do so, which will be
expanded upon in \sect{abstraction}{typeclass-lifting}.

\subsubsection{Threads}
\label{sec:abstraction-typeclass-threads}

Threads let a program do multiple things at once. Every program has at
least one thread, which starts where \verb|main| does and runs until
the program terminates. A thread is the basic unit of concurrency. It
lets us pretend (with parallelism, it might even be true!) that we're
computing multiple things at once.

We can start a new thread with the function:\footnote{This is a
  function named \texttt{fork} with a \emph{type signature}. Type
  signatures may contain typeclass constraints, type variables, type
  constructors (similar to generics in other languages), and concrete
  types. Here \texttt{ThreadId m} is a type constructor applied to the
  type variable \texttt{m}, which is constrained to be a type with an
  instance of \texttt{MonadConc}.}

\begin{haskellcode}
fork :: MonadConc m => m () -> m (ThreadId m)
\end{haskellcode}

This starts evaluating its argument in a separate thread. It also
gives us back a (monad-specific) \verb|ThreadId| value, which we can
use to kill the thread later on, if we want.

A thread can query its own \verb|ThreadId|:

\begin{haskellcode}
myThreadId :: MonadConc m => m (ThreadId m)
\end{haskellcode}

In a real machine, there are of course a number of processors and
cores. It may be that a particular application of concurrency is only
a net gain if every thread is operating on a separate core, so that
threads are not interrupting each other. The GHC runtime refers to the
number of Haskell threads that can run truly simultaneously as the
number of \emph{capabilities}. We can query this value, and fork
threads which are bound to a particular capability:

\begin{haskellcode}
getNumCapabilities :: MonadConc m => m Int
forkOn :: MonadConc m => Int -> m () -> m (ThreadId m)
\end{haskellcode}

% Layout hack - line is too long otherwise
The \verb|forkOn| function interprets the capability number modulo the
value returned by \\\verb|getNumCapabilities|.

\begin{departure}
  \verb|getNumCapabilities| is not required to return a true
  result. The testing instances return ``2'' despite executing
  everything in the same capability, to encourage more
  concurrency. The \verb|IO| instance does return a true result.
\end{departure}

Sometimes we just want the special case of evaluating something in a
separate thread, for which we can use \verb|spawn| (implemented in
terms of \verb|fork|):

\begin{haskellcode}
spawn :: MonadConc m => m a -> m (CVar m a)
\end{haskellcode}

This returns a \verb|CVar| (\emph{Concurrent Variable}), to which we
can apply \verb|readCVar|, blocking until the computation is done and
the value is stored.

Threads are scheduled non-deterministically, where every time the
runtime decides to perform a context switch, one of the runnable
threads will be executed. Sometimes, however, a thread may be runnable
but also waiting for something to happen. The programmer can provide a
clue to the scheduler that another thread should be tried instead:

\begin{haskellcode}
yield :: MonadConc m => m ()
\end{haskellcode}

This gives any other thread the opportunity to execute instead of the
yielding one, but it is not \emph{required} to cause a context switch
except on co-operative multitasking systems.

\subsubsection{Threading and the Foreign Function Interface}
\label{sec:abstraction-typeclass-ffithreads}

In order to accommodate Foreign Function Interface (FFI) calls which
may block, GHC provides a mechanism for \emph{binding} a Haskell
thread to an operating system thread. This allows FFI calls to be
managed by the operating system, unlike normal Haskell threads which
are managed by the runtime and multiplexed onto a smaller number of
operating system threads. This means that blocking FFI calls do not
necessarily block the entire program.

There is no \verb|MonadConc| equivalent of bound threads, as there
would be no way to reliably test this behaviour. Unfortunately, if
bound threads are required, \verb|IO| will have to be used.

A few predicates are provided for compatibility:\footnote{Function
  application in Haskell uses no special syntax, only juxtaposition,
  so \texttt{return False} is applying the value \texttt{False} to the
  function \texttt{return}. \texttt{return} here is used to inject a
  value into a monad, it is unfortunately named and has nothing to do
  with the return keyword in other languages.}

\begin{haskellcode}
rtsSupportsBoundThreads :: Bool
rtsSupportsBoundThreads = False

isCurrentThreadBound :: MonadConc m => m Bool
isCurrentThreadBound = return False
\end{haskellcode}

\subsubsection{Mutable State}
\label{sec:abstraction-typeclass-crefs}

Threading by itself is not really enough. We need to be able to
\emph{communicate} between threads: we've already seen an instance
of this with the \verb|spawn| function.

The simplest type of mutable shared state provided is the \verb|CRef|
(\emph{Concurrent Reference}). \verb|CRef|s are shared variables which
can be written to and read from:

\begin{haskellcode}
newCRef    :: MonadConc m => a -> m (CRef m a)
readCRef   :: MonadConc m => CRef m a -> m a
modifyCRef :: MonadConc m => CRef m a -> (a -> (a, b)) -> m b
writeCRef  :: MonadConc m => CRef m a -> a -> m ()
\end{haskellcode}

The \verb|modifyCRef| function is atomic. The \verb|readCRef| and
\verb|writeCRef| functions are not synchronised: it is possible for
one thread to read from a \verb|CRef| strictly after another thread
has written to it and to observe an old value!  This is expanded more
upon in \sect{abstraction}{mem}. To ensure that every thread sees a
value as soon as it is written there is a synchronised write function:

\begin{haskellcode}
atomicWriteCRef :: MonadConc m => CRef m a -> a -> m ()
\end{haskellcode}

However, synchronisation can slow down execution in a parallel
environment. Note that \verb|modifyCRef| is also synchronised.

\subsubsection{Compare-and-swap (CAS)}
\label{sec:abstraction-typeclass-cas}

As \verb|CRef|s correspond very closely to mutable memory locations,
there is also a compare-and-swap interface available. Compare-and-swap
is a synchronised atomic primitive which is used to update a location
in memory if and only if it has not been changed since some witness
value was produced. This role of this witness value is called a
\verb|Ticket| here:

\begin{haskellcode}
readForCAS :: MonadConc m => CRef m a -> m (Ticket m a)
peekTicket :: MonadConc m => Ticket m a -> m a
\end{haskellcode}

A \verb|Ticket| can be used to check if a \verb|CRef| has been written
to since it was produced, and can also be used to get the value that
was seen then.

\begin{haskellcode}
casCRef :: MonadConc m => CRef m a -> Ticket m a -> a -> m (Bool, Ticket m a)
\end{haskellcode}

The \verb|casCRef| function is synchronised, is strict in the value
written, and will replace the value within a \verb|CRef| if it hasn't
been modified since the \verb|Ticket| was produced. It returns an
indication of success and a \verb|Ticket| to use in future
operations. This operation is often used in the implementation of
lock-free synchronisation primitives.

There is also an equivalent of \verb|modifyCRef| using a
compare-and-swap. This behaves almost the same as the non-CAS version
but may be more performant in some cases, and is strict in the value
being written:

\begin{haskellcode}
modifyCRefCAS :: MonadConc m => CRef m a -> (a -> (a, b)) -> m b
\end{haskellcode}

\subsubsection{Mutual Exclusion}
\label{sec:abstraction-typeclass-cvars}

A \verb|CVar| is a shared variable under \emph{mutual exclusion}. It
has two possible states: \emph{full} or \emph{empty}. Writing to a
full \verb|CVar| blocks until it is empty, and reading or taking from
an empty \verb|CVar| blocks until it is full. There are also
non-blocking functions which return an indication of success:

\begin{haskellcode}
newEmptyCVar :: MonadConc m => m (CVar m a)
putCVar      :: MonadConc m => CVar m a -> a -> m ()
readCVar     :: MonadConc m => CVar m a -> m a
takeCVar     :: MonadConc m => CVar m a -> m a
tryPutCVar   :: MonadConc m => CVar m a -> a -> m Bool
tryTakeCVar  :: MonadConc m => CVar m a -> m (Maybe a)
\end{haskellcode}

Unfortunately, the mutual exclusion behaviour of \verb|CVar|s means
that computations can become \emph{deadlocked}. For example, deadlock
occurs if every thread tries to take from the same \verb|CVar|. The
GHC runtime can detect this in some situations (and will complain if
it does), and so can \dejafu{} in a more informative way.

\begin{departure}
  \dejafu{} can only detect deadlock to the same extent as GHC if
  every thread is annotated with which \verb|CVar|s it knows
  about. This is because GHC uses the garbage collector to solve this
  problem, which is out of the reach of \dejafu{}.
\end{departure}

There are also additional functions provided in the
Control\-.Concurrent\-.CVar and Control\-.Concurrent\-.CVar\-.Strict
modules.

\subsubsection{Exceptions}
\label{sec:abstraction-typeclass-excs}

Exceptions are a way to bail out of a computation early. Whether
they're a good solution to that problem is a question of style, but
they can be used to jump quickly to error handling code when
necessary. The basic functions for dealing with exceptions are:

\begin{haskellcode}
catch :: (Exception e, MonadConc m) => m a -> (e -> m a) -> m a
throw :: (Exception e, MonadConc m) => e -> m a
\end{haskellcode}

Where \verb|throw| causes the computation to jump back to the nearest
enclosing \verb|catch| capable of handling the particular
exception. As exceptions belong to a typeclass, rather than being a
concrete type, different \verb|catch| functions can be nested, to
handle different types of exceptions.

\begin{departure}
  The \verb|IO| \verb|catch| function can catch exceptions from pure
  code. This is not true in general for \verb|MonadConc| instances.
  So some things which work normally may not work in testing, and we
  risk false negatives. This is a small cost, however, as exceptions
  from pure code are things like pattern match failures and evaluating
  \verb|undefined|, which are arguably bugs.
\end{departure}

Exceptions can be used to kill a thread:

\begin{haskellcode}
throwTo :: (Exception e, MonadConc m) => ThreadId m -> e -> m ()
killThread :: MonadConc m => ThreadId m -> m ()
\end{haskellcode}

These functions block until the target thread is in an appropriate
state to receive the exception.

What if we don't want our threads to be subject to destruction in this
way? A thread also has a \emph{masking state}, which can be used to
block exceptions from other threads. There are three masking states:
\emph{unmasked}, in which a thread can have exceptions thrown to it;
\emph{interruptible}, in which a thread can only have exceptions
thrown to it if it is blocked; and \emph{uninterruptible}, in which a
thread cannot have exceptions thrown to it. When a thread is started,
it inherits the masking state of its parent. We can also execute a
subcomputation with a new masking state:\footnote{These functions take
  a \emph{higher-ranked} type. Removing the \texttt{forall} stuff, we
  have \texttt{(m a -> m a) -> m b}, which is a function which takes a
  function as an argument and returns a result. The \texttt{forall} is
  necessary because, without it, the concrete type that the variable
  \texttt{a} is unified with is fixed across \emph{all} usage sites,
  whereas with the \texttt{forall} it can be determined uniquely
  everywhere it is used.}

\begin{haskellcode}
mask :: MonadConc m => ((forall a. m a -> m a) -> m b) -> m b
uninterruptibleMask :: MonadConc m => ((forall a. m a -> m a) -> m b) -> m b
\end{haskellcode}

In both cases, the action evaluated is passed a function to reset the
masking state to the original one. A thread can be forked and given a
function to reset the masking state:

\begin{haskellcode}
forkWithUnmask :: MonadConc m => ((forall a. m a -> m a) -> m ())
  -> m (ThreadId m)
forkOnWithUnmask :: MonadConc m => Int -> ((forall a. m a -> m a) -> m ())
  -> m (ThreadId m)
\end{haskellcode}

We can also fork a thread and call a supplied function when the thread
is about to terminate, which is useful for informing the parent when a
child terminates, for example:

\begin{haskellcode}
forkFinally :: MonadConc m => m a -> (Either SomeException a -> m ())
  -> m (ThreadId m)
\end{haskellcode}

The \verb|SomeException| type is the top of the exception hierarchy,
and so can be used to catch all exceptions.

\subsubsection{Lifting Actions into \texttt{MonadConc}}
\label{sec:abstraction-typeclass-lifting}

If the programmer needs to make use of \verb|IO| actions, rather than
\verb|MonadConc| actions, then this can be achieved by adding a
\verb|MonadIO| context and using \verb|liftIO|. However, this can
easily compromise the results of testing, as the test runner cannot
peek inside \verb|IO| actions (that's why the typeclass exists in the
first place!). Thus, it is only safe if:

\begin{itemize}
\item \emph{The action is atomic and synchronised.}

  Otherwise the test framework will possibly miss schedules which lead
  to a bug.
\item \emph{The action is deterministic} (when executed as part of a
  computation with a deterministic schedule).

  Otherwise the fundamental assumption behind the testing methodology
  is false, and no guarantees about completeness can be made.
\item \emph{The action cannot block on the action of another thread.}

  Otherwise test execution may deadlock.
\end{itemize}


  \subsection{Software Transactional Memory}
  \label{sec:abstraction-stm}
  \verb|CVar|s are nice, until we need more than one, and find they need
to be kept synchronised. As we can only claim \emph{one} \verb|CVar|
atomically, it seems we need to introduce a \verb|CVar| to control
access to \verb|CVar|s! This is unwieldy and prone to bugs.

\emph{Software transactional memory} (STM) is the solution. STM uses
\verb|CTVar|s, or \emph{Concurrent Transactional Variables}, and is
based upon the idea of atomic \emph{transactions}. An STM transaction
consists of one or more operations over a collection of \verb|CTVar|s,
where a transaction may be aborted part-way through depending on their
values. If the transaction fails, \emph{none of its effects take
  place}, and the thread blocks until the transaction can
succeed. This means we need to limit the possible actions in an STM
transaction to those which can be safely undone and repeated, so we
have another typeclass, \verb|MonadSTM|.

\verb|CTVar|s always contain a value, as shown in the types of the
functions:

\begin{haskellcode}
newCTVar   :: MonadSTM s => a -> s (CTVar s a)
readCTVar  :: MonadSTM s => CTVar s a -> s a
writeCTVar :: MonadSTM s => CTVar s a -> a -> s ()
\end{haskellcode}

If we read a \verb|CTVar| and don't like the value it has, the
transaction can be aborted, and the thread will block until any of the
referenced \verb|CTVar|s have been mutated:

\begin{haskellcode}
retry :: MonadSTM s => s a
check :: MonadSTM s => Bool -> s ()
\end{haskellcode}

We can also try executing a transaction, and do something else if it
fails:

\begin{haskellcode}
orElse :: MonadSTM s => s a -> s a -> s a
\end{haskellcode}

The nice thing about STM transactions is that they \emph{compose}. We
can take small transactions and build bigger transactions from them,
and the whole is still executed atomically. This means we can do
complex state operations involving multiple shared variables without
worrying!

Each \verb|MonadConc| has an associated \verb|MonadSTM|, and can
execute transactions of it atomically:\footnote{Here \texttt{STMLike}
  is a \emph{type family}, it is used to relate the \texttt{MonadConc}
  and \texttt{MonadSTM} typeclasses.}

\begin{haskellcode}
atomically :: MonadConc m => STMLike m a -> m a
\end{haskellcode}

The instance of \verb|MonadConc| for \verb|IO| uses \verb|STM| as its
\verb|MonadSTM|.

STM can also use exceptions:

\begin{haskellcode}
throwSTM :: (Exception e, MonadSTM s) => e -> s a
catchSTM :: (Exception e, MonadSTM s) => s a -> (e -> s a) -> s a
\end{haskellcode}

If an exception propagates uncaught to the top of a transaction, that
transaction is aborted and the exception is re-thrown in the thread.

There are utility functions for \verb|CTVar|s provided in
Control\-.Concurrent\-.STM\-.CTVar, and an STM equivalent of
\verb|CVar|s in Control\-.Concurrent\-.STM\-.CTMVar.


  \subsection{Memory Model}
  \label{sec:abstraction-mem}
  There are three memory models supported in \dejafu{}:

\defineword{Sequential Consistency}{The most intuitive model: a
  program behaves as a simple interleaving of the actions in different
  threads. When a \texttt{CRef} is written to, that write is
  immediately visible to all threads.}

\defineword{Total Store Order (TSO)}{Each thread has a write buffer. A
  thread sees its writes immediately, but other threads will only see
  writes when they are committed, which may happen later. Writes are
  committed in the same order that they are created.}

\defineword{Partial Store Order (PSO)}{A relaxation of TSO where each
  thread has a write buffer for each \texttt{CRef}. A thread sees its
  writes immediately, but other threads will only see writes when they
  are committed, which may happen later. Writes to different
  \texttt{CRef}s are not necessarily committed in the same order that
  they are created.}

The memory model only makes a difference for unsynchronised
operations, such as \verb|readCRef|, \verb|writeCRef|, and
\verb|readForCAS|.

The default memory model for testing is TSO, as that most accurately
models the behaviour of modern x86 processors. The use of a relaxed
memory model does cause some blow-up in the number of schedules tested
when unsynchronised operations are used, but as most of the
concurrency primitives are synchronised this tends to be fairly
contained.


\section{Program Execution}
\label{chap:execution}
This chapter describes the test execution of a concurrent program
written using the \verb|MonadConc| abstraction, not the execution of
programs using GHC's actual concurrency primitives. Of course, for
correctness of testing, there should be a correspondence between these
two models, see \chap{correctness}.

The execution of a concurrent program is considered to be the
sequential stepwise execution of \emph{primitive actions}, the most
basic things that a computation can do.


  \subsection{Primitive Actions}
  \label{sec:execution-primops}
  There are currently 31 primitive actions used to construct the testing
instances of \verb|MonadConc|, one of which only arises when testing
under relaxed memory. These primitive actions contain a continuation,
allowing individual actions to be composed into larger execution
sequences. Each thread of execution consists of such a sequence,
terminated by the \verb|AStop| primitive, which has no continuation
and signals the termination of the thread.

Threads, \verb|CVar|s, \verb|CRef|s, and \verb|CTVar|s may be given
names, which are displayed in execution traces. If no name is given, a
unique numeric identifier is used instead.

\vspace{0.25cm}
\noindent \textsc{Threading} \vspace{-0.5\parskip}

\begin{primtable}
\defineprimT{AFork}{name (unmask \arr action) (thread\_id \arr action)}{%
  Create a new thread from the first action, and continue executing
  the current thread with the second.}
\defineprimT{AMyTId}{(thread\_id \arr action)}{%
  Continue execution of the current thread by querying the thread
  identifier.}
\defineprimT{AYield}{action}{%
  Execute the given action, but also signify to the scheduler that it
  may be worth running a different thread now.}
\defineprimT{AStop}{}{%
  Terminate the current thread.}
\end{primtable}

\noindent \textsc{\texttt{CRef}s} \vspace{-0.5\parskip}

\begin{primtable}
\defineprimT{ANewRef}{name a (cref a \arr action)}{%
  Construct a new \texttt{CRef} and give it to the continuation.}
\defineprimT{AReadRef}{(cref a) (a \arr action)}{%
  Read the currently visible value of a \texttt{CRef}.}
\defineprimT{AReadRefCas}{(cref a) (ticket a \arr action)}{%
  Produce a \texttt{Ticket} from the currently visible state of a
  \texttt{CRef}.}
\defineprimT{APeekTicket}{(ticket a) (a \arr action)}{%
  Get the value out of a \texttt{Ticket}.}
\defineprimT{AModRef}{(cref a) (a \arr (a, b)) (b \arr action)}{%
  Commit all pending writes and atomically modify the value within a
  \texttt{CRef}.}
\defineprimT{AModRefCas}{(cref a) (a \arr (a, b)) (b \arr action)}{%
  Commit all pending writes and atomically modify the value within a
  \texttt{CRef} using a compare-and-swap.}
\defineprimT{AWriteRef}{(cref a) a action}{%
  Update the value of a \texttt{CRef}. The updated value is visible to
  the current thread immediately.}
\defineprimT{ACasRef}{(cref a) (ticket a) a ((succeeded?, ticket a) \arr action)}{%
  Update the value of a \texttt{CRef} if it hasn't changed since the
  ticket was produced..}
\defineprimT{ACommit}{thread\_id cref\_id}{%
  Make the last write to the given \texttt{CRef} by that thread
  visible to all threads.}
\end{primtable}

\noindent \textsc{\texttt{CVar}s} \vspace{-0.5\parskip}

\begin{primtable}
\defineprimT{ANewVar}{name (cvar a \arr action)}{%
  Construct a new \texttt{CVar} and give it to the continuation.}
\defineprimT{APutVar}{(cvar a) a action}{%
  Block until the \texttt{CVar} is empty and put a value into it.}
\defineprimT{ATryPutVar}{(cvar a) a (succeeded? \arr action)}{%
  Try to put a value into the \texttt{CVar} without blocking.}
\defineprimT{AReadVar}{(cvar a) (a \arr action)}{%
  Block until the \texttt{CVar} is full and read its value.}
\defineprimT{ATakeVar}{(cvar a) (a \arr action)}{%
  Block until the \texttt{CVar} is full and take its value.}
\defineprimT{ATryTakeVar}{(cvar a) (Maybe a \arr action)}{%
  Try to take the value from a \texttt{CVar} without blocking.}
\end{primtable}

\noindent \textsc{Exceptions} \vspace{-0.5\parskip}

\begin{primtable}
\defineprimT{AThrow}{exception}{%
  Raises an exception in the current thread, terminating the current
  execution.}
\defineprimT{AThrowTo}{exception action}{%
  Raises an exception in the other thread, blocking if the other
  thread has exceptions masked.}
\defineprimT{ACatching}{(exception \arr handler) action continuation}{%
  Registers a new exception handler for the duration of the inner
  action.}
\defineprimT{APopCatching}{action}{%
  Remove the exception handler from the top of the stack.}
\defineprimT{AMasking}{masking\_state (unmask \arr action) continuation}{%
  Executes the inner action under a new masking state, and also gives
  it a function to reset the masking state.}
\defineprimT{AResetMask}{set? inner? masking\_state action}{%
  Sets the masking state.}
\end{primtable}

\noindent \textsc{Software Transactional Memory} \vspace{-0.5\parskip}

\begin{primtable}
\defineprimT{AAtom}{transaction continuation}{%
  Execute an STM transaction atomically.}
\defineprimT{SNew}{name a (ctvar a \arr action)}{%
  Create a new \texttt{CTVar} containing the given value.}
\defineprimT{SRead}{(ctvar a) (a \arr action)}{%
  Read the current value of a \texttt{CTVar}.}
\defineprimT{SWrite}{(ctvar a) a action}{%
  Update a \texttt{CTVar}.}
\defineprimT{SThrow}{exception}{%
  Throw an exception, aborting the current execution flow.}
\defineprimT{SCatch}{(exception \arr handler) action continuation}{%
  Registers a new exception handler for the duration of the action.}
\defineprimT{SRetry}{}{%
  Abort the current transaction.}
\defineprimT{SOrElse}{transaction transaction continuation}{%
  Try executing the first transaction, if it fails, execute the
  second.}
\end{primtable}

\noindent \textsc{Testing Annotations} \vspace{-0.5\parskip}

\begin{primtable}
\defineprimT{AKnowsAbout}{(Either cvar ctvar) action}{%
  Record that the thread has access to the given variable.}
\defineprimT{AForgets}{(Either cvar ctvar) action}{%
  Record that the thread no longer has access to the given variable.}
\defineprimT{AAllKnown}{action}{%
  Record that all variables the thread knows about have been
  reported.}
\end{primtable}

\noindent \textsc{Miscellaneous} \vspace{-0.5\parskip}

\begin{primtable}
\defineprimT{AReturn}{action}{%
  Execute the given action.}
\defineprimT{ALift}{monadic\_action}{%
  Execute an action from the underlying monad.}
\end{primtable}


  \subsection{Stepwise Execution}
  \label{sec:execution-stepwise}
  Execution of an entire computation proceeds in a stepwise manner: a
thread is chosen by the scheduler, its primitive action is executed,
and a new action is returned to be executed by that thread in the next
step. The simplest thing that a thread can do is to stop, which will
serve as a useful minimal example:

\begin{haskellcode}
stepStop = simple (kill tid threads) Stop
\end{haskellcode}

The effect of \verb|stepStop| is to: remove the current thread from
the map of live threads, and then return the new thread map and the
name of the action to appear in the trace (\verb|Stop|, here). The
\verb|simple| helper function is for actions which don't create any
new shared variables or threads, or have any relaxed-memory effects.

Another simple action that a thread can perform is \verb|AReturn|:

\begin{haskellcode}
stepReturn c = simple (goto c tid threads) Return
\end{haskellcode}

The effect of \verb|stepReturn| is to: extract the continuation of the
action and replace the continuation of the current thread with it. The
\verb|Right|\footnote{The type \texttt{Either a b} type is commonly
  used to represent computations that might fail with an error
  value. By convention \texttt{Left err} means that the computation
  failed with reason \texttt{err}, and \texttt{Right x} means that the
  computation succeeded, producing \texttt{x}.}  indicates that the
action completed successfully. There are a few different possible
failures, such as an uncaught exception, which will terminate the
current thread. If the main thread is terminated, the entire
computation terminates.

\subsubsection{Threading}
\label{sec:execution-stepwise-threading}

\begin{haskellcode}
type Threads n r s = Map ThreadId (Thread n r s)
\end{haskellcode}

A map is used to keep track of all current threads. There are helper
functions to manipulate this map: \verb|kill| and \verb|goto| are two;
another is \verb|launch|, used to create a new thread. This can be
seen in the implementation of \verb|AFork|:

\begin{haskellcode}
stepFork name a b = return (Right (threads', idSource', Fork newtid, wb)) where
  threads' = goto (b newtid) tid (launch tid newtid a threads)
  (idSource', newtid) = nextTId name idSource
\end{haskellcode}

The \verb|stepFork| function involves two modifications to the thread
map: firstly, a new thread is created (and inherits the masking state
of its parent); secondly the continuation of the current thread is
updated. Here \verb|simple| cannot be used, as the identifier source
is being modified.

In the implementation of \verb|AYield|, no special functionality is
needed:

\begin{haskellcode}
stepYield c = simple (goto c tid threads) Yield
\end{haskellcode}

Its effect is purely a scheduling concern; from the point of view of
updating the state of the system, it is no different to
\verb|AReturn|.

\subsubsection{\texttt{CRef}s and Relaxed Memory}
\label{sec:execution-stepwise-cref}

\begin{haskellcode}
newtype CRef r a = CRef (CRefId, r (Map ThreadId a, Integer, a))
\end{haskellcode}

A \verb|CRef| is implemented as a mutable reference containing a
\emph{globally visible} value, a counter of how many write commits
there have been, and a number of \emph{thread-specific} values. These
thread-specific values correspond to uncommitted writes. A
\verb|Ticket|, used in compare-and-swap operations, is a
\emph{witness} that a specific prior value was observed. Like threads,
a \verb|CRef| (and \verb|CVar|) can be given a name when it is
initially created.

There are three memory models supported by \dejafu{}. Each has a
different implementation for writing to a \verb|CRef|:

\begin{haskellcode}
stepWriteRef cref@(CRef (crid, _)) a c = case memtype of
\end{haskellcode}

The first model assumes sequential consistency. There are no relaxed
memory effects:\footnote{This is an emple of \emph{do notation}, which
  is a convenient synctatic sugar for composition of monadic actions.}

\begin{haskellcode}
  SequentialConsistency -> do
    writeImmediate cref a
    simple (goto c tid threads) (WriteRef crid)
\end{haskellcode}

The \verb|writeImmediate| function writes to the globally visible
value, and clears the thread-specific values.

Total store order (TSO) corresponds to an architecture where each
thread has its own cache. This matches modern x86 and x86\_64
processors. Writes made by a thread will be cached, but they will be
committed in that same order to main memory:

\begin{haskellcode}
  TotalStoreOrder -> do
    wb' <- bufferWrite wb tid cref a tid
    return (Right (goto c tid threads, idSource, WriteRef crid, wb'))
\end{haskellcode}

The \verb|bufferWrite| function appends a write to the relevant write
buffer, in this case the one corresponding to that thread.

Partial store order (PSO) is a more relaxed version of total store order,
where the writes a thread makes may not necessarily be committed in
order. It can be modelled by giving each \verb|CRef| a write buffer,
rather than each thread:

\begin{haskellcode}
  PartialStoreOrder -> do
    wb' <- bufferWrite wb crid cref a tid
    return (Right (goto c tid threads, idSource, WriteRef crid, wb'))
\end{haskellcode}

Both the TSO and PSO cases update the thread-specific map. A thread
will always see the writes it has made, but other threads may not.

The compare-and-swap write is a little different. It has the effect of
a memory barrier: any uncommitted writes to any \verb|CRef| are
committed before the CAS is done, and the result is immediately
globally visible. There is a \verb|synchronised| function for actions
with this barrier property:\footnote{The \texttt{\$} operator is
  function application, but with a very low precedence. This makes it
  convenient for avoiding parentheses, which can be more readable when
  multi-line expressions are involved.}

\begin{haskellcode}
stepCasRef cref@(CRef (crid, _)) tick a c = synchronised $ do
  (suc, tick') <- casCRef cref tid tick a
  simple (goto (c (suc, tick')) tid threads) (CasRef crid suc)
\end{haskellcode}
%$

The \verb|casCRef| function here generates a new \verb|Ticket|,
compares with the supplied one, and then swaps the value. It is
provided, rather than the logic be included verbatim, as it is used
again in the implementation of \verb|stepModRefCas|.

The implementation of \verb|synchronised| is as follows:

\begin{haskellcode}
synchronised ma = do
  writeBarrier wb
  res <- ma
  case res of
    Right (threads', idSource', act', _) -> return
      (Right (threads', idSource', act', emptyBuffer))
    _ -> return res
\end{haskellcode}

Here \verb|writeBarrier| commits all cached writes. The action is then
executed, and an empty write buffer returned. So \verb|simple| can be
used in the implementation of \verb|stepModRef| despite the write
buffer being changed.

Cached writes can be committed to the globally visible value (at which
point the thread-specific values disappear) by executing an
\verb|ACommit| action:

\begin{haskellcode}
stepCommit t c = do
  wb' <- case memtype of
    TotalStoreOrder   -> commitWrite wb t
    PartialStoreOrder -> commitWrite wb c
  return (Right (threads, idSource, CommitRef t c, wb'))
\end{haskellcode}

Note how the invocation of \verb|commitWrite| differs between the
cases: under TSO, the cache corresponding to the thread is used;
whereas under PSO, the cache corresponding to the \verb|CRef| is
used. There is no case for sequential consistency here, as commit
actions are not explicitly introduced by the program under test; they
are introduced by the test runner when executing under a relaxed
memory model. See \sect{execution}{scheduling}.

Reading a reference is quite simple:

\begin{haskellcode}
stepReadRef cref@(CRef (crid, _)) c = do
  val <- readCRef cref tid
  simple (goto (c val) tid threads) (ReadRef crid)

stepReadRefCas cref@(CRef (crid, _)) c = do
  tick <- readForTicket cref tid
  simple (goto (c tick) tid threads) (ReadRefCas crid)
\end{haskellcode}

The \verb|readCRef| function checks if there is a cached value for the
current thread and, if so, returns it. Otherwise it returns the
globally visible value. The \verb|readForTicket| function behaves
similarly, but returns a \verb|Ticket| rather than the current value.

\subsubsection{Exceptions}
\label{sec:execution-stepwise-exception}

A thread has both a stack of exception handlers, and a masking
state. The handler stack affects all exceptions raised in the thread,
whereas the masking state only affects exceptions raised by
\verb|AThrowTo|.

\begin{haskellcode}
stepCatching h ma c = simple threads' Catching where
  a     = runCont ma      (APopCatching . c)
  e exc = runCont (h exc) (APopCatching . c)
  threads' = goto a tid (catching e tid threads)
\end{haskellcode}

Note the addition of \verb|APopCatching| at the ends of the enclosed
action and the handler. This ensures that the handler is popped from
the stack whether an exception is thrown or not.

When an exception is thrown, it may not be able to be handled by the
topmost handler, as there are exceptions of many types:

\begin{haskellcode}
stepThrow e = case propagate e tid threads of
    Just threads' -> simple threads' Throw
    Nothing -> return (Left UncaughtException)
\end{haskellcode}

The \verb|propagate| function pops from the stack of exception
handlers until one is found capable of handling that type of
exception. It then jumps to the handler, and returns the new thread
map. If no handler was found, the thread is killed by the uncaught
exception.

Throwing an exception to another thread is significantly more
complicated, and is also a \verb|synchronised| operation:

\begin{haskellcode}
stepThrowTo t e c = synchronised $
  let threads' = goto c tid threads
      blocked  = block (OnMask t) tid threads
  in if interruptible (lookup t threads)
     then case propagate e t threads' of
            Just threads'' -> simple threads'' (ThrowTo t)
            Nothing
              | t == 0    -> return (Left UncaughtException)
              | otherwise -> simple (kill t threads') (ThrowTo t)
     else simple blocked (BlockedThrowTo t)
\end{haskellcode}
%$

Firstly, whether the thread is interruptible is checked. If not, the
current thread is blocked. If it is interruptible, then the exception
is propagated through its handler stack. If a handler is found, the
thread jumps to it, throwing away whatever it was going to do next. If
a handler is not found, the thread is killed. If the main thread is
killed, the entire computation terminates.

Like \verb|ACatching|, the \verb|AMasking| action introduces an
additional action into its continuation:

\begin{haskellcode}
stepMasking m ma c = simple threads' (SetMasking False m) where
  a = runCont (ma umask) (AResetMask False False m' . c)
  m' = masking (lookup tid threads)
  umask mb = do
    resetMask True m'
    b <- mb
    resetMask False m
    return b
  resetMask typ ms = cont (\k -> AResetMask typ True ms (k ()))
  threads' = goto a tid (mask m tid threads)
\end{haskellcode}

\subsubsection{Software Transactional Memory}
\label{sec:execution-stepwise-stm}

As STM transactions are atomic, the implementation is comparatively
simple. They are still implemented in terms of a step function, but it
is just iterated until termination.

Firstly, the transaction is executed:

\begin{haskellcode}
stepAtom stm c = synchronised $ do
  (res, newctvid) <- runstm stm (nextCTVId idSource)
  let idSource'   = idSource { nextCTVId = newctvid }
  case res of
\end{haskellcode}
%$

There are now three possible results:

\begin{enumerate}
\item The transaction succeeds. All threads blocked on \verb|CTVar|s
  which were modified are woken:

\begin{haskellcode}
    Success readen written val
      let (threads', woken) = wake (OnCTVar written) threads
      in return (Right (goto (c val) tid threads', idSource', STM woken, wb))
\end{haskellcode}

\item The transaction aborts due to \verb|retry|. The thread is
  blocked until any of the read \verb|CTVar|s are modified:

\begin{haskellcode}
    Retry touched ->
      let threads' = block (OnCTVar touched) tid threads
      in return (Right (threads', idSource, BlockedSTM, wb))
\end{haskellcode}

\item The transaction aborts due to an uncaught exception. The
  exception is thrown in the thread:

\begin{haskellcode}
    Exception e -> stepThrow e
\end{haskellcode}

\end{enumerate}


  \subsection{Scheduling}
  \label{sec:execution-scheduling}
  When there are multiple, non-blocked, threads available, the choice of
which one to execute next is made by the scheduler.

A scheduler is represented as a pure function, and is supplied as a
parameter when testing. This allows for deterministic results and,
just as importantly, allows for computing a list of scheduling
decisions in advance, designed to try to provoke the system into a new
state. This is the basis for the systematic concurrency testing
implementation.

\begin{haskellcode}
type Scheduler s = s
  -> Maybe (ThreadId, ThreadAction)
  -> NonEmpty (ThreadId, NonEmpty Lookahead)
  -> (ThreadId, s)
\end{haskellcode}

In order to make nontrivial decisions, a scheduler maintains some
state, of type \verb|s|. This could be, for example, a random number
generator:

\begin{haskellcode}
randomSched :: RandomGen g => Scheduler g
randomSched g _ threads = (threads' !! choice, g') where
  (choice, g') = randomR (0, length threads' - 1) g
  threads'     = map fst (toList threads)
\end{haskellcode}

The initial state is supplied when the execution begins, and the final
state is returned when it terminates. Use of this state is, of course,
not mandatory, as a simple round-robin scheduler illustrates:

\begin{haskellcode}
roundRobinSched :: Scheduler ()
roundRobinSched _ Nothing _ = (0, ())
roundRobinSched _ (Just (prior, _)) threads
  | prior >= maximum threads' = (minimum threads', ())
  | otherwise = (minimum (filter (>prior) threads'), ())

  where threads' = map fst (toList threads)
\end{haskellcode}

A scheduler is also given information about the state of the system:
what the last thread it scheduled did (this is \verb|Nothing| if this
is the first step of the computation), and what every runnable thread
in the system will do in the next few steps. Here \verb|NonEmpty| is
the type of non-empty lists,\footnote{And \texttt{toList} converts a
  \texttt{NonEmpty a} to a \texttt{[a]}.} to give a type-level
guarantee that there \emph{are} threads to run: if there are no
runnable threads, the execution terminates, signalling a deadlock
condition.

The \verb|ThreadAction| type is a record of what has been done, and
the \verb|Lookahead| type is a slightly simpler view of what will
happen. The two types cannot be the same, because in general the
effect of performing a primitive action at some point in the future
cannot be determined, due to interactions between threads.

\subsubsection{Phantom Threads}
\label{sec:execution-scheduling-phantom}

In a sequentially consistent memory model, the set of runnable threads
is exactly the set of threads created by \verb|AFork| which are not
blocked.

Under relaxed memory, however, this is not the case. In order to model
the nondeterministic committing of \verb|CRef| writes, for every
buffer with an uncommitted write (threads, under TSO; \verb|CRef|s,
under PSO), a \emph{phantom thread} is created, and added to the
runnable set. A phantom thread is a thread with only one action:
\verb|ACommit|. These threads do not exist in the same way that other
threads do, they are never added to the thread map, they only exist in
order for the scheduler to determine when commits happen.

This may seem like an odd approach: why create new not-quite-threads
in order to model relaxed memory? The advantage is that systematic
concurrency testing techniques assume there is only one source of
nondeterminism: the scheduler. If a second source is added, such as
when writes are committed, it is difficult to integrate this with
existing algorithms. By using phantom threads, the two sources of
nondeterminism are unified, and existing algorithms just work. This
approach was suggested by \citep{rdpor}.

This approach also ensures that \verb|ACommit| actions are never
introduced under a sequentially-consistent memory model.


\section{Systematic Concurrency Testing}
\label{chap:sct}
This chapter describes the test execution of a concurrent program
written using the \verb|MonadConc| abstraction, not the execution of
programs using GHC's actual concurrency primitives. Of course, for
correctness of testing, there should be a correspondence between these
two models, see \chap{correctness}.

The execution of a concurrent program is considered to be the
sequential stepwise execution of \emph{primitive actions}, the most
basic things that a computation can do.


  \subsection{Schedule Bounding}
  \label{sec:sct-bounding}
  Schedule bounding is an \emph{incomplete} approach to SCT. Each
sequence of scheduling decisions is associated with a \emph{bound
  value}, limiting the results of some \emph{bound function}. Such a
function could be the number of pre-emptive context switches, for
example. Schedule bounding was introduced in \citep{pbound}, and came
from work in the model checking field.

Here are three common bound functions in use today:

\definewordc{Pre-emption Bounding}{pbound}{%
  The number of pre-emptive context switches is bounded.}

\definewordc{Fair Bounding}{fbound}{%
  The difference between the number of times different threads call
  \texttt{yield} is bounded.}

\definewordc{Delay Bounding}{dbound}{%
  The number of deviations from a deterministic scheduler is bounded.}

Both pre-emption bounding and delay bounding have empirical evidence,
in \citep{empirical}, showing that small bounds are good for finding
bugs in many real-world programs.

Fair bounding is used to handle programs which make use of lock-free
constructs such as spinlocks. A spinlock may be implemented like so:

\begin{haskellcode}
lock p var = spin where
  spin = do
    x <- readCRef var
    unless (p x) (yield >> spin)
\end{haskellcode}

Here, a \verb|CRef| is read from repeatedly. Each time some predicate on its value
is not satisfied, the thread yields and tries again. This can easily
give rise to infinitely long executions: simply don't execute any
other thread after the \verb|yield|, as it doesn't \emph{force} a
context switch. Fair bounding bounds the difference between the number
of times that threads have called \verb|yield|: if the thread that has
yielded the fewest times has done so 1 time, and the thread that has
yielded the most times has done so 10 times, then the bound value is
9.

Strictly speaking, schedule bounding refers to trying only those
schedules with a bound value equal to some fixed parameter. A variant
of this is \emph{iterative} bounding, where this parameter is
increased from zero up to some limit. Another variant is where an
inequality, rather than an equality, is used. This explores the same
schedules as iterative bounding, but doesn't impose the same ordering
properties over schedules tried. In practice, ``schedule bounding''
typically refers to this third type, unless specified otherwise.

\dejafu{} uses a combination of pre-emption and fair bounding, with a
default pre-emption bound of 2 and a fair bound of 5, in order to
handle gracefully any computations which use spinlocking
techniques. The default pre-emption bound was chosen based on
empirical evidence.


  \subsection{Partial-order Reduction}
  \label{sec:sct-por}
  Partial-order reduction is a \emph{complete} approach to SCT. It is
based on the insight that, when comparing different execution traces,
only the relative ordering of \emph{dependent} actions is
important. Two actions are dependent if the order in which they are
performed could affect the result of the program:

\definewordc{Dependency Relation}{dpor}{%
  Let $\mathcal T$ be the set of transitions in a concurrent system. A
  binary, reflexive, and symmetric relation $\mathcal D \subseteq
  \mathcal T \times \mathcal T$ is a valid \emph{dependency relation}
  iff, for all $t_{1}, t_{2} \in \mathcal T$, $(t_{1}, t_{2}) \notin
  \mathcal D$ ($t_{1}$ and $t_{2}$ are \emph{independent}) implies
  that the following properties hold for all program states $s$:

  \begin{enumerate}
  \item if $t_{1}$ is enabled in $s$ and $s \xrightarrow{t_{1}} s'$,
    then $t_{2}$ is enabled in $s$ iff $t_{2}$ is enabled in $s'$; and

  \item if $t_{1}$ and $t_{2}$ are enabled in $s$, then there is a
    unique state $s'$ such that $s \xrightarrow{t_{1}t_{2}} s'$ and $s
    \xrightarrow{t_{2}t_{1}} s'$.
  \end{enumerate}}

In other words, independent transitions cannot enable or disable each
other, and enabled independent transitions commute. Rather than use
this relational definition directly, typically instead a collection of
conditions sufficient for dependency are identified. These conditions
are determined by what sorts of things the concurrent system under
test can express.

Typically for the presentation of algorithms, a very simple core
concurrent language of just reads and writes is shown. This gives rise
to the following dependency condition:

\begin{align*}
  x \dependent y \iff& \mathrm{thread\_id}(x) = \mathrm{thread\_id}(y) \lor\\
    &\left(\mathrm{variable}(x) = \mathrm{variable}(y)
     \land \left(\mathrm{is\_write}(x) \lor \mathrm{is\_write}(y)\right)\right)
\end{align*}

Where $x \dependent y$ is read as ``$x$ and $y$ are dependent''. This
choice of notation would suggest a symbol $\leftrightarrow$ meaning
independence, but that doesn't seem to be used.

The dependency relation for \dejafu{} is rather more complex than
this, as there are more actions than just reads and writes, however it
can be simplified to a few quite general conditions over different
sorts of reads and writes, with some remaining special cases.

These special cases are:

\begin{haskellcode}
dependent (t1, a1) (t2, a2) = case (a1, a2) of
  (Lift, Lift)   -> True
  (ThrowTo t, _) -> t == t2
  (_, ThrowTo t) -> t == t1
  (STM _, STM _) -> True
\end{haskellcode}

\begin{itemize}
\item Two lifts from the underlying monad are always dependent, as in
  general this allows arbitrary I/O to be performed. The only
  restriction over I/O is that, given a fixed schedule, the I/O is
  deterministic.

\item Throwing an exception to a thread is dependent with anything, as
  all actions can be pre-empted by an exception.

\item STM transactions are always dependent. This final case could
  probably be refined to STM transactions which have some overlap in
  the \verb|CTVar|s they modify, but this is an optimisation which has
  not yet been tried.
\end{itemize}

The general cases are defined in terms of synchronised and
unsynchronised actions. Synchronised actions commit all pending
\verb|CRef| writes, and do not have any relaxed memory
properties. Unsynchronised actions do not have this property.

\begin{haskellcode}
dependentActions memtype buf a1 a2 = case (a1, a2) of
  (UnsynchronisedRead  r1, UnsynchronisedWrite r2)
    -> r1 == r2 && memtype == SequentialConsistency
  (UnsynchronisedWrite r1, UnsynchronisedWrite r2)
    -> r1 == r2 && memtype == SequentialConsistency

  (UnsynchronisedRead r1, _)
    | isBarrier a2 -> isBuffered buf r1

  _ ->
    same crefOf && (isSynchronised a1 || isSynchronised a2)
    || same cvarOf
\end{haskellcode}

\begin{itemize}
\item A read and write to the same \verb|CRef| are only dependent
  under a sequentially-consistent memory model, as are two writes.

\item An unsynchronised read is dependent with an action that imposes
  a memory barrier if there are buffered writes to the variable being
  read from.

\item Any two actions on the same \verb|CRef| where at least one of
  them will cause a commit are dependent.

\item Any two actions on the same \verb|CVar| are dependent.
\end{itemize}

Characterising the execution of a concurrent program by the ordering
of its dependent actions only gives us a \emph{partial order} over the
actions in the entire program. An execution trace may be just one
possible \emph{total} order corresponding to the same partial
order. The goal of partial-order reduction, then, is to eliminate
these redundant total orders by intelligently making scheduling
decisions to permute the order of dependent actions.

This can be done by executing a program with a deterministic
scheduler, and then examining the trace, the total order, for
\emph{backtracking points}. A backtracking point is a place in the
execution where multiple dependent choices were available, and only
one was tried. The exploration of the state space continues by making
the same scheduling decisions up to that point, and then making a
different decision. This process of doing partial-order reduction
based on information gathered at run-time, rather than static
analysis, is called \emph{dynamic partial-order reduction} (DPOR).

% Implementation in dejafu

In an imperative language, DPOR is usually done by executing the
program under test stepwise in a recursive function, where each stack
frame has a set of decisions still to try, and this is mutated by
later calls when a backtracking point is identified. As \dejafu{} is a
Haskell library, this is not a very natural way to formulate anything,
and so a different approach was taken.

\dejafu{} explicitly constructs a graph structure in memory, where
each path from the root to a leaf corresponds to one complete
execution. Forks in the tree correspond to places where multiple
decisions have been tried. The operation proceeds like so:

\begin{haskellcode}
sctBounded memtype bf run = go initialState where
  go state = case next state of
    Just decisions -> do
      (result, s, trace) <- run decisions
      let bpoints = findBacktrack memtype s trace
      let newBPOR = todo bf bpoints (grow memtype trace state)

      go newState >>= ((result, trace) :)

    Nothing -> return []
\end{haskellcode}

Here \verb|next| returns a schedule prefix; \verb|run| executes the
computation with a given sequence of initial scheduling decisions,
returning the final result, the final scheduler state (which includes
a tentative list of bracktracking points), and the execution trace;
\verb|findBacktrack| identifies a list of actual backtracking points
from these tentative ones; \verb|grow| adds the trace to the tree
structure; and \verb|todo| adds the newly-identified backtracking
points. It is also the responsibility of \verb|todo| to ensure these
new backtracking points do not cause schedules exceeding the bound to
be generated; the \verb|bf| function is the bound function, expressed
as a predicate.

The entire process terminates when \verb|next| returns \verb|Nothing|,
which means that there are no unexplored backtracking points left.

\subsection{Integration with Schedule Bounding}
\label{sec:sct-por-bounding}

The na\"{\i}ve way to integrate DPOR with schedule bounding would be
to first use partial-order techniques to prune the search space, and
then to additionally filter things out with schedule bounding.

Unfortunately, this is unsound. This approach misses parts of the
search space reachable within the bound. This is because the
introduction of the bound introduces new dependencies between actions,
which cannot be determined \emph{a priori}. The solution is to add
\emph{conservative} backtracking points to account for the bound in
addition to any normal backtracking points that are identified. Where
to insert these depends on the bound function.

In the case of pre-emption bounding, it suffices to try all
possibilities at the last context switch before a normal backtracking
point. This is because context switches influence the number of
pre-emptions needed to reach a given program state, depending on which
thread gets scheduled. As pre-emption bounding has been found
empirically to be successful with a low number of threads, and DPOR is
already eliminating a lot of possibilities, this is not in practice a
huge additional cost.

\subsection{Integration with Relaxed Memory}
\label{sec:sct-por-relaxed}

Due to the use of phantom threads, explained in
\sect{execution}{scheduling}, almost nothing needs to be done to
support relaxed memory!

The \verb|dependentActions| function has some knowledge of it in order
to make less pessimistic decisions, as otherwise the assumption would
have to be made that there are always uncommitted writes. The only
other change is related to the integration with schedule bounding: a
pre-emption immediately before (or immediately after) a phantom thread
is free.


\section{Correctness}
\label{chap:correctness}
This chapter describes the test execution of a concurrent program
written using the \verb|MonadConc| abstraction, not the execution of
programs using GHC's actual concurrency primitives. Of course, for
correctness of testing, there should be a correspondence between these
two models, see \chap{correctness}.

The execution of a concurrent program is considered to be the
sequential stepwise execution of \emph{primitive actions}, the most
basic things that a computation can do.


  \subsection{Correct Execution}
  \label{sec:correctness-execution}
  Correctness of execution asks whether the result of an arbitrary
execution of \dejafu{}'s testing implementation can be obtained in
reality.  Furthermore, do all real-world executions correspond to a
possible execution under \dejafu{}? To put it more simply:

\begin{itemize}
\item Is the behaviour of the primitive functions the same?

\item Is the granularity of scheduling decisions the same?
\end{itemize}

Both of these come with the caveat that the behaviour can be
different, as long as this difference can't be observed.

\subsubsection{Primitives}
\label{sec:correctness-execution-primops}

The method of implementing the members of the \verb|MonadConc|
typeclass that would be most amenable to proof would be to implement
analogues of the GHC primitives directly, and implement everything
else in terms of these. This matches how actual Haskell is
implemented, and would lend itself to establishing a formal
correspondence between the \dejafu{} primitives and the GHC
primitives, and the higher-level \verb|MonadConc| functions and the
higher-level functions in Control.\-Concurrent.

This approach was not taken, however. Firstly, it would tie the
implementation and correctness of \dejafu{} very closely to the
implementation of GHC; in principle the implementation of GHC's
concurrency primitives could be completely changed but the behaviour
preserved. Secondly, it would restrict \dejafu{} to a very specific
type of concurrency, supporting low-level operations, which may not
map to all interesting implementations of concurrency.

Instead, a reimplementation of GHC's concurrency based on the
documented and observable behaviour of the various functions was
done. This allows observing the behaviour of a program and
determining, intuitively, whether it is correct or not; but it's not
so good for proof. The matter is complicated by there being no
standard for concurrent Haskell, there is only what GHC provides.

The correctness of operations using \verb|CRef|s is complicated even
further, as the behaviour of these depends on the underlying memory
model. Total store order was chosen to be the default, as it is what
x86 processors do with unsynchronised memory accesses. But a
\verb|CRef| is more complicated than a simple memory cell: it is a
pointer to a cell, which can be moved around in garbage collection,
and it is accessed through primitive operations more complicated than
simple loads and stores. The lack of a standard, or even comprehensive
documentation, means that to discover the memory model for
\verb|CRef|s, the compilation of \verb|IORef| functions must be traced
through GHC from Haskell source to machine code. As GHC uses C{-}{-}
as an intermediary language, this may also require determining a
memory model for C{-}{-}.

Finally, there are some intended departures from the behaviour of
GHC's behaviour documented in \sect{abstraction}{typeclass}:

\begin{itemize}
\item \verb|getNumCapabilities| is not required to return a true
  result.

\item Deadlock detection can only function to the same extent as GHC
  if every thread is annotated with which \verb|CVar|s and
  \verb|CTVar|s it knows about, as \dejafu{} cannot use the garbage
  collector for this task.

\item \verb|catch| is not required to be able to catch exceptions from
  pure code.
\end{itemize}

\subsubsection{Scheduling}
\label{sec:correctness-execution-scheduling}

The stepwise implementation allows for a scheduling decision to be
made between each primitive action, which doesn't quite correspond to
how GHC handles scheduling:

\quot{GHC implements pre-emptive multitasking: the execution of
  threads are interleaved in a random fashion. More specifically, a
  thread may be pre-empted whenever it allocates some memory, which
  unfortunately means that tight loops which do no allocation tend to
  lock out other threads (this only seems to happen with pathological
  benchmark-style code, however).}{controlConcurrent}

That is, GHC allows for pre-emptions to occur whilst evaluating pure
code, which the stepwise executor does not. So there are executions
involving the pre-emption of the evaluation of non-terminating
expressions which are possible under GHC but not under
\dejafu{}. Whether this can be used to produce different outputs is
less clear.


  \subsection{Correct Testing}
  \label{sec:correctness-testing}
  Correctness of testing asks whether the schedule prefixes generated by
the partial-order reduction mechanism are valid, and are there any
results possible in real-world execution for which no schedule will be
generated? This is definitely \emph{not} the same as asking if every
real-world schedule will be generated by the testing framework, as
that is precisely what partial-order reduction tries to avoid.

\subsection{Prefix Validity}
\label{sec:correctness-testing-validity}

Multiple executions with different schedules are stored internally as
a tree, with each path from the root to a leaf corresponding to a
complete execution:

\begin{haskellcode}
data BPOR = BPOR
  { runnable :: Set ThreadId
  , todo     :: Set Threadid
  , done     :: Map ThreadId BPOR
  , action   :: Maybe ThreadAction
  }
\end{haskellcode}

The \verb|runnable| field is the set of all threads runnable at that
point; \verb|todo| is the decisions still to try making; \verb|done|
is the decisions already made; and \verb|action| is what was done at
this step. \verb|action| is a \verb|Maybe| value, because initially no
action has been performed, as the computation hasn't yet started.

There is a unique initial state, where only the initial thread is
runnable and nothing has been done:

\begin{haskellcode}
initialState :: BPOR
initialState = BPOR (singleton 0) (singleton 0) empty Nothing
\end{haskellcode}

There are some basic well-formedness invariants associated with a
\verb|BPOR| value:

\begin{itemize}
\item Every decision in the to-do set is possible:
  $\mathrm{todo} \subseteq \mathrm{runnable}$

\item Every decision in the done map is possible:
  $\dom \mathrm{done} \subseteq \mathrm{runnable}$

\item No decision that has been done is in the to-do set:
  $\mathrm{todo} \cap \dom \mathrm{done} = \varnothing$

\item These properties hold recursively:
  $\forall p \in \ran \mathrm{done}.~\wellf{p}$
\end{itemize}

Some work has been started in the Isabelle/HOL theorem prover to
formalise part of the recursive loop in \verb|sctBounded| (see
\sect{sct}{por}) and to prove that the invariants are preserved,
assuming that the stepwise executor is correct. It is hoped that
schedule prefix validity will follow from this. Specifically, the
things to be proved are:

\begin{itemize}
\item A prefix produced by \verb|next| is valid if the \verb|BPOR|
  tree is well-formed; furthermore, it consists of a sequence of
  decisions that have already been made and is terminated by a single
  decision from a to-do set.

\item \verb|next| returns \verb|Nothing| if and only if the
  \verb|todo| field of every node in the \verb|BPOR| tree is empty.

\item \verb|grow| adds the information in a trace to the \verb|BPOR|
  tree, making no other changes.
\end{itemize}

Data structure invariants are an important property to verify, as if
they are broken any assumptions made in the rest of the code cannot be
trusted.

\subsection{Result Completeness}
\label{sec:correctness-testing-complete}

The simplest notion of completeness of interest here is that for all
results possible by executing a given program for real, the
partial-order reduction framework can give that result.

However, as schedule bounding is involved, this is clearly not the
case. Therefore, there are two different notions of completeness which
are of interest:

\begin{itemize}
\item If the bounds are all set to $\infty$, all results possible
  under real execution show up under \dejafu{} execution.

\item For all sets of bounds, all results possible under real
  execution subject to those bounds show up under \dejafu{} execution
  with the same bounds.
\end{itemize}

The former corresponds to the correctness of partial-order reduction
with no bounds, and does not imply the latter. The latter implies the
former, and is the more interesting property. The latter is what we
really want of our testing framework.

The proof would need to proceed by first showing that the dependency
relation is correct: that only actions related by the dependency
relation can influence each others results. It's not clear how to
approach this, as it relies on the implementation of the actions. Once
the correctness of the dependency relation is established, it must be
shown that the partial-order reduction only prunes schedules where
there is no dependency.


\section{Case Studies}
\label{chap:casestudies}
This chapter describes the test execution of a concurrent program
written using the \verb|MonadConc| abstraction, not the execution of
programs using GHC's actual concurrency primitives. Of course, for
correctness of testing, there should be a correspondence between these
two models, see \chap{correctness}.

The execution of a concurrent program is considered to be the
sequential stepwise execution of \emph{primitive actions}, the most
basic things that a computation can do.


  \subsection{auto-update}
  \label{sec:casestudies-autoupdate}
  The \emph{auto-update} library runs tasks periodically, but only if
needed. For example, a single worker thread may get the time every
second and store it to a shared \verb|IORef|, rather than have many
threads starting within a second of each other all get the time
independently \citep{autoupdate}. Despite the core functionality being
very simple, two race conditions were noticed by users inspecting the
code in October 2014.

\begin{figure}[h!]
  \captionsetup{format=fnoline}
  \centering
  \begin{cminted}{haskell}
data UpdateSettings a = UpdateSettings
    { updateFreq           :: Int
    , updateSpawnThreshold :: Int
    , updateAction         :: IO a
    }

defaultUpdateSettings :: UpdateSettings ()
defaultUpdateSettings = UpdateSettings
    { updateFreq           = 1000000
    , updateSpawnThreshold = 3
    , updateAction         = return ()
    }

mkAutoUpdate :: UpdateSettings a -> IO (IO a)
mkAutoUpdate us = do
    currRef      <- newIORef Nothing
    needsRunning <- newEmptyMVar
    lastValue    <- newEmptyMVar

    void $ forkIO $ forever $ do
        takeMVar needsRunning

        a <- catchSome $ updateAction us

        writeIORef currRef $ Just a
        void $ tryTakeMVar lastValue
        putMVar lastValue a

        threadDelay $ updateFreq us

        writeIORef currRef Nothing
        void $ takeMVar lastValue

    return $ do
        mval <- readIORef currRef
        case mval of
            Just val -> return val
            Nothing -> do
                void $ tryPutMVar needsRunning ()
                readMVar lastValue

catchSome :: IO a -> IO a
catchSome act = catch act $
  \e -> return $ throw (e :: SomeException)
  \end{cminted}
  \caption{\emph{auto-update} implementation}
  \label{fig:example-autoupdate}
\end{figure}


The entire implementation, excluding comments and imports, is
reproduced in Figure \ref{fig:example-autoupdate}. The
\verb|mkAutoUpdate| function spawns a worker thread, which performs
the update action at the given frequency, only if the
\verb|needsRunning| flag has been set. It returns an action to attempt
to read the current result, if necessary blocking until there is
one. The transformation to the \verb|MonadConc| typeclass is mostly
simple, however the \verb|threadDelay| must be wrapped inside a call
to \verb|liftIO|.

The simpler race condition occurs if the reading thread is pre-empted
by the worker thread after putting into \verb|needsRunning|, and does
not run again until after the delay has passed. In this case the
worker thread can become blocked on taking for a second time from
\verb|needsRunning|. The reader thread will be unable to read from
\verb|lastValue| as the worker thread emptied it as the last action it
performed. The race condition can be exhibited with the following
test:

\begin{minted}{haskell}
test :: (MonadConc m, MonadIO m) => m ()
test = do
  auto <- mkAutoUpdate defaultUpdateSettings
  auto
\end{minted}

This test was chosen as it is one of the simplest things a user may
wish to do with the library: to create the worker, and to then read
the computed value. The output of testing shows the different results
that were found, with a sample trace leading to each one:

\begin{verbatim}
> autocheckIO test
[fail] Never Deadlocks (checked: 1)
        [deadlock] S0--------S1-----------S0-
[pass] No Exceptions (checked: 9)
[fail] Consistent Result (checked: 8)
        [deadlock] S0-----P1-S0---S1-----------S0-
        () S0--------S1---------P0---
False
\end{verbatim}

The \verb|autocheckIO| function is used to search for deadlocks,
uncaught exceptions in the main thread, and nondeterminism in
results. It allows the use of \verb|liftIO|, there is an
\verb|autocheck| function which does not.

``S$n$'' indicates that thread $n$ began executing after the prior one
blocked, ``P$n$'' indicates that thread $n$ pre-empted the prior
one. It is also possible to obtain a richer data structure with a
clearer account of what happened.

To read this trace, it is helpful to look at the ``S'' points, rather
than to count the steps. Using this tactic, we can see that a deadlock
occurs if the initial thread runs until it blocks (which is the
\verb|readMVar lastValue| call), then thread 1 runs until it blocks
(the second time \verb|takeMVar needsRunning| is reached). At this
point the initial thread is runnable, as it was unblocked by the write
to \verb|lastValue|. This is scheduled, and immediately blocks because
thread 1 took the value. Both threads are now blocked, and so the
computation is deadlocked.

This deadlock may arise in any use of the library, as it depends only
on the timing of the delay, and not on the computation performed. If
the call to \verb|threadDelay| completes before the reading thread has
resumed execution, this situation will arise.

The more complex race condition arises if \verb|readMVar| isn't
atomic, as in GHC versions before 7.8. The \verb|readMVar| function
used to be a combination of a take and a put. In this case an old
value can be returned if the read of \verb|lastValue| is pre-empted
between these two operations, as shown in this test:

\begin{minted}{haskell}
test :: (MonadConc m, MonadIO m) => m Int
test = do
  var  <- newCRef 0
  let action = modifyCRef var $ \x -> (x+1, x)
  auto <- mkAutoUpdate $ defaultUpdateSettings { updateAction = action }
  auto
  auto
\end{minted}

Here \verb|auto| is called twice to update the counter variable
twice. Actually reproducing this bug requires a new \verb|readCVar|
function be written, which has the same behaviour as the old
\verb|readMVar| function. Exhibiting this bug requires three
pre-emptions. As we need to supply our own bounds, we cannot use
\verb|autocheckIO|. The more \verb|dejafuIO'| function allows for the
memory model and bounds to be specified:

\begin{verbatim}
> let bounds = defaultBounds { preEmptionBound = Just 3 }
> dejafuIO' TotalStoreOrder bounds test ("Consistent Result", alwaysSame)
[fail] Consistent Result (checked: 23)
        [deadlock] S0------P1-S0---S1-----------S0-
        0 S0---------S1--------P0-----
        1 S0---------S1---------P0---P1--------P0---
\end{verbatim}

Here we see two different (non-deadlocking) results. The issue is the
\verb|readMVar| in the main thread and the \verb|tryTakeMVar| followed
by \verb|putMVar| in the other. If the main thread takes from the
\verb|MVar| before the \verb|tryTake|, it will retrieve the prior
value. This in itself is not a problem. However, if the main thread
then puts that value back between the \verb|tryTake| and the
\verb|put|, the \verb|put| will block, and not be able to store the
updated value.

Despite the bugs being rather simple, one not requiring any
pre-emptions at all to trigger, they both arose in practice. How easy
it is to make mistakes when implementing concurrent programs!


  \subsection{Search Party}
  \label{sec:casestudies-searchparty}
  The \emph{Search Party}\footnote{\github{barrucadu}{search-party}}
library supports speculative parallelism in generate-and-test search
problems. It is motivated by the consideration that: if multiple
acceptable solutions exist, it may not matter which one is
returned. Initially, only single results could be returned, but
support for returning all results was later added, incorrectly,
introducing a bug.

The key piece of code causing the problem was this part of the worker
loop:

\begin{minted}{haskell}
case maybea of
  Just a -> do
    atomically $ do
      val <- tryTakeCTMVar res
      case val of
        Just (Just as) -> putCTMVar res $ Just (a:as)
        _ -> putCTMVar res $ Just [a]
    unless shortcircuit $
      process remaining res
  Nothing -> process remaining res
\end{minted}

Here \verb|maybea| is a value indicating whether the computation just
evaluated was successful. The intended behaviour is that, if a
computation is successful, its result is added to the list in the
\verb|res| \verb|CTMVar|. This \verb|CTMVar| is exposed indirectly to
the user of the library, as it is blocked upon when the final result
of the search is requested.

There are some small tests in \emph{Search Party}, verifying that
deadlocks and exceptions don't arise, and that results are as
expected. Upon introducing this new functionality, tests began to fail
with differing result lists returned for different schedules,
prompting the test:\footnote{The \texttt{representative} function
  picks only one trace for each unique result.}

\begin{minted}{haskell}
checkResultLists :: Eq a => Predicate [a]
checkResultLists = representative (alwaysTrue2 check) where
  check (Right (Just as)) (Right (Just bs)) = as `elem` permutations bs
  check a b = a == b
\end{minted}

Given this predicate, we can very clearly see the problem:

\begin{verbatim}
> dejafu (runFind $ [0..2] @! const True) ("Result Lists", checkResultLists)

[fail] Result Lists (checked: 145)
        Just [0] S0-----S1---------S3-------S0-------
        Just [1] S0-----S1---------S3---P2-------S0-------
        Just [1,0] S0-----S1---------S3-------S2-------S0-------
        Just [0,1] S0-----S1---------S3---P2-------S3----S0-------
False
\end{verbatim}

The problem was a lack of any indication that a list-producing
computation had finished. As results were written directly to the
\verb|CTMVar|, partial result lists could be read depending on how the
worker threads and the main thread were interleaved.

In this case, fixing the failure did not require any interactive
debugging. Only one place had been modified in introducing the new
functionality, and the bug was found by re-reading the code with the
possibility of error in mind. However, the ability to produce a test
case which reliably reproduces the problem gives confidence that it
will not be accidentally reintroduced.


  \subsection{The Par Monad}
  \label{sec:casestudies-parmonad}
  The \verb|Par| monad \citep{parmonad} is a library providing a
traditional-looking concurrency abstraction, providing the programmer
with threads and mutable state, however it maintains determinism by
restricting its shared variables to one write, and operations to read
block until a value has been written. Thus, \verb|Par|'s \verb|IVar|s
are \emph{futures}, not \emph{mutable} state. \verb|Par| uses a
work-stealing scheduler running on multiple operating system threads,
fully evaluating values on their own threads before inserting them
into an \verb|IVar|. Despite its limitations, the \verb|Par| monad can
be very effective in speeding up pure code.

The following example maps a function in parallel over a list, fully
evaluating it. Of course, laziness is generally what is desired in
Haskell programs, but often it is known that an entire result will
definitely be needed:

\begin{minted}{haskell}
parMap :: NFData b => (a -> b) -> [a] -> [b]
parMap f as = runPar $ do
  bs <- mapM (spawnP . f) as
  mapM get bs
\end{minted}
%$

However, with a lack of multi-write shared variables and non-blocking
reads, \verb|Par| is unsuitable for long-lived concurrent programs
with a central shared state. It could not be used to implement a
multi-threaded work-stealing scheduler, such as the one underpinning
\verb|Par| itself. The library provides a number of different
schedulers, the default being the ``trace'' scheduler. Due to reports
of potential deadlocks with the ``direct'' scheduler from a year ago
\citep{parreddit}, it was tested with \dejafu{}.

To reduce the effort in modifying the code, only the direct
dependencies of the ``direct'' scheduler were modified, the rest of
the library being left unchanged. This resulted in four files needing
change: two from the
\emph{abstract-deque}\footnote{\hackage{abstract-deque}} package and
two from the \emph{monad-par}\footnote{\hackage{monad-par}} package.

Converting \emph{monad-par} to use \dejafu{} was quite simple. All
relevant types were parametrised by the underlying monad, all
functions had a \verb|MonadConc| context added, functions were swapped
for their \dejafu{} alternatives, and a \verb|runPar'| function was
added:

\begin{minted}{haskell}
runPar' :: MonadConc m => Par m a -> m a
\end{minted}

Some simplifications were made in the conversion process:

\begin{itemize}
\item \verb|Par| normally uses the
  \emph{mwc-random}\footnote{\hackage{mwc-random}} package when
  performing its internal scheduling. This was initially replaced with
  a constant function, and then a \verb|StdGen|.

\item Behaviour of the \verb|Par| scheduler can be configured using
  cpp, but only the default configuration was tested.
\end{itemize}

Figure \ref{fig:example-parmonad-sched} shows the original and
converted scheduler initialisation code. As can be seen, they are very
similar, even though this is a core component of a rather
sophisticated library, where the types have been changed.

Converting the \emph{abstract-deque} package proved a little more
challenging, as the typeclass interface requires knowledge of both the
queue type and the monad results are produced in. This issue was
solved by use of type families:

\begin{minted}{haskell}
class MonadConc (MConc d) => DequeClass d where
  type MConc d :: * -> *

  newQ :: MConc d (d elt)
  ...
\end{minted}

This solution is not ideal as it adds explicit knowledge of
\verb|MonadConc| to the \verb|DequeClass| typeclass, but it suffices
for testing purposes.

With the constant value `PRNG', a deadlock was discovered. It only
arises after 200 queries. Given that the range of values is from 0 to
the number of capabilities, and the probability is uniformly
distributed, the probability of an actual deadlock is about $4 \times
10^{-121}$ on a quad-core computer. No deadlocks were discovered when
using the \verb|StdGen| generator, with a variety of initial seeds
tried. If there is still a deadlock, it may require more than 2
capabilities to manifest.

\begin{landscape}
\begin{figure*}[t]
  \captionsetup{format=fnoline}
  \centering
  \begin{minipage}[t]{0.49\linewidth}
    \begin{minted}{haskell}
makeScheds :: Int -> IO [Sched]
makeScheds main = do
  caps <- getNumCapabilities
  workpools <- replicateM caps R.newQ
  rngs <- replicateM caps
            (Random.create >>= newHotVar)
  idle <- newHotVar []

  sessionFinished <- newHotVar False
  let sess = [Session baseSessionID sessionFinished]
  sessionStacks <- mapM newHotVar
                     (replicate caps sess)
  activeSessions <- newHotVar S.empty
  sessionCounter <- newHotVar (baseSessionID + 1)
  let allscheds =
       [ Sched { no=x, idle, isMain=(x==main),
                 workpool=wp, scheds=allscheds,
                 rng=rng, sessions=stck,
                 activeSessions=activeSessions,
                 sessionCounter=sessionCounter
               }
         | x    <- [0 .. caps-1]
         | wp   <- workpools
         | rng  <- rngs
         | stck <- sessionStacks
       ]
  return allscheds
    \end{minted}
    \caption*{Original}
  \end{minipage}
  \begin{minipage}[t]{0.49\linewidth}
    \begin{minted}{haskell}
makeScheds :: MonadConc m => Int -> m [Sched m]
makeScheds main = do
  caps <- getNumCapabilities
  workpools <- replicateM caps R.newQ
  rngs <- replicateM caps
            (newHotVar (mkStdGen 0))
  idle <- newHotVar []

  sessionFinished <- newHotVar False
  let sess = [Session baseSessionID sessionFinished]
  sessionStacks <- mapM newHotVar
                     (replicate caps sess)
  activeSessions <- newHotVar S.empty
  sessionCounter <- newHotVar (baseSessionID + 1)
  let allscheds =
       [ Sched { no=x, idle, isMain=(x==main),
                 workpool=wp, scheds=allscheds,
                 rng=rng, sessions=stck,
                 activeSessions=activeSessions,
                 sessionCounter=sessionCounter
               }
         | x    <- [0 .. caps-1]
         | wp   <- workpools
         | rng  <- rngs
         | stck <- sessionStacks
       ]
  return allscheds
    \end{minted}
    \caption*{\dejafu{}}
  \end{minipage}
  \caption{Par ``direct'' scheduler initialisation}
  \label{fig:example-parmonad-sched}
\end{figure*}
\end{landscape}

\section{Practical Usage}
\label{chap:practice}
This chapter describes the test execution of a concurrent program
written using the \verb|MonadConc| abstraction, not the execution of
programs using GHC's actual concurrency primitives. Of course, for
correctness of testing, there should be a correspondence between these
two models, see \chap{correctness}.

The execution of a concurrent program is considered to be the
sequential stepwise execution of \emph{primitive actions}, the most
basic things that a computation can do.


  \subsection{Alternatives to Existing Libraries}
  \label{sec:practice-alternatives}
  There are a number of popular Haskell libraries specifically for
concurrency. One of these is the
\emph{async}\footnote{\hackage{async}} library, for expressing
asynchronous computations. This library is intended to be a
higher-level and safer way of expressing asynchronous computations
than using \verb|forkIO| and \verb|MVar|s directly. It provides two
main functions to execute an action asynchronously:

\begin{haskellcode}
async :: IO a -> IO (Async a)
withAsync :: IO a -> (Async a -> IO b) -> IO b
\end{haskellcode}

Both of these fork the computation into a separate thread, providing
this \verb|Async| value, containing an \verb|MVar| which can be
blocked on in order to retrieve the value. In addition,
\verb|withAsync| kills the thread if the inner action completes before
it, to help prevent resource leaks.

There is a further abstraction atop \verb|Async|, called
\verb|Concurrently|, which has Functor, Applicative, and Alternative
instances, and represents an action which can be composed with other
actions and execute concurrently. The concurrency is achieved by
having \verb|(<*>)| execute each action asynchronously. There was a
Monad instance for \verb|Concurrently|, but this broke the laws, as
\verb|ap| was not the same as
\verb|(<*>)|\footnote{\url{https://github.com/simonmar/async/pull/26}}. This
was due to \verb|ap| executing its arguments sequentially, as that is
all which can be done with \verb|(>>=)|.

This bug could have been discovered through testing, but only
probabilistically. If \emph{async} were written using
\verb|MonadConc|, the relevant laws could have been specified as unit
tests and checked and the bug could have been caught before it showed
up in user code. Furthermore, by using \verb|IO| directly, it is not
possible to write a generic \verb|MonadConc| action which makes use of
\emph{async}, which is very unfortunate.

To address both of these issues, there is an \emph{async-dejafu}
package, which provides almost the same API as \emph{async}, but is
parameterised by a \verb|MonadConc|, giving functions like this:

\begin{haskellcode}
async :: MonadConc m => m a -> m (Async m a)
withAsync :: MonadConc m => m a -> (Async m a -> m b) -> m b
\end{haskellcode}

There is a test suite using \dejafu{}, whereas the test suite for
\emph{async} just runs most tests a single time, although one of them
is run 1000 times. Using \dejafu{} here to automatically seek out
interesting schedules is a much more principled approach.

Not all of the features of \emph{async} are supported by
\emph{async-dejafu}: as \verb|MonadConc| does not support bound
threads, those functions that use them have been omitted.

Of course, \emph{async} is just one library, and providing an
alternative library people will have to switch to is far from
optimal. However, until library authors start to use \dejafu{} and
\verb|MonadConc| directly, such alternatives will be needed to answer
the question ``why should I use this if I can't use it with all of my
familiar tools?''


  \subsection{Integration with Testing Frameworks}
  \label{sec:practice-integration}
  There are two popular libraries for unit testing in Haskell,
\emph{HUnit}\footnote{\hackage{HUnit}} and
\emph{tasty}\footnote{\hackage{tasty}}. From the perspective of the
user, both libraries are very similar, but from the perspective of the
implementer, they have different approaches to integration. Packages
\emph{hunit-dejafu}\footnote{\hackage{hunit-dejafu}} and
\emph{tasty-dejafu}\footnote{\hackage{tasty-dejafu}} provide
integration with both.

These packages provide a common set of testing functions, an analogue
of Test.DejaFu but constructing values representing individual tests
which the frameworks can run, rather than executing and printing
results directly:

\begin{haskellcode}
testAuto    :: (Eq a, Show a) => (forall t. ConcST t a) -> Test
testDejafu  :: Show a => (forall t. ConcST t a) -> String -> Predicate a -> Test
testDejafus :: Show a => (forall t. ConcST t a) -> [(String, Predicate a)] -> Test
\end{haskellcode}

Here \verb|Test| is the type of individual tests, from
\emph{HUnit}. The \emph{tasty} library uses \verb|TestTree|, which has
a similar purpose; it also uses \verb|TestName| rather than
\verb|String|. To complete the set, there are variants of these
functions for \verb|ConcIO|, and also taking the schedule bounds and
memory type as parameters. All of the testing functions are
implemented in terms of \verb|testDejafus'| and \verb|testDejafusIO'|.

The \emph{test-framework}\footnote{\hackage{test-framework}} library
is also in common use, however it supports integration with
\emph{HUnit}, and so needs no special support. of its own.

\subsubsection{HUnit}
\label{sec:practice-integration-hunit}

Tests in \emph{HUnit} are just thinly wrapped \verb|IO ()| actions,
which can be grouped together into collections and given names. The
testing model is very simple: a test fails if and only if it produces
some output. Test-running functions throw an exception if they fail,
terminating the rest of the test case.

\begin{haskellcode}
test :: Show a => MemType -> Bounds -> (forall t. ConcST t a)
  -> [(String, Predicate a)] -> Test
test memtype cb conc tests = case map toTest tests of
  [t] -> t
  ts  -> TestList ts
  where
    toTest (name, p) = TestLabel name . TestCase . assertString . showErr $ p traces
    traces = sctBound memtype cb conc
\end{haskellcode}
%$

Here, each \verb|(String, Predicate a)| pair is turned into a separate
test case. If there is only one, it is returned directly, otherwise
the test cases are grouped together into a \verb|TestList|.

The \verb|assertString| function is provided by \emph{HUnit}. The test
fails if its string argument is non-empty, \verb|showErr| here is a
function to pretty-print the failures.

\subsubsection{tasty}
\label{sec:practice-integration-tasty}

In contrast to the simple function-based method of \emph{HUnit},
\emph{tasty} has a much more complex approach. It defines a typeclass
\verb|IsTest| of things which can be converted to a unit test:

\begin{haskellcode}
test :: Show a => MemType -> Bounds -> (forall t. ConcST t a)
  -> [(TestName, Predicate a)] -> TestTree
test memtype cb conc tests = case map toTest tests of
  [t] -> t
  ts  -> testGroup "Deja Fu Tests" ts
  where
    toTest (name, p) = singleTest name $ ConcTest traces p
    traces = sctBound memtype cb conc
\end{haskellcode}
%$

This is very similar to the \emph{HUnit} approach, but instead of
constructing a test value directly, it constructs an intermediate
\verb|ConcTest| value. Note also that \emph{tasty} does not allow
nameless test lists.

\begin{haskellcode}
data ConcTest where
  ConcTest :: Show a => [(Either Failure a, Trace)] -> Predicate a -> ConcTest
  deriving Typeable

instance IsTest ConcTest where
  testOptions = return []

  run _ (ConcTest traces p) _ =
    let err = showErr $ p traces
     in return $ if null err then testPassed "" else testFailed err
\end{haskellcode}

The \emph{tasty} library is definitely more featureful than
\emph{HUnit}, but this comes at the cost of additional complexity for
developers trying to integrate new functionality.


\section{Future Research \& Conclusions}
\label{chap:conclusions}
This chapter describes the test execution of a concurrent program
written using the \verb|MonadConc| abstraction, not the execution of
programs using GHC's actual concurrency primitives. Of course, for
correctness of testing, there should be a correspondence between these
two models, see \chap{correctness}.

The execution of a concurrent program is considered to be the
sequential stepwise execution of \emph{primitive actions}, the most
basic things that a computation can do.


  \subsection{Community Reception}
  \label{sec:conclusions-reception}
  A talk about \dejafu{} was given at the 2015 Haskell Symposium. The
response was generally positive and the relaxed memory work was
initiated following discussions about how to integrate \dejafu{} with
libraries like \emph{monad-par}\footnote{\hackage{monad-par}} and
\emph{lvish}\footnote{\hackage{lvish}}.

Whilst no packages use \dejafu{} yet, the released version is quite
primitive compared to the current developmental version, and it is
hoped that the next release (and announcement) will inspire some
interest.


  \subsection{Related Work}
  \label{sec:conclusions-related}
  % Story: history of the algorithm, focussing on the development of
% PULSE (and its probabilistic approach) in particular, and then less
% in-depth surveys of the available tools in general. Finish with a
% brief note of the blog post which inspired the typeclass approach
% used in dejafu.

There is a tension in testing concurrent programs between
\emph{verification} and \emph{bug-finding}. For the former,
completeness is desirable, whereas for the latter completeness can be
sacrificed if the number of defects found in non-contrived examples is
not affected much. Furthermore, by sacrificing completeness, speed can
be gained, which is of great importance for developers running a test
suite as they develop.

% DPOR

Partial-order methods were first introduced in \citep{por}, which also
introduced the insight that a concurrent execution can be thought of
as a \emph{partial-order} of the dependent actions in the
system. Initially, these methods were based on a static analysis of
the program under test. Further developments in \citep{dpor} discuss
how the information needed for partial-order methods can be obtained
at runtime, and shows that this often leads to a reduction in the
amount of work done. This is because the static analysis is
necessarily \emph{conservative}, whereas the dynamic analysis has much
more complete information available to it.

Meanwhile, a different approach to testing concurrent programs was
being explored in \citep{pbound}, where executions exceeding some
pre-determined \emph{bound} were simply not done. Completeness was
sacrificed in return for more rapid results of testing, on the
assumption that (later to be validated by empirical studies such as
\citep{empirical}) that test cases could be written in such a way that
this wasn't a problem.

It was later shown in \citep{bpor} that these two approaches,
partial-order reduction and schedule bounding, can be unified. The
result is necessarily incomplete, however it can reduce the number of
executions tried to a far greater extent than either of the two
component methods alone. With the evidence that schedule bounding
isn't a problem in practice for testing, this became an enticing
method.

An assumption of key importance in concurrency testing is that all
nondeterminism arises from the scheduler. Most other sources, such as
random number generators, can be controlled for by (for example) using
a fixed seed. However, in the quest for ever more performance,
hardware manufacturers imposed \emph{relaxed memory} architectures on
programmers, where reads and writes done in parallel can give results
impossible under sequential consistency. \citep{rdpor} showed how this
additional source of nondeterminism can be handled, by modelling a
single level of cache (which corresponds to total-store order or
partial-store order) as simply a separate thread, committing writes to
memory.

% Probabilistic DPOR

A different approach to reducing the work done under a pure
partial-order reduction approach was taken in \citep{rapos}, which
uses random scheduling. Partial-order reduction is used to prune the
search space, but random decisions are made where there are still
multiple choices available. Random scheduling itself does not
necessarily work very well, as some partial orders have more
corresponding total orders than others, hence pruning the search space
like this is an effective way to increase the bug-finding ability of
random scheduling.

This work was furthered in \citep{racefuzzer}, which does away with
the partial-order reduction entirely in favour of a simpler race
condition detection approach. The algorithm consists of two phases:
firstly, all pairs of possibly-racing operations are computed;
secondly, for each pair, execution proceeds with a random
scheduler. When one of the identified statements is about to be
executed, that thread is instead postponed until another thread is
about to execute the other statement, the race is then randomly
resolved and execution continues. Rather than exploring all partial
orders, this approach is a probabilistic one, but is guaranteed to
only explore \emph{racing} partial orders. This approach has an
advantage in programs which have many non-racy partial orders, where
randomly choosing between them does not reliably produce a bug.

% PULSE

\textsc{Pulse} \citep{pulse} is a user-level scheduler for Erlang
programs implementing co-operative multi-tasking and an
instrumentation process which automatically modifies existing programs
to call out to this scheduler. \textsc{Pulse} works by only allowing
one of its threads to operate at a time, and to make scheduling
decisions around actions with side-effects: such as a process
receiving a message. It also allows interaction with uninstrumented
functions, which are treated as atomic, allowing tested subsystems to
be composed without needing to test the entire thing again. This is
not possible in general in \dejafu{} due to the support for relaxed
memory, which Erlang does not have. Scheduling decisions are made
randomly, using a provided seed, and a complete execution trace is
returned, which can be rendered into a graphical form showing the
interactions between threads to aid debugging.

Although \textsc{Pulse} is not a concurrency testing tool as such, it
is a core component of one, and testing can be done by simply trying
different random seeds. The authors report that the graphical traces
can often suggest potential race conditions which have not been
directly produced to a human reader.

Sen's 2008 work was then used in \citep{procrastination} to improve
race condition detection in Erlang. \textsc{Pulse} is used to generate
an execution trace, which is then examined for possible race
conditions, which are delayed and randomly resolved as in Sen's
work. The authors reported that this results in new bugs being found,
although in the cases where the procrastination was not necessary to
find the bug, performance degrades. This is because one test with
procrastination is actually several program executions with different
schedules.

% Empirical Studies

% Neither of these include POR techniques. Not sure if particularly
% relevant here, although definitely relevant for wider lit review.

%\todoinline{empirical: Concurrency testing using schedule bounding: an empirical
%  study}

%\todoinline{empirical2: Concurrency Testing Using Controlled Schedulers: An
%  Empirical Study}

% Typeclass Approach

Whilst the \verb|MonadConc| typeclass was structured to be similar to
the standard concurrency primitives, the inspiration for this
approach, and the basic idea behind how to do SCT in Haskell, was
provided by \citep{typeclass}. However, both the family of primitives
and the approach to testing have been significantly advanced.


  \subsection{Future Work}
  \label{sec:conclusions-future}
  There are a number of areas available for further exploration. 

\begin{itemize}
\item \textbf{Verification of \dejafu{}} \hfill

  Work has already begun on one aspect of this, the formalisation in
  Isabelle/HOL of prefix validity in the SCT implementation. The other
  open issues of verification are discussed further in
  \chap{correctness}, but to summarise, these are: correctness of
  primitive actions; granularity of scheduling decisions; generated
  schedule prefix validity; and result completeness.

\item \textbf{Memory model for GHC Haskell / C{-}{-}} \hfill

  In order to fully validate the testing stepwise executor, a
  formalism of the memory model of the primitives used is
  necessary. One way to approach this would be a formalism for all of
  the GHC Haskell primitives. As these are written in C{-}{-}, which
  has no memory model, a formalism of that would also be necessary.

  Work on formalising the C++11 memory model in \citep{c++11} may be
  of use here.

\item \textbf{Generating test cases for concurrent APIs} \hfill

  The QuickSpec tool, introduced in \citep{quickspec}, can generate
  laws that a collection of functions appear to hold based on random
  testing. It can be used as a way to easily generate test cases, if
  the user filters the output to laws that \emph{should} hold, rather
  than those which merely \emph{accidentally} hold.

  Given that concurrent programs are now easily testable, some
  QuickSpec-like tool which can generate laws about a
  concurrency-using API would be interesting and useful.

\item \textbf{Multi-level memory caching} \hfill

  The current approach taken for modelling relaxed memory assumes only
  a single level of cache. This works well for x86 processors, but not
  for other devices, such as GPUs. GPUs group cores together where
  each core has a cache, and each group also has a cache. This means
  writes can be visible to some but not all threads.

  A simple way to model this would be to make group assignment static,
  and to have more types of commit. This would require some
  implementation change, but is not a large difference in
  algorithm. The situation becomes much more complex if group
  assignment is \emph{not} static however, as this then introduces
  another source of nondeterminism.

\item \textbf{Application to distributed systems} \hfill

  There is no reason why different threads in a concurrent program
  need to operate on the same physical machine, as long as the
  programmer cannot detect this.

  The major difficulty is the possibility of communication
  \emph{failure}, which cannot happen when operating on a single
  machine. Another is the memory model. A single level of cache
  corresponds roughly to a central server with all communication going
  through it, rather than between nodes directly. This can be
  alleviated with multiple levels of caching, but still results in
  undesirable centralisation.

  Work on modelling concurrent data stores as replicated
  eventually-consistent data types in \citep{replicated} may be
  relevant.
\end{itemize}


\if@openright
  \cleardoublepage
\else
  \clearpage
\fi

\bibliography{references}
\bibliographystyle{plainnat}

\end{document}