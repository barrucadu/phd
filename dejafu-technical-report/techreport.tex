\documentclass[openright, dottedtoc, headinclude, footinclude=true, a4paper, numbers=noenddot, fontsize=10pt]{scrreprt}

\title{D\'{e}j\`{a} Fu Technical Report}
\author{Michael Walker}

%TC:group minted      [ignore] xall
%TC:group haskellcode [ignore] xall

% Stuff cargo-culted from jmct's thesis
\usepackage[eulermath, pdfspacing, nochapters]{classicthesis}

\usepackage{setspace}
\onehalfspacing

\usepackage{graphicx}
\usepackage{caption}
\usepackage{mathtools}
\usepackage{fancyvrb}
\usepackage{hyperref}
\usepackage{subcaption}
\usepackage{alltt}
\usepackage{color}

\usepackage{microtype}
\usepackage{xargs}
\usepackage{amsmath}
\usepackage{amssymb}
\usepackage{latexsym}
\usepackage{textcomp}
\usepackage{multicol}
\usepackage{multirow}
\usepackage{url}
\usepackage{siunitx}
\usepackage{appendix}
\usepackage{nomencl}
\usepackage[top=3cm,bottom=3cm]{geometry}

\usepackage{tikz}

\usepackage{csquotes}

\usepackage{minted}
\usemintedstyle{trac}
\newminted{haskell}{}

\usepackage{haskell}

\usepackage[vlined, boxed]{algorithm2e}

\hypersetup{
 colorlinks,
 citecolor=Red,
 linkcolor=Black,
 urlcolor=Blue
}

\usepackage[square]{natbib}

% Definitions

\newcommand{\defineword}[2]{%
\begin{description}%
    \item{\textbf{#1}} \hfill \\%
        {#2}%
\end{description}%
}

\newcommand{\definewordc}[3]{%
\begin{description}%
    \item{\textbf{#1}} \citep{#2}\hfill \\%
        {#3}%
\end{description}%
}

% Front-matter

\usepackage{tocloft}
\setcounter{tocdepth}{1}

\DeclareCaptionFormat{myformat}{#1#2#3\hrulefill}
\DeclareCaptionFormat{fnoline}{#1#2#3}
\captionsetup[figure]{format=myformat}

\newcommand{\frontmatter}{\cleardoublepage\pagenumbering{roman}}

\newcommand{\mainmatter}{\cleardoublepage\pagenumbering{arabic}}

\renewcommand*\thesection{\arabic{section}}
\setcounter{tocdepth}{2}

% Chapter quotes

\makeatletter
\newenvironment{chapquote}[2][2em]
  {\setlength{\@tempdima}{#1}%
   \def\chapquote@author{#2}%
   \parshape 1 \@tempdima \dimexpr\textwidth-2\@tempdima\relax%
   \itshape}
  {\par\normalfont\hfill--\
    \chapquote@author\hspace*{\@tempdima}\par\noindent\hrulefill\\[1cm]}
\makeatother

% Todo notes
\usepackage[colorinlistoftodos,prependcaption,textsize=small]{todonotes}
\let\marginpar\oldmarginpar

\newcommandx{\tocite}[2][1=]{\todo[linecolor=red,backgroundcolor=red!25,bordercolor=red,#1]{Cite: #2}}
\newcommandx{\todoinline}[1]{\todo[inline]{#1}}
\newcommandx{\todofig}[1]{\todo[inline]{Make figure: #1}}

% Departures in \chap{abstraction}:
\usepackage{framed}
\usepackage{amsthm}

\newenvironment{justspacing}{%
\def\FrameCommand{\hspace{0pt}}%
\MakeFramed {\advance\hsize-\width \FrameRestore}}%
{\endMakeFramed}

\newenvironment{departure}%
{\begin{justspacing}%
\noindent%
\paragraph{Departure}}%
{\hfill $\qed$ \end{justspacing}}

% Landscape pages
\usepackage{pdflscape}

% Misc
\newcommand{\quot}[2]{``\textit{#1}'' \cite{#2}}
\newcommand{\sect}[2]{\S\ref{sec:#1-#2}}
\newcommand{\chap}[1]{Section \ref{chap:#1}}
\newcommand{\dejafu}{D\'{e}j\`{a} Fu}
\newcommand{\defineprim}[3]{\defineword{#1 \textit{#2}}{#3}}
\newcommand{\dependent}{\nleftrightarrow}
\newcommand{\arr}{$\rightarrow$ }
\newcommand{\hackage}[1]{\url{https://hackage.haskell.org/package/#1}}
\newcommand{\github}[2]{\url{https://github.com/#1/#2}}
\newcommand{\dom}[1]{\mathbf{dom}~#1}
\newcommand{\ran}[1]{\mathbf{ran}~#1}
\newcommand{\wellf}[1]{\mathbf{well\_formed}~#1}

\begin{document}

\frontmatter

\begin{titlepage}
\begin{center}

\textsc{\Huge \dejafu{}: A Concurrency Testing Library for Haskell}\\[0.25cm]

\large{\github{barrucadu}{dejafu}}\\[0.5cm]

\textsc{\Large Michael Walker}
\end{center}

\vspace{3.5cm}

\centerline{\rule{40pt}{1pt} \textsc{\large Abstract} \rule{40pt}{1pt}}
\vspace{0.1cm}

Out of work in the formal verification and model checking field has
grown the topic of \emph{systematic concurrency testing} (SCT), also
known as \emph{stateless model checking}. This technique allows the
reliable, deterministic, and rigorous testing of concurrent programs,
and has enjoyed some success in the imperative and object-oriented
settings. We propose that the functional world could also benefit from
systematic concurrency testing, as GHC Haskell in particular provides
a very rich set of concurrency primitives.

We have developed a library for writing testable concurrent Haskell
programs, using a typeclass-abstraction to select based on the context
of use the concrete implementation to use: the primitives provided by
the run-time system, or emulated versions provided as part of a
testing framework.

This report discusses the design and implementation of this library,
called \dejafu{}, including some case studies and the community
reception of the initial version presented at the 2015 Haskell
Symposium.

\vspace{0.1cm}
\centerline{\rule{130pt}{1pt}}

\vfill

\begin{center}
Department of Computer Science\par
University of York \\[1,3cm]

November 2015
\end{center}
\end{titlepage}

\tableofcontents

%\listoffigures
%\listoftables

\mainmatter

\section{Introduction}
\label{chap:intro}
\begin{chapquote}{Terry Pratchett, Thief of Time\nocite{pratchett2001}}
  [\dejafu{} is] A martial art in which the user's limbs move in time
  as well as space, [\ldots] It is best described as ``the feeling
  that you have been kicked in the head this way before''.
\end{chapquote}

Once the scheduling behaviour of a program can be directed at will,
there is the ability to implement \emph{systematic}
testing. Systematic concurrency testing (SCT) comprises a family of
techniques, all with the same general aim: to try to find bugs in
concurrent programs, more reliably than running a program several
times. Within this scope, there are techniques which are
\emph{complete}, in that they find all possible results a program
could produce; and \emph{incomplete}, which do not make such a
guarantee.

SCT works by providing an initial sequence of scheduling decisions
intended to put the program into a new state. After this point some
deterministic scheduler is used, and the final trace examined to
produce new initial sequences. Typically the assumption is made that
all executions are \emph{terminating}: all possible sequences of
scheduling decisions will lead to a termination by deadlock or
otherwise. Another common assumption is that there is a \emph{finite}
number of possible schedules: this forbids finite but arbitrarily long
executions, as can be created with constructs such as spinlocks.

Systematic testing terminates when there are no more unique initial
sequences possible.


  \subsection{Parallelism vs Concurrency}
  \label{sec:intro-parconc}
  It is worth clarifying at this early stage some terminology which will
be frequently used throughout this report:

\defineword{Concurrency}{A programming methodology, using concepts
  such as threads, locks, and mutable variables to structure
  programs.}

\defineword{Parallelism}{An implementation detail, where a
  multiplicity of hardware components are used to execute distinct
  pieces of code simultaneously.}

Concurrency does not require parallelism, as demonstrated by the
single-core, single-processor computers of yore. Similarly,
parallelism does not require concurrency, as demonstrated by the
data-parallel x86 assembly instructions such as \verb|PMULHUW|, which
computes an element-wise multiplication of two vectors, each
multiplication in parallel.

Unrestricted concurrency is explicit and \emph{semantically visible}
\citep{concurrent}. The interleaved execution of threads, when
combined with mutable state, gives rise to nondeterminism. Semaphores
and locks give rise to termination errors in the form of deadlock and
livelock. Parallelism, in particular the parallel evaluation of
expressions, is \emph{semantically invisible} in a language without
side-effects.

Concurrency is often implemented using parallelism, and indeed a
concurrency abstraction can be used to guarantee parallelism (given
suitable hardware), for example by having the ability to restrict the
execution of individual threads to given processor cores.

Parallelism is largely outside the scope of this report, although it
does make an appearance in the discussion of relaxed memory in
\chap{abstraction}.


  \subsection{Testing Concurrent Programs}
  \label{sec:intro-testing}
  Systematic concurrency testing (SCT) \citep{dpor, pbound, heisenbugs,
  empirical} is a way of tackling the problem of nondeterminism when
writing tests. It aims to test a large number of schedules, whilst
typically also making use of local knowledge of the program to reduce
the number of schedules needed to be confident of an accurate
result. By testing many schedules, we can increase our confidence that
any bugs which have not been found are unlikely to be exhibited.

SCT overcomes the scheduling problem by forcing a concurrent program
to use a scheduler implemented as part of the testing framework:
either by overriding the concurrency primitives of the language, or by
modifying the program under test to call out to this new scheduler (as
in PULSE \citep{pulse}).

Once the scheduler is under control, schedules can be recorded and
replayed, giving reproducibility. Furthermore, by observing which
scheduling decisions are available at each decision point, possible
schedules can be systematically explored, making different decisions
on subsequent executions. Common methods of choosing schedules to take
are random \citep{empirical}, schedule bounding \citep{pbound}, and
partial-order reduction \citep{dpor}. The latter of these is
\emph{complete}: partial-order reduction will find all distinct
program states given enough time, in a more intelligent way than just
trying all schedules.


  \subsection{Scope}
  \label{sec:intro-scope}
  In the qualifying dissertation, the plan was to spend the months
October 2015 to July 2016 producing a tool for verifying concurrent
programs written in a core functional language.

Formal verification was investigated, starting with the formalisation
of the core data structures and functions of \dejafu{}. This was
attempted in Isabelle. Unfortunately, due to the nature of the core
data structure, this seemed to have limited prospects of success,
without significantly restructuring the implementation. The testing
framework in \dejafu{} is built around an $n$-ary prefix
tree. Formulations of recursive data structures in Isabelle tend to
have a static number of branches as a part of their definition,
whereas the \dejafu{} tree structure does not. This led to a lot of
difficulty in expressing the structure in a way amenable to automated
reasoning, and was abandoned as being too complicated for the return.

Discussions following the presentation of \dejafu{} at the Haskell
Symposium suggested further work. This resulted in support for
modelling computations with relaxed memory effects, and new basic
operations (see \sect{progress-dejafu}).

Now the intent is to return the focus to the area of testing, rather
than verification, see \chap{plan}.


  \subsection{Contributions}
  \label{sec:intro-contribs}
  The primary contribution of this report is a library for testing
concurrent Haskell programs based on a somewhat novel algorithm for
reducing the number of schedules to test.

Our contributions can be seen as follows:

\begin{itemize}
\item Existing results from the concurrency testing world have been
  ported to functional programming, an area historically without much
  in this topic, and verified to apply.

\item A partial-order reduction algorithm for systematic concurrency
  testing based on a combination of two others: bounded partial-order
  reduction \cite{bpor} and relaxed-memory DPOR \cite{rdpor}.
\end{itemize}


  \subsection{Report Roadmap}
  \label{sec:intro-roadmap}
  The report is divided into two broad parts:

Part I explores the problem of testing concurrent programs and how it
can be done, culminating in case studies of \dejafu{} applied to two
instances of pre-existing code, and one custom library in
\chap{casestudies}. In \chap{abstraction} we discuss our typeclass
abstraction for concurrency ans how it relates to GHC's standard
concurrency API in terms of functionality. \chap{execution} explains
how, given a monadic action polymorphic in the monad (as long as it
belongs to our typeclass) we can execute it with a given scheduler,
and \chap{sct} extends this to cover a systematic exploration of the
space of all schedules.

Part II discusses the real-world impact of this work with
\chap{practice} discussing the usage of \dejafu{} in combination with
existing code, and \chap{conclusions} summarising the community
reception to the idea and what is still to be done.


\section{Concurrency Abstraction}
\label{chap:abstraction}
Once the scheduling behaviour of a program can be directed at will,
there is the ability to implement \emph{systematic}
testing. Systematic concurrency testing (SCT) comprises a family of
techniques, all with the same general aim: to try to find bugs in
concurrent programs, more reliably than running a program several
times. Within this scope, there are techniques which are
\emph{complete}, in that they find all possible results a program
could produce; and \emph{incomplete}, which do not make such a
guarantee.

SCT works by providing an initial sequence of scheduling decisions
intended to put the program into a new state. After this point some
deterministic scheduler is used, and the final trace examined to
produce new initial sequences. Typically the assumption is made that
all executions are \emph{terminating}: all possible sequences of
scheduling decisions will lead to a termination by deadlock or
otherwise. Another common assumption is that there is a \emph{finite}
number of possible schedules: this forbids finite but arbitrarily long
executions, as can be created with constructs such as spinlocks.

Systematic testing terminates when there are no more unique initial
sequences possible.


  \subsection{The \texttt{MonadConc} Typeclass}
  \label{sec:abstraction-typeclass}
  Readers already familiar with Haskell's concurrency primitives may
find it enough to skim this section noting the syntactic differences
in the \dejafu{} variant.

\begin{departure}
  The few departures from the semantics of the traditional concurrency
  abstraction are highlighted like this.
\end{departure}

The \verb|MonadConc| typeclass has an instance for \verb|IO|, and so
existing code using only the functions generalised over can be made
suitable for testing quite simply. Existing code which makes use of
more functionality may require a light dusting of \verb|liftIO| where
it is safe to do so, which will be expanded upon in
\sect{abstraction}{typeclass-lifting}.

\subsection{Threads}
\label{sec:abstraction-typeclass-threads}

Threads let a program do multiple things at once. Every program has at
least one thread, which starts where \verb|main| does and runs until
the program terminates. A thread is the basic unit of concurrency. It
lets us pretend (with parallelism, it might even be true!) that we're
computing multiple things at once.

We can start a new thread with the function:

\begin{haskellcode}
fork :: MonadConc m => m () -> m (ThreadId m)
\end{haskellcode}

This starts evaluating its argument in a separate thread. It also
gives us back a (monad-specific) \verb|ThreadId| value, which we can
use to kill the thread later on, if we want.

A thread can query its own \verb|ThreadId|:

\begin{haskellcode}
myThreadId :: MonadConc m => m (ThreadId m)
\end{haskellcode}

In a real machine, there are of course a number of processors and
cores. It may be that a particular application of concurrency is only
a net gain if every thread is operating on a separate core, so that
threads are not interrupting each other. The GHC runtime refers to the
number of Haskell threads that can run truly simultaneously as the
number of \emph{capabilities}. We can query this value, and fork
threads which are bound to a particular capability:

\begin{haskellcode}
getNumCapabilities :: MonadConc m => m Int
forkOn :: MonadConc m => Int -> m () -> m (ThreadId m)
\end{haskellcode}

The \verb|forkOn| function interprets the capability number modulo the
value returned by \verb|getNumCapabilities|.

\begin{departure}
  \verb|getNumCapabilities| is not required to return a true
  result. The testing instances return ``2'' despite executing
  everything in the same capability, to encourage more
  concurrency. The \verb|IO| instance does return a true result.
\end{departure}

Sometimes we just want the special case of evaluating something in a
separate thread, for which we can use \verb|spawn| (implemented in
terms of \verb|fork|):

\begin{haskellcode}
spawn :: MonadConc m => m a -> m (CVar m a)
\end{haskellcode}

This returns a \verb|CVar| (\emph{Concurrent Variable}), to which we
can apply \verb|readCVar|, blocking until the computation is done and
the value is stored.

Threads are scheduled non-deterministically, where every time the
runtime decides to perform a context switch, one of the runnable
threads will be executed. Sometimes, however, a thread may be runnable
but also waiting for something to happen. The programmer can provide a
clue to the scheduler that another thread should be tried instead:

\begin{haskellcode}
yield :: MonadConc m => m ()
\end{haskellcode}

This gives any other thread the opportunity to execute instead of the
yielding one, but it is not \emph{required} to cause a context switch
except on co-operative multitasking systems.

\subsubsection{Threading and the Foreign Function Interface}
\label{sec:abstraction-typeclass-threads-ffi}

In order to accommodate Foreign Function Interface (FFI) calls which
may block, GHC provides a mechanism for \emph{binding} a Haskell
thread to an operating system thread. This allows FFI calls to be
managed by the operating system, unlike normal Haskell threads which
are managed by the runtime and multiplexed onto a smaller number of
operating system threads. This means that blocking FFI calls do not
necessarily block the entire program.

There is no \verb|MonadConc| equivalent of bound threads, as there
would be no way to reliably test this behaviour. Unfortunately, if
bound threads are required, \verb|IO| will have to be used.

A few predicates are provided for compatibility:

\begin{haskellcode}
rtsSupportsBoundThreads :: Bool
rtsSupportsBoundThreads = False

isCurrentThreadBound :: MonadConc m => m Bool
isCurrentThreadBound = return False
\end{haskellcode}

\subsection{Mutable State}
\label{sec:abstraction-typeclass-crefs}

Threading by itself is not really enough. We need to be able to
\emph{communicate} between threads: we've already seen an instance
of this with the \verb|spawn| function.

The simplest type of mutable shared state provided is the \verb|CRef|
(\emph{Concurrent Reference}). \verb|CRef|s are shared variables which
can be written to and read from:

\begin{haskellcode}
newCRef    :: MonadConc m => a -> m (CRef m a)
readCRef   :: MonadConc m => CRef m a -> m a
modifyCRef :: MonadConc m => CRef m a -> (a -> (a, b)) -> m b
writeCRef  :: MonadConc m => CRef m a -> a -> m ()
\end{haskellcode}

The \verb|readCRef| and \verb|writeCRef| functions are not
synchronised: it is possible for one thread to read from a \verb|CRef|
strictly after another thread has written to it and to observe an old
value!  This is expanded more upon in \sect{abstraction}{mem}. To
ensure that every thread sees a value as soon as it is written there
is a synchronised write function:

\begin{haskellcode}
atomicWriteCRef :: MonadConc m => CRef m a -> a -> m ()
\end{haskellcode}

However, synchronisation can slow down execution in a parallel
environment. Note that \verb|modifyCRef| is also synchronised.

As \emph{any} thread can write at \emph{any} time, we risk threads
overwriting each other's work! At least \verb|modifyCRef| is atomic:
no thread can update it between the value being read and the new value
being stored, as could happen if \verb|readCRef| and \verb|writeCRef|
were composed. Even so, \verb|CRef|s quickly fall down if we want to
do anything complicated.

\subsection{Mutual Exclusion}
\label{sec:abstraction-typeclass-cvars}

A \verb|CVar| is a shared variable under \emph{mutual exclusion}. It
has two possible states: \emph{full} or \emph{empty}. Writing to a
full \verb|CVar| blocks until it is empty, and reading or taking from
an empty \verb|CVar| blocks until it is full. There are also
non-blocking functions which return an indication of success:

\begin{haskellcode}
newEmptyCVar :: MonadConc m => m (CVar m a)
putCVar      :: MonadConc m => CVar m a -> a -> m ()
tryPutCVar   :: MonadConc m => CVar m a -> a -> m Bool
readCVar     :: MonadConc m => CVar m a -> m a
takeCVar     :: MonadConc m => CVar m a -> m a
tryTakeCVar  :: MonadConc m => CVar m a -> m (Maybe a)
\end{haskellcode}

Unfortunately, the mutual exclusion behaviour of \verb|CVar|s means
that computations can become \emph{deadlocked}. For example, deadlock
occurs if every thread tries to take from the same \verb|CVar|. The
GHC runtime can detect this in some situations (and will complain if
it does), and so can \dejafu{} in a more informative way.

\begin{departure}
  \dejafu{} can only detect deadlock to the same extent as GHC if
  every thread is annotated with which \verb|CVar|s it knows
  about. This is because GHC uses the garbage collector to solve this
  problem, which is out of the reach of \dejafu{}.
\end{departure}

There are also additional functions provided in the
Control\-.Concurrent\-.CVar and Control\-.Concurrent\-.CVar\-.Strict
modules.

\subsection{Exceptions}
\label{sec:abstraction-typeclass-excs}

Exceptions are a way to bail out of a computation early. Whether
they're a good solution to that problem is a question of style, but
they can be used to jump quickly to error handling code when
necessary. The basic functions for dealing with exceptions are:

\begin{haskellcode}
catch :: (Exception e, MonadConc m) => m a -> (e -> m a) -> m a
throw :: (Exception e, MonadConc m) => e -> m a
\end{haskellcode}

Where \verb|throw| causes the computation to jump back to the nearest
enclosing \verb|catch| capable of handling the particular
exception. As exceptions belong to a typeclass, rather than being a
concrete type, different \verb|catch| functions can be nested, to
handle different types of exceptions.

\begin{departure}
  The IO \verb|catch| function can catch exceptions from pure
  code. This is not true in general for MonadConc instances.  So some
  things which work normally may not work in testing, and we risk
  false negatives. This is a small cost, however, as exceptions from
  pure code are things like pattern match failures and evaluating
  \verb|undefined|, which are arguably bugs.
\end{departure}

Exceptions can be used to kill a thread:

\begin{haskellcode}
throwTo :: (Exception e, MonadConc m) => ThreadId m -> e -> m ()
killThread :: MonadConc m => ThreadId m -> m ()
\end{haskellcode}

These functions block until the target thread is in an appropriate
state to receive the exception.

What if we don't want our threads to be subject to destruction in this
way? A thread also has a \emph{masking state}, which can be used to
block exceptions from other threads. There are three masking states:
\emph{unmasked}, in which a thread can have exceptions thrown to it;
\emph{interruptible}, in which a thread can only have exceptions
thrown to it if it is blocked; and \emph{uninterruptible}, in which a
thread cannot have exceptions thrown to it. When a thread is started,
it inherits the masking state of its parent. We can also execute a
subcomputation with a new masking state:

\begin{haskellcode}
mask :: MonadConc m
  => ((forall a. m a -> m a) -> m b) -> m b
uninterruptibleMask :: MonadConc m
  => ((forall a. m a -> m a) -> m b) -> m b
\end{haskellcode}

A thread can be forked and given a function to reset the masking
state:

\begin{haskellcode}
forkWithUnmask :: MonadConc m
  => ((forall a. m a -> m a) -> m ()) -> m (ThreadId m)
forkOnWithUnmask :: MonadConc m => Int
  -> ((forall a. m a -> m a) -> m ()) -> m (ThreadId m)
\end{haskellcode}

We can also fork a thread and call a supplied function when the thread
is about to terminate, which is useful for informing the parent when a
child terminates, for example:

\begin{haskellcode}
forkFinally :: MonadConc m => m a
  -> (Either SomeException a -> m ()) -> m (ThreadId m)
\end{haskellcode}

The \verb|SomeException| type is the top of the exception hierarchy,
and so can be used to catch all exceptions.

\subsection{Lifting Actions into \texttt{MonadConc}}
\label{sec:abstraction-typeclass-lifting}

If the programmer needs to make use of \verb|IO| actions, rather than
\verb|MonadConc| actions, then this can be achieved by adding a
\verb|MonadIO| context and using \verb|liftIO|. However, this can
easily compromise the results of testing, as the test runner cannot
peek inside \verb|IO| actions (that's why the typeclass exists in the
first place!). Thus, it is only safe if:

\begin{itemize}
\item \emph{The action is atomic and synchronised.}

  Otherwise the test framework will possibly miss schedules which lead
  to a bug.
\item \emph{The action is deterministic} (when executed as part of a
  computation with a deterministic schedule).

  Otherwise the fundamental assumption behind the testing methodology
  is false, and no guarantees about completeness can be made.
\item \emph{The action cannot block on the action of another thread.}

  Otherwise test execution may deadlock.
\end{itemize}

State transformer actions can also be lifted directly into a
\verb|MonadConc| computation using the \verb|primitive| function from
\verb|PrimMonad|. This is only safe if the requirements above hold.


  \subsection{Software Transactional Memory}
  \label{sec:abstraction-stm}
  \verb|CVar|s are nice, until we need more than one, and find they need
to be kept synchronised. As we can only claim \emph{one} \verb|CVar|
atomically, it seems we need to introduce a \verb|CVar| to control
access to \verb|CVar|s! But that would be unwieldy and prone to bugs.

\emph{Software transactional memory} (STM) \citep{stm} is the
solution. STM uses \verb|CTVar|s, or \emph{Concurrent Transactional
  Variables}, and is based upon the idea of atomic
\emph{transactions}. An STM transaction consists of one or more
operations over a collection of \verb|CTVar|s, where a transaction may
be aborted part-way through depending on their values. If the
transaction fails, \emph{none of its effects take place}, and the
thread blocks until the transaction can succeed. This means we need to
limit the possible actions in an STM transaction to those which can be
safely undone and repeated, so we have another typeclass,
\verb|MonadSTM|.

\verb|CTVar|s always contain a value, as shown in the types of the
functions:

\begin{haskellcode}
newCTVar   :: MonadSTM s => a -> s (CTVar s a)
readCTVar  :: MonadSTM s => CTVar s a -> s a
writeCTVar :: MonadSTM s => CTVar s a -> a -> s ()
\end{haskellcode}

If we read a \verb|CTVar| and don't like the value it has, the
transaction can be aborted, and the thread will block until at least
one of the referenced \verb|CTVar|s has been mutated:

\begin{haskellcode}
retry :: MonadSTM s => s a
check :: MonadSTM s => Bool -> s ()
\end{haskellcode}

We can also try executing a transaction, and do something else if it
fails:

\begin{haskellcode}
orElse :: MonadSTM s => s a -> s a -> s a
\end{haskellcode}

The nice thing about STM transactions is that they \emph{compose}. We
can take small transactions and build bigger transactions from them,
and the whole is still executed atomically. This means we can do
complex state operations involving multiple shared variables without
worrying!

Each \verb|MonadConc| has an associated \verb|MonadSTM|, and can
execute transactions of it atomically:\footnote{Here \texttt{STMLike}
  is a \emph{type family}, it is used to relate the \texttt{MonadConc}
  and \texttt{MonadSTM} typeclasses.}

\begin{haskellcode}
atomically :: MonadConc m => STMLike m a -> m a
\end{haskellcode}

The instance of \verb|MonadConc| for \verb|IO| uses \verb|STM| as its
\verb|MonadSTM|.

STM can also use exceptions:

\begin{haskellcode}
throwSTM :: (Exception e, MonadSTM s) => e -> s a
catchSTM :: (Exception e, MonadSTM s) => s a -> (e -> s a) -> s a
\end{haskellcode}

If an exception propagates uncaught to the top of a transaction, that
transaction is aborted and the exception is re-thrown in the thread.

There are utility functions for \verb|CTVar|s provided in
Control\-.Concurrent\-.STM\-.CTVar, and an STM equivalent of
\verb|CVar|s in Control\-.Concurrent\-.STM\-.CTMVar.


  \subsection{Memory Model}
  \label{sec:abstraction-mem}
  There are three memory models supported in \dejafu{}:

\defineword{Sequential Consistency}{The most intuitive model: a
  program behaves as a simple interleaving of the actions in different
  threads. When a \texttt{CRef} is written to, that write is
  immediately visible to all threads.}

\defineword{Total Store Order (TSO)}{Each thread has a write buffer. A
  thread sees its writes immediately, but other threads will only see
  writes when they are committed, which may happen later. Writes are
  committed in the same order that they are created.}

\defineword{Partial Store Order (PSO)}{A relaxation of TSO where each
  thread has a write buffer for each \texttt{CRef}. A thread sees its
  writes immediately, but other threads will only see writes when they
  are committed, which may happen later. Writes to different
  \texttt{CRef}s are not necessarily committed in the same order that
  they are created.}

The memory model only makes a difference for unsynchronised
operations, such as \verb|readCRef|, \verb|writeCRef|, and
\verb|readForCAS|.

The default memory model for testing is TSO, as that most accurately
models the behaviour of modern x86 processors. The use of a relaxed
memory model does cause some blow-up in the number of schedules tested
when unsynchronised operations are used, but as most of the
concurrency primitives are synchronised this tends to be fairly
contained.


\section{Program Execution}
\label{chap:execution}
Once the scheduling behaviour of a program can be directed at will,
there is the ability to implement \emph{systematic}
testing. Systematic concurrency testing (SCT) comprises a family of
techniques, all with the same general aim: to try to find bugs in
concurrent programs, more reliably than running a program several
times. Within this scope, there are techniques which are
\emph{complete}, in that they find all possible results a program
could produce; and \emph{incomplete}, which do not make such a
guarantee.

SCT works by providing an initial sequence of scheduling decisions
intended to put the program into a new state. After this point some
deterministic scheduler is used, and the final trace examined to
produce new initial sequences. Typically the assumption is made that
all executions are \emph{terminating}: all possible sequences of
scheduling decisions will lead to a termination by deadlock or
otherwise. Another common assumption is that there is a \emph{finite}
number of possible schedules: this forbids finite but arbitrarily long
executions, as can be created with constructs such as spinlocks.

Systematic testing terminates when there are no more unique initial
sequences possible.


  \subsection{Primitive Actions}
  \label{sec:execution-primops}
  There are currently 31 primitive actions used to construct the testing
instances of \verb|MonadConc|, one of which only arises when testing
under relaxed memory. These primitive actions contain a continuation,
allowing individual actions to be composed into larger execution
sequences. Each thread of execution consists of such a sequence,
terminated by the \verb|AStop| primitive, which has no continuation
and signals the termination of the thread.

\vspace{0.25cm}
\noindent \textsc{Threading} \vspace{-0.5\parskip}

\begin{primtable}
\defineprimT{AFork}{(unmask \arr action) (thread\_id \arr action)}{%
  Create a new thread from the first action, and continue executing
  the current thread with the second.}
\defineprimT{AMyTId}{(thread\_id \arr action)}{%
  Continue execution of the current thread by querying the thread
  identifier.}
\defineprimT{AYield}{action}{%
  Execute the given action, but also signify to the scheduler that it
  may be worth running a different thread now.}
\defineprimT{AStop}{}{%
  Terminate the current thread.}
\end{primtable}

\noindent \textsc{\texttt{CRef}s} \vspace{-0.5\parskip}

\begin{primtable}
\defineprimT{ANewRef}{a (cref a \arr action)}{%
  Construct a new \texttt{CRef} and give it to the continuation.}
\defineprimT{AReadRef}{(cref a) (a \arr action)}{%
  Read the currently visible value of a \texttt{CRef}.}
\defineprimT{AReadRefCas}{(cref a) (ticket a \arr action)}{%
  Produce a \texttt{Ticket} from the currently visible state of a
  \texttt{CRef}.}
\defineprimT{APeekTicket}{(ticket a) (a \arr action)}{%
  Get the value out of a \texttt{Ticket}.}
\defineprimT{AModRef}{(cref a) (a \arr (a, b)) (b \arr action)}{%
  Commit all pending writes and atomically modify the value within a
  \texttt{CRef}.}
\defineprimT{AModRefCas}{(cref a) (a \arr (a, b)) (b \arr action)}{%
  Commit all pending writes and atomically modify the value within a
  \texttt{CRef} using a compare-and-swap.}
\defineprimT{AWriteRef}{(cref a) a action}{%
  Update the value of a \texttt{CRef}. The updated value is visible to
  the current thread immediately.}
\defineprimT{ACasRef}{(cref a) (ticket a) a ((succeeded?, ticket a) \arr action)}{%
  Update the value of a \texttt{CRef} if it hasn't changed since the
  ticket was produced..}
\defineprimT{ACommit}{thread\_id cref\_id}{%
  Make the last write to the given \texttt{CRef} by that thread
  visible to all threads.}
\end{primtable}

\noindent \textsc{\texttt{CVar}s} \vspace{-0.5\parskip}

\begin{primtable}
\defineprimT{ANewVar}{cvar a \arr action}{%
  Construct a new \texttt{CVar} and give it to the continuation.}
\defineprimT{APutVar}{(cvar a) a action}{%
  Block until the \texttt{CVar} is empty and put a value into it.}
\defineprimT{ATryPutVar}{(cvar a) a (succeeded? \arr action)}{%
  Try to put a value into the \texttt{CVar} without blocking.}
\defineprimT{AReadVar}{(cvar a) (a \arr action)}{%
  Block until the \texttt{CVar} is full and read its value.}
\defineprimT{ATakeVar}{(cvar a) (a \arr action)}{%
  Block until the \texttt{CVar} is full and take its value.}
\defineprimT{ATryTakeVar}{(cvar a) (Maybe a \arr action)}{%
  Try to take the value from a \texttt{CVar} without blocking.}
\end{primtable}

\noindent \textsc{Exceptions} \vspace{-0.5\parskip}

\begin{primtable}
\defineprimT{AThrow}{exception}{%
  Raises an exception in the current thread, terminating the current
  execution.}
\defineprimT{AThrowTo}{exception action}{%
  Raises an exception in the other thread, blocking if the other
  thread has exceptions masked.}
\defineprimT{ACatching}{(exception \arr handler) action continuation}{%
  Registers a new exception handler for the duration of the inner
  action.}
\defineprimT{APopCatching}{action}{%
  Remove the exception handler from the top of the stack.}
\defineprimT{AMasking}{masking\_state (unmask \arr action) continuation}{%
  Executes the inner action under a new masking state, and also gives
  it a function to reset the masking state.}
\defineprimT{AResetMask}{set? inner? masking\_state action}{%
  Sets the masking state.}
\end{primtable}

\noindent \textsc{Software Transactional Memory} \vspace{-0.5\parskip}

\begin{primtable}
\defineprimT{AAtom}{transaction continuation}{%
  Execute an STM transaction atomically.}
\defineprimT{SNew}{a (ctvar a \arr action)}{%
  Create a new \texttt{CTVar} containing the given value.}
\defineprimT{SRead}{(ctvar a) (a \arr action)}{%
  Read the current value of a \texttt{CTVar}.}
\defineprimT{SWrite}{(ctvar a) a action}{%
  Update a \texttt{CTVar}.}
\defineprimT{SThrow}{exception}{%
  Throw an exception, aborting the current execution flow.}
\defineprimT{SCatch}{(exception \arr handler) action continuation}{%
  Registers a new exception handler for the duration of the action.}
\defineprimT{SRetry}{}{%
  Abort the current transaction.}
\defineprimT{SOrElse}{transaction transaction continuation}{%
  Try executing the first transaction, if it fails, execute the
  second.}
\end{primtable}

\noindent \textsc{Testing Annotations} \vspace{-0.5\parskip}

\begin{primtable}
\defineprimT{AKnowsAbout}{(Either cvar ctvar) action}{%
  Record that the thread has access to the given variable.}
\defineprimT{AForgets}{(Either cvar ctvar) action}{%
  Record that the thread no longer has access to the given variable.}
\defineprimT{AAllKnown}{action}{%
  Record that all variables the thread knows about have been
  reported.}
\end{primtable}

\noindent \textsc{Miscellaneous} \vspace{-0.5\parskip}

\begin{primtable}
\defineprimT{AReturn}{action}{%
  Execute the given action.}
\defineprimT{ALift}{monadic\_action}{%
  Execute an action from the underlying monad.}
\end{primtable}


  \subsection{Stepwise Execution}
  \label{sec:execution-stepwise}
  Each thread is represented as a sequence of primitive actions, where
the continuation of one action is the next action that a thread will
take.

Execution of an entire computation proceeds in a stepwise manner: a
thread is chosen by the scheduler, its primitive action is executed,
and a new action is returned to be executed by that thread in the next
step. In order to model all the possible effects, this step function
returns a completely new set of threads and identifier source.

Normally only the chosen thread will be modified, but this allows, for
example, \verb|AFork| to create new threads, and \verb|AThrowTo| to
terminate other threads. The identifier source is used to produce
fresh thread, \verb|CRef|, \verb|CVar|, and \verb|CTVar| identifiers.

After the scheduler is consulted to choose a thread, a function is
called to evaluate the next step of that thread. The simplest thing
that a thread can do is to stop, which will serve as a useful
example:

\begin{haskellcode}
stepStop = simple (kill tid threads) Stop
\end{haskellcode}

The \verb|simple| function is defined as follows:

\begin{haskellcode}
simple threads' act = return (Right (threads', idSource, act, wb))
\end{haskellcode}

The effect of \verb|stepStop| can be read as: remove the current
thread from the map of live threads (the \verb|kill tid threads| bit);
and then return the new thread map and the name of the action to
appear in the trace (\verb|Stop|, here). \verb|simple| is a helper
function for actions which don't modify the identifier source or have
any relaxed-memory effects.

The \verb|Right|\footnote{The type \texttt{Either a b} type is
  commonly used to represent computations that might fail with an
  error value. By convention \texttt{Left err} means that the
  computation failed with reason \texttt{err}, and \texttt{Right x}
  means that the computation succeeded, producing \texttt{x}.}
indicates that the action completed successfully. There are a few
different possible failures, such as an uncaught exception, which will
terminate the current thread. If the main thread is terminated like
so, the entire computation terminates with that failure.

Another simple action that a thread can perform is \verb|AReturn|:

\begin{haskellcode}
stepReturn c = simple (goto c tid threads) Return
\end{haskellcode}

The effect of \verb|stepReturn| can be read as: extract the
continuation of the action and replace the continuation of the current
thread with it.

\subsection{Threading}
\label{sec:execution-stepwise-threading}

Threads are represented as a record type:

\begin{haskellcode}
data Thread n r s = Thread
  { continuation :: Action n r s
  , blocking     :: Maybe BlockedOn
  , handlers     :: [Handler n r s]
  , masking      :: MaskingState
  , known        :: [Either CVarId CTVarId]
  , fullknown    :: Bool
  }
\end{haskellcode}

The \verb|continuation| field contains the action to execute in the
next step. \verb|blocking| records whether the thread is blocked and,
if so, what it is waiting for. \verb|handlers| is the stack of
exception handlers. \verb|masking| is the masking state. \verb|known|
is the collection of \verb|CVar|s and \verb|CTVar|s the thread is
known to have access to. \verb|allknown| indicates whether
\verb|_concAllKnown| has been called in this thread. If
\verb|allknown| is \verb|True| for all threads, then detection of
deadlocks only involving a subset of the threads is possible.

Furthermore, a simple map is used to keep track of all the threads
currently extant:

\begin{haskellcode}
type Threads n r s = Map ThreadId (Thread n r s)
\end{haskellcode}

There are a number of functions to ease manipulating this structure,
\verb|kill| and \verb|goto| are two. Another is \verb|launch|, used to
create a new thread:

\begin{haskellcode}
stepFork a b = return result where
  result = Right (threads', idSource', Fork newtid, wb)

  threads' = goto (b newtid) tid (launch tid newtid a threads)
  (idSource', newtid) = nextTId idSource
\end{haskellcode}

This is somewhat more complex than the two examples seen before, as it
involves two modifications to the thread map: firstly, a new thread is
created (and inherits the masking state of its parent), secondly the
continuation of the current thread is updated. Here \verb|simple|
cannot be used, as the identifier source is being modified.

Both \verb|AMyTId| and \verb|AYield| follow the \verb|simple| pattern:

\begin{haskellcode}
stepMyTId c = simple (goto (c tid) tid threads) MyThreadId
stepYield c = simple (goto c tid threads) Yield
\end{haskellcode}

Note that \verb|AYield| does not have any special implementation
here. Its effect is purely a scheduling concern; from the point of
view of updating the state of the system, it is no different to
\verb|AReturn|.

\subsection{\texttt{CRef}s and Relaxed Memory}
\label{sec:execution-stepwise-cref}

\begin{haskellcode}
newtype CRef r a = CRef (CRefId, r (Map ThreadId a, Integer, a))
\end{haskellcode}

A \verb|CRef| is implemented as a mutable reference containing a
\emph{globally visible} value, a counter of how many write commits
there have been, and a number of \emph{thread-specific} values. These
thread-specific values correspond to uncommitted writes, and so only
show up when using relaxed memory.\footnote{The
  \texttt{name@(pattern)} syntax is called an \emph{as-pattern}. The
  name before the \texttt{@} can refers to the entire value.}

\begin{haskellcode}
newtype Ticket a = Ticket (CRefId, Integer, a)
\end{haskellcode}

A \verb|Ticket| just keeps track of the \verb|CRef| it was produced
for, what the write count was when it was produced, and the
thread-specific value seen by its creating thread. Checking if a
\verb|CRef| has been modified since the creation of a \verb|Ticket|
becomes very simple with this implementation: the write counts are
compared.

\begin{haskellcode}
stepWriteRef cref@(CRef (crid, _)) a c = case memtype of
\end{haskellcode}

There are three memory models supported by \dejafu{}, each of which
has a different implementation for writing to a \verb|CRef|. Firstly,
sequential consistency. This does not have any relaxed memory
effects:\footnote{This is an emple of \emph{do notation}, which is a
  convenient synctatic sugar for composition of monadic functions.}

\begin{haskellcode}
  SequentialConsistency -> do
    writeImmediate cref a
    simple (goto c tid threads) (WriteRef crid)
\end{haskellcode}

The \verb|writeImmediate| function writes to the globally visible
value, and clears the thread-specific values.

Total store order (TSO) corresponds to an architecture where each
thread has its own cache: writes made by a thread will be cached, but
they will be committed in that same order to main memory:

\begin{haskellcode}
  TotalStoreOrder -> do
    wb' <- bufferWrite wb tid cref a tid
    return $ Right
      (goto c tid threads, idSource, WriteRef crid, wb')
\end{haskellcode}
%$

The \verb|bufferWrite| function appends a write to the relevant write
buffer, in this case the one corresponding to that thread. Total store
order corresponds to modern x86 and x86\_64 processors.

Partial store order (PSO) is a more relaxed version of total store order,
where the writes a thread makes may not necessarily be committed in
order. It can be modelled by giving each \verb|CRef| a write buffer,
rather than each thread:

\begin{haskellcode}
  PartialStoreOrder -> do
    wb' <- bufferWrite wb crid cref a tid
    return $ Right
      (goto c tid threads, idSource, WriteRef crid, wb')
\end{haskellcode}
%$

Both the TSO and PSO cases update the thread-specific map. A thread
will always see the writes it has made, but other threads may not.

The compare-and-swap write is a little different, as this has the
effect of being a memory barrier: any uncommitted writes to any
\verb|CRef| are committed before the CAS is done, and the result is
immediately globally visible. There is a \verb|synchronised| function
for actions which have this barrier property:\footnote{The \texttt{\$}
  operator is function application, but with a very low
  precedence. This makes it convenient for avoiding parentheses, which
  can be more readable when multi-line expressions are involved.}

\begin{haskellcode}
stepCasRef cref@(CRef (crid, _)) tick a c = synchronised $ do
  (suc, tick') <- casCRef cref tid tick a
  simple (goto (c (suc, tick')) tid threads) (CasRef crid suc)
\end{haskellcode}
%$

The \verb|casCRef| function here generates a new \verb|Ticket|,
compares the write counts, and then swaps the value. It is provided,
rather than the logic be included verbatim, as it is used again in the
implementation of \verb|stepModRefCas|.

The implementation of \verb|synchronised| is as follows:

\begin{haskellcode}
synchronised ma = do
  writeBarrier wb
  res <- ma

  case res of
    Right (threads', idSource', act', _) -> return
      (Right (threads', idSource', act', emptyBuffer))
    _ -> return res
\end{haskellcode}

Here \verb|writeBarrier| commits all cached writes. The action is then
executed, and an empty write buffer returned. This is why
\verb|simple| can be used in the implementation of \verb|stepModRef|
despite the write buffer being changed.

Cached writes can be committed to the globally visible value (at which
point the thread-specific values disappear) by executing an
\verb|ACommit| action:

\begin{haskellcode}
stepCommit c t = do
  wb' <- case memtype of
    TotalStoreOrder   -> commitWrite wb t
    PartialStoreOrder -> commitWrite wb c

  return (Right (threads, idSource, CommitRef t c, wb'))
\end{haskellcode}

The \verb|commitWrite| function is used here. Note how the invocation
differs between the cases: under TSO, the cache corresponding to the
thread is used; whereas under PSO, the cache corresponding to the
\verb|CRef| is used. There is no case for sequential consistency here,
as commit actions are not explicitly introduced by the program under
test; they are introduced by the test runner when executing under a
relaxed memory model. This is expanded upon in
\sect{execution}{scheduling}.

\begin{haskellcode}
stepModRef cref@(CRef (crid, _)) f c = synchronised $ do
  val <- readCRef cref tid
  let (new, ret) = f val
  writeImmediate cref new
  simple (goto (c ret) tid threads) (ModRef crid)

stepModRefCas cref@(CRef (crid, _)) f c = synchronised $ do
  tick@(Ticket (_, _, val)) <- readForTicket cref tid
  let (new, ret) = f val
  casCRef cref tid tick new
  simple (goto (c ret) tid threads) (ModRefCas crid)
\end{haskellcode}

The modification is made globally visible by the use of
\verb|writeImmediate|. Here \verb|casCRef| is used in the
implementation \verb|stepModRefCas|, because it is strict in the value
written, which \verb|writeImmediate| is not.

Reading a reference is quite simple:

\begin{haskellcode}
stepReadRef cref@(CRef (crid, _)) c = do
  val <- readCRef cref tid
  simple (goto (c val) tid threads) (ReadRef crid)

stepReadRefCas cref@(CRef (crid, _)) c = do
  tick <- readForTicket cref tid
  simple (goto (c tick) tid threads) (ReadRefCas crid)
\end{haskellcode}

The \verb|readCRef| function checks if there is a cached value for
that thread and, if so, returns it. Otherwise it returns the globally
visible value. The \verb|readForTicket| function behaves similarly,
but returns a \verb|Ticket| rather than the current value.

Creating a new \verb|CRef| looks a little more involved, but it is
really quite simple. Firstly, a new mutable reference containing the
given value and no thread-specific values is created; then this is
packaged up into a \verb|CRef| by giving it a unique identifier;
finally the thread is given it:
\pagebreak
\begin{haskellcode}
stepNewRef a c = do
  ref <- newRef (empty, 0, a)

  let (idSource', newcrid) = nextCRId idSource
  let threads' = goto (c (CRef (newcrid, ref))) tid threads

  return (Right (threads', idSource', NewRef newcrid, wb))
\end{haskellcode}

\subsection{\texttt{CVar}s}
\label{sec:execution-stepwise-cvar}

As there are no relaxed memory issues to worry about, the \verb|CVar|
implementation is in many respects simpler than that for
\verb|CRef|. However, \verb|CVar|s have their own unique features:
specifically, blocking. Attempting to read or take from an empty
\verb|CVar| blocks the thread, and attempting to put into a full
\verb|CVar| does the same.

Firstly, creating a new \verb|CVar| is almost identical to creating a
new \verb|CRef|:

\begin{haskellcode}
stepNewVar c = do
  ref <- newRef Nothing

  let (idSource', newcvid) = nextCVId idSource
  let threads' = knows [Left newcvid] tid
                 (goto (c (CVar (newcvid, ref))) tid threads)

  return (Right (threads', idSource', New newcvid, wb))
\end{haskellcode}

There is a difference, however. The \verb|knows| function is used to
record that a thread has a reference to a \verb|CVar| or \verb|CTVar|,
which can be used to improve deadlock detection.

Fortunately there is a lot of similarity between the \verb|CVar|
functions, which makes them easy to follow.

\begin{haskellcode}
stepPutVar cvar@(CVar (cvid, _)) a c = synchronised $ do
  (success, threads', woken) <-
    putIntoCVar cvar a c tid threads
  simple threads' $ if success
    then Put cvid woken else BlockedPut cvid

stepTryPutVar cvar@(CVar (cvid, _)) a c = synchronised $ do
  (success, threads', woken) <-
    tryPutIntoCVar cvar a c tid threads
  simple threads' (TryPut cvid success woken)
\end{haskellcode}
%$

Note that these functions are all \verb|synchronised|, and so commit
\verb|CRef| writes. All \verb|CVar| actions (other than
\verb|ANewVar|) are.

\pagebreak
\begin{haskellcode}
stepReadVar cvar@(CVar (cvid, _)) c = synchronised $ do
  (success, threads', _) <-
    readFromCVar cvar c tid threads
  simple threads' $ if success
    then Read cvid else BlockedRead cvid

stepTakeVar cvar@(CVar (cvid, _)) c = synchronised $ do
  (success, threads', woken) <-
    takeFromCVar cvar c tid threads
  simple threads' $ if success
    then Take cvid woken else BlockedTake cvid

stepTryTakeVar cvar@(CVar (cvid, _)) c = synchronised $ do
  (success, threads', woken) <-
    tryTakeFromCVar cvar c tid threads
  simple threads' (TryTake cvid success woken)
\end{haskellcode}
%$

The \verb|putInto|/\verb|readFrom|/\verb|takeFromCVar| functions, and
their \verb|try| variants, handle waking threads which are blocked in
the appropriate way on that \verb|CVar|. All such threads get woken at
once, and a list of them is returned to be included in the execution
trace. This is somewhat different to how GHC does things, where
threads blocked on an \verb|MVar| are woken up in a FIFO order to
guarantee fairness. The current behaviour was chosen because there is
no standard for Haskell concurrency, and so that ordering is only an
implementation detail which could, conceivably, be changed in the
future if another were judged more desirable.

\subsection{Exceptions}
\label{sec:execution-stepwise-exception}

A thread has both a stack of exception handlers, and a masking
state. The handler stack affects all exceptions raised in the thread,
whereas the masking state only affects exceptions raised by
\verb|AThrowTo|.

\begin{haskellcode}
stepCatching h ma c = simple threads' Catching where
  a     = runCont ma      (APopCatching . c)
  e exc = runCont (h exc) (APopCatching . c)

  threads' = goto a tid (catching e tid threads)
\end{haskellcode}

This introduces the \verb|APopCatching| action, at the end of both the
enclosed action, and at the end of the handler. This is necessary
because actions are executed one at a time, and so we cannot just run
the entire inner computation in one go and then check the result for
an uncaught exception.

\begin{haskellcode}
stepPopCatching a = simple threads' PopCatching where
  threads' = goto a tid (uncatching tid threads)
\end{haskellcode}

The \verb|catching| and \verb|uncatching| functions are used to modify
the handler stack, corresponding to push and pop operations.

When an exception is thrown, it may not be able to be handled by the
topmost handler, as there are exceptions of many types:

\begin{haskellcode}
stepThrow e = case propagate e tid threads of
    Just threads' -> simple threads' Throw
    Nothing -> return (Left UncaughtException)
\end{haskellcode}

The \verb|propagate| function pops from the stack of exception
handlers until one is found capable of handling that type of
exception. It then jumps to the handler, and returns the new thread
map. If no handler was found, the thread is killed by the uncaught
exception.

Throwing an exception to another thread is significantly more
complicated, and is also a \verb|synchronised| operation:

\begin{haskellcode}
stepThrowTo t e c = synchronised $
  let threads' = goto c tid threads
      blocked  = block (OnMask t) tid threads
  in if interruptible (lookup t threads)
     then case propagate e t threads' of
            Just threads'' -> simple threads'' (ThrowTo t)
            Nothing
              | t == 0 -> return (Left UncaughtException)
              | otherwise ->
                simple (kill t threads') (ThrowTo t)
     else simple blocked (BlockedThrowTo t)
\end{haskellcode}
%$

Firstly, whether the thread is interruptible is checked. If it's not,
the current thread is blocked. If it is interruptible, then the
exception is propagated through its handler stack. If a handler is
found, the thread jumps to it, throwing away whatever it was going to
do next. If a handler is not found, the thread is killed. If the
thread is killed and is the main thread, the entire computation
terminates.

\begin{haskellcode}
stepMasking m ma c = simple threads' (SetMasking False m)
  where
  a = runCont (ma umask) (AResetMask False False m' . c)

  m' = masking (lookup tid threads)
  umask mb = do
    resetMask True m'
    b <- mb
    resetMask False m
    return b
  resetMask typ ms = cont (\k -> AResetMask typ True ms (k ()))

  threads' = goto a tid (mask m tid threads)
\end{haskellcode}

Similarly to \verb|ACatching|, \verb|AMasking| introduces an
additional action into its continuation: \verb|AResetMask|, which
returns the masking state to what it originally was. It also
constructs a function to execute an action with the original masking
state, the \verb|umask| function.

\begin{haskellcode}
stepResetMask b1 b2 m c = simple threads' action where
  action   = (if b1 then SetMasking else ResetMasking) b2 m
  threads' = goto c tid (mask m tid threads)
\end{haskellcode}

\subsection{Software Transactional Memory}
\label{sec:execution-stepwise-stm}

As STM transactions are atomic, the implementation is vastly
simplified. They are still implemented in terms of a step function,
but it is just iterated until termination.

Firstly, the transaction is executed:

\begin{haskellcode}
stepAtom stm c = synchronised $ do
  (res, newctvid) <- runstm stm (nextCTVId idSource)
  let idSource'   = idSource { nextCTVId = newctvid }
  case res of
\end{haskellcode}
%$

There are now three possible results: the transaction succeeded; the
transaction aborted due to calling \verb|retry|; or the transaction
aborted due to an uncaught exception.

If the transaction succeeds, all threads blocked on \verb|CTVar|s
which were modified are woken:

\begin{haskellcode}
    Success readen written val
      let (threads', woken) = wake (OnCTVar written) threads
      in return (Right
         (goto (c val) tid threads', idSource', STM woken, wb))
\end{haskellcode}

If the transaction aborts due to \verb|retry|, the thread is blocked
until any of the read \verb|CTVar|s are modified:

\begin{haskellcode}
    Retry touched ->
      let threads' = block (OnCTVar touched) tid threads
      in return (Right (threads', idSource, BlockedSTM, wb))
\end{haskellcode}

If the transaction aborts due to an uncaught exception, the exception
is thrown in the thread:

\begin{haskellcode}
    Exception e -> stepThrow e
\end{haskellcode}

There are 9 primitive actions used to implement STM transactions,
discussed in \sect{execution}{primops}. The implementation of two of
them, \verb|SNew| and \verb|SLift|, are virtually identical to similar
primitives discussed elsewhere, and so will not be discussed here. The
remaining primitives are \verb|SRead|, \verb|SWrite|, \verb|SCatch|,
\verb|SOrElse|, \verb|SRetry|, \verb|SThrow|, and \verb|SStop|.

\begin{haskellcode}
stepRead (CTVar (ctvid, ref)) c = do
  val <- readRef ref
  return (c val, nothing, [ctvid], [])
\end{haskellcode}

The most obvious difference is that there is only one thread of
control. Moreover, a new transaction is built up to \emph{undo} what
has already been done (\verb|nothing| in this case), so that a
transaction can be reverted. Furthermore, lists of the \verb|CTVar|s
read from and written to are constructed.

There are no relaxed memory effects in STM transactions, so reading
and writing is incredibly simple.

\begin{haskellcode}
stepWrite (CTVar (ctvid, ref)) a c = do
  old <- readRef ref
  writeRef ref a
  return (c, writeRef ref old, [], [ctvid])
\end{haskellcode}

Here we see the inverse transaction being built up: to undo a write,
write the old value.

The handling of exceptions is vastly simplified, as \verb|SCatch| can
just execute the entire inner transaction and examine the result:

\begin{haskellcode}
stepCatch h stm c = onFailure stm c
  (\readen -> return (SRetry, nothing, readen, []))
  (\exc    -> case fromException exc of
    Just exc' -> transaction (h exc') c
    Nothing   -> return (SThrow exc, nothing, [], []))
\end{haskellcode}

Here \verb|onFailure| and \verb|transaction| are functions to do
different things for the three different possible results of a
transaction:

\begin{haskellcode}
transaction stm onSuccess = onFailure stm onSuccess
  (\readen -> return (SRetry, nothing, readen, []))
  (\exc    -> return (SThrow exc, nothing, [], []))

onFailure stm onSuccess onRetry onException = do
  (res, undo) <- doTransaction stm
  case res of
    Success readen written val -> return
      (onSuccess val, undo, readen, written)

    Retry readen  -> onRetry readen
    Exception exc -> onException exc
\end{haskellcode}

The effect of \verb|stepCatch|, then, is to run the entire inner
transaction. If it throws an exception, and the exception is of the
right type for the handler, the handler is executed. If the handler
throws an exception, it is not dealt with. If the exception is not of
the right type for the handler, it propagates upwards.

The implementation of \verb|SOrElse| is actually quite similar to
\verb|SCatch|: it runs the entire first transaction and, if it aborts
due to a \verb|retry|, runs the second:

\begin{haskellcode}
stepOrElse a b c = onFailure a c
  (\_   -> transaction b c)
  (\exc -> return (SThrow exc, nothing, [], []))
\end{haskellcode}

The terminating cases are implemented very simply:

\begin{haskellcode}
stepRetry   = return (SRetry,   nothing, [], [])
stepThrow e = return (SThrow e, nothing, [], [])
stepStop    = return (SStop,    nothing, [], [])
\end{haskellcode}

Termination is achieved by checking if the next action a thread will
perform is one of these three.

\subsection{Testing Annotations}
\label{sec:execution-stepwise-annotations}

Normally \dejafu{} can only detect a deadlock when \emph{every} thread
is blocked. However, it may be the case that a smaller collection of
threads are deadlocked, if they are all blocked on a \verb|CVar| which
no thread outside the collection has a reference to, for example. GHC
can do this sort of deadlock detection using its garbage collector,
and can signal to threads when they are blocked.

\dejafu{} does not have access to the garbage collector, and so relies
on programmer-provided hints about which \verb|CVar|s and
\verb|CTVar|s are known about by which threads.

\begin{haskellcode}
stepKnowsAbout v c = simple
  (knows   [v] tid (goto c tid threads)) KnowsAbout

stepForgets    v c = simple
  (forgets [v] tid (goto c tid threads)) Forgets

stepAllKnown     c = simple
  (fullknown   tid (goto c tid threads)) AllKnown
\end{haskellcode}

The \verb|knows| and \verb|forgets| functions are used to modify the
set of variables that a thread is known to have a reference to. The
\verb|fullknown| function indicates that this set is complete.

If every thread is in a fully-known state, then the deadlock detection
algorithm is enhanced to: for a given thread blocked on a \verb|CVar|
or \verb|CTVar|, if no other thread \emph{which is not also blocked on
  the same thing} has a reference to that variable, then the thread is
deadlocked.

\subsection{Lifting from the Underlying Monad}
\label{sec:execution-stepwise-lift}

Because all of the step functions are defined in terms of the
underlying monad, lifting an action is incredibly simple:

\begin{haskellcode}
stepLift na = do
  a <- na
  simple (goto a tid threads) Lift
\end{haskellcode}


  \subsection{Scheduling}
  \label{sec:execution-scheduling}
  When there are multiple non-blocked threads available, the choice of
which one to execute next is made by the scheduler.

A scheduler is represented as a pure function, and is supplied as a
parameter when testing. Doing things this way allows for deterministic
results and, just as importantly, allows for computing a list of
scheduling decisions in advance, designed to try to provoke the system
into a new state. This explicit computation of schedules is the basis
for the systematic concurrency testing implementation.

\begin{haskellcode}
type Scheduler s = s
  -> Maybe (ThreadId, ThreadAction)
  -> NonEmpty (ThreadId, NonEmpty Lookahead)
  -> (Maybe ThreadId, s)
\end{haskellcode}

The \verb|Maybe| return value can be used by the schedule to signal
that the execution should be aborted. In order to make nontrivial
decisions, a scheduler maintains some state, of type \verb|s|. This
could be, for example, a random number generator:

\begin{haskellcode}
randomSched :: RandomGen g => Scheduler g
randomSched g _ threads = (Just (threads' !! choice), g') where
  (choice, g') = randomR (0, length threads' - 1) g
  threads'     = map fst (toList threads)
\end{haskellcode}

The initial state is supplied when the execution begins, and the final
state is returned when it terminates. Use of this state is, of course,
not mandatory, as a simple round-robin scheduler illustrates:

\begin{haskellcode}
roundRobinSched :: Scheduler ()
roundRobinSched _ Nothing _ = (Just 0, ())
roundRobinSched _ (Just (prior, _)) threads
  | prior >= maximum threads' = (Just (minimum threads'), ())
  | otherwise = (Just (minimum (filter (>prior) threads')), ())
  where threads' = map fst (toList threads)
\end{haskellcode}

A scheduler is also given information about the state of the system:
what the last thread it scheduled did (this is \verb|Nothing| if this
is the first step of the computation), and what every runnable thread
in the system will do in the next few steps. Here \verb|NonEmpty| is
the type of non-empty lists,\footnote{And \texttt{toList} converts a
  \texttt{NonEmpty a} to a \texttt{[a]}.} to give a type-level
guarantee that there \emph{are} threads to run: if there are no
runnable threads, the execution terminates, signalling a deadlock
condition.

The \verb|ThreadAction| type is a record of what has happened. The
\verb|Lookahead| type is a slightly simpler view of what will
happen. The two types cannot be the same, because in general the
effect of performing a primitive action at some point in the future
cannot be determined, due to interactions between threads.

\subsubsection{Phantom Threads}
\label{sec:execution-scheduling-phantom}

In a sequentially consistent memory model, the set of runnable threads
is exactly the set of threads created by \verb|AFork| which are not
blocked.

Under relaxed memory, however, this is not the case. In order to model
the nondeterminism of \verb|CRef| writes, for every buffer with an
uncommitted write (threads, under TSO; \verb|CRef|s, under PSO), a
\emph{phantom thread} is created, and added to the runnable set. A
phantom thread is a thread with only one action: \verb|ACommit|. These
threads are never added to the thread map, they only exist in order
for the scheduler to determine when commits happen.

This may seem like an odd approach: why create new not-quite-threads
in order to model relaxed memory? The advantage is that systematic
concurrency testing techniques assume there is only one source of
nondeterminism: the scheduler. If a second source is added, such as
when writes are committed, it is difficult to integrate this with
existing algorithms. By using phantom threads, the two sources of
nondeterminism are unified, and existing algorithms just work. The
phantom thread approach was suggested by \citep{rdpor}.


\section{Systematic Concurrency Testing}
\label{chap:sct}
Once the scheduling behaviour of a program can be directed at will,
there is the ability to implement \emph{systematic}
testing. Systematic concurrency testing (SCT) comprises a family of
techniques, all with the same general aim: to try to find bugs in
concurrent programs, more reliably than running a program several
times. Within this scope, there are techniques which are
\emph{complete}, in that they find all possible results a program
could produce; and \emph{incomplete}, which do not make such a
guarantee.

SCT works by providing an initial sequence of scheduling decisions
intended to put the program into a new state. After this point some
deterministic scheduler is used, and the final trace examined to
produce new initial sequences. Typically the assumption is made that
all executions are \emph{terminating}: all possible sequences of
scheduling decisions will lead to a termination by deadlock or
otherwise. Another common assumption is that there is a \emph{finite}
number of possible schedules: this forbids finite but arbitrarily long
executions, as can be created with constructs such as spinlocks.

Systematic testing terminates when there are no more unique initial
sequences possible.


  \subsection{Schedule Bounding}
  \label{sec:sct-bounding}
  Schedule bounding is an \emph{incomplete} approach to SCT. Each
sequence of scheduling decisions is associated with a \emph{bound
  value}, by some \emph{bound function}. Such a function could be the
number of pre-emptive context switches, for example. Schedule bounding
was introduced in \citep{pbound}, and came from work in the model
checking field.

There are some common bound functions in use today:

\definewordc{Pre-emption Bounding}{pbound}{%
  The number of pre-emptive context switches is bounded.}

\definewordc{Fair Bounding}{fbound}{%
  The difference between the number of times different threads call
  \texttt{yield} is bounded.}

\definewordc{Delay Bounding}{dbound}{%
  The number of deviations from a deterministic scheduler is bounded.}

Both pre-emption bounding and delay bounding have empirical evidence,
in \citep{empirical}, showing that small bounds are good for finding
bugs in many real-world programs.

Fair bounding is used to handle programs which make use of lock-free
constructs such as spinlocks. A spinlock may be implemented like so:

\begin{haskellcode}
lock p var = spin where
  spin = do
    x <- readCRef var
    unless (p x) (yield >> spin)
\end{haskellcode}

Here, a \verb|CRef| is read from and, if some predicate on its value
is not satisfied, the thread yields and tries again. This can easily
give rise to infinitely long executions: simply don't execute any
other thread after the \verb|yield|, as it doesn't \emph{force} a
context switch. Fair bounding bounds the difference between the number
of times that threads have called \verb|yield|: if the thread that has
yielded the fewest times has done so 1 time, and the thread that has
yielded the most times has done so 10 times, then the bound value is
9.

Strictly speaking, schedule bounding refers to trying only those
schedules with a bound value equal to some fixed parameter. A variant
of this is \emph{iterative} bounding, where this parameter is
increased from zero up to some limit. Another variant is where an
inequality, rather than an equality, is used. This explores the same
schedules as iterative bounding, but doesn't impose the same ordering
properties over schedules tried. In practice, ``schedule bounding''
typically refers to this third type, unless specified otherwise.

\dejafu{} uses a combination of pre-emption and fair bounding, with a
pre-emption bound of 2 and a fair bound of 5, in order to gracefully
handle computations which use spinlocking techniques. The pre-emption
bound was chosen based on empirical evidence, but the fair bound was
picked fairly arbitrarily.


  \subsection{Partial-order Reduction}
  \label{sec:sct-por}
  Partial-order reduction is a \emph{complete} approach to SCT. It is
based on the insight that, when comparing different execution traces,
only the relative ordering of \emph{dependent} actions is
important. Two actions are dependent if the order in which they are
performed could affect the result of the program:

\definewordc{Dependency Relation}{dpor}{%
  Let $\mathcal T$ be the set of transitions in a concurrent system. A
  binary, reflexive, and symmetric relation $\mathcal D \subseteq
  \mathcal T \times \mathcal T$ is a valid \emph{dependency relation}
  iff, for all $t_{1}, t_{2} \in \mathcal T$, $(t_{1}, t_{2}) \notin
  \mathcal D$ ($t_{1}$ and $t_{2}$ are \emph{independent})
  following properties hold for all program states $s$:

  \begin{enumerate}
  \item if $t_{1}$ is enabled in $s$ and $s \xrightarrow{t_{1}} s'$,
    then $t_{2}$ is enabled in $s$ iff $t_{2}$ is enabled in $s'$; and

  \item if $t_{1}$ and $t_{2}$ are enabled in $s$, then there is a
    unique state $s'$ such that $s \xrightarrow{t_{1}t_{2}} s'$ and $s
    \xrightarrow{t_{2}t_{1}} s'$.
  \end{enumerate}}

In other words, independent transitions cannot enable or disable each
other, and enabled independent transitions commute. Rather than use
this relational definition directly, typically some sufficient
conditions for dependency are identified. These conditions are
determined by what sorts of things the concurrent system under test
can express.

Typically, the presentation of algorithms assumes a very simple core
concurrent language of just reads and writes. This gives rise to the
following dependency condition:

\begin{align*}
  x \dependent y \iff& \mathrm{thread\_id}(x) = \mathrm{thread\_id}(y) \lor\\
    &\left(\mathrm{variable}(x) = \mathrm{variable}(y)
     \land \left(\mathrm{is\_write}(x) \lor \mathrm{is\_write}(y)\right)\right)
\end{align*}

Where $x \dependent y$ is read as ``$x$ and $y$ are dependent''. This
choice of notation would suggest a symbol $\leftrightarrow$ meaning
independence, but that doesn't seem to be used.

The dependency relation for \dejafu{} is rather more complex than
this, as there are more actions than just reads and writes. However it
can be simplified to a few quite general conditions over different
sorts of reads and writes, with some remaining special cases.

These special cases are:

\begin{haskellcode}
dependent (t1, a1) (t2, a2) = case (a1, a2) of
  (Lift, Lift)   -> True
  (ThrowTo t, _) -> t == t2
  (_, ThrowTo t) -> t == t1
  (STM _, STM _) -> True
\end{haskellcode}

\begin{itemize}
\item Two lifts from the underlying monad are always dependent, as in
  general a monad may allow arbitrary I/O to be performed. The only
  restriction over I/O is that, given a fixed schedule, the I/O is
  deterministic.

\item Throwing an exception to a thread is dependent with anything, as
  all actions can be pre-empted by an exception.

\item STM transactions are always dependent. This final case could
  probably be refined to STM transactions which have some overlap in
  the \verb|CTVar|s they modify, but this is an optimisation which has
  not yet been tried.
\end{itemize}

Furthermore, as a Haskell program terminates when the main thread
terminates, there is a dependency between the last action in a trace
(whatever it may be) and \emph{everything} else.

The general cases are defined in terms of synchronised and
unsynchronised actions. Synchronised actions commit all pending
\verb|CRef| writes.

\begin{haskellcode}
dependentActions memtype buf a1 a2 = case (a1, a2) of
  (UnsynchronisedRead  r1, UnsynchronisedWrite r2) -> r1 == r2
  (UnsynchronisedWrite r1, UnsynchronisedRead  r2) -> r1 == r2
  (UnsynchronisedWrite r1, UnsynchronisedWrite r2) -> r1 == r2
  (UnsynchronisedRead r1, _) | isBarrier a2 -> isBuffered buf r1
  (_, UnsynchronisedRead r2) | isBarrier a1 -> isBuffered buf r2
  _ -> same crefOf && (isSynchronised a1 || isSynchronised a2) || same cvarOf
\end{haskellcode}

\begin{itemize}
\item A read and write to the same \verb|CRef| are dependent, as are
  two writes. The reason for this dependence even under a relaxed
  memory model is because writes give rise to commits, which \emph{do}
  synchronise.

\item An unsynchronised read from a variable is dependent with an
  action that imposes a memory barrier if there are buffered writes to
  the same variable.

\item Any two actions on the same \verb|CRef| where at least one of
  them will cause a commit are dependent.

\item Any two actions on the same \verb|CVar| are dependent.
\end{itemize}

Characterising the execution of a concurrent program by the ordering
of its dependent actions only gives us a \emph{partial order} over the
actions in the entire program. An execution trace may be just one
possible \emph{total} order corresponding to the same partial
order. The goal of partial-order reduction, then, is to eliminate
these redundant total orders by intelligently making scheduling
decisions to permute the order of dependent actions.

This can be done by executing a program with a deterministic
scheduler, and then examining the trace, the total order, for
\emph{backtracking points}. A backtracking point is a place in the
execution where multiple dependent choices were available, and only
one was tried. The exploration of the state space continues by making
the same scheduling decisions up to that point, and then making a
different decision. This process of doing partial-order reduction
based on information gathered at run-time, rather than static
analysis, is called \emph{dynamic partial-order reduction} (DPOR).

% Implementation in dejafu

In an imperative language, DPOR is usually done by executing the
program under test stepwise in a recursive function, where each stack
frame has a set of decisions still to try, and this is mutated by
later calls when a backtracking point is identified. As \dejafu{} is a
Haskell library, this is not a very natural way to formulate anything,
and so a different approach was taken.

\dejafu{} explicitly constructs a tree in memory, where each path from
the root to a leaf corresponds to one complete execution. Forks in the
tree correspond to places where multiple decisions have been
tried. The operation proceeds as follows:

\begin{haskellcode}
sctBounded memtype bf run = go initialState where
  go state = case next state of
    Just decisions -> do
      (result, s, trace) <- run decisions
      let bpoints = findBacktrack memtype s trace
      let newBPOR = todo bf bpoints (grow memtype trace state)
      ((result, trace) :) <$> go newState
    Nothing -> return []
\end{haskellcode}
%$

Here \verb|next| returns a schedule prefix; \verb|run| executes the
computation with a given sequence of initial scheduling decisions,
returning the final result, the final scheduler state (which includes
a tentative list of bracktracking points), and the execution trace;
\verb|findBacktrack| identifies a list of actual backtracking points
from these tentative ones; \verb|grow| adds the trace to the tree
structure; and \verb|todo| adds the newly-identified backtracking
points. It is also the responsibility of \verb|todo| to ensure these
new backtracking points do not cause schedules exceeding the bound to
be generated; the \verb|bf| function is the bound function, expressed
as a predicate.

The entire process terminates when \verb|next| returns \verb|Nothing|,
as there are no unexplored backtracking points left.

\subsubsection{Integration with Schedule Bounding}
\label{sec:sct-por-bounding}

The na\"{\i}ve way to integrate DPOR with schedule bounding would be
to first use partial-order techniques to prune the search space, and
then to additionally filter things out with schedule bounding.

Unfortunately, this is unsound. This approach misses parts of the
search space reachable within the bound. This is because the
introduction of the bound introduces new dependencies between actions,
which cannot be determined \emph{a priori}. The solution is to add
\emph{conservative} backtracking points to account for the bound in
addition to any normal backtracking points that are identified. Where
to insert these depends on the bound function.

In the case of pre-emption bounding, it suffices to try all
possibilities at the last context switch before a normal backtracking
point. This is because context switches influence the number of
pre-emptions needed to reach a given program state, depending on which
thread gets scheduled. As pre-emption bounding has been found
empirically to be successful with a low number of threads, and DPOR is
already eliminating a lot of possibilities, this is not in practice a
huge additional cost.

\subsubsection{Integration with Relaxed Memory}
\label{sec:sct-por-relaxed}

Due to the use of phantom threads, explained in
\sect{execution}{scheduling}, almost nothing needs to be done to
support relaxed memory!

The \verb|dependentActions| function has some knowledge of relaxed
memory in order to make less pessimistic decisions, as otherwise the
assumption would have to be made that there are always uncommitted
writes. The only other change is related to the integration with
schedule bounding: a pre-emption immediately before (or immediately
after) a phantom thread is free.


\section{Correctness}
\label{chap:correctness}
Once the scheduling behaviour of a program can be directed at will,
there is the ability to implement \emph{systematic}
testing. Systematic concurrency testing (SCT) comprises a family of
techniques, all with the same general aim: to try to find bugs in
concurrent programs, more reliably than running a program several
times. Within this scope, there are techniques which are
\emph{complete}, in that they find all possible results a program
could produce; and \emph{incomplete}, which do not make such a
guarantee.

SCT works by providing an initial sequence of scheduling decisions
intended to put the program into a new state. After this point some
deterministic scheduler is used, and the final trace examined to
produce new initial sequences. Typically the assumption is made that
all executions are \emph{terminating}: all possible sequences of
scheduling decisions will lead to a termination by deadlock or
otherwise. Another common assumption is that there is a \emph{finite}
number of possible schedules: this forbids finite but arbitrarily long
executions, as can be created with constructs such as spinlocks.

Systematic testing terminates when there are no more unique initial
sequences possible.


  \subsection{Correct Execution}
  \label{sec:correctness-execution}
  Correctness of execution asks whether the result of an arbitrary
execution of \dejafu{}'s testing implementation happen in reality.
Furthermore, do all real-world executions correspond to a possible
execution under \dejafu{}? To put it more simply:

\begin{itemize}
\item Is the behaviour of the primitive functions the same?

\item Is the granularity of scheduling decisions the same?
\end{itemize}

Both of these come with the caveat that the behaviour can be
different, as long as this difference can't be observed.

\subsubsection{Primitives}
\label{sec:correctness-execution-primops}

The method of implementing the members of the \verb|MonadConc|
typeclass that would be most amenable to proof would be to implement
analogues of the GHC primitives directly, and implement everything
else in terms of these. This matches how actual Haskell is
implemented, and would lend itself to establishing a formal
correspondence between the \dejafu{} primitives and the GHC
primitives, and the higher-level \verb|MonadConc| functions and the
higher-level functions in Control.\-Concurrent.

This approach was not taken, however. Firstly, it ties the
implementation and correctness of \dejafu{} very closely to the
implementation of GHC; in principle the implementation of GHC's
concurrency primitives could be completely changed but the behaviour
preserved. Secondly, this restricts \dejafu{} to a very specific type
of concurrency, supporting low-level operations, which may not map to
all interesting implementations of concurrency.

Instead, a reimplementation of GHC's concurrency based on the
documented and observable behaviour of the various functions was
done. This allows observing the behaviour of a program and
determining, intuitively, whether it is correct or not; but it's not
so good for proof. The matter is complicated by there being no
standard for concurrent Haskell, there is only what GHC provides.

The correctness of operations using \verb|CRef|s is complicated even
further, as the behaviour of these depends on the underlying memory
model. Total store order was chosen to be the default, as it is what
x86 processors do with unsynchronised memory accesses, but a
\verb|CRef| is more complicated than a simple memory cell: it is a
pointer to a cell, which can be moved around in garbage collection,
and it is accessed through primitive operations more complicated than
simple loads and stores. The lack of a standard, or even comprehensive
documentation, means that in order to formally establish the memory
model for \verb|CRef|s, the compilation of \verb|IORef| functions must
be traced through GHC from Haskell source to machine code. As GHC uses
C{-}{-} as an intermediary language, this may also require determining
a memory model for C{-}{-}.

Finally, there are some intended departures from the behaviour of
GHC's behaviour documented in \sect{abstraction}{typeclass}:

\begin{itemize}
\item \verb|getNumCapabilities| is not required to return a true
  result.

\item Deadlock detection can only function to the same extent as GHC
  if every thread is annotated with which \verb|CVar|s and
  \verb|CTVar|s it knows about, as \dejafu{} cannot use the garbage
  collector for this task.

\item \verb|catch| is not required to be able to catch exceptions from
  pure code.
\end{itemize}

\subsubsection{Scheduling}
\label{sec:correctness-execution-scheduling}

The stepwise implementation allows for a scheduling decision to be
made between each primitive action, which doesn't quite correspond to
how GHC handles scheduling:

\quot{GHC implements pre-emptive multitasking: the execution of
  threads are interleaved in a random fashion. More specifically, a
  thread may be pre-empted whenever it allocates some memory, which
  unfortunately means that tight loops which do no allocation tend to
  lock out other threads (this only seems to happen with pathological
  benchmark-style code, however).}{controlConcurrent}

That is, GHC allows for pre-emptions to occur whilst evaluating pure
code, which the stepwise executor does not. There are executions
involving the pre-emption of the evaluation of non-terminating
expressions which are possible under GHC but not under \dejafu{}, but
whether this can be used to produce different outputs is less clear.


  \subsection{Correct Testing}
  \label{sec:correctness-testing}
  Systematic concurrency testing (SCT) \citep{dpor, pbound, heisenbugs,
  empirical} is a way of tackling the problem of nondeterminism when
writing tests. It aims to test a large number of schedules, whilst
typically also making use of local knowledge of the program to reduce
the number of schedules needed to be confident of an accurate
result. By testing many schedules, we can increase our confidence that
any bugs which have not been found are unlikely to be exhibited.

SCT overcomes the scheduling problem by forcing a concurrent program
to use a scheduler implemented as part of the testing framework:
either by overriding the concurrency primitives of the language, or by
modifying the program under test to call out to this new scheduler (as
in PULSE \citep{pulse}).

Once the scheduler is under control, schedules can be recorded and
replayed, giving reproducibility. Furthermore, by observing which
scheduling decisions are available at each decision point, possible
schedules can be systematically explored, making different decisions
on subsequent executions. Common methods of choosing schedules to take
are random \citep{empirical}, schedule bounding \citep{pbound}, and
partial-order reduction \citep{dpor}. The latter of these is
\emph{complete}: partial-order reduction will find all distinct
program states given enough time, in a more intelligent way than just
trying all schedules.


\section{Case Studies}
\label{chap:casestudies}
Once the scheduling behaviour of a program can be directed at will,
there is the ability to implement \emph{systematic}
testing. Systematic concurrency testing (SCT) comprises a family of
techniques, all with the same general aim: to try to find bugs in
concurrent programs, more reliably than running a program several
times. Within this scope, there are techniques which are
\emph{complete}, in that they find all possible results a program
could produce; and \emph{incomplete}, which do not make such a
guarantee.

SCT works by providing an initial sequence of scheduling decisions
intended to put the program into a new state. After this point some
deterministic scheduler is used, and the final trace examined to
produce new initial sequences. Typically the assumption is made that
all executions are \emph{terminating}: all possible sequences of
scheduling decisions will lead to a termination by deadlock or
otherwise. Another common assumption is that there is a \emph{finite}
number of possible schedules: this forbids finite but arbitrarily long
executions, as can be created with constructs such as spinlocks.

Systematic testing terminates when there are no more unique initial
sequences possible.


  \subsection{auto-update}
  \label{sec:casestudies-autoupdate}
  The \emph{auto-update} library runs tasks periodically, but only if
needed. For example, a single worker thread may get the time every
second and store it to a shared \verb|IORef|, rather than have many
threads starting within a second of each other all get the time
independently \citep{autoupdate}. Despite the core functionality being
very simple, two race conditions were noticed by users inspecting the
code in October 2014.

\begin{figure}[t]
  \captionsetup{format=fnoline}
  \centering
  \begin{minted}{haskell}
data UpdateSettings a = UpdateSettings
    { updateFreq           :: Int
    , updateSpawnThreshold :: Int
    , updateAction         :: IO a
    }

defaultUpdateSettings :: UpdateSettings ()
defaultUpdateSettings = UpdateSettings
    { updateFreq           = 1000000
    , updateSpawnThreshold = 3
    , updateAction         = return ()
    }

mkAutoUpdate :: UpdateSettings a -> IO (IO a)
mkAutoUpdate us = do
    currRef      <- newIORef Nothing
    needsRunning <- newEmptyMVar
    lastValue    <- newEmptyMVar

    void $ forkIO $ forever $ do
        takeMVar needsRunning

        a <- catchSome $ updateAction us

        writeIORef currRef $ Just a
        void $ tryTakeMVar lastValue
        putMVar lastValue a

        threadDelay $ updateFreq us

        writeIORef currRef Nothing
        void $ takeMVar lastValue

    return $ do
        mval <- readIORef currRef
        case mval of
            Just val -> return val
            Nothing -> do
                void $ tryPutMVar needsRunning ()
                readMVar lastValue

catchSome :: IO a -> IO a
catchSome act = catch act $
  \e -> return $ throw (e :: SomeException)
  \end{minted}
  \caption{\emph{auto-update} implementation}
  \label{fig:example-autoupdate}
\end{figure}

The entire implementation, excluding comments and imports, is
reproduced in Figure \ref{fig:example-autoupdate}. The
\verb|mkAutoUpdate| function spawns a worker thread, which performs
the update action at the given frequency, only if the
\verb|needsRunning| flag has been set. It returns an action to attempt
to read the current result, demanding one be computed and blocking
until it has been done if there isn't one.

The simpler race condition occurs if the reading thread is pre-empted
by the worker thread after putting into \verb|needsRunning|, and does
not run again until after the delay has passed. In this case the
worker thread can become blocked on taking for a second time from
\verb|needsRunning|. The reader thread will be unable to read from
\verb|lastValue| as the worker thread emptied it as the last action it
performed. The transformation to the \verb|MonadConc| typeclass is
mostly simple, however the \verb|threadDelay| must be wrapped inside a
call to \verb|liftIO|. The first race condition can be exhibited with
the following test:

\begin{minted}{haskell}
test :: (MonadConc m, MonadIO m) => m ()
test = do
  auto <- mkAutoUpdate defaultUpdateSettings
  auto
\end{minted}

The output is as we would expect, knowing the bug is present:

\begin{verbatim}
> autocheckIO test
[fail] Never Deadlocks (checked: 1)
        [deadlock] S0--------S1-----------S0-
[pass] No Exceptions (checked: 9)
[fail] Consistent Result (checked: 8)
        [deadlock] S0-----P1-S0---S1-----------S0-
        () S0--------S1---------P0---
False
\end{verbatim}

This deadlock may arise in any use of the library, as it depends only
on the timing of the delay, and not on the computation performed.

The more complex race condition arises if \verb|readMVar| isn't
atomic, as in GHC versions before 7.8. In this case an old value can
be returned if the read of \verb|lastValue| is pre-empted between the
internal take and put operations, as shown in this test:

\begin{minted}{haskell}
test :: (MonadConc m, MonadIO m) => m Int
test = do
  var  <- newCRef 0
  auto <- mkAutoUpdate $ defaultUpdateSettings
    { updateAction = modifyCRef var (\x -> (x+1, x)) }

  auto
  auto
\end{minted}
%$

Here \verb|auto| is called twice to update the counter variable
twice. Actually reproducing this bug requires a new \verb|readCVar|
function be written, as the library does not currently provide an
option for non-atomic reads. Exhibiting this bug requires three
pre-emptions:

\begin{verbatim}
> dejafuIO' TotalStoreOrder 3 5 test ("Consistent Result", alwaysSame)
[fail] Consistent Result (checked: 23)
        [deadlock] S0------P1-S0---S1-----------S0-
        0 S0---------S1--------P0-----
        1 S0---------S1---------P0---P1--------P0---
\end{verbatim}

Despite the bugs being rather simple, one not requiring any
pre-emptions at all to trigger, they both arose in practice. How easy
it is to make mistakes when implementing concurrent programs!


  \subsection{Search Party}
  \label{sec:casestudies-searchparty}
  The Search Party library supports speculative parallelism in
generate-and-test search problems. It is motivated by the
consideration that if multiple acceptable solutions exist, it may not
matter which one is returned. It was originally developed as a case
study for \dejafu{}, but was also presented as a poster at the 2015
Haskell Symposium.

The library provides a collection of combinators used to express a
generate-and-test problem, which are executed in using a concurrent
producer/consumer pattern. Unfortunately, how to efficiently
parallelise a list comprehension is less than obvious to a new user of
the library.

Whilst the efficient and automatic parallelisation of functional
programs based purely on static analysis is not feasible
\citep{autopar}, it may be possible to construct a ``good enough''
analysis for list comprehensions. This would be realised in the form
of Template Haskell ``search comprehensions'', which would look like
list comprehensions but expand to some appropriate Search Party
formulation of the computation. If reasonably-successful heuristics
could be developed to determine how to break up the problem, then this
could be a cheap way for a programmer to speed up their code.

\paragraph{Timeline:}

\begin{description}
\item[End of Sep 2016] Implement search comprehensions.

\item[End of Dec 2016] Develop heuristics for splitting up list
  comprehensions into Search Party-using parallel computations.
\end{description}

\paragraph{Success Criteria:}

The tool produces as-good-as-human results in some nontrivial
cases. As the programmer will need to manually introduce a search
comprehension, occasionally producing worse results is not such a
problem, as bad comprehensions can be disabled.

\paragraph{Publications:}

This would significantly expand the current Search Party paper, which
was rejected from the 2015 Haskell Symposium, and make it worth
resubmitting.


  \subsection{The Par Monad}
  \label{sec:casestudies-parmonad}
  The \verb|Par| monad \citep{parmonad} is a library providing a
traditional-looking concurrency abstraction, providing the programmer
with threads and mutable state, however it maintains determinism by
restricting its shared variables to one write, and operations to read
block until a value has been written. Thus, \verb|Par|'s \verb|IVar|s
are \emph{futures}, not \emph{mutable} state. \verb|Par| uses a
work-stealing scheduler running on multiple operating system threads,
fully evaluating values on their own threads before inserting them
into an \verb|IVar|. Despite its limitations, the \verb|Par| monad can
be very effective in speeding up pure code.

The following example maps a function in parallel over a list, fully
evaluating it. Of course, laziness is generally what is desired in
Haskell programs, but often it is known that an entire result will
definitely be needed:

\begin{minted}{haskell}
parMap :: NFData b => (a -> b) -> [a] -> [b]
parMap f as = runPar $ do
  bs <- mapM (spawnP . f) as
  mapM get bs
\end{minted}
%$

However, with a lack of multi-write shared variables and non-blocking
reads, \verb|Par| is unsuitable for long-lived concurrent programs
with a central shared state. It could not be used to implement a
multi-threaded work-stealing scheduler, such as the one underpinning
\verb|Par| itself. The library provides a number of different
schedulers, the default being the ``trace'' scheduler. Due to reports
of potential deadlocks with the ``direct'' scheduler from a year ago,
it was tested with \dejafu{}.

% removed: \citep{parreddit}

To reduce the effort in modifying the code, only the direct
dependencies of the ``direct'' scheduler were modified, the rest of
the library being left unchanged. This resulted in four files needing
change: two from the
\emph{abstract-deque}\footnote{\hackage{abstract-deque}} package and
two from the \emph{monad-par}\footnote{\hackage{monad-par}} package.

Converting \emph{monad-par} to use \dejafu{} was quite simple. All
relevant types were parametrised by the underlying monad, all
functions had a \verb|MonadConc| context added, functions were swapped
for their \dejafu{} alternatives, and a \verb|runPar'| function was
added:

\begin{minted}{haskell}
runPar' :: MonadConc m => Par m a -> m a
\end{minted}

Some simplifications were made in the conversion process:

\begin{itemize}
\item \verb|Par| normally uses the
  \emph{mwc-random}\footnote{\hackage{mwc-random}} package when
  performing its internal scheduling. This was initially replaced with
  a constant function, and then a \verb|StdGen|.

\item Behaviour of the \verb|Par| scheduler can be configured using
  cpp, but only the default configuration was tested.
\end{itemize}

Figure \ref{fig:example-parmonad-sched} shows the original and
converted scheduler initialisation code. As can be seen, they are very
similar, even though this is a core component of a rather
sophisticated library, where the types have been changed.

Converting the \emph{abstract-deque} package proved a little more
challenging, as the typeclass interface requires knowledge of both the
queue type and the monad results are produced in. This issue was
solved by use of type families:

\begin{minted}{haskell}
class MonadConc (MConc d) => DequeClass d where
  type MConc d :: * -> *

  newQ :: MConc d (d elt)
  ...
\end{minted}

This solution is not ideal as it adds explicit knowledge of
\verb|MonadConc| to the \verb|DequeClass| typeclass, but it suffices
for testing purposes.

With the constant value `PRNG', a deadlock was discovered. It only
arises after 200 queries. Given that the range of values is from 0 to
the number of capabilities, and the probability is uniformly
distributed, the probability of an actual deadlock is about $4 \times
10^{-121}$ on a quad-core computer. No deadlocks were discovered when
using the \verb|StdGen| generator, with a variety of initial seeds
tried. If there is still a deadlock, it may require more than 2
capabilities to manifest.

\begin{landscape}
\begin{figure*}[t]
  \captionsetup{format=fnoline}
  \centering
  \begin{minipage}[t]{0.49\linewidth}
    \begin{minted}{haskell}
makeScheds :: Int -> IO [Sched]
makeScheds main = do
  caps <- getNumCapabilities
  workpools <- replicateM caps R.newQ
  rngs <- replicateM caps
            (Random.create >>= newHotVar)
  idle <- newHotVar []

  sessionFinished <- newHotVar False
  let sess = [Session baseSessionID sessionFinished]
  sessionStacks <- mapM newHotVar
                     (replicate caps sess)
  activeSessions <- newHotVar S.empty
  sessionCounter <- newHotVar (baseSessionID + 1)
  let allscheds =
       [ Sched { no=x, idle, isMain=(x==main),
                 workpool=wp, scheds=allscheds,
                 rng=rng, sessions=stck,
                 activeSessions=activeSessions,
                 sessionCounter=sessionCounter
               }
         | x    <- [0 .. caps-1]
         | wp   <- workpools
         | rng  <- rngs
         | stck <- sessionStacks
       ]
  return allscheds
    \end{minted}
    \caption*{Original}
  \end{minipage}
  \begin{minipage}[t]{0.49\linewidth}
    \begin{minted}{haskell}
makeScheds :: MonadConc m => Int -> m [Sched m]
makeScheds main = do
  caps <- getNumCapabilities
  workpools <- replicateM caps R.newQ
  rngs <- replicateM caps
            (newHotVar (mkStdGen 0))
  idle <- newHotVar []

  sessionFinished <- newHotVar False
  let sess = [Session baseSessionID sessionFinished]
  sessionStacks <- mapM newHotVar
                     (replicate caps sess)
  activeSessions <- newHotVar S.empty
  sessionCounter <- newHotVar (baseSessionID + 1)
  let allscheds =
       [ Sched { no=x, idle, isMain=(x==main),
                 workpool=wp, scheds=allscheds,
                 rng=rng, sessions=stck,
                 activeSessions=activeSessions,
                 sessionCounter=sessionCounter
               }
         | x    <- [0 .. caps-1]
         | wp   <- workpools
         | rng  <- rngs
         | stck <- sessionStacks
       ]
  return allscheds
    \end{minted}
    \caption*{\dejafu{}}
  \end{minipage}
  \caption{Par ``direct'' scheduler initialisation}
  \label{fig:example-parmonad-sched}
\end{figure*}
\end{landscape}

\section{Practical Usage}
\label{chap:practice}
Once the scheduling behaviour of a program can be directed at will,
there is the ability to implement \emph{systematic}
testing. Systematic concurrency testing (SCT) comprises a family of
techniques, all with the same general aim: to try to find bugs in
concurrent programs, more reliably than running a program several
times. Within this scope, there are techniques which are
\emph{complete}, in that they find all possible results a program
could produce; and \emph{incomplete}, which do not make such a
guarantee.

SCT works by providing an initial sequence of scheduling decisions
intended to put the program into a new state. After this point some
deterministic scheduler is used, and the final trace examined to
produce new initial sequences. Typically the assumption is made that
all executions are \emph{terminating}: all possible sequences of
scheduling decisions will lead to a termination by deadlock or
otherwise. Another common assumption is that there is a \emph{finite}
number of possible schedules: this forbids finite but arbitrarily long
executions, as can be created with constructs such as spinlocks.

Systematic testing terminates when there are no more unique initial
sequences possible.


  \subsection{Alternatives to Existing Libraries}
  \label{sec:practice-alternatives}
  There are a number of popular Haskell libraries specifically for
concurrency. One of these is the
\emph{async}\footnote{\hackage{async}} library, for expressing
asynchronous computations. This library is intended to be a
higher-level and safer way of expressing asynchronous computations
than using \verb|forkIO| and \verb|MVar|s directly. It provides two
main functions to execute an action asynchronously:

\begin{haskellcode}
async :: IO a -> IO (Async a)
withAsync :: IO a -> (Async a -> IO b) -> IO b
\end{haskellcode}

Both of these fork the computation into a separate thread, providing
this \verb|Async| value, containing an \verb|MVar| which can be
blocked on in order to retrieve the value. In addition,
\verb|withAsync| kills the thread if the inner action completes before
it, to help prevent resource leaks.

There is a further abstraction atop \verb|Async|, called
\verb|Concurrently|, which has Functor, Applicative, and Alternative
instances, and represents an action which can be composed with other
actions and execute concurrently. The concurrency is achieved by
having \verb|(<*>)| execute each action asynchronously. There was a
Monad instance for \verb|Concurrently|, but this broke the laws, as
\verb|ap| was not the same as
\verb|(<*>)|\footnote{\url{https://github.com/simonmar/async/pull/26}}. This
was due to \verb|ap| executing its arguments sequentially, as that is
all which can be done with \verb|(>>=)|.

This bug could have been discovered through testing, but only
probabilistically. If \emph{async} were written using
\verb|MonadConc|, the relevant laws could have been specified as unit
tests and checked and the bug could have been caught before it showed
up in user code. Furthermore, by using \verb|IO| directly, it is not
possible to write a generic \verb|MonadConc| action which makes use of
\emph{async}, which is very unfortunate.

To address both of these issues, there is an \emph{async-dejafu}
package, which provides almost the same API as \emph{async}, but is
parameterised by a \verb|MonadConc|, giving functions like this:

\begin{haskellcode}
async :: MonadConc m => m a -> m (Async m a)
withAsync :: MonadConc m => m a -> (Async m a -> m b) -> m b
\end{haskellcode}

There is a test suite using \dejafu{}, whereas the test suite for
\emph{async} just runs most tests a single time, although one of them
is run 1000 times. Using \dejafu{} here to automatically seek out
interesting schedules is a much more principled approach.

Not all of the features of \emph{async} are supported by
\emph{async-dejafu}: as \verb|MonadConc| does not support bound
threads, those functions that use them have been omitted.

Of course, \emph{async} is just one library, and providing an
alternative library people will have to switch to is far from
optimal. However, until library authors start to use \dejafu{} and
\verb|MonadConc| directly, such alternatives will be needed to answer
the question ``why should I use this if I can't use it with all of my
familiar tools?''


  \subsection{Integration with Testing Frameworks}
  \label{sec:practice-integration}
  There are two popular libraries for unit testing in Haskell,
\emph{HUnit}\footnote{\hackage{HUnit}} and
\emph{tasty}\footnote{\hackage{tasty}}. From the perspective of the
user, both libraries are very similar, but from the perspective of the
implementer, they have different approaches to integration. Packages
providing integration with both, called \emph{hunit-dejafu} and
\emph{tasty-dejafu} are provided.

Both packages provide a common set of testing functions, an analogue
of Test.\-DejaFu but constructing values representing individual tests
which the frameworks can run, rather than executing and printing
results directly:

\begin{haskellcode}
testAuto    :: (Eq a, Show a) => (forall t. ConcST t a) -> Test
testDejafu  :: Show a => (forall t. ConcST t a) -> String -> Predicate a -> Test
testDejafus :: Show a => (forall t. ConcST t a) -> [(String, Predicate a)] -> Test
\end{haskellcode}

Here \verb|Test| is the type of individual tests, from
\emph{HUnit}. \emph{tasty} uses \verb|TestTree|, which has a similar
purpose; it also uses \verb|TestName| rather than \verb|String|. To
complete the set, variants of these functions for \verb|ConcIO|, and
also taking the schedule bounds and memory type as parameters, are
provided. All of the testing functions are implemented in terms of
\verb|testDejafus'| and \verb|testDejafusIO'|.

The \emph{test-framework}\footnote{\hackage{test-framework}} library
is also in common use, however it supports integration with
\emph{HUnit}, and so needs no special support.

\subsection{HUnit}
\label{sec:practice-integration-hunit}

Tests in \emph{HUnit} are just a thin wrapper around an \verb|IO ()|
action, which can be grouped together into collections and given
names. The testing model is very simple: a test fails if and only if
it produces some output. There are a number of provided testing
functions, which throw an exception if they fail, terminating the rest
of the test case.

\begin{haskellcode}
test :: Show a => MemType -> Bounds -> (forall t. ConcST t a)
  -> [(String, Predicate a)] -> Test
test memtype cb conc tests = case map toTest tests of
  [t] -> t
  ts  -> TestList ts

  where
    toTest (name, p) = TestLabel name . TestCase . assertString . showErr $ p traces

    traces = sctBound memtype cb conc
\end{haskellcode}
%$

Here, each \verb|(String, Predicate a)| pair is turned into a separate
test case. If there is only one, it is returned directly, otherwise
they are grouped together into a \verb|TestList|. A \verb|TestList|
\emph{can} consist of only one entry, but making this distinction
results in a closer correspondance between the generated \verb|Test|
and the call to the testing function which produced it.

The \verb|assertString| function is provided by \emph{HUnit}. If the
provided string is non-empty (\verb|showErr| here is a function to
pretty-print the failures, if any) the test fails.

\subsection{tasty}
\label{sec:practice-integration-tasty}

In contrast to the simple function-based method of \emph{HUnit},
\emph{tasty} has a much more complex approach based on a typeclass of
things which can be converted to a unit test:

\begin{haskellcode}
test :: Show a => MemType -> Bounds -> (forall t. ConcST t a)
  -> [(TestName, Predicate a)] -> TestTree
test memtype cb conc tests = case map toTest tests of
  [t] -> t
  ts  -> testGroup "Deja Fu Tests" ts

  where
    toTest (name, p) = singleTest name $ ConcTest traces p

    traces = sctBound memtype cb conc
\end{haskellcode}
%$

This is very similar to the \emph{HUnit} approach, however instead of
constructing a test value directly, it constructs an intermediate
\verb|ConcTest| value. Note also that \emph{tasty} does not allow
nameless test lists. The \verb|singleTest| function takes a value
which is a member of the \verb|IsTest| typeclass, and uses that to
construct a \verb|TestTree|:

\begin{haskellcode}
data ConcTest where
  ConcTest :: Show a => [(Either Failure a, Trace)] -> Predicate a -> ConcTest
  deriving Typeable

instance IsTest ConcTest where
  testOptions = return []

  run _ (ConcTest traces p) _ =
    let err = showErr $ p traces
     in return $ if null err then testPassed "" else testFailed err
\end{haskellcode}

\emph{tasty} allows for passing tests to also have output associated
with them, through the \verb|testPassed| function, which is only
displayed if the tests are executed with sufficient
verbosity. Furthermore, tests can have options associated with them,
which can be set when the test is executed. Neither of these features
are used, as this was largely just a port of
\emph{hunit-dejafu}. \emph{tasty} is definitely a more featureful
library than \emph{HUnit}, but this comes at the cost of additional
complexity for developers trying to integrate new functionality.


\section{Future Research \& Conclusions}
\label{chap:conclusions}
Once the scheduling behaviour of a program can be directed at will,
there is the ability to implement \emph{systematic}
testing. Systematic concurrency testing (SCT) comprises a family of
techniques, all with the same general aim: to try to find bugs in
concurrent programs, more reliably than running a program several
times. Within this scope, there are techniques which are
\emph{complete}, in that they find all possible results a program
could produce; and \emph{incomplete}, which do not make such a
guarantee.

SCT works by providing an initial sequence of scheduling decisions
intended to put the program into a new state. After this point some
deterministic scheduler is used, and the final trace examined to
produce new initial sequences. Typically the assumption is made that
all executions are \emph{terminating}: all possible sequences of
scheduling decisions will lead to a termination by deadlock or
otherwise. Another common assumption is that there is a \emph{finite}
number of possible schedules: this forbids finite but arbitrarily long
executions, as can be created with constructs such as spinlocks.

Systematic testing terminates when there are no more unique initial
sequences possible.


  \subsection{Related Work}
  \label{sec:conclusions-related}
  % Story: history of the algorithm, focussing on the development of
% PULSE (and its probabilistic approach) in particular, and then less
% in-depth surveys of the available tools in general. Finish with a
% brief note of the blog post which inspired the typeclass approach
% used in dejafu.

There is a tension in testing concurrent programs between
\emph{verification} and \emph{bug-finding}. For verification,
completeness is desirable, whereas for testing completeness can be
sacrificed if the number of defects found in non-contrived examples is
not affected much. Furthermore, by sacrificing completeness, speed can
be gained, which is of great importance for developers running a test
suite repeatedly as development proceeds.

% DPOR

Partial-order methods were first introduced in \citep{por}, which also
introduced the insight that a concurrent execution can be thought of
as a \emph{partial-order} of the dependent actions in the
system. Initially, these methods were based on a static analysis of
the program under test. Further developments in \citep{dpor} enabled
the information needed for partial-order methods to be obtained at
run-time system, often leading to a reduction in the amount of work
done. The static analysis is necessarily \emph{conservative}, whereas
the dynamic analysis has much more complete information available to
it.

A different approach to testing concurrent programs was explored in
\citep{pbound}, where executions exceeding some pre-determined
\emph{bound} are simply not done. Completeness is sacrificed in return
for more rapid results of testing, on the assumption (later validated
by empirical studies such as \citep{empirical}) that realistic test
cases could find bugs within a small bound.

It was later shown in \citep{bpor} that these two approaches,
partial-order reduction and schedule bounding, can be unified. The
result is necessarily incomplete. However it can reduce the number of
executions tried to a far greater extent than either of the two
component methods alone. With the evidence that schedule bounding
isn't a problem in practice for testing, this became a widely-adopted
method.

An assumption of key importance in concurrency testing is that all
nondeterminism arises from the scheduler. Most other sources, such as
random number generators, can be handled by (for example) using a
fixed seed. However, in the quest for ever more performance, hardware
manufacturers imposed \emph{relaxed memory} architectures on
programmers, where reads and writes done in parallel can give results
impossible under sequential consistency.

\citep{rdpor} showed how this additional source of nondeterminism can
be handled, by modelling a single level of cache (which corresponds to
total-store order or partial-store order) as simply a separate thread,
committing writes to memory.

% Probabilistic DPOR

A different approach to reducing the work done as a refinement of a
pure partial-order reduction approach was taken in \citep{rapos},
which uses random scheduling. Partial-order reduction is used to prune
the search space, but random decisions are made where there are still
multiple choices available. Random scheduling itself does not
necessarily work very well, as some partial orders have more total
order refinements than others, hence pruning the search space in
partial-order reduction was reported to be an effective way to
increase the bug-finding ability of random scheduling.

This work was furthered in \citep{racefuzzer}, which does away with
the partial-order reduction entirely in favour of a simpler race
condition detection approach. The algorithm consists of two phases:
firstly, all pairs of possibly-racing operations are computed;
secondly, for each pair, execution proceeds with a random
scheduler. When one of the identified statements is about to be
executed, that thread is instead postponed until another thread is
about to execute the other statement, the race is then randomly
resolved and execution continues. Rather than exploring all partial
orders, this approach is a probabilistic one, but is guaranteed to
explore only \emph{racing} partial orders. This approach has an
advantage in programs which have many non-racy partial orders, where
randomly choosing between them does not reliably produce a bug.

% PULSE

\textsc{Pulse} \citep{pulse} is a user-level scheduler for Erlang
programs implementing co-operative multi-tasking. An instrumentation
process automatically modifies existing programs to call out to this
scheduler. \textsc{Pulse} works by only allowing one of the threads to
operate at a time, and makes scheduling decisions around actions with
side-effects: such as a process receiving a message. It also allows
interaction with uninstrumented functions, which are treated as
atomic, allowing tested subsystems to be composed without exploring
interleavings within the subsystem. This is not possible in general in
\dejafu{} due to the support for relaxed memory, which Erlang does not
have. \textsc{Pulse} scheduling decisions are made randomly, using a
provided seed, and a complete execution trace is returned, which can
be rendered into a graphical form showing the interactions between
threads to aid debugging. Although \textsc{Pulse} is not a concurrency
testing tool as such, it is a core component of one, and testing can
be done by simply trying different random seeds. The authors report
that the graphical traces can often suggest potential race conditions
which have not been evident to a human reader.

Sen's 2008 work was then used in \citep{procrastination} to improve
race condition detection in Erlang. \textsc{Pulse} is used to generate
an execution trace, which is then examined for possible race
conditions, which are delayed and randomly resolved as in Sen's
work. The authors reported that improvements can result in new bugs
being found, although in the cases where the procrastination was not
necessary to find the bug, performance degrades. This is because one
test with procrastination is actually several program executions with
different schedules.

% Empirical Studies

% Neither of these include POR techniques. Not sure if particularly
% relevant here, although definitely relevant for wider lit review.

%\todoinline{empirical: Concurrency testing using schedule bounding: an empirical
%  study}

%\todoinline{empirical2: Concurrency Testing Using Controlled Schedulers: An
%  Empirical Study}

% Typeclass Approach

Although the \verb|MonadConc| typeclass was structured to be similar
to the standard concurrency primitives, the inspiration for this
approach, and the basic idea behind how to do SCT in Haskell, was
provided by \citep{typeclass}. However, both the family of primitives
and the approach to testing have been significantly advanced.

% Prior Publication

An earlier version of this work was published as \citep{dejafu}. The
version discussed in this publication does not make any use of
partial-order reduction, relying solely on schedule bounding to prune
the search space. As a result it suffers from state explosion as
programs under test become larger, and is less applicable to
real-world applications. Furthermore, it lacks any support for relaxed
memory. This was originally a design decision, on the assumption that
most real-world concurrent Haskell programs do not use such behaviour,
but initial feedback led to this decision being revised.


  \subsection{Future Work}
  \label{sec:conclusions-future}
  There are a number of areas available for further exploration. 

\begin{description}
\item[Verification of \dejafu{}] \hfill

  Work has already begun on one aspect of this, the formalisation in
  Isabelle/HOL of prefix validity in the SCT implementation. The other
  open issues of verification are discussed further in
  \chap{correctness}, but to summarise, these are: correctness of
  primitive actions; granularity of scheduling decisions; generated
  schedule prefix validity; and result completeness.

\item[Memory model for GHC Haskell / C{-}{-}] \hfill

  In order to fully validate the testing stepwise executor, a
  formalism of the memory model of the primitives used is
  necessary. One way to approach this would be a formalism for all of
  the GHC Haskell primitives. As these are written in C{-}{-}, which
  has no memory model, a formalism of that would also be necessary.

  Work on formalising the C++11 memory model in \citep{c++11} may be
  of use here.

\item[Generating test cases for concurrent APIs] \hfill

  The QuickSpec tool, introduced in \citep{quickspec}, can generate
  laws that a collection of functions appear to hold based on random
  testing. It can be used as a way to easily generate test cases, if
  the user filters the output to laws that \emph{should} hold, rather
  than those which merely \emph{accidentally} hold.

  Given that concurrent programs are now easily testable, some
  QuickSpec-like tool which can generate laws about a
  concurrency-using API would be interesting and useful.

\item[Multi-level memory caching] \hfill

  The current approach taken for modelling relaxed memory assumes only
  a single level of cache. This works well for x86 processors, but not
  for other devices, such as GPUs. GPUs group cores together where
  each core has a cache, and each group also has a cache. This means
  writes can be visible to some but not all threads.

  A simple way to model this would be to make group assignment static,
  and to have more types of commit. This would require some
  implementation change, but is not a large difference in
  algorithm. The situation becomes much more complex if group
  assignment is \emph{not} static however, as this then introduces
  another source of nondeterminism.

\item[Application to distributed systems] \hfill

  There is no reason why different threads in a concurrent program
  need to operate on the same physical machine, as long as the
  programmer cannot detect this.

  The major difficulty is the possibility of communication
  \emph{failure}, which cannot happen when operating on a single
  machine. Another is the memory model. A single level of cache
  corresponds roughly to a central server with all communication going
  through it, rather than between nodes directly. This can be
  alleviated with multiple levels of caching, but still results in
  undesirable centralisation.

  Work on modelling concurrent data stores as replicated
  eventually-consistent data types in \citep{replicated} may be
  relevant.
\end{description}


\if@openright
  \cleardoublepage
\else
  \clearpage
\fi

\bibliography{references}
\bibliographystyle{plainnat}

\end{document}