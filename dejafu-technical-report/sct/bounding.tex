Schedule bounding is an \emph{incomplete} approach to SCT. Each
sequence of scheduling decisions is associated with a \emph{bound
  value}, by some \emph{bound function}. Such a function could be the
number of pre-emptive context switches, for example. Schedule bounding
was introduced in \citep{pbound}, and came from work in the model
checking field.

There are some common bound functions in use today:

\definewordc{Pre-emption Bounding}{pbound}{%
  The number of pre-emptive context switches is bounded.}

\definewordc{Fair Bounding}{fbound}{%
  The difference between the number of times different threads call
  \texttt{yield} is bounded.}

\definewordc{Delay Bounding}{dbound}{%
  The number of deviations from a deterministic scheduler is bounded.}

Both pre-emption bounding and delay bounding have empirical evidence,
in \citep{empirical}, showing that small bounds are good for finding
bugs in many real-world programs.

Fair bounding is used to handle programs which make use of lock-free
constructs such as spinlocks. A spinlock may be implemented like so:

\begin{haskellcode}
lock p var = spin where
  spin = do
    x <- readCRef var
    unless (p x) (yield >> spin)
\end{haskellcode}

Here, a \verb|CRef| is read from and, if some predicate on its value
is not satisfied, the thread yields and tries again. This can easily
give rise to infinitely long executions: simply don't execute any
other thread after the \verb|yield|, as it doesn't \emph{force} a
context switch. Fair bounding bounds the difference between the number
of times that threads have called \verb|yield|: if the thread that has
yielded the fewest times has done so 1 time, and the thread that has
yielded the most times has done so 10 times, then the bound value is
9.

Strictly speaking, schedule bounding refers to trying only those
schedules with a bound value equal to some fixed parameter. A variant
of this is \emph{iterative} bounding, where this parameter is
increased from zero up to some limit. Another variant is where an
inequality, rather than an equality, is used. This explores the same
schedules as iterative bounding, but doesn't impose the same ordering
properties over schedules tried. In practice, ``schedule bounding''
typically refers to this third type, unless specified otherwise.

\dejafu{} uses a combination of pre-emption and fair bounding, with a
pre-emption bound of 2 and a fair bound of 5, in order to gracefully
handle computations which use spinlocking techniques. The pre-emption
bound was chosen based on empirical evidence, but the fair bound was
picked fairly arbitrarily.
