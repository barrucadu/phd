Once the scheduling behaviour of a program can be directed at will,
there is the ability to implement \emph{systematic}
testing. Systematic concurrency testing (SCT) comprises a family of
techniques, all with the same general aim: to try to find bugs in
concurrent programs, more reliably than running a program several
times. Within this scope, there are techniques which are
\emph{complete}, in that they find all possible results a program
could produce; and \emph{incomplete}, which do not make such a
guarantee.

SCT works by providing an initial sequence of scheduling decisions
intended to put the program into a new state. After this point some
deterministic scheduler is used, and the final trace examined to
produce new initial sequences. Typically the assumption is made that
all executions are \emph{terminating}: all possible sequences of
scheduling decisions will lead to a termination by deadlock or
otherwise. Another common assumption is that there is a \emph{finite}
number of possible schedules: this forbids finite but arbitrarily long
executions, as can be created with constructs such as spinlocks.

Systematic testing terminates when there are no more unique initial
sequences possible.
