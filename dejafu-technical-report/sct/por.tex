Partial-order reduction is a \emph{complete} approach to SCT. It is
based on the insight that, when comparing different execution traces,
only the relative ordering of \emph{dependent} actions is
important. Two actions are dependent if the order in which they are
performed could affect the result of the program:

\definewordc{Dependency Relation}{dpor}{%
  Let $\mathcal T$ be the set of transitions in a concurrent system. A
  binary, reflexive, and symmetric relation $\mathcal D \subseteq
  \mathcal T \times \mathcal T$ is a valid \emph{dependency relation}
  iff, for all $t_{1}, t_{2} \in \mathcal T$, $(t_{1}, t_{2}) \notin
  \mathcal D$ ($t_{1}$ and $t_{2}$ are \emph{independent}) implies
  that the following properties hold for all program states $s$:

  \begin{enumerate}
  \item if $t_{1}$ is enabled in $s$ and $s \xrightarrow{t_{1}} s'$,
    then $t_{2}$ is enabled in $s$ iff $t_{2}$ is enabled in $s'$; and

  \item if $t_{1}$ and $t_{2}$ are enabled in $s$, then there is a
    unique state $s'$ such that $s \xrightarrow{t_{1}t_{2}} s'$ and $s
    \xrightarrow{t_{2}t_{1}} s'$.
  \end{enumerate}}

In other words, independent transitions cannot enable or disable each
other, and enabled independent transitions commute. Rather than use
this relational definition directly, typically instead a collection of
conditions sufficient for dependency are identified. These conditions
are determined by what sorts of things the concurrent system under
test can express.

Typically for the presentation of algorithms, a very simple core
concurrent language of just reads and writes is shown. This gives rise
to the following dependency condition:

\begin{align*}
  x \dependent y \iff& \mathrm{thread\_id}(x) = \mathrm{thread\_id}(y) \lor\\
    &\left(\mathrm{variable}(x) = \mathrm{variable}(y)
     \land \left(\mathrm{is\_write}(x) \lor \mathrm{is\_write}(y)\right)\right)
\end{align*}

Where $x \dependent y$ is read as ``$x$ and $y$ are dependent''. This
choice of notation would suggest a symbol $\leftrightarrow$ meaning
independence, but that doesn't seem to be used.

The dependency relation for \dejafu{} is rather more complex than
this, as there are more actions than just reads and writes, however it
can be simplified to a few quite general conditions over different
sorts of reads and writes, with some remaining special cases.

These special cases are:

\begin{haskellcode}
dependent (t1, a1) (t2, a2) = case (a1, a2) of
  (Lift, Lift)   -> True
  (ThrowTo t, _) -> t == t2
  (_, ThrowTo t) -> t == t1
  (STM _, STM _) -> True
\end{haskellcode}

\begin{itemize}
\item Two lifts from the underlying monad are always dependent, as in
  general this allows arbitrary I/O to be performed. The only
  restriction over I/O is that, given a fixed schedule, the I/O is
  deterministic.

\item Throwing an exception to a thread is dependent with anything, as
  all actions can be pre-empted by an exception.

\item STM transactions are always dependent. This final case could
  probably be refined to STM transactions which have some overlap in
  the \verb|CTVar|s they modify, but this is an optimisation which has
  not yet been tried.
\end{itemize}

Furthermore, as a Haskell program terminates when the main thread
terminates, there is a dependency between the last action in a trace
(whatever it may be) and \emph{everything} else.

The general cases are defined in terms of synchronised and
unsynchronised actions. Synchronised actions commit all pending
\verb|CRef| writes, and do not have any relaxed memory
properties. Unsynchronised actions do not have this property.

\begin{haskellcode}
dependentActions memtype buf a1 a2 = case (a1, a2) of
  (UnsynchronisedRead  r1, UnsynchronisedWrite r2) -> r1 == r2
  (UnsynchronisedWrite r1, UnsynchronisedRead  r2) -> r1 == r2
  (UnsynchronisedWrite r1, UnsynchronisedWrite r2) -> r1 == r2

  (UnsynchronisedRead r1, _) | isBarrier a2 -> isBuffered buf r1
  (_, UnsynchronisedRead r2) | isBarrier a1 -> isBuffered buf r2

  _ -> same crefOf && (isSynchronised a1 || isSynchronised a2) || same cvarOf
\end{haskellcode}

\begin{itemize}
\item A read and write to the same \verb|CRef| are dependent, as are
  two writes. The reason for this dependence even under a relaxed
  memory model is because writes give rise to commits, which \emph{do}
  synchronise.

\item An unsynchronised read is dependent with an action that imposes
  a memory barrier if there are buffered writes to the variable being
  read from.

\item Any two actions on the same \verb|CRef| where at least one of
  them will cause a commit are dependent.

\item Any two actions on the same \verb|CVar| are dependent.
\end{itemize}

Characterising the execution of a concurrent program by the ordering
of its dependent actions only gives us a \emph{partial order} over the
actions in the entire program. An execution trace may be just one
possible \emph{total} order corresponding to the same partial
order. The goal of partial-order reduction, then, is to eliminate
these redundant total orders by intelligently making scheduling
decisions to permute the order of dependent actions.

This can be done by executing a program with a deterministic
scheduler, and then examining the trace, the total order, for
\emph{backtracking points}. A backtracking point is a place in the
execution where multiple dependent choices were available, and only
one was tried. The exploration of the state space continues by making
the same scheduling decisions up to that point, and then making a
different decision. This process of doing partial-order reduction
based on information gathered at run-time, rather than static
analysis, is called \emph{dynamic partial-order reduction} (DPOR).

% Implementation in dejafu

In an imperative language, DPOR is usually done by executing the
program under test stepwise in a recursive function, where each stack
frame has a set of decisions still to try, and this is mutated by
later calls when a backtracking point is identified. As \dejafu{} is a
Haskell library, this is not a very natural way to formulate anything,
and so a different approach was taken.

\dejafu{} explicitly constructs a graph structure in memory, where
each path from the root to a leaf corresponds to one complete
execution. Forks in the tree correspond to places where multiple
decisions have been tried. The operation proceeds like so:

\begin{haskellcode}
sctBounded memtype bf run = go initialState where
  go state = case next state of
    Just decisions -> do
      (result, s, trace) <- run decisions
      let bpoints = findBacktrack memtype s trace
      let newBPOR = todo bf bpoints (grow memtype trace state)

      ((result, trace) :) <$> go newState

    Nothing -> return []
\end{haskellcode}
%$

Here \verb|next| returns a schedule prefix; \verb|run| executes the
computation with a given sequence of initial scheduling decisions,
returning the final result, the final scheduler state (which includes
a tentative list of bracktracking points), and the execution trace;
\verb|findBacktrack| identifies a list of actual backtracking points
from these tentative ones; \verb|grow| adds the trace to the tree
structure; and \verb|todo| adds the newly-identified backtracking
points. It is also the responsibility of \verb|todo| to ensure these
new backtracking points do not cause schedules exceeding the bound to
be generated; the \verb|bf| function is the bound function, expressed
as a predicate.

The entire process terminates when \verb|next| returns \verb|Nothing|,
which means that there are no unexplored backtracking points left.

\subsection{Integration with Schedule Bounding}
\label{sec:sct-por-bounding}

The na\"{\i}ve way to integrate DPOR with schedule bounding would be
to first use partial-order techniques to prune the search space, and
then to additionally filter things out with schedule bounding.

Unfortunately, this is unsound. This approach misses parts of the
search space reachable within the bound. This is because the
introduction of the bound introduces new dependencies between actions,
which cannot be determined \emph{a priori}. The solution is to add
\emph{conservative} backtracking points to account for the bound in
addition to any normal backtracking points that are identified. Where
to insert these depends on the bound function.

In the case of pre-emption bounding, it suffices to try all
possibilities at the last context switch before a normal backtracking
point. This is because context switches influence the number of
pre-emptions needed to reach a given program state, depending on which
thread gets scheduled. As pre-emption bounding has been found
empirically to be successful with a low number of threads, and DPOR is
already eliminating a lot of possibilities, this is not in practice a
huge additional cost.

\subsection{Integration with Relaxed Memory}
\label{sec:sct-por-relaxed}

Due to the use of phantom threads, explained in
\sect{execution}{scheduling}, almost nothing needs to be done to
support relaxed memory!

The \verb|dependentActions| function has some knowledge of it in order
to make less pessimistic decisions, as otherwise the assumption would
have to be made that there are always uncommitted writes. The only
other change is related to the integration with schedule bounding: a
pre-emption immediately before (or immediately after) a phantom thread
is free.
