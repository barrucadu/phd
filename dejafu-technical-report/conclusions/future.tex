There are a number of areas available for further exploration. 

\textbf{Verification of \dejafu{}}\hspace{0.3cm} Work has already
begun on the formalisation in Isabelle/HOL of prefix validity in the
SCT implementation. The other open issues of verification were
discussed in \chap{correctness}, but to summarise, these are:
correctness of primitive actions; granularity of scheduling decisions;
generated schedule prefix validity; and result completeness.

\textbf{Memory model for GHC Haskell / C{-}{-}}\hspace{0.3cm} In order
to fully validate the testing stepwise executor, a formulation of the
memory model is necessary. One way to approach this would be a
formulation for all of the GHC Haskell primitives. As these are
written in C{-}{-}, which has no memory model, a formulation of that
would also be necessary. Work on formalising the C++11 memory model in
\citep{c++11} may be of use here.

\textbf{Generating test cases for concurrent APIs}\hspace{0.3cm} The
QuickSpec tool, introduced in \citep{quickspec}, can generate laws
that a collection of functions appear to hold based on random
testing. It can be used as a way to obtain suitable test cases, if the
user selects the generated laws that \emph{should} hold, rather than
those which merely \emph{accidentally} hold. A QuickSpec-like tool
which can generate laws about a concurrency-using API would be
interesting and useful.

\textbf{Multi-level memory caching}\hspace{0.3cm} The current approach
taken for modelling relaxed memory assumes only a single level of
cache. This works well for x86 processors. Other devices, such as
GPUs, group cores together where each core has a cache, and each group
also has a cache. This means writes can be visible to some but not all
threads. A simple way to model this would be to make group assignment
static, and to have more types of commit. This would require some
implementation change, but is not a large difference in algorithm. The
situation becomes much more complex if group assignment is \emph{not}
static, introducing yes another source of nondeterminism.

\textbf{Application to distributed systems}\hspace{0.3cm} Different
threads in a concurrent program need not operate on the same machine,
as long as the programmer cannot detect this. The major difficulty is
the possibility of \emph{communication failure}, which cannot happen
when operating on a single machine. Another is the memory model. A
single level of cache corresponds roughly to a central server with all
communication going through it, rather than between nodes
directly. This can be alleviated with multiple levels of caching, but
still results in undesirable centralisation. Work on modelling
concurrent data stores as replicated eventually-consistent data types
in \citep{replicated} may be relevant.
