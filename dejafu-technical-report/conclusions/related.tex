% Story: history of the algorithm, focussing on the development of
% PULSE (and its probabilistic approach) in particular, and then less
% in-depth surveys of the available tools in general. Finish with a
% brief note of the blog post which inspired the typeclass approach
% used in dejafu.

There is a tension in testing concurrent programs between
\emph{verification} and \emph{bug-finding}. For the former,
completeness is desirable, whereas for the latter completeness can be
sacrificed if the number of defects found in non-contrived examples is
not affected much. Furthermore, by sacrificing completeness, speed can
be gained, which is of great importance for developers running a test
suite as they develop.

% DPOR

Partial-order methods were first introduced in \citep{por}, which also
introduced the insight that a concurrent execution can be thought of
as a \emph{partial-order} of the dependent actions in the
system. Initially, these methods were based on a static analysis of
the program under test. Further developments in \citep{dpor} discuss
how the information needed for partial-order methods can be obtained
at run-time system, and shows that this often leads to a reduction in
the amount of work done. This is because the static analysis is
necessarily \emph{conservative}, whereas the dynamic analysis has much
more complete information available to it.

Meanwhile, a different approach to testing concurrent programs was
being explored in \citep{pbound}, where executions exceeding some
pre-determined \emph{bound} were simply not done. Completeness was
sacrificed in return for more rapid results of testing, on the
assumption that (later to be validated by empirical studies such as
\citep{empirical}) that test cases could be written in such a way that
this wasn't a problem.

It was later shown in \citep{bpor} that these two approaches,
partial-order reduction and schedule bounding, can be unified. The
result is necessarily incomplete, however it can reduce the number of
executions tried to a far greater extent than either of the two
component methods alone. With the evidence that schedule bounding
isn't a problem in practice for testing, this became an enticing
method.

An assumption of key importance in concurrency testing is that all
nondeterminism arises from the scheduler. Most other sources, such as
random number generators, can be controlled for by (for example) using
a fixed seed. However, in the quest for ever more performance,
hardware manufacturers imposed \emph{relaxed memory} architectures on
programmers, where reads and writes done in parallel can give results
impossible under sequential consistency. \citep{rdpor} showed how this
additional source of nondeterminism can be handled, by modelling a
single level of cache (which corresponds to total-store order or
partial-store order) as simply a separate thread, committing writes to
memory.

% Probabilistic DPOR

A different approach to reducing the work done under a pure
partial-order reduction approach was taken in \citep{rapos}, which
uses random scheduling. Partial-order reduction is used to prune the
search space, but random decisions are made where there are still
multiple choices available. Random scheduling itself does not
necessarily work very well, as some partial orders have more
corresponding total orders than others, hence pruning the search space
like this is an effective way to increase the bug-finding ability of
random scheduling.

This work was furthered in \citep{racefuzzer}, which does away with
the partial-order reduction entirely in favour of a simpler race
condition detection approach. The algorithm consists of two phases:
firstly, all pairs of possibly-racing operations are computed;
secondly, for each pair, execution proceeds with a random
scheduler. When one of the identified statements is about to be
executed, that thread is instead postponed until another thread is
about to execute the other statement, the race is then randomly
resolved and execution continues. Rather than exploring all partial
orders, this approach is a probabilistic one, but is guaranteed to
only explore \emph{racing} partial orders. This approach has an
advantage in programs which have many non-racy partial orders, where
randomly choosing between them does not reliably produce a bug.

% PULSE

\textsc{Pulse} \citep{pulse} is a user-level scheduler for Erlang
programs implementing co-operative multi-tasking and an
instrumentation process which automatically modifies existing programs
to call out to this scheduler. \textsc{Pulse} works by only allowing
one of its threads to operate at a time, and to make scheduling
decisions around actions with side-effects: such as a process
receiving a message. It also allows interaction with uninstrumented
functions, which are treated as atomic, allowing tested subsystems to
be composed without needing to test the entire thing again. This is
not possible in general in \dejafu{} due to the support for relaxed
memory, which Erlang does not have. Scheduling decisions are made
randomly, using a provided seed, and a complete execution trace is
returned, which can be rendered into a graphical form showing the
interactions between threads to aid debugging.

Although \textsc{Pulse} is not a concurrency testing tool as such, it
is a core component of one, and testing can be done by simply trying
different random seeds. The authors report that the graphical traces
can often suggest potential race conditions which have not been
directly produced to a human reader.

Sen's 2008 work was then used in \citep{procrastination} to improve
race condition detection in Erlang. \textsc{Pulse} is used to generate
an execution trace, which is then examined for possible race
conditions, which are delayed and randomly resolved as in Sen's
work. The authors reported that this results in new bugs being found,
although in the cases where the procrastination was not necessary to
find the bug, performance degrades. This is because one test with
procrastination is actually several program executions with different
schedules.

% Empirical Studies

% Neither of these include POR techniques. Not sure if particularly
% relevant here, although definitely relevant for wider lit review.

%\todoinline{empirical: Concurrency testing using schedule bounding: an empirical
%  study}

%\todoinline{empirical2: Concurrency Testing Using Controlled Schedulers: An
%  Empirical Study}

% Typeclass Approach

Whilst the \verb|MonadConc| typeclass was structured to be similar to
the standard concurrency primitives, the inspiration for this
approach, and the basic idea behind how to do SCT in Haskell, was
provided by \citep{typeclass}. However, both the family of primitives
and the approach to testing have been significantly advanced.
