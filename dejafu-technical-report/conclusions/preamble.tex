This work was inspired by attending a talk at INVEST
2014\footnote{\url{http://wp.doc.ic.ac.uk/verificationgroup/event/workshop-introduction-to-verification-and-testing-invest-2014/}}
by Paul Thompson, then a Ph.D student at Imperial College London, on
systematic concurrency testing, the talk mentioned tools for languages
like Java and C, but functional languages were not mentioned at
all. The questions to be answered, then, were:

\begin{itemize}
\item \textbf{Can concurrency testing techniques from the imperative
    and object-oriented worlds be applied in the functional world?}
  \hfill

  The answer to this would seem to be a resounding ``yes''. There was
  some difficulty in implementing these techniques in a purely
  functional setting, as the algorithms are typically expressed in
  terms of mutable state, but this was overcome.

\item \textbf{Does the purely functional setting allow for new
    techniques to be developed?} \hfill

  Initially it was hoped that the lack of side-effects in regular
  evaluation, amongst other things, would allow for new techniques to
  be developed. Unfortunately, as concurrency testing explicitly cares
  about \emph{execution} rather than \emph{evaluation} (although these
  are one and the same in most languages), this does not seem to be
  the case.
\end{itemize}

In \citep{dejafu}, we asked if the cost of a programmer needing to
write their code in terms of \verb|MonadConc| rather than \verb|IO|
was too high, and would discourage people. This is still a concern,
but with the development of libraries to integrate with or replace
others, we hope that the use of \dejafu{} will appear attractive
enough to overcome this.

The contributions of this work are:

\begin{itemize}
\item a generalisation of a large subset of the GHC concurrency
  abstraction;

\item a library called \dejafu{} for the systematic testing of
  concurrent Haskell programs, including those using relaxed-memory
  effects.
\end{itemize}
